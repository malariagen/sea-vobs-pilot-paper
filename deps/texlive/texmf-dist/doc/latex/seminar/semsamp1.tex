\listfiles
%%
%% This file may be distributed and/or modified under the conditions of
%% the LaTeX Project Public License, either version 1.2 of this license
%% or (at your option) any later version.  The latest version of this
%% license is in:
%% 
%%    http://www.latex-project.org/lppl.txt
%% 
%% and version 1.2 or later is part of all distributions of LaTeX version
%% 1999/12/01 or later.
%%
%% BEGIN semsamp1.tex
%
% This is a sample document for seminar.sty, v0.93 (and maybe later).
%
% Try this with and without the article option:

\documentclass[fancybox,article]{seminar}

\def\printlandscape{\special{landscape}}    % Works with dvips.

\articlemag{1}

%\twoup   % Try me.

\newpagestyle{327}%
  {Economics 327 \hspace{\fill}\rightmark \hspace{\fill}\thepage}{}%
\pagestyle{327}

\markright{Choice under uncertainty}

\slideframe{Oval}

\newcommand{\heading}[1]{%
  \begin{center}
    \large\bf
    \shadowbox{#1}%
  \end{center}
  \vspace{1ex minus 1ex}}

\newcommand{\BF}[1]{{\bf #1:}\hspace{1em}\ignorespaces}

\begin{document}


\begin{slide}
\heading{A heading}

One thing this example illustrates is how the {\tt article} style option is
good for printing slides two-up, for distribution to a seminar audience or
class, or just for proofreading.

\BF{Definition}
$p$ (weakly) first-order stochastically dominates $q$ if for every $\bar z\in
Z$,
\[
  p(z\leq \bar z) \leq q(z\leq \bar z)
\]
\end{slide}


\begin{slide}
\heading{Problems with stochastic dominance as a DT}

\begin{center}
  \begin{tabular}{|r|l|}\hline
    $z$ & $p(z)$\\ \hline
    \$999 & .01\\ \hline
    \$1,000,000 & .99 \\ \hline
  \end{tabular}%
  \hspace{1cm}%
  \begin{tabular}{|r|l|}\hline
    $z$ & $q(z)$\\ \hline
    \$1,000 & 1\\ \hline
  \end{tabular}
\end{center}
\end{slide}

\begin{slide}
\heading{Candidate Theory \#3: Expected utility}

Let $Z$ be an arbitrary set of outcomes. Let $u:Z\rightarrow R$ be a utility
representation of the DM's preferences over the elements of $Z$ as certain
outcomes. (I.e., $u(y)\geq u(z)$ iff $y \geq z$.)

\end{slide}


\begin{slide}
\heading{Expected utility \& the St.\ Petersburg Paradox}

This can get around even St.\ Petersburg Paradox, because we don't require
that utility be linear in money:

\begin{center}
  \begin{tabular}{r|c|c|c|c|c}\cline{2-6}
    Prize & \$2 & \$4 & \$8 & \$16 & $\ldots$\\ \cline{2-6}
    $u(z)=\log_2(z)$ & 1 & 2 & 3 & 4 & $\ldots$ \\ \cline{2-6}
    Prob. & 1/2 &  1/4 & 1/8 & 1/16 & $\ldots$\\ \cline{2-6}
  \end{tabular}
\end{center}

Expected utility is $\sum_{k=1}^\infty k/2^k = 2$, and so lottery gives same
expected utility as getting \$4 for sure.
\end{slide}


\end{document}
%% END semsamp1.tex
