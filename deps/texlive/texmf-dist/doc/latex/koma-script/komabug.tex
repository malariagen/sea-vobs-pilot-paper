% ======================================================================
% komabug.tex
% Copyright (c) Markus Kohm, 1995-2009
%
% This file is part of the LaTeX2e KOMA-Script bundle.
%
% This work may be distributed and/or modified under the conditions of
% the LaTeX Project Public License, version 1.3c of the license.
% The latest version of this license is in
%   http://www.latex-project.org/lppl.txt
% and version 1.3c or later is part of all distributions of LaTeX
% version 2005/12/01 or later and of this work.
%
% This work has the LPPL maintenance status "author-maintained".
%
% The Current Maintainer and author of this work is Markus Kohm.
%
% This work consists of all files listed in manifest.txt.
% ----------------------------------------------------------------------
% komabug.tex
% Copyright (c) Markus Kohm, 1995-2009
%
% Dieses Werk darf nach den Bedingungen der LaTeX Project Public Lizenz,
% Version 1.3c, verteilt und/oder veraendert werden.
% Die neuste Version dieser Lizenz ist
%   http://www.latex-project.org/lppl.txt
% und Version 1.3c ist Teil aller Verteilungen von LaTeX
% Version 2005/12/01 oder spaeter und dieses Werks.
%
%
% Dieses Werk hat den LPPL-Verwaltungs-Status "author-maintained"
% (allein durch den Autor verwaltet).
%
% Der Aktuelle Verwalter und Autor dieses Werkes ist Markus Kohm.
%
% Dieses Werk besteht aus den in manifest.txt aufgefuehrten Dateien.
% ======================================================================
%%
%% @ is a letter
%%
\catcode`\@=11

%%
%% Grab the initex file list
%%
%% If this file is called via
%%     latex "%%
%% This is file `latexbug.sty',
%% generated with the docstrip utility.
%%
%% The original source files were:
%%
%% latexbug.dtx  (with options: `package')
%% 
%% This is a generated file.
%% 
%% The source is maintained by the LaTeX Project team and bug
%% reports for it can be opened at
%%   http://latex-project.org/bugs/
%% or
%%   https://github.com/latex3/latexbug/
%% 
%% Copyright 2016
%% The LaTeX3 Project and any individual authors listed elsewhere
%% in this file.
%% 
%% This file was generated from file(s) of the Standard LaTeX `latexbug' module.
%% -----------------------------------------------------------------------------
%% 
%% It may be distributed and/or modified under the
%% conditions of the LaTeX Project Public License, either version 1.3c
%% of this license or (at your option) any later version.
%% The latest version of this license is in
%%    http://www.latex-project.org/lppl.txt
%% and version 1.3c or later is part of all distributions of LaTeX
%% version 2005/12/01 or later.
%% 




\NeedsTeXFormat{LaTeX2e}
\ProvidesPackage{latexbug}[2017/10/12 v1.0d Bug-classification]
\def\latexbug@empty{}
\def\latexbug@us{us}
\def\latexbug@usstar{us*}
\def\latexbug@ignore{ignore}
\def\Contact{\MessageBreak \@spaces}
\begingroup
\long\def\latexbug@loop #1 = #2 , {% spaces matter
  \global\@namedef{latexbug@@#1}{#2}%
  \ifx\latexbug@empty#1\else\expandafter\latexbug@loop\fi}
\latexbug@loop
latexbug.sty = us ,
regression-test = us ,
alltt.sty = us ,
ansinew.def = us ,
applemac.def = us ,
article.cls = us ,
article.sty = us ,
ascii.def = us ,
bezier.sty = us ,
bk10.clo = us ,
bk11.clo = us ,
bk12.clo = us ,
book.cls = us ,
book.sty = us ,
cp1250.def = us ,
cp1252.def = us ,
cp1257.def = us ,
cp437.def = us ,
cp437de.def = us ,
cp850.def = us ,
cp852.def = us ,
cp858.def = us ,
cp865.def = us ,
decmulti.def = us ,
doc.sty = us ,
docstrip.tex = us ,
exscale.sty = us ,
fix-cm.sty = us ,
fixltx2e.sty = us ,
flafter.sty = us ,
fleqn.clo = us ,
fleqn.sty = us ,
fltrace.sty = us ,
fontenc.sty = us ,
fontmath.cfg = us ,
fontmath.ltx = us ,
fonttext.cfg = us ,
fonttext.ltx = us ,
graphpap.sty = us ,
hyphen.ltx = us ,
idx.tex = us ,
ifthen.sty = us ,
inputenc.sty = us ,
lablst.tex = us ,
latex.ltx = us ,
latex209.def = us ,
latexbug.tex = us ,
latexrelease.sty = us ,
latexsym.sty = us ,
latin1.def = us ,
latin10.def = us ,
latin2.def = us ,
latin3.def = us ,
latin4.def = us ,
latin5.def = us ,
latin9.def = us ,
lcyenc.dfu = us ,
leqno.clo = us ,
leqno.sty = us ,
letter.cls = us ,
letter.sty = us ,
lppl.tex = us ,
ltluatex.lua = us ,
ltluatex.tex = us ,
ltnews.cls = us ,
ltxcheck.tex = us ,
ltxdoc.cls = us ,
ltxguide.cls = us ,
ly1enc.dfu = us ,
macce.def = us ,
makeidx.sty = us ,
minimal.cls = us ,
newlfont.sty = us ,
next.def = us ,
nfssfont.tex = us ,
oldlfont.sty = us ,
omlcmm.fd = us ,
omlcmr.fd = us ,
omlenc.def = us ,
omllcmm.fd = us ,
omscmr.fd = us ,
omscmsy.fd = us ,
omsenc.def = us ,
omsenc.dfu = us ,
omslcmsy.fd = us ,
omxcmex.fd = us ,
omxlcmex.fd = us ,
openbib.sty = us ,
ot1cmdh.fd = us ,
ot1cmfib.fd = us ,
ot1cmfr.fd = us ,
ot1cmr.fd = us ,
ot1cmss.fd = us ,
ot1cmtt.fd = us ,
ot1cmvtt.fd = us ,
ot1enc.def = us ,
ot1enc.dfu = us ,
ot1lcmss.fd = us ,
ot1lcmtt.fd = us ,
ot2enc.dfu = us ,
ot4enc.def = us ,
preload.cfg = us ,
preload.ltx = us ,
proc.cls = us ,
proc.sty = us ,
report.cls = us ,
report.sty = us ,
sample2e.tex = us ,
sfonts.def = us ,
shortvrb.sty = us ,
showidx.sty = us ,
size10.clo = us ,
size11.clo = us ,
size12.clo = us ,
slides.cls = us ,
slides.def = us ,
slides.sty = us ,
small2e.tex = us ,
source2e.tex = us ,
syntonly.sty = us ,
t1cmdh.fd = us ,
t1cmfib.fd = us ,
t1cmfr.fd = us ,
t1cmr.fd = us ,
t1cmss.fd = us ,
t1cmtt.fd = us ,
t1cmvtt.fd = us ,
t1enc.def = us ,
t1enc.dfu = us ,
t1enc.sty = us ,
t1lcmss.fd = us ,
t1lcmtt.fd = us ,
t2aenc.dfu = us ,
t2benc.dfu = us ,
t2cenc.dfu = us ,
testpage.tex = us ,
texsys.cfg = us ,
textcomp.sty = us ,
tracefnt.sty = us ,
ts1cmr.fd = us ,
ts1cmss.fd = us ,
ts1cmtt.fd = us ,
ts1cmvtt.fd = us ,
ts1enc.def = us ,
ts1enc.dfu = us ,
ucmr.fd = us ,
ucmss.fd = us ,
ucmtt.fd = us ,
ulasy.fd = us ,
ullasy.fd = us ,
utf8-test.tex = us ,
utf8.def = us ,
utf8enc.dfu = us ,
utf8test.tex = us ,
x2enc.dfu = us ,
afterpage.sty = us ,
array.sty = us ,
bm.sty = us ,
calc.sty = us ,
dcolumn.sty = us ,
delarray.sty = us ,
e.tex = us ,
enumerate.sty = us ,
fontsmpl.sty = us ,
fontsmpl.tex = us ,
ftnright.sty = us ,
h.tex = us ,
hhline.sty = us ,
indentfirst.sty = us ,
layout.sty = us ,
longtable.sty = us ,
multicol.sty = us ,
q.tex = us ,
r.tex = us ,
rawfonts.sty = us ,
s.tex = us ,
showkeys.sty = us ,
somedefs.sty = us ,
tabularx.sty = us ,
thb.sty = us ,
thc.sty = us ,
thcb.sty = us ,
theorem.sty = us ,
thm.sty = us ,
thmb.sty = us ,
thp.sty = us ,
trace.sty = us ,
varioref.sty = us ,
verbatim.sty = us ,
verbtest.tex = us ,
x.tex = us ,
xr.sty = us ,
xspace.sty = us ,
UKenglish.sty = us ,
USenglish.sty = us ,
afrikaans.sty = us ,
albanian.sty = us ,
american.sty = us ,
austrian.sty = us ,
babel.def = us ,
babel.sty = us ,
bahasa.sty = us ,
bahasam.sty = us ,
basque.sty = us ,
blplain.tex = us ,
bplain.tex = us ,
breton.sty = us ,
british.sty = us ,
bulgarian.sty = us ,
catalan.sty = us ,
croatian.sty = us ,
czech.sty = us ,
danish.sty = us ,
dutch.sty = us ,
english.sty = us ,
esperanto.sty = us ,
estonian.sty = us ,
finnish.sty = us ,
francais.sty = us ,
galician.sty = us ,
germanb.sty = us ,
greek.sty = us ,
hebrew.sty = us ,
hyphen.cfg = us ,
icelandic.sty = us ,
interlingua.sty = us ,
irish.sty = us ,
italian.sty = us ,
latin.sty = us ,
lsorbian.sty = us ,
luababel.def = us ,
magyar.sty = us ,
naustrian.sty = us ,
ngermanb.sty = us ,
nil.ldf = us ,
norsk.sty = us ,
plain.def = us ,
polish.sty = us ,
portuges.sty = us ,
romanian.sty = us ,
russianb.sty = us ,
samin.sty = us ,
scottish.sty = us ,
serbian.sty = us ,
slovak.sty = us ,
slovene.sty = us ,
spanish.sty = us ,
swedish.sty = us ,
switch.def = us ,
turkish.sty = us ,
ukraineb.sty = us ,
usorbian.sty = us ,
welsh.sty = us ,
xebabel.def = us ,
color.cfg = us ,
color.sty = us ,
dvipdf.def = us ,
dvipdfmx.def = us ,
dvips.def = us ,
dvipsnam.def = us ,
dvipsone.def = us ,
dviwin.def = us ,
emtex.def = us ,
epsfig.sty = us ,
graphics.cfg = us ,
graphics.sty = us ,
graphicx.sty = us ,
keyval.sty = us ,
lscape.sty = us ,
luatex.def = us ,
pctex32.def = us ,
pctexhp.def = us ,
pctexps.def = us ,
pctexwin.def = us ,
pdftex.def = us ,
pdftex.def = us ,
tcidvi.def = us ,
trig.sty = us ,
truetex.def = us ,
xetex.def = us ,
oldgerm.sty = us ,
ot1panr.fd = us ,
ot1pss.fd = us ,
pandora.sty = us ,
uyfrak.fd = us ,
uygoth.fd = us ,
uyinit.fd = us ,
uyswab.fd = us ,
amsbsy.sty = us ,
amscd.sty = us ,
amsgen.sty = us ,
amsmath.sty = us ,
amsopn.sty = us ,
amstex.sty = us ,
amstext.sty = us ,
amsxtra.sty = us ,
amsart.cls = us ,
amsbook.cls = us ,
amsbooka.sty = us ,
amsdtx.cls = us ,
amsldoc.cls = us ,
amsmidx.sty = us ,
amsproc.cls = us ,
amsthm.sty = us ,
upref.sty = us ,
amsfonts.sty = us ,
amssymb.sty = us ,
cmmib57.sty = us ,
eucal.sty = us ,
eufrak.sty = us ,
euscript.sty = us ,
ueuex.fd = us ,
ueuf.fd = us ,
ueur.fd = us ,
ueus.fd = us ,
umsa.fd = us ,
umsb.fd = us ,
8rbch.fd = us ,
8rpag.fd = us ,
8rpbk.fd = us ,
8rpcr.fd = us ,
8rphv.fd = us ,
8rpnc.fd = us ,
8rppl.fd = us ,
8rptm.fd = us ,
8rput.fd = us ,
8rpzc.fd = us ,
avant.sty = us ,
bookman.sty = us ,
chancery.sty = us ,
charter.sty = us ,
courier.sty = us ,
helvet.sty = us ,
mathpazo.sty = us ,
mathpple.sty = us ,
mathptm.sty = us ,
mathptmx.sty = us ,
newcent.sty = us ,
omlbch.fd = us ,
omlpag.fd = us ,
omlpbk.fd = us ,
omlpcr.fd = us ,
omlphv.fd = us ,
omlpnc.fd = us ,
omlppl.fd = us ,
omlptm.fd = us ,
omlptmcm.fd = us ,
omlput.fd = us ,
omlpzc.fd = us ,
omlzplm.fd = us ,
omlzpple.fd = us ,
omlztmcm.fd = us ,
omsbch.fd = us ,
omspag.fd = us ,
omspbk.fd = us ,
omspcr.fd = us ,
omsphv.fd = us ,
omspnc.fd = us ,
omsppl.fd = us ,
omsptm.fd = us ,
omsput.fd = us ,
omspzc.fd = us ,
omspzccm.fd = us ,
omszplm.fd = us ,
omszpple.fd = us ,
omsztmcm.fd = us ,
omxpsycm.fd = us ,
omxzplm.fd = us ,
omxzpple.fd = us ,
omxztmcm.fd = us ,
ot1bch.fd = us ,
ot1pag.fd = us ,
ot1pbk.fd = us ,
ot1pcr.fd = us ,
ot1phv.fd = us ,
ot1pnc.fd = us ,
ot1ppl.fd = us ,
ot1pplj.fd = us ,
ot1pplx.fd = us ,
ot1ptm.fd = us ,
ot1ptmcm.fd = us ,
ot1put.fd = us ,
ot1pzc.fd = us ,
ot1zplm.fd = us ,
ot1zpple.fd = us ,
ot1ztmcm.fd = us ,
palatino.sty = us ,
pifont.sty = us ,
t1bch.fd = us ,
t1pag.fd = us ,
t1pbk.fd = us ,
t1pcr.fd = us ,
t1phv.fd = us ,
t1pnc.fd = us ,
t1ppl.fd = us ,
t1pplj.fd = us ,
t1pplx.fd = us ,
t1ptm.fd = us ,
t1put.fd = us ,
t1pzc.fd = us ,
times.sty = us ,
ts1bch.fd = us ,
ts1pag.fd = us ,
ts1pbk.fd = us ,
ts1pcr.fd = us ,
ts1phv.fd = us ,
ts1pnc.fd = us ,
ts1ppl.fd = us ,
ts1pplj.fd = us ,
ts1pplx.fd = us ,
ts1ptm.fd = us ,
ts1put.fd = us ,
ts1pzc.fd = us ,
ufplm.fd = us ,
ufplmbb.fd = us ,
upsy.fd = us ,
upzd.fd = us ,
utopia.sty = us ,
expl3.sty  = us* ,
l3sort.sty = us* ,
l3regex.sty = us* ,
l3tl-analysis.sty = us* ,
xfrac.sty = us* ,
xparse.sty = us* ,
xtemplate.sty = us* ,
xfrac.sty = us* ,
xcoffins.sty = us* ,
xgalley.sty = us* ,
l3keys2e.sty = us* ,
expl3-code.tex = ignore ,
blindtext.sty = ignore ,
etoolbox.sty = ignore ,
lipsum.sty   = ignore ,
beamer.cls   = Joseph Wright
               \Contact https://github.com/josephwright/beamer/issues ,
fontspec.sty = Will Robertson
               \Contact https://github.com/wspr/fontspec/issues ,
geometry.sty = Hideo Umeki
               \Contact <latexgeometry [at] gmail [dot] com> ,
hpdftex.def  = ignore ,
hluatex.def  = ignore ,
hxetex.def   = ignore ,
hyperref.cfg = ignore ,
hyperref.sty = Heiko Oberdiek
               \Contact https://github.com/ho-tex/hyperref/issues ,
luatex85.sty   = Joseph Wright
               \Contact https://github.com/josephwright/luatex85/issues ,
pd1enc.def   = ignore ,
puenc.def    = ignore ,
siunitx.sty  = Joseph Wright
               \Contact https://github.com/josephwright/siunitx/issues ,
basque.ldf      = Juan M. Aguirregabiria
                  \Contact <http://tp.lc.ehu.es/jma.html> ,
belarusian.ldf  = Aleksey Novodvorsky, Andrew Shadura
                  \Contact <andrew [at] shadura [dot] me> ,
bosnian.ldf     = Samir Halilcevic %% Halil^^c4^^8devi^^c4^^87
                  \Contact <samir [dot] halilcevic [at] fet [dot] ba> ,
bulgarian.ldf   = Georgi N. Boshnakov
                  \Contact <Georgi Boshnakov [at] manchester [dot] ac [dot] uk> ,
croatian.ldf    = Ivan Kokan
                  \Contact <ivan [dot] kokan [at] gmail [dot] com> ,
estonian.ldf    = Jaan Vajakas
                  \Contact <jaanvajakas [at] hot [dot] ee> ,
frenchb.ldf     = Daniel Flipo
                  \Contact <daniel [dot] flipo [at] free [dot] fr> ,
friulan.ldf     = Claudio Beccari
                  \Contact <claudio [dot] beccari [at] gmail [dot] com> ,
georgian.ldf    = Levan Shoshiashvili
                  \Contact <shoshia [at] hotmail [dot] com> ,
austrian.ldf    = Juergen Spitzmueller
                  \Contact <juergen [at] spitzmueller [dot] org> ,
german.ldf      = Juergen Spitzmueller
                  \Contact <juergen [at] spitzmueller [dot] org> ,
germanb.ldf     = Juergen Spitzmueller
                  \Contact <juergen [at] spitzmueller [dot] org> ,
naustrian.ldf   = Juergen Spitzmueller
                  \Contact <juergen [at] spitzmueller [dot] org> ,
ngerman.ldf     = Juergen Spitzmueller
                  \Contact <juergen [at] spitzmueller [dot] org> ,
ngermanb.ldf    = Juergen Spitzmueller
                  \Contact <juergen [at] spitzmueller [dot] org> ,
nswissgerman.ldf = Juergen Spitzmueller
                  \Contact <juergen [at] spitzmueller [dot] org> ,
swissgerman.ldf = Juergen Spitzmueller
                  \Contact <juergen [at] spitzmueller [dot] org> ,
greek.ldf       = Guenter Milde
                  \Contact <milde [at] users [dot] sf [dot] net> ,
magyar.ldf      = Peter Szabo %% P^^c3^^a9ter Szab^^c3^^b3
                  \Contact <http://www.math.bme.hu/latex/> ,
italian.ldf     = Claudio Beccari
                  \Contact <claudio [dot] beccari [at] gmail [dot] com> ,
japanese.ldf    = Japanese TEX Development Community
                  \Contact <https://github.com/texjporg/babel-japanese> ,
latin.ldf       = Claudio Beccari
                  \Contact <claudio [dot] beccari [at] gmail [dot] com> ,
macedonian.ldf  = Stojan Trajanovski
                  \Contact <stojan [dot] trajanovski [at] gmail [dot] com> ,
occitan.ldf     = Cedric Valmary %% C^^c3^^a9dric Valmary
                  \Contact <cvalmary [at] yahoo [dot] fr> ,
piedmontese.ldf = Claudio Beccari
                  \Contact <claudio [dot] beccari [at] gmail [dot] com> ,
pinyin.ldf      = Werner Lemberg
                  \Contact <wl [at] gnu [dot] org> ,
romansh.ldf     = Claudio Beccari
                  \Contact <claudio [dot] beccari [at] gmail [dot] com> ,
russain.ldf     = Igor A [dot]  Kotelnikov
                  \Contact <kia999 [at] mail [dot] ru> ,
serbianc.ldf    = Filip Brcic
                  \Contact <brcha [at] users [dot] sourceforge [dot] net> ,
spanglish.ldf   = J [dot]  Luis Rivera
                  \Contact <jlrn77 [at] gmail [dot] com> ,
spanish.ldf     = Javier Bezos
                  \Contact <http://www.texnia.com/contact.html> ,
thaicjk.ldf     = Werner Lemberg
                  \Contact <wl [at] gnu [dot] org> ,
ukraineb.ldf    = Sergiy Ponomarenko
                  \Contact <sergiy [dot] pono^^c2^^admarenko [at] gmail [dot] com> ,
vietnamese.ldf  = Werner Lemberg
                  \Contact <wl [at] gnu [dot] org> ,
{\latexbug@empty} = {} ,
\endgroup
\let\latexbug@addtofilelist\@addtofilelist
\def\latexbug@zzzz{}
\def\latexbug@expl{}
\let\latexbug@process@table\process@table
\def\process@table{\global\let\@addtofilelist\latexbug@addtofilelist
  \latexbug@process@table
  \ifx\latexbug@zzzz\latexbug@empty
    \ifx\latexbug@expl\latexbug@empty
    \else
      \PackageError{latexbug}%
        {LaTeX3 file(s)\MessageBreak
\MessageBreak
This test files uses the LaTeX3 file(s)\MessageBreak
\MessageBreak
        ==============\MessageBreak
        \latexbug@expl
        ==============\MessageBreak
        \MessageBreak
You should report bugs in these packages\MessageBreak
at the LaTeX3 GitHub site,\MessageBreak
https://github.com/latex3/latex3/issues\MessageBreak
(Or remove them from your example,\MessageBreak
if they are not necessary to\MessageBreak
exhibit the problem).\MessageBreak
}{Please correct your test file prior
                     to submitting the bug report.\MessageBreak
                     Otherwise it is likely to be rejected!}%
    \fi
  \else
      \PackageError{latexbug}%
                   {Third-party file(s)\MessageBreak
\MessageBreak
This test file uses third-party file(s)\MessageBreak
\MessageBreak
        ==============\MessageBreak
        \latexbug@zzzz
        ==============\MessageBreak
        \MessageBreak
So you should contact the authors\MessageBreak
of these files, not the LaTeX Team!\MessageBreak
(Or remove the packages that load\MessageBreak
them, if they are not necessary to\MessageBreak
exhibit the problem).\MessageBreak
\MessageBreak
If you think the bug is in core LaTeX\MessageBreak
(as maintained by the LaTeX Team) but\MessageBreak
these files are needed to demonstrate\MessageBreak
the problem, please continue and mention\MessageBreak
this explicitly in your bug report}{Please correct your test file prior
                     to submitting the bug report.\MessageBreak
                     Otherwise it is likely to be rejected!}%
  \fi
}
\def\@addtofilelist#1{%
  \expandafter\latexbug@iftoplevel\@currnamestack {}\@nil
  \begingroup
    \xdef\latexbug@x{#1}% TODO: one-level sanitize
    \expandafter
    \ifx\csname latexbug@@\latexbug@x\endcsname\relax
      \def\latexbug@y{}%
    \else
      \xdef\latexbug@y{\csname latexbug@@\latexbug@x\endcsname}%
    \fi
    \ifx\latexbug@y\latexbug@us\else
      \ifx\latexbug@y\latexbug@usstar
        \begingroup
          \let\MessageBreak\relax
          \xdef\latexbug@expl{%
            \latexbug@expl\latexbug@x
            \MessageBreak
          }%
        \endgroup
      \else
        \ifx\latexbug@y\latexbug@ignore\else
           {\let\MessageBreak\relax
             \xdef\latexbug@zzzz{\latexbug@zzzz\latexbug@x
               \ifx\latexbug@y\latexbug@empty\else
                 \space\space -> \space \latexbug@y\fi
               \MessageBreak}}%
        \fi
      \fi
    \fi
    \endgroup
  \fi
  \latexbug@addtofilelist{#1}}
\def\latexbug@iftoplevel #1#2\@nil{%
  \def\next{#1}%
  \ifx\next\@empty
}
\endinput
%%
%% End of file `latexbug.sty'.
" or some
%% similar command sequence rather than
%%     latex latexbug
%% then the debugging info in \reserved@a will already have been lost.
%% This might not matter, but if it does we may ask the user to resubmit
%% the report.
\ifx\reserved@b\@undefined
  \ifx\reserved@a\@gobble
    \def\@inputfiles{NONE}
  \else
    \let\@inputfiles\reserved@a
  \fi
\else
  \def\@inputfiles{LOST}
\fi

%%
%% Output stream to produce the bug report template.
%%
\newif\ifinteractive\interactivetrue
\newwrite\@msg
\immediate\openout\@msg=\jobname.msg
\immediate\write\@msg{%
An fatal error occured before writing anything to the message file.^^J%
You should have a look at \jobname.log to see the reason of the error.^^J%
^^J%
Maybe you didn't use an interactive terminal to run ``latex komabug''.^^J%
At this case you should try an interactive terminal or add^^J%
\space\space\space\space\string\interactivefalse^^J%
to your local ``komabug.cnf'' to create an empty message.^^J%
You have to edit that message file after creating the empty message.^^J%
}
\immediate\closeout\@msg\let\msg\m@ne

%%
%% We have no end line char (so we have no paragraphs)
%%
\endlinechar=-1

%%
%% Check that LaTeX2e is being used.
%%
\ifx\undefined\newcommand
 \newlinechar`\^^J%
 \immediate\write17{^^J%
    You must use LaTeX2e to generate the bug report!^^J^^J%
    Sie muessen LaTeX2e verwenden, um die Fehlermeldung zu erzeugen!}%

 \let\relax\end
\else
 \def\@tempa{LaTeX2e}\ifx\@tempa\fmtname\else
  \immediate\write17{^^J%
   You must use LaTeX2e to generate the bug report!^^J^^J%
   Aeltere Versionen von LaTeX werden nicht unterstuetzt.^^J%
   Sie muessen LaTeX2e verwenden, um die Fehlermeldung zu erzeugen!}%
  \let\relax\@@end
\fi\fi

%%
%% \wmsg writes to the terminal, and the .msg file
%% \wmsg* just writes to the .msg file
%% \typeout just writes to the terminal
%%

\def\wmsg{%
  \ifnum\msg<0\relax\let\msg\@msg\immediate\openout\msg=\jobname.msg\fi
  \begingroup
    \@ifstar {\interactivefalse\@wmsg}{\@wmsg}
}

\def\@wmsg#1{%
    \ifinteractive\immediate\write17{#1}\fi%
    \immediate\write\msg{#1}%
  \endgroup
}

%%
%% Prompt for an answer from the user, if the answer is not
%% provided by the cfg file.
%%

\def\readifnotknown#1{%
 \@ifundefined{#1}%
    {{\message{#1> }%
     \catcode`\^^I=12 \let\do\@makeother\dospecials
     \global\read\m@ne t\expandafter o\csname#1\endcsname}}%
    {\message{\csname#1\endcsname}}}

%%
%% Get number
%% #1 = message
%% #2 = max value
%% --> \answer
%%
\def\scannumber#1{\@tempswatrue
  \typeout{`#1'}%
  \def\@tempa{\expandafter\@scannumber\expandafter0#1\@nil}%
  \edef\@tempa{\@tempa}%
  \count@\@tempa\relax
  \if@tempswa\else\count@\z@\relax\fi
}%
\def\@scannumber#1{%
  \ifx\@nil#1\else
    \if 0#1#1\else
      \ifnum 0<0#1 #1\else
        \noexpand\@tempswafalse%
      \fi
    \fi
    \expandafter\@scannumber
  \fi
}%
\def\ReadNumber#1#2{%
  \typeout{#1}%
  \ifenglish
    \message{Input the corresponding number between 1 and #2:  }%
  \else
    \message{Geben Sie die betreffende Zahl zwischen 1 und #2 ein:  }%
  \fi
  \count@=\z@\relax
  \read\m@ne to \answer
  \scannumber{\answer}%
  \ifnum\count@>\number#2\relax
    \count@=\z@\relax
  \fi
%  \typeout{\the\count@ > 0?}%
  \ifnum\count@>\z@\relax
%    \typeout{YES}%
    \advance\count@ by \m@ne
    \edef\answer{\the\count@}%
    \message{^^J}%
  \else
%    \typeout{NO}%
    \ifenglish
      \typeout{Value not valid!}%
    \else
      \typeout{Wert nicht im erlaubten Bereich!}%
    \fi
    \pause
    \def\@tempa{\ReadNumber{#1}{#2}}%
    \expandafter\@tempa
  \fi
}

%%
%% Get Yes or No
%% #1 Question
%% --> \if@tempswa (yes is true, no is false)
%%
\def\ReadYesNo#1{%
  \typeout{#1}%
  \ifenglish
    \message{Answer `yes' or `no':  }%
    \read\m@ne to \answer
  \else
    \message{Antworten Sie bitte mit `ja' oder `nein': }%
    \read\m@ne to \answer
    \edef\answer{\uccode`\expandafter\@car\answer\@nil}
    \ifnum \answer=`J \def\answer{\uccode`Y}\fi
  \fi
  \def\@tempa{\message{^^J}}%
  \ifnum \answer=`Y \@tempswatrue
  \else
    \ifnum \answer=`N \@tempswafalse
    \else
      \ifenglish
        \typeout{Answer not valid!}%
      \else
        \typeout{Antwort unverstaendlich!}%
      \fi
      \def\@tempa{\ReadYesNo{#1}}%
    \fi
  \fi
  \@tempa
}

%%
%% Pause so messages do not scroll off screen.
%%
\def\pause{%
  \ifinteractive
    \ifenglish
      \message{Press <return> to continue. }%
    \else
      \message{Mit der <Return>-Taste geht es weiter. }%
    \fi
    \read\m@ne to \@tempa
    \message{^^J}%
  \fi
}

%%
%% german or english report generator
%%
\newif\ifenglish\englishfalse

%%
%% Opening Banner.
%%

\InputIfFileExists{komabug.cfg}{%
  \ifenglish
    \typeout{** using komabug.cfg **}%
  \else
    \typeout{** komabug.cfg wird verwendet **}%
  \fi
}{}

\ifenglish
  \typeout{^^J%
    ============================================================^^J%
    ^^J%
    KOMA bug report generator^^J%
    =========================^^J%
    Running this file through LaTeX generates a formular ``\jobname.msg''^^J%
    containing a bug report to KOMA-Script bundle.^^J^^J%
    * Please use german, if possible.^^J \space
      If you're not able to use german, write the report in english.^^J%
    * Please write a short report not a large one.^^J%
    * Please tell me everything, which may be important.^^J}
  \pause
\else
  \typeout{^^J%
    ============================================================^^J%
    ^^J%
    KOMA-Script Fehlermeldungsgenerator^^J%
    ===================================^^J%
    Die Bearbeitung dieser Datei mit LaTeX erzeugt das Formular \jobname.msg,^^J%
    um Fehlermeldungen zum KOMA-Script-Paket zu melden.^^J^^J%
    * Schreiben Sie Ihre Meldung nach Moeglichkeit in Deutsch.^^J \space
      Notfalls ist auch Englisch moeglich.^^J%
    * Bitte fassen Sie sich kurz.^^J%
    * Bitte halten Sie keine Information zurueck, die moeglicherweise^^J \space
    wichtig sein koennte.^^J}%
  \typeout{%
    If you prefere english, you may write a file ``komabug.cfg'' with\space 
    contents:^^J\space
    \string\englishtrue^^J%
    at same folder as ``komabug.tex'' and restart ``latex komabug.tex''.^^J}%
  \englishtrue\pause\englishfalse
\fi

%%
%% if \interactivefalse just make a blank template.
%%

\ifinteractive
  \ifenglish
    \ReadYesNo{%
      There are two kinds of using this generator.^^J%
      At the interactive mode, you have to answer questions. At the other^^J%
      mode an empty formular will be generated, you have to fill using^^J%
      an editor.^^J^^J%
      Do you want an interactive session?
    }%
  \else
    \ReadYesNo{%
      Dieser Generator kann auf zwei Arten verwendet werden.^^J%
      Im interaktiven Betrieb, werden Ihnen Fragen zur direkten
      Beantwortung^^J%
      gestellt. Ansonsten wird ein leeres Formular erzeugt, das Sie dann
      mit^^J% 
      einem Editor ausfuellen muessen.^^J^^J%
      Wollen Sie eine interaktive Sitzung?
    }%
  \fi
  \if@tempswa\interactivetrue\else\interactivefalse\fi
\fi

%%
%% Fatal error
%%
\def\fatalerror#1{%
  \ifenglish
    \errhelp{This error is fatal! You cannot continue.^^J%
      You should terminate using `x' and restart ``latex komabug''.}%
    \errmessage{#1}%
  \else
    \errhelp{Dieser Fehler erlaubt keine Fortsetzung der Bearbeitung.^^J%
      Sie sollten mit `x' abbrechen und ``latex komabug'' neu starten.}%
    \errmessage{#1}%
  \fi
  \batchmode\csname @@end\endcsname
  \fatalerror{#1}%
}

%%
%% Try to get Version
%%
\def\GetFileVersionWithExtend#1#2{%
  \def\ProvidesFile##1{\@ifnextchar [{\@P@F}{\@P@F[1996/10/31 ]}}%
  \def\@P@F[##1 ##2]{\xdef\komaversion{##1}\csname endinput\endcsname}%
  \let\ProvidesClass\ProvidesFile
  \let\ProvidesPackage\ProvidesFile
  \InputIfFileExists{#1#2}{}{%
    \ifenglish
      \def\noscrclass{%
        ! File ``#1#2'' not found!^^J%
        ! The file must be at the same folder like ``komabug.tex''^^J%
        ! or must be readable by LaTeX to get the version information!}%
    \else
      \def\noscrclass{%
        ! Die Datei ``#1#2'' konnte nicht gefunden werden!^^J%
        ! Zur Bestimmung der aktuellen Version ist es unbedingt^^J%
        ! erforderlich, dass diese Datei sich im selben Verzeichnis^^J%
        ! wie ``komabug.tex'' befindet oder zumindest von LaTeX^^J%
        ! gefunden werden kann!}%
    \fi
    \errmessage{\noscrclass}
    \ifenglish
      \errhelp{Terminate TeX using `x' and restart komabug}
    \else
      \errhelp{Beenden Sie TeX mit `x' und starten Sie komabug neu}
    \fi
  }%
}
\def\GetVersion#1{%
  \GetFileVersionWithExtend{#1}\@empty
}

\ifinteractive
  \ifenglish
    \ReadNumber{%
      There are several categories, related to several parts and files^^J%
      of the KOMA-Script bundle:^^J^^J
      1) Installation of KOMA-Script^^J
      2) KOMA-Script manual^^J
      3) Basics (scrbook, scrreprt, scrartcl, scrlttr2, typearea, scrlfile)^^J
      4) Pagestyle definition (scrpage2)^^J
      5) Time or date (scrtime, scrdate)^^J
      6) Address file handling (scraddr)^^J
      7) Obsolete (scrlettr, scrpage)^^J%
    }{7}%
  \else
    \ReadNumber{%
      Verschiedene Bereiche werden von diesem Generator unterstuetzt, die^^J%
      sich auf verschiedene Dateien im KOMA-Script-Paket beziehen:^^J^^J
      1) Auspacken und Installieren von KOMA-Script^^J
      2) KOMA-Script-Anleitung^^J
      3) Grundfunktion (scrbook, scrreprt, scrartcl, scrlttr2, typearea,
      scrlfile)^^J
      4) Definition von Seitenstilen (scrpage2)^^J
      5) Zeit oder Datum (scrtime, scrdate)^^J
      6) Umgang mit Adressdateien (scraddr)^^J
      7) Obsoletes (scrlettr, scrpage)^^J%
    }{7}%
  \fi

  \ifcase\expandafter\number\answer
    % 1
    \def\category{Installation}\GetVersion{scrkernel-version.dtx}
    \let\categoryversion\komaversion
  \or
    % 2
    \def\category{Manual}\GetVersion{scrartcl.cls}
    \let\categoryversion\komaversion
  \or
    % 3
    \def\category{Basics}\GetVersion{scrartcl.cls}
    \let\categoryversion\komaversion
  \or
    % 4
    \def\category{Addons, scrpage2}\GetVersion{scrpage2.sty}%
    \let\categoryversion\komaversion\GetVersion{scrartcl.cls}
  \or
    % 5
    \def\category{Addons, scrtime/date}\GetVersion{scrtime.sty}%
    \let\categoryversion\komaversion\GetVersion{scrartcl.cls}
  \or
    % 6
    \def\category{Addons, scraddr}\GetVersion{scraddr.sty}%
    \let\categoryversion\komaversion\GetVersion{scrartcl.cls}
  \or
    % 7
    \ifenglish
      \typeout{These obsolete parts of KOMA-Script are unsupported!^^J%
        You should use scrlttr2 instead of scrlettr and scrpage2 instead of
        scrpage.}
    \else
      \typeout{Fuer diese obsoleten Teile von KOMA-Script gibt es keinen
        Support mehr!^^J%
        Sie sollten scrlttr2 an Stelle von scrlettr bzw. scrpage2 an Stelle
        von^^J%
        scrpage verwenden.}
    \fi
    \batchmode\csname @@end\endcsname
  \fi
\else% \ifinteractive
  \ifenglish
    \def\category{<PLEASE REPLACE BY ONE OF `Installation', `Manual',
      `Basics', `Addons, scrpage2', `Addons, scrtime/date', `Addons,
      scraddr'.>}
    \def\categoryversion{<PLEASE REPLACE BY VERSION DATE FROM USED CLASS OR
      PACKAGE.>}
  \else
    \def\category{<BITTE DURCH EINE DER ANGABEN `Installation', `Manual',
      `Basics', `Addons, scrpage2', `Addons, scrtime/date', `Addons,
      scraddr' ERSETZEN.>}
    \def\categoryversion{<BITTE DURCH DAS VERSIONSDATUM AUS DER ENTSPRECHENDEN
      KLASSE BZW. DEM ENTSPRECHENDEN PAKET ERSETZEN.>}
  \fi
  \GetVersion{scrartcl.cls}
\fi

\ifinteractive
  \ifenglish
    \typeout{^^J%
      ===========================================================^^J%
      ^^J%
      Please give a one line description (< 50 chars) of your problem.%
      ^^J^^J%
      If your using email for sending the report, please use this^^J%
      description as subject, too:%
      ^^J \@spaces\@spaces\space
      |<------------------------------------------------>|}
  \else
    \typeout{^^J%
      ============================================================^^J%
      ^^J%
      Bitte eine einzeilige (< 50 Zeichen) Beschreibung des Problems.%
      ^^J^^J%
      Wenn Sie fuer die Meldung eMail verwenden, setzen Sie diese
      Beschreibung^^J%
      bitte auch als Betreff (`Subject') der eMail ein:%
      ^^J \@spaces\@spaces\space
      |<------------------------------------------------>|}
  \fi
  \loop
    \let\synopsis\relax
    \readifnotknown{synopsis}
    \ifx\synopsis\@empty
  \repeat
\else%\ifinteractive
  \ifenglish
    \def\synopsis{<PLEASE REPLACE BY SHORT ONE-LINE DESCRIPTION.>}
  \else
    \def\synopsis{<BITTE DURCH EINE KURZE, EINZEILIGE BESCHREIBUNG ERSETZEN.>}
  \fi
\fi

%%
%% Header in the msg file.
%%
\ifenglish
  \wmsg*{^^J%
    KOMA bug report.^^J%
    \ifinteractive Interactive \else Formular \fi
    generated using komabug.tex at
    \space\number\year-\two@digits\month-\two@digits\day.^^J%
    ^^J
    You may send this message to komascript@gmx.info.^^J%
    If you do so, please use subject:^^J%
    \space KOMA-BUG:\space\synopsis^^J%
    ============================================================^^J
  }
\else
  \wmsg*{^^J%
    KOMA-Fehlermeldung.^^J%
    \ifinteractive Interaktiv \else Formular \fi
    erzeugt mit komabug.tex am
    \space\number\year-\two@digits\month-\two@digits\day.^^J%
    ^^J%
    Die Meldung kann per E-Mail an komascript@gmx.info^^J%
    verschickt werden.^^J%
    Bitte verwenden Sie dabei als Betreff:^^J%
    \space KOMA-BUG:\space\synopsis^^J%
    ============================================================^^J
  }
\fi

%%
%% Category of bug, obtained earlier but put out now, after the header.
%%
\wmsg{>Bereich: \category}
\wmsg{>Version: \categoryversion}

%%
%% synopsis of bug, obtained earlier but put out now, after the header.
%%
\wmsg{>Betreff: \synopsis}

\begingroup
 \global\let\format\@empty
 \gdef\hyphenation{standard}
 \def\immediate#1#{\xdef\hyphenation}
 \def\typeout#1{\xdef\format{\format#1}}
 \the\everyjob
\endgroup

\wmsg{>Format: \format}

\wmsg{>KOMA-Script: \komaversion}

\ifinteractive
%%
%% if interactive, \wread reads a line (verbatim) and write it to the
%% .msg file, until a blank line is entered.
%%
  \def\wread{{%
      \catcode`\^^I=12
      \let\do\@makeother\dospecials
      \let\lastanswer\answer
      \message{=> }\read\m@ne to \answer
      \ifx\lastanswer\@empty
        \let\lastanswer\answer
      \fi
      \ifx\lastanswer\@empty
      \else
        \immediate\write\msg{\answer}
        \expandafter\wread
      \fi
    }%
  }
\else
%%
%% If non-interactive, \wread just writes a blank line to the .msg file,
%% and \wmsg does not write to the terminal.
%%
  \def\wread{\wmsg{}}
\fi

%%
%% \copytomsg copies the contents of a file into the .msg file.
%% (at least it does it as well as TeX can, so there may be
%% transcription problems with 8-bit characters).
%%
%% It does a line count, and complains if the test file is
%% too large.

\chardef\inputfile=15

\newcount\linecount

\def\copytomsg#1{{%
    \endlinechar=-1
    \def\do##1{\catcode`##1=11}%
    \dospecials
    \global\linecount\z@
    \openin\inputfile#1\relax
    \def\thefile{#1}%
    \@copytomsg
    \closein\inputfile}}

\def\@copytomsg{%
   \ifeof\inputfile
      \typeout{*** \thefile\space Zeilen = \the\linecount}
   \else
      \global\advance\linecount\@ne
      \read\inputfile to \inputline
      \ifx\inputline\@empty
         \wmsg*{}
      \else
         \wmsg*{\inputline}
      \fi
      \expandafter\@copytomsg
   \fi}


%%
%% Test the age of the current format.
%%
\def\getage#1/#2/#3\@nil{%
  \count@\year
  \advance\count@-#1\relax
  \multiply\count@ by 12\relax
  \advance\count@\month
  \advance\count@-#2\relax}
%
\expandafter\getage\fmtversion\@nil
%%
%% \count@ should now be the age of the format in months.
%%
\ifnum\count@>24
  \ifenglish
    \def\oldformat{^^J%
      ! Your LaTeX installation is older than two years.^^J%
      ! Updating woul dbe a good idea before sending this report.^^J%
      ! You should compare the date of the package with the date of^^J
      ! the used LaTeX version. If LaTeX is more than two years older^^J
      ! than KOMA-Script, this could be the reason of the problem.}
    \errhelp{If you want to continue, press <return>.}
  \else
    \def\oldformat{^^J%
      ! Ihre LaTeX-Installation ist ueber zwei Jahre alt.^^J%
      ! Bitte denken Sie ueber ein Update nach, ehe Sie diese Meldung^^J%
      ! abschicken.^^J%
      ! Vergleichen Sie wenigstens das Datum des Paketes mit dem Datum^^J%
      ! der LaTeX-Version. LaTeX sollte nicht mehr als zwei Jahre aelter^^J%
      ! als KOMA-Script sein. Ansonsten koennte der Fehler in einer^^J%
      ! Unvertraeglichkeit zwischen Format- und Paketversion liegen.}
%%
%% Put the message in a macro to improve the look of the error mesage.
%%
%
    \errhelp{Wenn Sie dennoch fortfahren wollen, druecken Sie einfach <Return>.}
  \fi
  \errmessage{\oldformat}
\fi
%
\expandafter\getage\komaversion\@nil
\ifnum\count@>18
  \ifenglish
    \def\oldversion{^^J%
      ! Your KOMA-Script installation is older than one year.^^J%
      ! You should check for an update, bevor you're sending this message.^^J
      ! You should compare the date of the package with the date of^^J
      ! the used LaTeX version. If LaTeX is years younger than KOMA-Script,^^J
      ! this could be the reason of the problem.}
    \errhelp{If you want to continue, press <return>.}
  \else
    \def\oldversion{^^J%
      ! Ihre KOMA-Script-Version ist ueber ein Jahr alt.^^J%
      ! Bitte denken Sie ueber ein Update nach, ehe Sie diese Meldung^^J%
      ! abschicken.^^J%
      ! Vergleichen Sie wenigstens das Datum des Paketes mit dem Datum^^J%
      ! der LaTeX-Version. LaTeX sollte nicht Jahre juenger als KOMA-Script sein.^^J%
      ! Ansonsten koennte der Fehler in einer Unvertraeglichkeit zwischen^^J%
      ! Format- und Paketversion liegen.}
    \errhelp{Wenn Sie dennoch fortfahren wollen, druecken Sie einfach <Return>.}
  \fi
  \errmessage{\oldversion} 
\fi


%%
%% Now use \wmsg and \wread for each of the multi-line fields
%% in the form. Currently one-line fields use \read directly.
%%
\ifinteractive
  \ifenglish
    \typeout{^^JYour name:}
  \else
    \typeout{^^JIhr Name:}
  \fi
  \readifnotknown{name}
\else
  \ifx\name\undefined
    \ifenglish
      \def\name{<REPLACE THIS BY YOUR NAME>}
    \else
      \def\name{<GEBEN SIE IHREN NAMEN EIN>}
    \fi
  \fi
\fi


\ifinteractive
  \ifenglish
    \typeout{^^JYour address (email if possible):}
  \else
    \typeout{^^JIhre Adresse (eMail bevorzugt):}
  \fi
  \readifnotknown{address}
\else
  \ifx\address\undefined
    \ifenglish
      \def\address{<PEPLACE THIS BE YOUR (EMAIL-)ADDRESS>}
    \else
      \def\address{<GEBEN SIE IHRE (EMAIL-)ADRESSE EIN>}
    \fi
  \fi
\fi


\ifinteractive
  \ifenglish
    \typeout{^^JYour computer system (z. B. Atari, Linux, Mac, Win98SE):}
  \else
    \typeout{^^JDas verwendete Computersystem (z. B. Atari, Linux, Mac, Win98SE):}
  \fi
  \readifnotknown{computersystem}
\else
  \ifx\computersystem\undefined
    \ifenglish
      \def\computersystem{<REPLACE THIS BY YOUR COMPUTERSYSTEM>}
    \else
      \def\computersystem{<GEBEN SIE IHR VERWENDETES COMPUTERSYSTEM EIN>}
    \fi
  \fi
\fi


\wmsg*{>Adresse: \name\space<\address>}

%%
%% >Organisation: is really a GNATS multiline field
%% but we treat it as a one-line field.
%%
\wmsg*{>Organisation: \ifx\organisation\undefined
                        \ifx\organization\undefined\else
                           \organization
                        \fi
                       \else
                         \organisation
                       \fi}


%%
%% Test which format is being used. These fields are completed
%% automatically even if the blank template is being produced.
%%

\wmsg*{>Voraussetzungen:}
\wmsg*{ \string\@TeXversion: \meaning\@TeXversion
        \ifx\@TeXversion\@@undefined
         \space (Standardeinstellung fuer TeX3.141 und spaeter)\fi}
\wmsg*{ \string\@currdir: \meaning\@currdir}
\wmsg*{ \string\input@path: \meaning\input@path
        \ifx\input@path\@@undefined
         \space (Standardeinstellung)\fi}
\wmsg*{ System: \computersystem}

\wmsg*{>Beschreibung:}
\ifinteractive
  \ifenglish
    \typeout{%
      Description of your problem:^^J^^J%
      \@spaces You can use multiple lines (each is askes by the prompt^^J%
      \@spaces ``=>'').^^J%
      \@spaces Use two blank lines to finish your answer.}
  \else
    \typeout{%
      Beschreibung des Problems:^^J^^J%
      \@spaces Die Beantwortung dieser Frage kann mehrere Zeilen^^J%
      \@spaces einnehmen (jede wird durch die Eingabeaufforderung^^J%
      \@spaces ``=>'' eingeleitet).^^J%
      \@spaces Durch zwei aufeinanderfolgende Leerzeilen wird die^^J%
      \@spaces Antwort beenden.}
  \fi
\else
  \ifenglish
    \wmsg{<REPLACE THIS BY YOUR DESCRIPTION OF THE PROBLEM>}
  \else
    \wmsg{<GEBEN SIE HIER IHRE PROBLEMBESCHREIBUNG EIN>}
  \fi
\fi
\wread


%%
%% insertion of the test file
%%
\ifinteractive
  \ifenglish
    \typeout{^^J%
      Name of a short self describing file, which shows the problem^^J%
      (file should be as short as possible, not more than 60 lines):^^J^^J%
      If the file is not at current directory please enter the hole^^J%
      name (directory inclusive), so LaTeX may load it.^^J^^J%
      If no testfile exists, because your not reporting a bug at a class^^J%
      or package, simply press <return>.}
  \else
    \typeout{^^J%
      Name einer KURZEN, SELBSTERKLAERENDEN Datei, bei der das Problem^^J%
      auftritt (die Datei sollte wirklich so kurz wie moeglich sein,^^J%
      nicht mehr als 60 Zeilen):^^J^^J%
      Damit LaTeX diese Datei einlesen kann, geben Sie bitte den kompletten^^J%
      Namen einschliesslich des Verzeichnisses an, falls die Datei nicht im^^J%
      aktuellen Verzeichnis zu finden ist.^^J^^J%
      Falls Sie keinen Fehler in einer der Klassen oder Pakete melden und^^J%
      daher keine Testdatei existiert, druecken Sie einfach <Return>.}
  \fi
  \message{filename> }\read\m@ne to \filename
\else
   \let\filename\@empty
\fi

%%
%% Try to find the .tex file and .log file
%%


\ifx\filename\@empty
  \ifinteractive
    \ifenglish
      \ReadNumber{^^J^^JNo Testfile.^^J^^J%
        Three kinds of reports are possible:^^J^^J%
        1) SW bug:^^J\@spaces
        Software bug, you have to add a test file!^^J
        2) DOC bug:^^J\@spaces
        Bug at manual or you do not understand the manual.^^J
        3) Ask for change:^^J\@spaces
        Not a bug, but you'd like a change or simply help.^^J}{3}
    \else
      \ReadNumber{^^J^^JKeine Testdatei.^^J^^J%
        Drei Arten von Meldungen werden unterstuetzt:^^J^^J%
        1) SW-Fehler:^^J\@spaces
        Software-Fehler, unbedingt eine Testdatei beilegen!^^J
        2) DOC-Fehler:^^J\@spaces
        Fehler in oder unverstaendliche Anleitung.^^J
        3) Aenderungswunsch:^^J\@spaces
        Kein Fehler sondern eine Frage nach Aenderung oder Hilfe.^^J}{3}
    \fi
  \else
    \def\answer{0}%
  \fi
\else
  \def\answer{0}%
  \filename@parse\filename

  \IfFileExists{\filename}{\edef\samplefile{\filename}%
    \IfFileExists{\filename@area\filename@base.log}{%
      \edef\logfile{\filename@area\filename@base.log}%
    }{%
      \IfFileExists{\filename@area\filename@base.lis}{%
        \edef\logfile{\filename@area\filename@base.lis}%
      }{%
        \ifenglish
          \typeout{^^J%
            Log file ``\filename@area\filename@base.log'' not found.^^J%
            Please add the log file at ``\jobname.msg''.}%
        \else
          \typeout{^^J%
            Logdatei ``\filename@area\filename@base.log'' nicht gefunden.^^J%
            Bitte ergaenzen Sie das Beispiel in ``\jobname.msg''.}%
        \fi
        \pause
      }%
    }%
  }{%
    \ifenglish
      \typeout{^^J%
        Test file ``\filename'' not found.^^J%
        Please add the test file at ``\jobname.msg''.}
    \else
      \typeout{^^J%
        Beispieldatei ``\filename'' nicht gefunden.^^J%
        Bitte ergaenzen Sie das Beispiel in ``\jobname.msg''.}
    \fi
    \pause
  }%
\fi

\ifcase\expandafter\number\answer
  \ifinteractive\wmsg{>Unterbereich: SW-Fehler}\fi
  \wmsg*{%
    Beispieldatei, die das Problem verdeutlicht:^^J%
    ============================================}
  \ifx\samplefile\undefinedcommand
    \ifenglish
      \typeout{^^J! 
        Please add test file and log file at your report message!}% 
    \else
      \typeout{^^J! 
        Bitte ergaenzen Sie Beispiel- und LOG-Datei in der Meldung!}%
    \fi
    \pause
    \ifenglish
      \wmsg*{<REPLACE THIS BY YOUR TEST FILE>}
    \else
      \wmsg*{<HIER TESTDATEI EINFUEGEN>}
    \fi
  \else
    %%
    %% The example file goes here:
    %%
    \copytomsg{\samplefile}
    \ifnum\linecount>60
      \ifenglish
        \typeout{%
          ^^J%
          !!! Your test file has \the\linecount\space lines.^^J%
          !!! Large test files make it difficult to find the cause of the
          problem:^^J% 
          !!! ^^J%
          !!! Please decrease your test file as much as possible.^^J}
      \else
        \typeout{%
          ^^J%
          !!! Ihre Testdatei ist \the\linecount\space Zeilen lang.^^J%
          !!! So grosse Testdateien erschweren die Ursachenfindung:^^J%
          !!! ^^J%
          !!! Bitte verkleinern Sie Ihre Testdatei, soweit das ueberhaupt^^J%
          !!! moeglich ist, so dass das Problem gerade noch auftritt.^^J}
      \fi
      \pause
    \fi
  \fi
  \wmsg*{^^J%
    LOG-Datei vom LaTeX-Lauf der Beispieldatei:^^J%
    ===========================================}
  \ifx\logfile\undefinedcommand
    \ifenglish
      \typeout{^^J%
        Log file of your test file not found.^^J%
        Please add the log file of your test file at ``\jobname.msg''.}
     \wmsg*{<REPLACE THIS BY THE LOG OF YOUR TEST FILE>}
    \else
      \typeout{^^J%
        Log-Datei zur Testdatei nicht gefunden.^^J%
        Bitte ergaenzen Sie die LOG-Datei in ``\jobname.msg''.}
      \wmsg*{<HIER LOG ZUR TESTDATEI EINFUEGEN >}
    \fi
    \pause
  \else
    \copytomsg{\logfile}
  \fi
\or
  \wmsg{>Unterbereich: DOC-Fehler}
 \or
  \wmsg{>Unterbereich: Aenderungswunsch}
\fi

%%
%% Closing Banner.
%%
\typeout{^^J%
============================================================}

\ifinteractive
  \ifenglish
    \typeout{^^J%
      You may edit file ``\jobname.msg'' for additional changes.}
  \else
    \typeout{^^J%
      Weiteren Aenderungen koennen sie direkt in der Datei^^J%
      ``\jobname.msg'' mit Hilfe eines Editors vornehmen.}
  \fi
\else
  \ifenglish
    \typeout{^^J%
      The formular of the report will be saved to ``\jobname.msg''.^^J%
      Please use an editor to replace all information fields, before^^J%
      sending it.}
  \else
    \typeout{^^J%
      Das Formular fuer die Erstellung der Meldung wurde in die^^J%
      Datei ``\jobname.msg'' geschrieben, die Sie bitte mit Hilfe^^J%
      eines Editors ergaenzen, bevor Sie sie abschicken.}
  \fi
\fi

\let\ifinteractivetrue\iftrue
\ifenglish
  \typeout{^^J%
    Please send ``\jobname.msg'' to komascript@gmx.info using subject:^^J%
    \@spaces ``KOMA-BUG:\space\synopsis''^^J%
    ^^J%
    Thank you for spending time for a bug report.}
\else
  \typeout{^^J%
    Wenn Sie ueber eMail verfuegen, so senden Sie ``\jobname.msg''^^J%
    bitte an komascript@gmx.info.^^J%
    Verwenden Sie dabei bitte als Betreff (Subject):^^J%
    \@spaces ``KOMA-BUG:\space\synopsis''^^J%
    ^^J%
    Danke, dass Sie sich die Zeit genommen haben.}
\fi
\wmsg*{^^J%
  ============================================================^^J
  Ende der KOMA-Fehlermeldung.^^J%
  ============================================================}

%%
%% Close the .msg output stream.
%%
\immediate\closeout\msg

%%
%% This is the TeX primitive \end command.
%%
\@@end

%%% Local Variables: 
%%% mode: latex
%%% TeX-master: t
%%% End: 
