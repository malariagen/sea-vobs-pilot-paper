\documentclass[11pt,a4paper,english]{article}

% To compile:
% (pdf)latex mptrees-doc.tex
% mpost mptrees-doc.mp
% mpost mptrees-doc.mp
% (pdf)latex mptrees-doc.tex

  \usepackage{calc,ifthen}

  \usepackage[utf8]{inputenc}
  \usepackage[T1]{fontenc}
  \usepackage{lmodern}
  \usepackage{textcomp}
  \usepackage{mathtools}

  \usepackage{geometry}
  \geometry{twoside,hmargin=2cm,vmargin={1.5cm,1.8cm},includefoot}
  
  \usepackage[bottom]{footmisc}

  \usepackage{mflogo}
  

  \usepackage{multicol}
  \setlength{\multicolsep}{3pt}
  \setlength{\columnsep}{0pt}
  
  \usepackage[toc]{multitoc}
  
  \usepackage{enumitem}
  \setlist[description]{font=\ttfamily\bfseries\color{blue}}
  
  \usepackage[svgnames]{xcolor}

  \usepackage{graphicx}
  \usepackage{ifpdf}
    \ifpdf
      \DeclareGraphicsRule{*}{mps}{*}{}
    \fi

  \usepackage{url}
  \usepackage{verbatim}
  \usepackage{fancyvrb}
  
%  \usepackage{array}
%  \usepackage{tabularx}
%  \renewcommand{\tabularxcolumn}[1]{>{\arraybackslash}m{#1}}
  
  \usepackage{tcolorbox}
  \tcbset{colframe=black,boxsep=0pt,left=3pt,right=3pt,top=0pt,bottom=0pt,boxrule=0.4pt,colback=LightGoldenrod}
  
  \usepackage{listings}
  \lstset{columns=flexible,%
          language=MetaPost,%
          showstringspaces=false,%
          basicstyle=\ttfamily,
          literate={é}{{\'e}}1}


  \usepackage{babel}
  \usepackage[colorlinks=true,urlcolor=blue]{hyperref}



%%%%%%%%%%%%%%%%%%% moreverb.sty
\makeatletter
\newwrite\verbatim@out
  \immediate\openout \verbatim@out \jobname.mp
\newwrite\temp@code
\def\verbatimwrite{%
  \@bsphack
  \immediate\openout \temp@code figtmp.mp
  \let\do\@makeother\dospecials
  \catcode`\^^M\active \catcode`\^^I=12
  \def\verbatim@processline{%
    \immediate\write\verbatim@out
      {\the\verbatim@line}%
    \immediate\write\temp@code
          {\the\verbatim@line}
     }%
  \immediate\write\temp@code{beginfig(\thenumfig)}
  \verbatim@start}

\def\endverbatimwrite{%
  \immediate\closeout\temp@code
  \@esphack
  }%
\makeatother
%%%%%%%%%%%%%%%%%%%%%%%%%%%%%%

\newlength{\largeurcode}
\newlength{\largeurfig}
\newcounter{numfig}
\setcounter{numfig}{-1}

\newcommand{\codedeuxcol}{%
                  \begin{tcolorbox}
                     \begin{minipage}[c]{\largeurcode}
                       \begin{multicols}{2}
                         \lstinputlisting{figtmp.mp}
                       \end{multicols}
                     \end{minipage}
                   \end{tcolorbox}
                     \par\noindent
                    }
                    
                    
\newcommand{\codehoriz}{%
                   \begin{minipage}[c]{\largeurcode}
                        \lstinputlisting[frame=single,frameround=tttt,backgroundcolor=\color{LightGoldenrod}]{figtmp.mp}
                    \end{minipage}
                    \hfill
                 }

\newcommand{\codevert}{%
                   {\centering
                   \begin{minipage}[c]{\largeurcode}
                        \lstinputlisting[frame=single,frameround=tttt,backgroundcolor=\color{LightGoldenrod}]{figtmp.mp}
                    \end{minipage}
                    \par\noindent}
                 }

\makeatletter
\newenvironment{codefigure}[3]%
               {%
               \refstepcounter{numfig}
               \setlength{\largeurcode}{#2\linewidth}
               \ifnum#1=0
                  \setlength{\largeurfig}{\linewidth-\largeurcode-2\fboxsep-2\fboxrule}
                  \def\inscode{\codehoriz}
               \else
                  \setlength{\largeurfig}{\linewidth}
                  \def\inscode{\codevert}
                  \ifnum#3=1
                   \relax
                  \else
                    \def\inscode{\codedeuxcol}
                  \fi
               \fi
                 \immediate\write\verbatim@out{beginfig(\thenumfig)}
               \verbatimwrite
               }%
               {%
                 \immediate\write\temp@code{endfig;}
                 \immediate\write\verbatim@out{endfig;}
                   
               \endverbatimwrite
               \par
               \noindent
                 \inscode
               \IfFileExists{\jobname.\thenumfig}%
                   {\begin{minipage}[c]{\largeurfig}
                        \centering \includegraphics{\jobname.\thenumfig}\par
                    \end{minipage}}{}
               \par}
\makeatother

\newenvironment{codecache}%
            {\verbatimwrite}{\endverbatimwrite}
            
\newcounter{reptmp}


\newcounter{repexemple}
\setcounter{repexemple}{0}

\newcommand{\debutexemple}{\par\vspace{2ex}
                   \refstepcounter{repexemple}
                   \noindent {\bfseries \color{blue!20!black} 
                   Example \arabic{repexemple}}
                   \par\nopagebreak\vspace{1ex}
                   }


\newenvironment{exemple}[1][0.5]%
                   {\debutexemple
                   \codefigure{0}{#1}{1}
                   }%
                   {\endcodefigure
                   \par\vspace{2ex}}

\newenvironment{exemplev}[2][1]%
                   {\debutexemple
                   \codefigure{1}{#1}{#2}
                   }%
                   {\endcodefigure
                   \par\vspace{2ex}}
                   
                   
\newenvironment{figreperedoc}%
                  {\refstepcounter{numfig}
                   \verbatimwrite}%
                   {%
                   \endverbatimwrite%
                   \IfFileExists{\jobname.\thenumfig}%
                      {\includegraphics{\jobname.\thenumfig}}{}
                   }



\begin{document}
\title{Documentation of \texttt{mptrees.mp}}
\date{\today}
\author{Olivier \textsc{Péault}%
\footnote{E-mail : \href{mailto:o.peault@posteo.net}{\texttt{o.peault@posteo.net}}}}
\maketitle

\setcounter{tocdepth}{2}

\setlength{\columnsep}{25pt}
\tableofcontents
\setlength{\columnsep}{10pt}


\newcommand{\ijc}{[i][j]}

\begin{codecache}
input mptrees;
%input latexmp;
setupLaTeXMP(inputencoding="utf8",
             packages="fourier,amsmath,nicefrac,babel[english]");
dirtree:=0;
\end{codecache}




\section{Overview}

This package is intended to simplify the drawing of probability trees with \MP. It provides one main command and several parameters to control the output.

It can be used in standalone files with two compilations (\verb|latexmp| package is loaded) but it can also be used with Lua\LaTeX{} and \verb|luamplib| package.


\begin{description}
\item[tree\ijc (dim1,dim2,...)(ev1,prob1,ev2,prob2,...)] probability tree located in column \verb|i| and row \verb|j| (see figure below). \verb|dim1|, \verb|dim2|,... can be numerics or pairs and control the dimension of the tree. \verb|ev1|, \verb|prob1|... can be strings or pictures and will be printed (using \verb|latexmp| if strings) at the end of the edge (the event) and above the edge (the probability).

%\item[shiftev, scaleprob, posprob, typeprob, proboffset, dirtree] numerics controlling the scale and position of the probability label, the direction of the tree...

\end{description}

\begin{center}
\begin{figreperedoc}
beginfig(0)
draw tree[1][1](3cm,2cm)("$A$","$p$","$\overline{A}$","$q$");
draw tree[2][1](3cm,1cm)("$B$","$\nicefrac{1}{4}$","$\overline{B}$","$\nicefrac{3}{4}$");
draw tree[2][2](3cm,1cm)("$B$","$0.5$","$\overline{B}$","$0.5$");
draw tree[3][2](3cm,1cm)("$C$","$c$","$\overline{C}$","$1-c$");
draw tree[4][2](3cm,1cm)("$D$","$a$","$\overline{D}$","$b$");
drawoptions(dashed evenly);
draw (Orig_arbre[2][1]+(0,0.5cm))--(Orig_arbre[2][2]-(0,0.5cm))--(Orig_arbre[2][2]-(3.5cm,0.5cm))--(Orig_arbre[2][1]+(-3.5cm,0.5cm))--cycle withcolor red;
label.bot(textext("\verb|tree[1][1]|"),0.5[Orig_arbre[2][2]-(0,0.5cm),Orig_arbre[2][2]-(3.5cm,0.5cm)]) withcolor red;
draw (Orig_arbre[3][1]+(0,0.5cm))--(Orig_arbre[3][2]-(0,0.5cm))--(Orig_arbre[2][1]-(0,1cm))--(Orig_arbre[2][1]+(0,1cm))--cycle withcolor 0.5green;
label.top(textext("\verb|tree[2][1]|"),0.5[Orig_arbre[3][1]+(0,0.5cm),Orig_arbre[2][1]+(0,1cm)]) withcolor 0.5green;
draw (Orig_arbre[3][3]+(0,0.5cm))--(Orig_arbre[3][4]-(0,0.5cm))--(Orig_arbre[2][2]-(0,1cm))--(Orig_arbre[2][2]+(0,1cm))--cycle withcolor 0.5green;
label.bot(textext("\verb|tree[2][2]|"),0.5[Orig_arbre[3][4]-(0,0.5cm),Orig_arbre[2][2]-(0,1cm)]) withcolor 0.5green;
draw (Orig_arbre[4][1]+(0,0.5cm))--(Orig_arbre[4][2]-(0,0.5cm))--(Orig_arbre[3][2]-(0,1cm))--(Orig_arbre[3][2]+(0,1cm))--cycle withcolor blue;
label.bot(textext("\verb|tree[3][2]|"),0.5[Orig_arbre[4][2]-(0,0.5cm),Orig_arbre[3][2]-(0,1cm)]) withcolor blue;
draw (Orig_arbre[5][1]+(0,0.5cm))--(Orig_arbre[5][2]-(0,0.5cm))--(Orig_arbre[4][2]-(0,1cm))--(Orig_arbre[4][2]+(0,1cm))--cycle withcolor red;
label.bot(textext("\verb|tree[4][2]|"),0.5[Orig_arbre[5][2]-(0,0.5cm),Orig_arbre[4][2]-(0,1cm)]) withcolor red;
endfig;

\end{figreperedoc}

\end{center}


\section{Trees}
\subsection{Different kinds of trees}

\begin{description}
\item[tree\ijc (width,vspace)(ev1,prob1,ev2,prob2,...)] regular tree where \verb|width| is the horizontal width of the tree and \verb|vspace| the vertical space between two consecutive nodes.

\begin{codecache}
string dim[];
dim[4]:="4 cm";dim[3]:="3 cm";dim[2]:="2 cm";dim[1]:="1 cm";
dim[15]:="1.5 cm";dim[25]:="2.5 cm";dim[11]:="-1 cm";
dim[22]:="-2 cm";dim[7]:="7cm";

\end{codecache}

\begin{codecache}
extra_endfig:="drawoptions(withcolor red);";
extra_endfig:=extra_endfig & "drawdblarrow ((0,0)--(4cm,0)) shifted (0,2.25cm);label.top(textext(dim[4]),(2cm,2.25cm));";
extra_endfig:=extra_endfig & "drawdblarrow ((0,0)--(0,2.5cm)) shifted (Orig_arbre[2][2]-(5cm,0));label.lft(textext(dim[25]),(-1cm,0));";
extra_endfig:=extra_endfig & "drawoptions(withcolor 0.5green);";
extra_endfig:=extra_endfig & "drawdblarrow ((0,0)--(3cm,0)) shifted (Orig_arbre[2][1]+(0,1cm));label.top(textext(dim[3]),(6cm,2.25cm));";
extra_endfig:=extra_endfig & "drawdblarrow ((0,0)--(0,1.5cm)) shifted (Orig_arbre[3][2]+(0.25cm,0));label.rt(textext(dim[15]),Orig_arbre[3][2]+(0.25cm,0.75cm));";
extra_endfig:=extra_endfig & "drawdblarrow ((0,0)--(0,1cm)) shifted (Orig_arbre[3][4]+(0.25cm,0));label.rt(textext(dim[1]),Orig_arbre[3][4]+(0.25cm,0.5cm));";
extra_endfig:=extra_endfig & "drawdblarrow ((0,0)--(0,1cm)) shifted (Orig_arbre[3][5]+(0.25cm,0));label.rt(textext(dim[1]),Orig_arbre[3][5]+(0.25cm,0.5cm));";

\end{codecache}
 
{\small \begin{exemplev}{1}
draw tree[1][1](4cm,2.5cm)("$A_1$","$\nicefrac{1}{3}$","$A_2$","$\nicefrac{2}{3}$");
draw tree[2][1](3cm,1.5cm)("$B$","$\nicefrac{1}{4}$","$C$","$\nicefrac{3}{4}$");
draw tree[2][2](3cm,1cm)("$D$","$p$","$E$","$q$","$F$","$r$");
\end{exemplev}
}

\item[tree\ijc (width,vspace1,vspace2...)(ev1,prob1,ev2,prob2,...)]  tree where \verb|width| is the horizontal width of the tree while each \verb|vspace| indicates the vertical space between the node and the origin of the tree.

\begin{codecache}
extra_endfig:="drawoptions(withcolor 0.5green);";
extra_endfig:=extra_endfig & "draw Orig_arbre[2][1] -- (Orig_arbre[2][1]+(4.5cm,0)) dashed evenly;";
extra_endfig:=extra_endfig & "drawdblarrow ((0,0)--(3cm,0)) shifted (Orig_arbre[2][1]+(0,2.25cm));label.top(textext(dim[3]),(5cm,3.25cm));";
extra_endfig:=extra_endfig & "drawarrow ((0,-2cm)--(0,0)) shifted (Orig_arbre[3][1]+(0.25cm,0));label.rt(textext(dim[2]),Orig_arbre[3][1]+(0.25cm,-0.7cm));";
extra_endfig:=extra_endfig & "drawarrow ((0,-1cm)--(0,0)) shifted (Orig_arbre[3][2]+(0.5cm,0));label.rt(textext(dim[1]),Orig_arbre[3][2]+(0.5cm,-0.5cm));";
extra_endfig:=extra_endfig & "drawarrow ((0,1cm)--(0,0)) shifted (Orig_arbre[3][3]+(0.75cm,0));label.rt(textext(dim[11]),Orig_arbre[3][3]+(0.75cm,0.5cm));";
extra_endfig:=extra_endfig & "drawarrow ((0,0)--(0,-2cm)) shifted (Orig_arbre[3][4]+(0.25cm,0));label.rt(textext(dim[22]),Orig_arbre[3][4]+(0.25cm,-1cm));";
\end{codecache}
\begin{exemplev}{1}
draw tree[1][1](3cm,2cm)("$A$","$p$","$\overline{A}$","$q$");
draw tree[2][1](3cm,2cm,1cm,-1cm)("$B$","$p$","$C$","$q$","$D$","$r$");
draw tree[2][2](3cm,0cm,-2cm)("$E$","$0.5$","$F$","$0.5$");
\end{exemplev}

\begin{codecache}
extra_endfig:="";
\end{codecache}

\item[tree\ijc (pair1,pair2,...)(ev1,prob1,ev2,prob2,...)] tree where \verb|pair1|, \verb|pair2|... indicate the coordinates of each node from the origin of the tree.


\begin{codecache}
extra_endfig:="drawoptions(withcolor 0.5green);";
extra_endfig:=extra_endfig & "draw Orig_arbre[2][1] -- (Orig_arbre[2][1]+(5cm,0)) dashed evenly;";
extra_endfig:=extra_endfig & "drawarrow ((0,0)--(3cm,0)) shifted (Orig_arbre[2][1]+(0,2.25cm));label.top(textext(dim[3]),(5cm,3.25cm));";
extra_endfig:=extra_endfig & "drawarrow ((0,-2cm)--(0,0)) shifted (Orig_arbre[3][1]+(0.25cm,0));label.rt(textext(dim[2]),Orig_arbre[3][1]+(0.25cm,-0.7cm));";
extra_endfig:=extra_endfig & "drawarrow ((0,0)--(4cm,0)) shifted (Orig_arbre[2][1]+(0,-1.25cm));label.bot(textext(dim[4]),(5.5cm,-0.25cm));";
extra_endfig:=extra_endfig & "drawarrow ((0,0)--(0,-1cm)) shifted (Orig_arbre[3][2]+(0,1cm));label.rt(textext(dim[11]),Orig_arbre[3][2]+(0,0.5cm));";
\end{codecache}

\begin{exemplev}{1}
draw tree[1][1](3cm,2cm)("$A$","$p$","$\overline{A}$","$1-p$");
draw tree[2][1]((3cm,2cm),(4cm,-1cm))("$B$","$q$","$C$","$r$");
\end{exemplev}

\begin{codecache}
extra_endfig:="";
\end{codecache}
\end{description}

\subsection{Start and end labels}
\begin{description}
\item[startlabel(s)] Print \verb|s| (can be a string or a picture) at the origin of the tree.

\begin{exemple}[0.65]
 draw startlabel("$S$");
 draw tree[1][1](3cm,2cm)("$A$","$p$","$B$","$q$");
\end{exemple}

\item[endlabel\ijc(s)] Print \verb|s| at the end of a branch. The space between the previous label ans \verb|s| is controlled by the numeric \verb|endlabelspace| which defaults to \verb|1cm|.

\begin{exemplev}{1}
 draw startlabel("$S$");
 draw tree[1][1](3cm,2cm)("$A$","$p$","$B$","$q$");
 draw tree[2][2](2cm,1cm)("$A$","$p$","$B$","$q$");
 draw endlabel[2][1]("$SA$");
 draw endlabel[3][1]("$SBA$");
 draw endlabel[3][2]("$SBB$");
\end{exemplev}

\end{description}

\section{Direction}
\begin{description}
\item[dirtree] All trees are construct horizontally by default. \verb|ditree| indicates the angle in degrees between the horizontal and the main direction of the tree. Default is $0$.



\begin{exemplev}{1}
dirtree:=135;
draw tree[1][1](3cm,2cm)("$A_1$","$a_1$","$A_2$","$a_2$");
draw tree[2][1](3cm,1cm)("$B$","$b$","$C$","$c$");
draw tree[2][2](3cm,1cm)("$D$","$p$","$E$","$q$");
\end{exemplev}


\begin{exemplev}{1}
dirtree:=-60;
draw tree[1][1](3cm,2cm)("$A_1$","$a_1$","$A_2$","$a_2$");
draw tree[2][1](3cm,1cm)("$B$","$b$","$C$","$c$");
draw tree[2][2](3cm,1cm)("$D$","$p$","$E$","$q$");
\end{exemplev}

\begin{codecache}
dirtree:=0;dirlabel:=0;
\end{codecache}

\item[dirlabel] With dirtree, the whole tree is rotated. With \verb|dirlabel|, only the position of the labels is changed so the given coordinates are the real ones. May be useful for vertical trees.

\begin{exemplev}{1}
dirlabel:=90;
draw tree[1][1]((-1cm,2cm),(1cm,2cm))("$A$","$p$","$B$","$q$");
draw tree[2][1]((-0.5cm,2cm),(0.5cm,2cm))("$C$","$c$","$D$","$d$");
draw tree[2][2]((-0.5cm,2cm),(0.5cm,2cm))("$E$","$e$","$F$","$f$");
\end{exemplev}

\begin{codecache}
dirlabel:=0;
\end{codecache}

\end{description}

\section{Dealing with alignment}
\begin{description}
\item[shiftev] The origin of each tree is located where the bounding box of the previous event's name ends. Thus subtrees may begin at different places. The numeric \verb|shiftev| indicates the horizontal space between the end of the edges and the beginning of following trees.

It can be used inside the first set of parameters of the tree (see example below) or as a global variable.


\begin{exemplev}{1}
 draw tree[1][1](80,120)("$A$","$0.5$","$\overline{A}$","$0.5$");
 draw tree[2][1](70,40)("Yes","$p$","No","$q$","Maybe","$r$");
 draw tree[2][2](70,40,"shiftev:=1.5cm")("Yes","$p$","No","$q$","Maybe","$r$");
 draw tree[3][1](50,20)("$B$","$b$","$C$","$c$");
 draw tree[3][2](50,20)("$B$","$b$","$C$","$c$");
 draw tree[3][3](50,20)("$B$","$b$","$C$","$c$");
 draw tree[3][4](50,20)("$B$","$b$","$C$","$c$");
 draw tree[3][5](50,20)("$B$","$b$","$C$","$c$");
 draw tree[3][6](50,20)("$B$","$b$","$C$","$c$");
\end{exemplev}


\item[abscoord] With the boolean \verb|abscoord| set to \verb|true|, all the coordinates are given from the origin of the \emph{first} tree instead of the origin of the subtree, which make easier the alignment of all the subtrees. 
\end{description}


\begin{codecache}
extra_endfig:="drawoptions(withcolor red);";
extra_endfig:=extra_endfig & "drawdblarrow ((0,0)--(7cm,0)) shifted (0,2.25cm);label.top(textext(dim[7]),(3.5cm,2.25cm));";
extra_endfig:=extra_endfig & "drawdblarrow ((0,0)--(0,2cm)) shifted (-0.5cm,0);label.lft(textext(dim[2]),(-0.5cm,1cm));";
\end{codecache}

\begin{exemplev}{1}
abscoord:=true;
draw tree[1][1](3cm,2cm)("$A$","$p$","Blabla","$q$");
draw tree[2][1]((7cm,2cm),(7cm,0.5cm))("$A$","$p$","$B$","$q$");
draw tree[2][2]((7cm,-0.5cm),(7cm,-2cm))("$A$","$p$","$B$","$q$");
\end{exemplev}

\begin{codecache}
extra_endfig:="";
abscoord:=false;
\end{codecache}


\section{Parameters}
\begin{description}


\item[scaleprob] numeric controlling the scale of the label above the edge (the probability). Default is $0.85$.

\begin{exemple}[0.65]
 scaleprob:=1.5;
 draw tree[1][1](3cm,2cm)("$A$","$p$","$B$","$q$");
\end{exemple}

\begin{codecache}
scaleprob:=0.85;
\end{codecache}


\item[scaleev] numeric controlling the scale of the label at the end of the edge (the event). Default is $1$.

\begin{exemple}[0.65]
 scaleev:=2;
 draw tree[1][1](3cm,2cm)("$A$","$p$","$B$","$q$");
\end{exemple}

\begin{codecache}
scaleev:=1;
\end{codecache}


\item[posprob] numeric controlling the position of the label above the edge. Default is $0.6$.


\begin{exemple}[0.65]
 posprob:=0.8;
 draw tree[1][1](3cm,2cm)("$A$","$p$","$B$","$q$");
\end{exemple}

\item[typeprob] numeric controlling how the label is printed. Values can be 1 (the default, label is printed above the edge), 2 (the label is printed on the edge), 3 (the label is printed above the edge and rotated) or 4 (the label is printed on the edge and rotated).

\begin{codecache}
posprob:=0.6;
\end{codecache}

\begin{exemple}[0.7]
 typeprob:=2;
 draw tree[1][1](3cm,2cm)("$A$","$p$","$B$","$1-p$");
\end{exemple}

\begin{exemple}[0.7]
 typeprob:=3;
 draw tree[1][1](3cm,2cm)("$A$","$p$","$B$","$1-p$");
\end{exemple}

\begin{exemple}[0.7]
 typeprob:=4;
 draw tree[1][1](3cm,2cm)("$A$","$p$","$B$","$1-p$");
\end{exemple}


\item[proboffset] numeric controlling the amount by which the label above the edge is offset. Default is \verb|labeloffset| (\verb|3bp|).

\begin{codecache}
typeprob:=1;
\end{codecache}

\begin{exemple}[0.75]
 draw tree[1][1](3cm,3cm)("$A$","$p+q+r$","$B$","$s$");
\end{exemple}



\begin{exemple}[0.75]
 proboffset:=6bp;
 draw tree[1][1](3cm,3cm)("$A$","$p+q+r$","$B$","$s$");
\end{exemple}

\begin{codecache}
proboffset:=3bp;
\end{codecache}

\item[edgearrow] When the boolean \verb|edgearrow| is set to true, edges end with an arrow. Default is \verb|false|.

\begin{exemple}[0.65]
 edgearrow:=true;
 draw tree[1][1](3cm,2cm)("$A$","$p$","$B$","$q$");
\end{exemple}

\begin{codecache}
edgearrow:=false;
\end{codecache}

\item[endedgeshift] vertical space added at the end of the edge. Default is $0$. Useful when various edges end at the same point

\begin{exemple}[0.65]
 draw startlabel("$S$");
 draw tree[1][1]((3cm,-1cm))("$A$","$p$");
\end{exemple}


\begin{exemple}[0.65]
 endedgeshift:=10;
 draw startlabel("$S$");
 draw tree[1][1]((3cm,-1cm))("$A$","$p$");
\end{exemple}

\begin{codecache}
endedgeshift:=0;
\end{codecache}

\end{description}


\section{Embedded code in \LaTeX{} files}

You can embed your code in \LaTeX{} files.

\subsection{With \texttt{pdftex}}

\medskip

\begin{minipage}[t]{0.45\linewidth}
{\centering \textbf{Using \texttt{emp} package}\par}



\verb|pdflatex myfile.tex|

\verb|mpost myfile.mp|

\verb|mpost myfile.mp|

\verb|pdflatex myfile.tex|


\begin{lstlisting}[frame=single,frameround=tttt,backgroundcolor=\color{LightSteelBlue},language={[LaTeX]TeX}]
\documentclass{article}
\usepackage{emp}
\usepackage{ifpdf}
 \ifpdf % allows pdflatex compilation
  \DeclareGraphicsRule{*}{mps}{*}{}
 \fi
\begin{document}
\begin{empfile}
\begin{empcmds}
 input mptrees;
\end{empcmds}
\begin{emp}(0,0)
 draw tree[1][1](3cm,3cm)(...);
\end{emp}
\end{empfile}
\end{document}
\end{lstlisting}
\end{minipage}
\hfill
\begin{minipage}[t]{0.45\linewidth}
{\centering \textbf{Using \texttt{mpgraphics} package}\par}


\verb|pdflatex -shell-escape myfile.tex|

\begin{lstlisting}[frame=single,frameround=tttt,backgroundcolor=\color{LightSteelBlue},language={[LaTeX]TeX}]
\documentclass{article}
\usepackage[runs=2]{mpgraphics}
\begin{document}
\begin{mpdefs}
 input mptrees;
\end{mpdefs}
\begin{mpdisplay}
 draw tree[1][1](3cm,3cm)(...);
\end{mpdisplay}
\end{document}
\end{lstlisting}
\end{minipage}

\subsection{With \texttt{luatex}}

\begin{center}
\begin{minipage}[t]{0.6\linewidth}
{\centering \textbf{Using Lua\LaTeX}\par}


\verb|lualatex myfile.tex|

\begin{lstlisting}[frame=single,frameround=tttt,backgroundcolor=\color{LightSteelBlue},language={[LaTeX]TeX}]
\documentclass{article}
\usepackage{fontspec}
\usepackage{luamplib}
\begin{document}
\everymplib{input mptrees;}
\begin{mplibcode}
beginfig(1);
 draw tree[1][1](3cm,3cm)("$A$","$p$","$B$","$q$");
endfig;
\end{mplibcode}
\end{document}
\end{lstlisting}
\end{minipage}
\end{center}

\section{Examples}

\begin{exemplev}[0.9]{1}
u:=0.4cm;
dirlabel:=90;
abscoord:=true;
endlabelspace:=0.5cm;
draw startlabel("S");
draw tree[1][1]((-5.5u,4u),(5.5u,8u))("NP","","VP","");
draw tree[2][1]((-8.5u,12u),(-2.5u,8u))("A","","NP","");
draw tree[2][2]((3.5u,12u),(7.5u,12u))("V","","Adv","");
draw tree[3][2]((-4.5u,12u),(-0.5u,12u))("A","","N","");
draw endlabel[3][1]("Colorless");
draw endlabel[4][1]("green");
draw endlabel[4][2]("ideas");
draw endlabel[3][3]("sleep");
draw endlabel[3][4]("furiously");
\end{exemplev}

\begin{codecache}
endlabelspace:=1cm;
\end{codecache}

\begin{exemplev}[0.9]{1}
u:=1cm;
dirlabel:=-90;
abscoord:=true;
scaleev:=2;
label.top(textext("\Large Tree diagram of $(2x+1)(x-1)$"),(0,1cm));
draw startlabel("$\times$");
draw tree[1][1]((-2u,-1.5u),(2u,-1.5u))("$+$","","$-$","");
draw tree[2][1]((-3u,-3.5u),(-1u,-3.5u))("$\times$","","$1$","");
draw tree[2][2]((1u,-3.5u),(3u,-3.5u))("$x$","","$2$","");
draw tree[3][1]((-4u,-5.5u),(-2u,-5.5u))("$2$","","$x$","");
\end{exemplev}

\begin{codecache}
dirlabel:=0;
abscoord:=false;
scaleev:=1;
\end{codecache}

\begin{exemplev}{1}
posprob:=0.5;
typeprob:=3;
shiftev:=1.5cm;
edgearrow:=true;
u:=0.2cm;
vardef paral = ((2,-2)--(6,2)--(0,2)--(-4,-2)--cycle) scaled u enddef;
vardef rhombus = ((3,0)--(0,6)--(-3,0)--(0,-6)--cycle) scaled u enddef;
vardef rectangle = ((3,5)--(-3,5)--(-3,-5)--(3,-5)--cycle) scaled u enddef;
vardef square = ((3,3)--(-3,3)--(-3,-3)--(3,-3)--cycle) scaled u enddef;
draw startlabel(paral);
draw tree[1][1](5cm,4cm)(rhombus,"Diagonals perpendicular",%
                            rectangle,"Diagonals of equal length");
endedgeshift:=5;
draw tree[2][1]((5cm,-2cm))("","Diagonals of equal length");
draw tree[2][2]((5cm,2cm))(square,"Diagonals perpendicular");
\end{exemplev}
 
\begin{codecache}
end
\end{codecache}

\makeatletter
\immediate\closeout\verbatim@out
\makeatother
\end{document}


