% \iffalse meta-comment
%
% Copyright 1993-2016
% The LaTeX3 Project and any individual authors listed elsewhere
% in this file.
%
% This file is part of the LaTeX base system.
% -------------------------------------------
%
% It may be distributed and/or modified under the
% conditions of the LaTeX Project Public License, either version 1.3c
% of this license or (at your option) any later version.
% The latest version of this license is in
%    http://www.latex-project.org/lppl.txt
% and version 1.3c or later is part of all distributions of LaTeX
% version 2005/12/01 or later.
%
% This file has the LPPL maintenance status "maintained".
%
% The list of all files belonging to the LaTeX base distribution is
% given in the file `manifest.txt'. See also `legal.txt' for additional
% information.
%
% The list of derived (unpacked) files belonging to the distribution
% and covered by LPPL is defined by the unpacking scripts (with
% extension .ins) which are part of the distribution.
%
% \fi
%
% \iffalse
%%% From File: ltxref.dtx
%
%<*driver>
% \fi
\ProvidesFile{ltxref.dtx}
             [2014/09/29 v1.1k LaTeX Kernel (Cross Referencing)]
% \iffalse
\documentclass{ltxdoc}
\GetFileInfo{ltxref.dtx}
\title{\filename}
\date{\filedate}
 \author{%
  Johannes Braams\and
  David Carlisle\and
  Alan Jeffrey\and
  Leslie Lamport\and
  Frank Mittelbach\and
  Chris Rowley\and
  Rainer Sch\"opf}

\begin{document}
 \MaintainedByLaTeXTeam{latex}
 \maketitle
 \DocInput{\filename}
\end{document}
%</driver>
% \fi
%
%
% \changes{v1.0c}{1994/03/29}
%     {Create file ltcntlen from parts of ltmiscen and ltherest.}
% \changes{v1.1a}{1994/05/19}
%     {Extract file ltxref from ltcntlen.}
% \changes{v1.1b}{1994/05/21}{Use new warning commands}
% \changes{v1.1c}{1994/05/25}{Modify documentation}
%
% \section{Cross Referencing}
%  The user writes  |\label|\marg{foo}  to define the following
%  cross-references:
%
%   |\ref|\marg{foo}: value of most recently incremented referencable
%             counter. in the current environment. (Chapter, section,
%             theorem and enumeration counters counters are
%             referencable, footnote counters are not.)
%
%   |\pageref|\marg{foo}: page number at which |\label{foo}|  command
%             appeared.  where  foo  can be any string of characters not
%             containing  `|\|', `|{|' or `|}|'.
%
%  Note: The scope of the |\label| command is delimited by environments,
%  so\\
%  |\begin{theorem} \label{foo} ... \end{theorem} \label{bar}|\\
%  defines |\ref{foo}| to be the theorem number and |\ref{bar}| to be
%  the current section number.
%
%  Note: |\label| does the right thing in terms of spacing -- i.e.,
%  leaving a space on both sides of it is equivalent to leaving
%  a space on either side.
%
%
% \StopEventually{}
%
% \subsection{Cross Referencing}
%
% \begin{oldcomments}
%    \begin{macrocode}
%<*2ekernel>
\message{x-ref,}
%    \end{macrocode}
%
%  This is implemented as follows.  A referencable counter  CNT  is
%  incremented by the command  \refstepcounter{CNT} , which sets
%  \@currentlabel == {CNT}{eval(\p@cnt\theCNT)}.   The command
%  \label{FOO} then writes the following on file \@auxout :
%        \newlabel{FOO}{{eval(\@currentlabel)}{eval(\thepage)}}
%
%  \ref{FOO} ==
%    BEGIN
%      if \r@foo undefined
%        then  @refundefined := G T
%              ??
%              Warning: 'reference foo on page ... undefined'
%        else  \@car \eval(\r@FOO)\@nil
%      fi
%    END
%
%  \pageref{foo} =
%    BEGIN
%      if \r@foo undefined
%        then  @refundefined := G T
%              ??
%              Warning: 'reference foo on page ... undefined'
%        else  \@cdr \eval(\r@FOO)\@nil
%      fi
%    END
%
% \end{oldcomments}
%
%  \begin{macro}{\G@refundefinedtrue}
% \changes{v1.1i}{1995/12/07}{Renamed (back) from \cs{G@refundefined}}
%  \begin{macro}{\@refundefined}
% \changes{v1.1h}{1995/10/24}{Switch for refundefined replaced}
%    This does not save on name-space (since \cs{G@refundefinedfalse}
%    was never needed) but it does make the implementation of such
%    one-way switches more consistent. The extra macro to make the
%    change is used since this change appears several times.
%
%    \textbf{Note} despite its name, |\G@refundefinedtrue| does
%    \emph{not} correspond to an |\if| command, and there is no
%    matching \ldots|false|. It would be more natural to call the
%    command |\G@refundefined| (as inspection of the change log will
%    reveal) but unfortunately such a change would break any package
%    that had defined a  |\ref|-like command that mimicked the
%    definition of |\ref|, calling |\G@refundefinedtrue|. Inspection
%    of the \TeX\ archives revealed several such packages, and so this
%    command has been named \ldots|true| so that the definition of
%    |\ref| need not be changed, and the packages will work without
%    change.
%    \begin{macrocode}
% \newif\ifG@refundefined
% \def\G@refundefinedtrue{\global\let\ifG@refundefined\iftrue}
% \def\G@refundefinedfalse{\global\let\ifG@refundefined\iffalse}
\def\G@refundefinedtrue{%
  \gdef\@refundefined{%
    \@latex@warning@no@line{There were undefined references}}}
\let\@refundefined\relax
%    \end{macrocode}
%  \end{macro}
%  \end{macro}
%
%    \begin{macrocode}
%    \end{macrocode}
%  \begin{macro}{\ref}
% \changes{LaTeX2e}{1993/12/11}{Macro reimplemented}
%  \begin{macro}{\pageref}
% \changes{LaTeX2e}{1993/12/11}{Macro reimplemented}
%  \begin{macro}{\@setref}
% \changes{LaTeX2e}{1993/12/11}{Macro added}
% \changes{v1.1h}{1995/10/24}{Switch for refundefined renamed}
% \changes{v1.1i}{1995/12/07}{Switch for refundefined restored}
%    Referencing a |\label|.
% RmS 91/10/25: added a few extra |\reset@font|,
%               as suggested by Bernd Raichle
%
% RmS 92/08/14: made |\ref| and |\pageref| robust
%
% RmS 93/09/08: Added setting of refundefined switch.
%    \begin{macrocode}
\def\@setref#1#2#3{%
  \ifx#1\relax
   \protect\G@refundefinedtrue
   \nfss@text{\reset@font\bfseries ??}%
   \@latex@warning{Reference `#3' on page \thepage \space
             undefined}%
  \else
   \expandafter#2#1\null
  \fi}
\def\ref#1{\expandafter\@setref\csname r@#1\endcsname\@firstoftwo{#1}}
\def\pageref#1{\expandafter\@setref\csname r@#1\endcsname
                                   \@secondoftwo{#1}}
%    \end{macrocode}
%  \end{macro}
%  \end{macro}
%  \end{macro}
%
%
%  \begin{macro}{\newlabel}
% \changes{v1.1b}{1994/05/21}{Use new warning commands}
% \changes{v1.1e}{1995/04/24}{Make \cs{@onlypreamble} for /1388.}
% \changes{v1.1e}{1995/06/19}
%      {Use \cs{@newl@bel} to share code with \cs{bibcite}}
% \changes{v1.1g}{1995/07/14}
%   {Remove \cs{@onlypreamble} so still defined in new \cs{enddocument}}
%    This command will be written to the \texttt{.aux} file to
%    pass label information from one run to another.
%  \begin{macro}{\@newl@bel}
%    The internal form of |\newlabel| and |\bibcite|. Note that this
%    macro does it's work inside a group. That way the local
%    assignments it needs to do don't clutter the save stack. This
%    prevents large documents with many labels to run out of save
%    stack.
% \changes{v1.1h}{1995/10/24}{Switch for multiplelabels replaced by
%    inline code}
% \changes{v1.1k}{2001/02/16}{Added an extra grouplevel (PR3250), jlb}
%    \begin{macrocode}
\def\@newl@bel#1#2#3{{%
  \@ifundefined{#1@#2}%
    \relax
    {\gdef \@multiplelabels {%
       \@latex@warning@no@line{There were multiply-defined labels}}%
     \@latex@warning@no@line{Label `#2' multiply defined}}%
  \global\@namedef{#1@#2}{#3}}}
%    \end{macrocode}
%
%    \begin{macrocode}
\def\newlabel{\@newl@bel r}
%    \end{macrocode}
%
%    \begin{macrocode}
\@onlypreamble\@newl@bel
%    \end{macrocode}
%  \end{macro}
%  \end{macro}
%
%  \begin{macro}{\if@multiplelabels}
% \changes{v1.1h}{1995/10/24}{Macro removed}
%  \begin{macro}{\@multiplelabels}
% \changes{v1.1h}{1995/10/24}{Switch for multiplelabels removed}
%    This is redefined to produce a warning if at least one label is
%    defined more than once. It is executed by the |\enddocument|
%    command.
%    \begin{macrocode}
\let \@multiplelabels \relax
%    \end{macrocode}
%  \end{macro}
%  \end{macro}
%
%  \begin{macro}{\label}
% \changes{v1.1d}{1994/11/04}{(ASAJ)Added \cs{protected@write}}
%  \begin{macro}{\refstepcounter}
% \changes{v1.1d}{1994/11/04}{(ASAJ)Added \cs{protected@edef}}
%    The commands |\label| and |\refstepcounter| have been changed to
%    allow |\protect|'ed commands to work properly.  For example,
%\begin{verbatim}
%   \def\thechapter{\protect\foo{\arabic{chapter}.\roman{section}}}
%\end{verbatim}
%    will cause a |\label{bar}| command to define |\ref{bar}| to expand
%    to something like |\foo{4.d}|.  Change made 20 Jul 88.
%
%    \begin{macrocode}
\def\label#1{\@bsphack
  \protected@write\@auxout{}%
         {\string\newlabel{#1}{{\@currentlabel}{\thepage}}}%
  \@esphack}
%    \end{macrocode}
%
%    \begin{macrocode}
\def\refstepcounter#1{\stepcounter{#1}%
    \protected@edef\@currentlabel
       {\csname p@#1\endcsname\csname the#1\endcsname}%
}
%    \end{macrocode}
%  \end{macro}
%  \end{macro}
%
%
%  \begin{macro}{\@currentlabel}
% For |\label| commands that come before any environment
%
%    \begin{macrocode}
\def\@currentlabel{}
%    \end{macrocode}
%  \end{macro}
%
%    \begin{macrocode}
%</2ekernel>
%    \end{macrocode}
%
% \subsection{An extension of counter referencing}
%
%
% At the moment a reference to a counter |foo| will generate the
% equivalent of |\p@foo\thefoo| although not quite in this form.  For
% some applications it would be nice of one could have |\thefoo| being
% an argument to |\p@foo| to be able to put material before and after
% the number generated by |\thefoo|. This can be easily achieved with
% a small change to one of the kernel commands as follows:
%
%\begin{verbatim}
%\def\refstepcounter#1{\stepcounter{#1}%
%    \protected@edef\@currentlabel
%       {\csname p@#1\expandafter\endcsname\csname the#1\endcsname}%
%}
%\end{verbatim}
%
% The trick is to ensure that |\csname the#1\endcsname| is turned into
% a single token before |\p@...| is expanded further. This way, if the
% |\p@...| command is a macro with one argument it will receive
% |\the...|. With the kernel code (i.e., without the |\expandafter|)
% it will instead pick up |\csname| which would be disastrous.
%
% Using |\expandafter| instead of braces delimiting the argument is
% better because, assuming that the |\p@...| command is not defined as
% a macro with one argument, the braces will stay and prohibit kerning
% that might otherwise happen between the glyphs generated by
% |\the...| and surrounding glyphs.
%
% We have refrained from making this change in the kernel code
% although for existing documents it would be 100\% backward
% compatible. The reason being that any class or package making use of
% this functionality would then horribly fail with older \LaTeX{}
% installations.
%
% Instead we suggest that people who are interested in using this
% functionality in a document class or package add the redefinition to
% the class file. To ensure that this redefinition is properly applied
% they might want to test for the original definition first, e.g.
%
%\begin{verbatim}
%\CheckCommand*\refstepcounter[1]{\stepcounter{#1}%
%    \protected@edef\@currentlabel
%       {\csname p@#1\endcsname\csname the#1\endcsname}%
%}
%\renewcommand*\refstepcounter[1]{\stepcounter{#1}%
%    \protected@edef\@currentlabel
%       {\csname p@#1\expandafter\endcsname\csname the#1\endcsname}%
%}
%\end{verbatim}
%
% \Finale
%
