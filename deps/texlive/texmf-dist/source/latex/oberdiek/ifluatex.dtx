% \iffalse meta-comment
%
% File: ifluatex.dtx
% Version: 2016/05/16 v1.4
% Info: Provides the ifluatex switch
%
% Copyright (C) 2007, 2009, 2010 by
%    Heiko Oberdiek <heiko.oberdiek at googlemail.com>
%    2016
%    https://github.com/ho-tex/oberdiek/issues
%
% This work may be distributed and/or modified under the
% conditions of the LaTeX Project Public License, either
% version 1.3c of this license or (at your option) any later
% version. This version of this license is in
%    http://www.latex-project.org/lppl/lppl-1-3c.txt
% and the latest version of this license is in
%    http://www.latex-project.org/lppl.txt
% and version 1.3 or later is part of all distributions of
% LaTeX version 2005/12/01 or later.
%
% This work has the LPPL maintenance status "maintained".
%
% This Current Maintainer of this work is Heiko Oberdiek.
%
% The Base Interpreter refers to any `TeX-Format',
% because some files are installed in TDS:tex/generic//.
%
% This work consists of the main source file ifluatex.dtx
% and the derived files
%    ifluatex.sty, ifluatex.pdf, ifluatex.ins, ifluatex.drv,
%    ifluatex-test1.tex, ifluatex-test2.tex, ifluatex-test3.tex.
%
% Distribution:
%    CTAN:macros/latex/contrib/oberdiek/ifluatex.dtx
%    CTAN:macros/latex/contrib/oberdiek/ifluatex.pdf
%
% Unpacking:
%    (a) If ifluatex.ins is present:
%           tex ifluatex.ins
%    (b) Without ifluatex.ins:
%           tex ifluatex.dtx
%    (c) If you insist on using LaTeX
%           latex \let\install=y% \iffalse meta-comment
%
% File: ifluatex.dtx
% Version: 2016/05/16 v1.4
% Info: Provides the ifluatex switch
%
% Copyright (C) 2007, 2009, 2010 by
%    Heiko Oberdiek <heiko.oberdiek at googlemail.com>
%    2016
%    https://github.com/ho-tex/oberdiek/issues
%
% This work may be distributed and/or modified under the
% conditions of the LaTeX Project Public License, either
% version 1.3c of this license or (at your option) any later
% version. This version of this license is in
%    http://www.latex-project.org/lppl/lppl-1-3c.txt
% and the latest version of this license is in
%    http://www.latex-project.org/lppl.txt
% and version 1.3 or later is part of all distributions of
% LaTeX version 2005/12/01 or later.
%
% This work has the LPPL maintenance status "maintained".
%
% This Current Maintainer of this work is Heiko Oberdiek.
%
% The Base Interpreter refers to any `TeX-Format',
% because some files are installed in TDS:tex/generic//.
%
% This work consists of the main source file ifluatex.dtx
% and the derived files
%    ifluatex.sty, ifluatex.pdf, ifluatex.ins, ifluatex.drv,
%    ifluatex-test1.tex, ifluatex-test2.tex, ifluatex-test3.tex.
%
% Distribution:
%    CTAN:macros/latex/contrib/oberdiek/ifluatex.dtx
%    CTAN:macros/latex/contrib/oberdiek/ifluatex.pdf
%
% Unpacking:
%    (a) If ifluatex.ins is present:
%           tex ifluatex.ins
%    (b) Without ifluatex.ins:
%           tex ifluatex.dtx
%    (c) If you insist on using LaTeX
%           latex \let\install=y% \iffalse meta-comment
%
% File: ifluatex.dtx
% Version: 2016/05/16 v1.4
% Info: Provides the ifluatex switch
%
% Copyright (C) 2007, 2009, 2010 by
%    Heiko Oberdiek <heiko.oberdiek at googlemail.com>
%    2016
%    https://github.com/ho-tex/oberdiek/issues
%
% This work may be distributed and/or modified under the
% conditions of the LaTeX Project Public License, either
% version 1.3c of this license or (at your option) any later
% version. This version of this license is in
%    http://www.latex-project.org/lppl/lppl-1-3c.txt
% and the latest version of this license is in
%    http://www.latex-project.org/lppl.txt
% and version 1.3 or later is part of all distributions of
% LaTeX version 2005/12/01 or later.
%
% This work has the LPPL maintenance status "maintained".
%
% This Current Maintainer of this work is Heiko Oberdiek.
%
% The Base Interpreter refers to any `TeX-Format',
% because some files are installed in TDS:tex/generic//.
%
% This work consists of the main source file ifluatex.dtx
% and the derived files
%    ifluatex.sty, ifluatex.pdf, ifluatex.ins, ifluatex.drv,
%    ifluatex-test1.tex, ifluatex-test2.tex, ifluatex-test3.tex.
%
% Distribution:
%    CTAN:macros/latex/contrib/oberdiek/ifluatex.dtx
%    CTAN:macros/latex/contrib/oberdiek/ifluatex.pdf
%
% Unpacking:
%    (a) If ifluatex.ins is present:
%           tex ifluatex.ins
%    (b) Without ifluatex.ins:
%           tex ifluatex.dtx
%    (c) If you insist on using LaTeX
%           latex \let\install=y% \iffalse meta-comment
%
% File: ifluatex.dtx
% Version: 2016/05/16 v1.4
% Info: Provides the ifluatex switch
%
% Copyright (C) 2007, 2009, 2010 by
%    Heiko Oberdiek <heiko.oberdiek at googlemail.com>
%    2016
%    https://github.com/ho-tex/oberdiek/issues
%
% This work may be distributed and/or modified under the
% conditions of the LaTeX Project Public License, either
% version 1.3c of this license or (at your option) any later
% version. This version of this license is in
%    http://www.latex-project.org/lppl/lppl-1-3c.txt
% and the latest version of this license is in
%    http://www.latex-project.org/lppl.txt
% and version 1.3 or later is part of all distributions of
% LaTeX version 2005/12/01 or later.
%
% This work has the LPPL maintenance status "maintained".
%
% This Current Maintainer of this work is Heiko Oberdiek.
%
% The Base Interpreter refers to any `TeX-Format',
% because some files are installed in TDS:tex/generic//.
%
% This work consists of the main source file ifluatex.dtx
% and the derived files
%    ifluatex.sty, ifluatex.pdf, ifluatex.ins, ifluatex.drv,
%    ifluatex-test1.tex, ifluatex-test2.tex, ifluatex-test3.tex.
%
% Distribution:
%    CTAN:macros/latex/contrib/oberdiek/ifluatex.dtx
%    CTAN:macros/latex/contrib/oberdiek/ifluatex.pdf
%
% Unpacking:
%    (a) If ifluatex.ins is present:
%           tex ifluatex.ins
%    (b) Without ifluatex.ins:
%           tex ifluatex.dtx
%    (c) If you insist on using LaTeX
%           latex \let\install=y\input{ifluatex.dtx}
%        (quote the arguments according to the demands of your shell)
%
% Documentation:
%    (a) If ifluatex.drv is present:
%           latex ifluatex.drv
%    (b) Without ifluatex.drv:
%           latex ifluatex.dtx; ...
%    The class ltxdoc loads the configuration file ltxdoc.cfg
%    if available. Here you can specify further options, e.g.
%    use A4 as paper format:
%       \PassOptionsToClass{a4paper}{article}
%
%    Programm calls to get the documentation (example):
%       pdflatex ifluatex.dtx
%       makeindex -s gind.ist ifluatex.idx
%       pdflatex ifluatex.dtx
%       makeindex -s gind.ist ifluatex.idx
%       pdflatex ifluatex.dtx
%
% Installation:
%    TDS:tex/generic/oberdiek/ifluatex.sty
%    TDS:doc/latex/oberdiek/ifluatex.pdf
%    TDS:doc/latex/oberdiek/test/ifluatex-test1.tex
%    TDS:doc/latex/oberdiek/test/ifluatex-test2.tex
%    TDS:doc/latex/oberdiek/test/ifluatex-test3.tex
%    TDS:source/latex/oberdiek/ifluatex.dtx
%
%<*ignore>
\begingroup
  \catcode123=1 %
  \catcode125=2 %
  \def\x{LaTeX2e}%
\expandafter\endgroup
\ifcase 0\ifx\install y1\fi\expandafter
         \ifx\csname processbatchFile\endcsname\relax\else1\fi
         \ifx\fmtname\x\else 1\fi\relax
\else\csname fi\endcsname
%</ignore>
%<*install>
\input docstrip.tex
\Msg{************************************************************************}
\Msg{* Installation}
\Msg{* Package: ifluatex 2016/05/16 v1.4 Provides the ifluatex switch (HO)}
\Msg{************************************************************************}

\keepsilent
\askforoverwritefalse

\let\MetaPrefix\relax
\preamble

This is a generated file.

Project: ifluatex
Version: 2016/05/16 v1.4

Copyright (C) 2007, 2009, 2010 by
   Heiko Oberdiek <heiko.oberdiek at googlemail.com>

This work may be distributed and/or modified under the
conditions of the LaTeX Project Public License, either
version 1.3c of this license or (at your option) any later
version. This version of this license is in
   http://www.latex-project.org/lppl/lppl-1-3c.txt
and the latest version of this license is in
   http://www.latex-project.org/lppl.txt
and version 1.3 or later is part of all distributions of
LaTeX version 2005/12/01 or later.

This work has the LPPL maintenance status "maintained".

This Current Maintainer of this work is Heiko Oberdiek.

The Base Interpreter refers to any `TeX-Format',
because some files are installed in TDS:tex/generic//.

This work consists of the main source file ifluatex.dtx
and the derived files
   ifluatex.sty, ifluatex.pdf, ifluatex.ins, ifluatex.drv,
   ifluatex-test1.tex, ifluatex-test2.tex, ifluatex-test3.tex.

\endpreamble
\let\MetaPrefix\DoubleperCent

\generate{%
  \file{ifluatex.ins}{\from{ifluatex.dtx}{install}}%
  \file{ifluatex.drv}{\from{ifluatex.dtx}{driver}}%
  \usedir{tex/generic/oberdiek}%
  \file{ifluatex.sty}{\from{ifluatex.dtx}{package}}%
  \usedir{doc/latex/oberdiek/test}%
  \file{ifluatex-test1.tex}{\from{ifluatex.dtx}{test1}}%
  \file{ifluatex-test2.tex}{\from{ifluatex.dtx}{test-reload1}}%
  \file{ifluatex-test3.tex}{\from{ifluatex.dtx}{test-reload2}}%
  \nopreamble
  \nopostamble
  \usedir{source/latex/oberdiek/catalogue}%
  \file{ifluatex.xml}{\from{ifluatex.dtx}{catalogue}}%
}

\catcode32=13\relax% active space
\let =\space%
\Msg{************************************************************************}
\Msg{*}
\Msg{* To finish the installation you have to move the following}
\Msg{* file into a directory searched by TeX:}
\Msg{*}
\Msg{*     ifluatex.sty}
\Msg{*}
\Msg{* To produce the documentation run the file `ifluatex.drv'}
\Msg{* through LaTeX.}
\Msg{*}
\Msg{* Happy TeXing!}
\Msg{*}
\Msg{************************************************************************}

\endbatchfile
%</install>
%<*ignore>
\fi
%</ignore>
%<*driver>
\NeedsTeXFormat{LaTeX2e}
\ProvidesFile{ifluatex.drv}%
  [2016/05/16 v1.4 Provides the ifluatex switch (HO)]%
\documentclass{ltxdoc}
\usepackage{holtxdoc}[2011/11/22]
\begin{document}
  \DocInput{ifluatex.dtx}%
\end{document}
%</driver>
% \fi
%
%
% \CharacterTable
%  {Upper-case    \A\B\C\D\E\F\G\H\I\J\K\L\M\N\O\P\Q\R\S\T\U\V\W\X\Y\Z
%   Lower-case    \a\b\c\d\e\f\g\h\i\j\k\l\m\n\o\p\q\r\s\t\u\v\w\x\y\z
%   Digits        \0\1\2\3\4\5\6\7\8\9
%   Exclamation   \!     Double quote  \"     Hash (number) \#
%   Dollar        \$     Percent       \%     Ampersand     \&
%   Acute accent  \'     Left paren    \(     Right paren   \)
%   Asterisk      \*     Plus          \+     Comma         \,
%   Minus         \-     Point         \.     Solidus       \/
%   Colon         \:     Semicolon     \;     Less than     \<
%   Equals        \=     Greater than  \>     Question mark \?
%   Commercial at \@     Left bracket  \[     Backslash     \\
%   Right bracket \]     Circumflex    \^     Underscore    \_
%   Grave accent  \`     Left brace    \{     Vertical bar  \|
%   Right brace   \}     Tilde         \~}
%
% \GetFileInfo{ifluatex.drv}
%
% \title{The \xpackage{ifluatex} package}
% \date{2016/05/16 v1.4}
% \author{Heiko Oberdiek\thanks
% {Please report any issues at https://github.com/ho-tex/oberdiek/issues}\\
% \xemail{heiko.oberdiek at googlemail.com}}
%
% \maketitle
%
% \begin{abstract}
% This package looks for \LuaTeX\ regardless of its mode
% and provides the switch \cs{ifluatex}. Also it makes
% \cs{luatexversion} available if it is not present.
% It works with \plainTeX\ or \LaTeX.
% \end{abstract}
%
% \tableofcontents
%
% \section{Documentation}
%
% The package \xpackage{ifluatex} can be used with both \plainTeX\
% and \LaTeX:
% \begin{description}
% \item[\plainTeX:] |\input ifluatex.sty|
% \item[\LaTeXe:]   |\usepackage{ifluatex}|
% \end{description}
%
% \DescribeMacro{\ifluatex}
% The package provides the switch \cs{ifluatex}:
% \begin{quote}
%   |\ifluatex|\\
%   \hspace{1.5em}\LuaTeX\ is running\\
%   |\else|\\
%   \hspace{1.5em}Without \LuaTeX\\
%   |\fi|
% \end{quote}
%
% Since version 0.39 \LuaTeX\ only provides \cs{directlua} at startup
% time. Also the syntax of \cs{directlua} changed in version 0.36.
% Thus the user might want to check the LuaTeX version.
% Therefore this package also makes \cs{luatexversion} and
% \cs{luatexrevision} available, if it is not yet done.
%
% If you want to detect the mode (DVI or PDF), then use package
% \xpackage{ifpdf}. \LuaTeX\ has inherited \cs{pdfoutput} from \pdfTeX.
%
% \StopEventually{
% }
%
% \section{Implementation}
%
%    \begin{macrocode}
%<*package>
%    \end{macrocode}
%
% \subsection{Reload check and package identification}
%    Reload check, especially if the package is not used with \LaTeX.
%    \begin{macrocode}
\begingroup\catcode61\catcode48\catcode32=10\relax%
  \catcode13=5 % ^^M
  \endlinechar=13 %
  \catcode35=6 % #
  \catcode39=12 % '
  \catcode44=12 % ,
  \catcode45=12 % -
  \catcode46=12 % .
  \catcode58=12 % :
  \catcode64=11 % @
  \catcode123=1 % {
  \catcode125=2 % }
  \expandafter\let\expandafter\x\csname ver@ifluatex.sty\endcsname
  \ifx\x\relax % plain-TeX, first loading
  \else
    \def\empty{}%
    \ifx\x\empty % LaTeX, first loading,
      % variable is initialized, but \ProvidesPackage not yet seen
    \else
      \expandafter\ifx\csname PackageInfo\endcsname\relax
        \def\x#1#2{%
          \immediate\write-1{Package #1 Info: #2.}%
        }%
      \else
        \def\x#1#2{\PackageInfo{#1}{#2, stopped}}%
      \fi
      \x{ifluatex}{The package is already loaded}%
      \aftergroup\endinput
    \fi
  \fi
\endgroup%
%    \end{macrocode}
%    Package identification:
%    \begin{macrocode}
\begingroup\catcode61\catcode48\catcode32=10\relax%
  \catcode13=5 % ^^M
  \endlinechar=13 %
  \catcode35=6 % #
  \catcode39=12 % '
  \catcode40=12 % (
  \catcode41=12 % )
  \catcode44=12 % ,
  \catcode45=12 % -
  \catcode46=12 % .
  \catcode47=12 % /
  \catcode58=12 % :
  \catcode64=11 % @
  \catcode91=12 % [
  \catcode93=12 % ]
  \catcode123=1 % {
  \catcode125=2 % }
  \expandafter\ifx\csname ProvidesPackage\endcsname\relax
    \def\x#1#2#3[#4]{\endgroup
      \immediate\write-1{Package: #3 #4}%
      \xdef#1{#4}%
    }%
  \else
    \def\x#1#2[#3]{\endgroup
      #2[{#3}]%
      \ifx#1\@undefined
        \xdef#1{#3}%
      \fi
      \ifx#1\relax
        \xdef#1{#3}%
      \fi
    }%
  \fi
\expandafter\x\csname ver@ifluatex.sty\endcsname
\ProvidesPackage{ifluatex}%
  [2016/05/16 v1.4 Provides the ifluatex switch (HO)]%
%    \end{macrocode}
%
% \subsection{Catcodes}
%
%    \begin{macrocode}
\begingroup\catcode61\catcode48\catcode32=10\relax%
  \catcode13=5 % ^^M
  \endlinechar=13 %
  \catcode123=1 % {
  \catcode125=2 % }
  \catcode64=11 % @
  \def\x{\endgroup
    \expandafter\edef\csname ifluatex@AtEnd\endcsname{%
      \endlinechar=\the\endlinechar\relax
      \catcode13=\the\catcode13\relax
      \catcode32=\the\catcode32\relax
      \catcode35=\the\catcode35\relax
      \catcode61=\the\catcode61\relax
      \catcode64=\the\catcode64\relax
      \catcode123=\the\catcode123\relax
      \catcode125=\the\catcode125\relax
    }%
  }%
\x\catcode61\catcode48\catcode32=10\relax%
\catcode13=5 % ^^M
\endlinechar=13 %
\catcode35=6 % #
\catcode64=11 % @
\catcode123=1 % {
\catcode125=2 % }
\def\TMP@EnsureCode#1#2{%
  \edef\ifluatex@AtEnd{%
    \ifluatex@AtEnd
    \catcode#1=\the\catcode#1\relax
  }%
  \catcode#1=#2\relax
}
\TMP@EnsureCode{10}{12}% ^^J
\TMP@EnsureCode{39}{12}% '
\TMP@EnsureCode{40}{12}% (
\TMP@EnsureCode{41}{12}% )
\TMP@EnsureCode{44}{12}% ,
\TMP@EnsureCode{45}{12}% -
\TMP@EnsureCode{46}{12}% .
\TMP@EnsureCode{47}{12}% /
\TMP@EnsureCode{58}{12}% :
\TMP@EnsureCode{60}{12}% <
\TMP@EnsureCode{94}{7}% ^
\TMP@EnsureCode{96}{12}% `
\edef\ifluatex@AtEnd{\ifluatex@AtEnd\noexpand\endinput}
%    \end{macrocode}
%
% \subsection{Macro for error messages}
%
%    \begin{macro}{\ifluatex@Error}
%    \begin{macrocode}
\begingroup\expandafter\expandafter\expandafter\endgroup
\expandafter\ifx\csname PackageError\endcsname\relax
  \def\ifluatex@Error#1#2{%
    \begingroup
      \newlinechar=10 %
      \def\MessageBreak{^^J}%
      \edef\x{\errhelp{#2}}%
      \x
      \errmessage{Package ifluatex Error: #1}%
    \endgroup
  }%
\else
  \def\ifluatex@Error{%
    \PackageError{ifluatex}%
  }%
\fi
%    \end{macrocode}
%    \end{macro}
%
% \subsection{Check for previously defined \cs{ifluatex}}
%
%    \begin{macrocode}
\begingroup
  \expandafter\ifx\csname ifluatex\endcsname\relax
  \else
    \edef\i/{\expandafter\string\csname ifluatex\endcsname}%
    \ifluatex@Error{Name clash, \i/ is already defined}{%
      Incompatible versions of \i/ can cause problems,\MessageBreak
      therefore package loading is aborted.%
    }%
    \endgroup
    \expandafter\ifluatex@AtEnd
  \fi%
\endgroup
%    \end{macrocode}
%
% \subsection{\cs{ifluatex}}
%
%    \begin{macro}{\ifluatex}
%    \begin{macrocode}
\let\ifluatex\iffalse
%    \end{macrocode}
%    \end{macro}
%
%    Test \cs{luatexversion}. Is it  defined and different from
%    \cs{relax}? Someone could have used \LaTeX\ internal
%    \cs{@ifundefined}, or something else involving.
%    Notice, \cs{csname} is executed inside a group for the test
%    to cancel the side effect of \cs{csname}.
%    \begin{macrocode}
\begingroup\expandafter\expandafter\expandafter\endgroup
\expandafter\ifx\csname luatexversion\endcsname\relax
\else
  \expandafter\let\csname ifluatex\expandafter\endcsname
                  \csname iftrue\endcsname
\fi
%    \end{macrocode}
%
% \subsection{Lua\TeX\ v0.39}
%
%     Starting with version 0.39 \LuaTeX\ wants to provide \cs{directlua}
%     as only primitive at startup time beyond vanilla \TeX's primitives.
%     Then \cs{directlua} exists, but \cs{luatexversion} cannot be found.
%     Unhappily also the syntax of \cs{directlua} changed in v0.36,
%     thus the user would want to check \cs{luatexversion}.
%     Therefore we make \cs{luatexversion} available using
%     \LuaTeX's Lua function |tex.enableprimitives|.
%
%    \begin{macrocode}
\ifluatex
\else
  \begingroup\expandafter\expandafter\expandafter\endgroup
  \expandafter\ifx\csname directlua\endcsname\relax
  \else
    \expandafter\let\csname ifluatex\expandafter\endcsname
                    \csname iftrue\endcsname
    \begingroup
      \newlinechar=10 %
      \endlinechar=\newlinechar%
      \ifnum0%
          \directlua{%
            if tex.enableprimitives then
              tex.enableprimitives('ifluatex', {'luatexversion'})
              tex.print('1')
            end
          }%
          \ifx\ifluatexluatexversion\@undefined\else 1\fi %
          =11 %
        \global\let\luatexversion\ifluatexluatexversion%
      \else%
        \ifluatex@Error{%
          Missing \string\luatexversion%
        }{%
          Update LuaTeX.%
        }%
      \fi%
    \endgroup%
  \fi
\fi
%    \end{macrocode}
%    \begin{macrocode}
\ifluatex
  \begingroup\expandafter\expandafter\expandafter\endgroup
  \expandafter\ifx\csname luatexrevision\endcsname\relax
    \ifnum\luatexversion<36 %
    \else
      \begingroup
        \ifx\luatexrevision\relax
          \let\luatexrevision\@undefined
        \fi
        \newlinechar=10 %
        \endlinechar=\newlinechar%
        \ifcase0%
            \directlua{%
              if tex.enableprimitives then
                tex.enableprimitives('ifluatex', {'luatexrevision'})
              else
                tex.print('1')
              end
            }%
            \ifx\ifluatexluatexrevision\@undefined 1\fi%
            \relax%
          \global\let\luatexrevision\ifluatexluatexrevision%
        \fi%
      \endgroup%
    \fi
    \begingroup\expandafter\expandafter\expandafter\endgroup
    \expandafter\ifx\csname luatexrevision\endcsname\relax
      \ifluatex@Error{%
        Missing \string\luatexrevision%
      }{%
        Update LuaTeX.%
      }%
    \fi
  \fi
\fi
%    \end{macrocode}
%
% \subsection{Protocol entry}
%
%     Log comment:
%    \begin{macrocode}
\begingroup
  \expandafter\ifx\csname PackageInfo\endcsname\relax
    \def\x#1#2{%
      \immediate\write-1{Package #1 Info: #2.}%
    }%
  \else
    \let\x\PackageInfo
    \expandafter\let\csname on@line\endcsname\empty
  \fi
  \x{ifluatex}{LuaTeX \ifluatex\else not \fi detected}%
\endgroup
%    \end{macrocode}
%    \begin{macrocode}
\ifluatex@AtEnd%
%    \end{macrocode}
%    \begin{macrocode}
%</package>
%    \end{macrocode}
%
% \section{Test}
%
% \subsection{Catcode checks for loading}
%
%    \begin{macrocode}
%<*test1>
%    \end{macrocode}
%    \begin{macrocode}
\catcode`\{=1 %
\catcode`\}=2 %
\catcode`\#=6 %
\catcode`\@=11 %
\expandafter\ifx\csname count@\endcsname\relax
  \countdef\count@=255 %
\fi
\expandafter\ifx\csname @gobble\endcsname\relax
  \long\def\@gobble#1{}%
\fi
\expandafter\ifx\csname @firstofone\endcsname\relax
  \long\def\@firstofone#1{#1}%
\fi
\expandafter\ifx\csname loop\endcsname\relax
  \expandafter\@firstofone
\else
  \expandafter\@gobble
\fi
{%
  \def\loop#1\repeat{%
    \def\body{#1}%
    \iterate
  }%
  \def\iterate{%
    \body
      \let\next\iterate
    \else
      \let\next\relax
    \fi
    \next
  }%
  \let\repeat=\fi
}%
\def\RestoreCatcodes{}
\count@=0 %
\loop
  \edef\RestoreCatcodes{%
    \RestoreCatcodes
    \catcode\the\count@=\the\catcode\count@\relax
  }%
\ifnum\count@<255 %
  \advance\count@ 1 %
\repeat

\def\RangeCatcodeInvalid#1#2{%
  \count@=#1\relax
  \loop
    \catcode\count@=15 %
  \ifnum\count@<#2\relax
    \advance\count@ 1 %
  \repeat
}
\def\RangeCatcodeCheck#1#2#3{%
  \count@=#1\relax
  \loop
    \ifnum#3=\catcode\count@
    \else
      \errmessage{%
        Character \the\count@\space
        with wrong catcode \the\catcode\count@\space
        instead of \number#3%
      }%
    \fi
  \ifnum\count@<#2\relax
    \advance\count@ 1 %
  \repeat
}
\def\space{ }
\expandafter\ifx\csname LoadCommand\endcsname\relax
  \def\LoadCommand{\input ifluatex.sty\relax}%
\fi
\def\Test{%
  \RangeCatcodeInvalid{0}{47}%
  \RangeCatcodeInvalid{58}{64}%
  \RangeCatcodeInvalid{91}{96}%
  \RangeCatcodeInvalid{123}{255}%
  \catcode`\@=12 %
  \catcode`\\=0 %
  \catcode`\%=14 %
  \LoadCommand
  \RangeCatcodeCheck{0}{36}{15}%
  \RangeCatcodeCheck{37}{37}{14}%
  \RangeCatcodeCheck{38}{47}{15}%
  \RangeCatcodeCheck{48}{57}{12}%
  \RangeCatcodeCheck{58}{63}{15}%
  \RangeCatcodeCheck{64}{64}{12}%
  \RangeCatcodeCheck{65}{90}{11}%
  \RangeCatcodeCheck{91}{91}{15}%
  \RangeCatcodeCheck{92}{92}{0}%
  \RangeCatcodeCheck{93}{96}{15}%
  \RangeCatcodeCheck{97}{122}{11}%
  \RangeCatcodeCheck{123}{255}{15}%
  \RestoreCatcodes
}
\Test
\csname @@end\endcsname
\end
%    \end{macrocode}
%    \begin{macrocode}
%</test1>
%    \end{macrocode}
%
% \section{Reload check for plain}
%
%    \begin{macrocode}
%<*test-reload1>
\input ifluatex.sty\relax
\input ifluatex.sty\relax
\csname @@end\endcsname\end
%</test-reload1>
%    \end{macrocode}
%
%    \begin{macrocode}
%<*test-reload2>
\input miniltx.tex\relax
\input ifluatex.sty\relax
\input ifluatex.sty\relax
\csname @@end\endcsname\end
%</test-reload2>
%    \end{macrocode}
%
% \section{Installation}
%
% \subsection{Download}
%
% \paragraph{Package.} This package is available on
% CTAN\footnote{\url{http://ctan.org/pkg/ifluatex}}:
% \begin{description}
% \item[\CTAN{macros/latex/contrib/oberdiek/ifluatex.dtx}] The source file.
% \item[\CTAN{macros/latex/contrib/oberdiek/ifluatex.pdf}] Documentation.
% \end{description}
%
%
% \paragraph{Bundle.} All the packages of the bundle `oberdiek'
% are also available in a TDS compliant ZIP archive. There
% the packages are already unpacked and the documentation files
% are generated. The files and directories obey the TDS standard.
% \begin{description}
% \item[\CTAN{install/macros/latex/contrib/oberdiek.tds.zip}]
% \end{description}
% \emph{TDS} refers to the standard ``A Directory Structure
% for \TeX\ Files'' (\CTAN{tds/tds.pdf}). Directories
% with \xfile{texmf} in their name are usually organized this way.
%
% \subsection{Bundle installation}
%
% \paragraph{Unpacking.} Unpack the \xfile{oberdiek.tds.zip} in the
% TDS tree (also known as \xfile{texmf} tree) of your choice.
% Example (linux):
% \begin{quote}
%   |unzip oberdiek.tds.zip -d ~/texmf|
% \end{quote}
%
% \paragraph{Script installation.}
% Check the directory \xfile{TDS:scripts/oberdiek/} for
% scripts that need further installation steps.
% Package \xpackage{attachfile2} comes with the Perl script
% \xfile{pdfatfi.pl} that should be installed in such a way
% that it can be called as \texttt{pdfatfi}.
% Example (linux):
% \begin{quote}
%   |chmod +x scripts/oberdiek/pdfatfi.pl|\\
%   |cp scripts/oberdiek/pdfatfi.pl /usr/local/bin/|
% \end{quote}
%
% \subsection{Package installation}
%
% \paragraph{Unpacking.} The \xfile{.dtx} file is a self-extracting
% \docstrip\ archive. The files are extracted by running the
% \xfile{.dtx} through \plainTeX:
% \begin{quote}
%   \verb|tex ifluatex.dtx|
% \end{quote}
%
% \paragraph{TDS.} Now the different files must be moved into
% the different directories in your installation TDS tree
% (also known as \xfile{texmf} tree):
% \begin{quote}
% \def\t{^^A
% \begin{tabular}{@{}>{\ttfamily}l@{ $\rightarrow$ }>{\ttfamily}l@{}}
%   ifluatex.sty & tex/generic/oberdiek/ifluatex.sty\\
%   ifluatex.pdf & doc/latex/oberdiek/ifluatex.pdf\\
%   test/ifluatex-test1.tex & doc/latex/oberdiek/test/ifluatex-test1.tex\\
%   test/ifluatex-test2.tex & doc/latex/oberdiek/test/ifluatex-test2.tex\\
%   test/ifluatex-test3.tex & doc/latex/oberdiek/test/ifluatex-test3.tex\\
%   ifluatex.dtx & source/latex/oberdiek/ifluatex.dtx\\
% \end{tabular}^^A
% }^^A
% \sbox0{\t}^^A
% \ifdim\wd0>\linewidth
%   \begingroup
%     \advance\linewidth by\leftmargin
%     \advance\linewidth by\rightmargin
%   \edef\x{\endgroup
%     \def\noexpand\lw{\the\linewidth}^^A
%   }\x
%   \def\lwbox{^^A
%     \leavevmode
%     \hbox to \linewidth{^^A
%       \kern-\leftmargin\relax
%       \hss
%       \usebox0
%       \hss
%       \kern-\rightmargin\relax
%     }^^A
%   }^^A
%   \ifdim\wd0>\lw
%     \sbox0{\small\t}^^A
%     \ifdim\wd0>\linewidth
%       \ifdim\wd0>\lw
%         \sbox0{\footnotesize\t}^^A
%         \ifdim\wd0>\linewidth
%           \ifdim\wd0>\lw
%             \sbox0{\scriptsize\t}^^A
%             \ifdim\wd0>\linewidth
%               \ifdim\wd0>\lw
%                 \sbox0{\tiny\t}^^A
%                 \ifdim\wd0>\linewidth
%                   \lwbox
%                 \else
%                   \usebox0
%                 \fi
%               \else
%                 \lwbox
%               \fi
%             \else
%               \usebox0
%             \fi
%           \else
%             \lwbox
%           \fi
%         \else
%           \usebox0
%         \fi
%       \else
%         \lwbox
%       \fi
%     \else
%       \usebox0
%     \fi
%   \else
%     \lwbox
%   \fi
% \else
%   \usebox0
% \fi
% \end{quote}
% If you have a \xfile{docstrip.cfg} that configures and enables \docstrip's
% TDS installing feature, then some files can already be in the right
% place, see the documentation of \docstrip.
%
% \subsection{Refresh file name databases}
%
% If your \TeX~distribution
% (\teTeX, \mikTeX, \dots) relies on file name databases, you must refresh
% these. For example, \teTeX\ users run \verb|texhash| or
% \verb|mktexlsr|.
%
% \subsection{Some details for the interested}
%
% \paragraph{Attached source.}
%
% The PDF documentation on CTAN also includes the
% \xfile{.dtx} source file. It can be extracted by
% AcrobatReader 6 or higher. Another option is \textsf{pdftk},
% e.g. unpack the file into the current directory:
% \begin{quote}
%   \verb|pdftk ifluatex.pdf unpack_files output .|
% \end{quote}
%
% \paragraph{Unpacking with \LaTeX.}
% The \xfile{.dtx} chooses its action depending on the format:
% \begin{description}
% \item[\plainTeX:] Run \docstrip\ and extract the files.
% \item[\LaTeX:] Generate the documentation.
% \end{description}
% If you insist on using \LaTeX\ for \docstrip\ (really,
% \docstrip\ does not need \LaTeX), then inform the autodetect routine
% about your intention:
% \begin{quote}
%   \verb|latex \let\install=y\input{ifluatex.dtx}|
% \end{quote}
% Do not forget to quote the argument according to the demands
% of your shell.
%
% \paragraph{Generating the documentation.}
% You can use both the \xfile{.dtx} or the \xfile{.drv} to generate
% the documentation. The process can be configured by the
% configuration file \xfile{ltxdoc.cfg}. For instance, put this
% line into this file, if you want to have A4 as paper format:
% \begin{quote}
%   \verb|\PassOptionsToClass{a4paper}{article}|
% \end{quote}
% An example follows how to generate the
% documentation with pdf\LaTeX:
% \begin{quote}
%\begin{verbatim}
%pdflatex ifluatex.dtx
%makeindex -s gind.ist ifluatex.idx
%pdflatex ifluatex.dtx
%makeindex -s gind.ist ifluatex.idx
%pdflatex ifluatex.dtx
%\end{verbatim}
% \end{quote}
%
% \section{Catalogue}
%
% The following XML file can be used as source for the
% \href{http://mirror.ctan.org/help/Catalogue/catalogue.html}{\TeX\ Catalogue}.
% The elements \texttt{caption} and \texttt{description} are imported
% from the original XML file from the Catalogue.
% The name of the XML file in the Catalogue is \xfile{ifluatex.xml}.
%    \begin{macrocode}
%<*catalogue>
<?xml version='1.0' encoding='us-ascii'?>
<!DOCTYPE entry SYSTEM 'catalogue.dtd'>
<entry datestamp='$Date$' modifier='$Author$' id='ifluatex'>
  <name>ifluatex</name>
  <caption>Provides the \ifluatex switch.</caption>
  <authorref id='auth:oberdiek'/>
  <copyright owner='Heiko Oberdiek' year='2007,2009,2010'/>
  <license type='lppl1.3'/>
  <version number='1.4'/>
  <description>
    The package looks for  LuaTeX regardless of its mode and provides
    the switch <tt>\ifluatex</tt>; it works with Plain TeX or LaTeX.
    <p/>
    The package is part of the <xref refid='oberdiek'>oberdiek</xref>
    bundle.
  </description>
  <documentation details='Package documentation'
      href='ctan:/macros/latex/contrib/oberdiek/ifluatex.pdf'/>
  <ctan file='true' path='/macros/latex/contrib/oberdiek/ifluatex.dtx'/>
  <miktex location='oberdiek'/>
  <texlive location='ifluatex'/>
  <install path='/macros/latex/contrib/oberdiek/oberdiek.tds.zip'/>
</entry>
%</catalogue>
%    \end{macrocode}
%
% \begin{History}
%   \begin{Version}{2007/12/12 v1.0}
%   \item
%     First public version.
%   \end{Version}
%   \begin{Version}{2009/04/10 v1.1}
%   \item
%     Test adopted for \LuaTeX\ 0.39.
%   \item
%     Makes \cs{luatexversion} available.
%   \end{Version}
%   \begin{Version}{2009/04/17 v1.2}
%   \item
%     Fixes (Manuel P\'egouri\'e-Gonnard).
%   \item
%     \cs{luatextrue} and \cs{luatexfalse} are no longer defined.
%   \item
%     Makes \cs{luatexrevision} available, too.
%   \end{Version}
%   \begin{Version}{2010/03/01 v1.3}
%   \item
%     Line ends fixed in case \cs{endlinechar} = \cs{newlinechar}.
%   \end{Version}
%   \begin{Version}{2016/05/16 v1.4}
%   \item
%     Documentation updates.
%   \end{Version}
% \end{History}
%
% \PrintIndex
%
% \Finale
\endinput

%        (quote the arguments according to the demands of your shell)
%
% Documentation:
%    (a) If ifluatex.drv is present:
%           latex ifluatex.drv
%    (b) Without ifluatex.drv:
%           latex ifluatex.dtx; ...
%    The class ltxdoc loads the configuration file ltxdoc.cfg
%    if available. Here you can specify further options, e.g.
%    use A4 as paper format:
%       \PassOptionsToClass{a4paper}{article}
%
%    Programm calls to get the documentation (example):
%       pdflatex ifluatex.dtx
%       makeindex -s gind.ist ifluatex.idx
%       pdflatex ifluatex.dtx
%       makeindex -s gind.ist ifluatex.idx
%       pdflatex ifluatex.dtx
%
% Installation:
%    TDS:tex/generic/oberdiek/ifluatex.sty
%    TDS:doc/latex/oberdiek/ifluatex.pdf
%    TDS:doc/latex/oberdiek/test/ifluatex-test1.tex
%    TDS:doc/latex/oberdiek/test/ifluatex-test2.tex
%    TDS:doc/latex/oberdiek/test/ifluatex-test3.tex
%    TDS:source/latex/oberdiek/ifluatex.dtx
%
%<*ignore>
\begingroup
  \catcode123=1 %
  \catcode125=2 %
  \def\x{LaTeX2e}%
\expandafter\endgroup
\ifcase 0\ifx\install y1\fi\expandafter
         \ifx\csname processbatchFile\endcsname\relax\else1\fi
         \ifx\fmtname\x\else 1\fi\relax
\else\csname fi\endcsname
%</ignore>
%<*install>
\input docstrip.tex
\Msg{************************************************************************}
\Msg{* Installation}
\Msg{* Package: ifluatex 2016/05/16 v1.4 Provides the ifluatex switch (HO)}
\Msg{************************************************************************}

\keepsilent
\askforoverwritefalse

\let\MetaPrefix\relax
\preamble

This is a generated file.

Project: ifluatex
Version: 2016/05/16 v1.4

Copyright (C) 2007, 2009, 2010 by
   Heiko Oberdiek <heiko.oberdiek at googlemail.com>

This work may be distributed and/or modified under the
conditions of the LaTeX Project Public License, either
version 1.3c of this license or (at your option) any later
version. This version of this license is in
   http://www.latex-project.org/lppl/lppl-1-3c.txt
and the latest version of this license is in
   http://www.latex-project.org/lppl.txt
and version 1.3 or later is part of all distributions of
LaTeX version 2005/12/01 or later.

This work has the LPPL maintenance status "maintained".

This Current Maintainer of this work is Heiko Oberdiek.

The Base Interpreter refers to any `TeX-Format',
because some files are installed in TDS:tex/generic//.

This work consists of the main source file ifluatex.dtx
and the derived files
   ifluatex.sty, ifluatex.pdf, ifluatex.ins, ifluatex.drv,
   ifluatex-test1.tex, ifluatex-test2.tex, ifluatex-test3.tex.

\endpreamble
\let\MetaPrefix\DoubleperCent

\generate{%
  \file{ifluatex.ins}{\from{ifluatex.dtx}{install}}%
  \file{ifluatex.drv}{\from{ifluatex.dtx}{driver}}%
  \usedir{tex/generic/oberdiek}%
  \file{ifluatex.sty}{\from{ifluatex.dtx}{package}}%
  \usedir{doc/latex/oberdiek/test}%
  \file{ifluatex-test1.tex}{\from{ifluatex.dtx}{test1}}%
  \file{ifluatex-test2.tex}{\from{ifluatex.dtx}{test-reload1}}%
  \file{ifluatex-test3.tex}{\from{ifluatex.dtx}{test-reload2}}%
  \nopreamble
  \nopostamble
  \usedir{source/latex/oberdiek/catalogue}%
  \file{ifluatex.xml}{\from{ifluatex.dtx}{catalogue}}%
}

\catcode32=13\relax% active space
\let =\space%
\Msg{************************************************************************}
\Msg{*}
\Msg{* To finish the installation you have to move the following}
\Msg{* file into a directory searched by TeX:}
\Msg{*}
\Msg{*     ifluatex.sty}
\Msg{*}
\Msg{* To produce the documentation run the file `ifluatex.drv'}
\Msg{* through LaTeX.}
\Msg{*}
\Msg{* Happy TeXing!}
\Msg{*}
\Msg{************************************************************************}

\endbatchfile
%</install>
%<*ignore>
\fi
%</ignore>
%<*driver>
\NeedsTeXFormat{LaTeX2e}
\ProvidesFile{ifluatex.drv}%
  [2016/05/16 v1.4 Provides the ifluatex switch (HO)]%
\documentclass{ltxdoc}
\usepackage{holtxdoc}[2011/11/22]
\begin{document}
  \DocInput{ifluatex.dtx}%
\end{document}
%</driver>
% \fi
%
%
% \CharacterTable
%  {Upper-case    \A\B\C\D\E\F\G\H\I\J\K\L\M\N\O\P\Q\R\S\T\U\V\W\X\Y\Z
%   Lower-case    \a\b\c\d\e\f\g\h\i\j\k\l\m\n\o\p\q\r\s\t\u\v\w\x\y\z
%   Digits        \0\1\2\3\4\5\6\7\8\9
%   Exclamation   \!     Double quote  \"     Hash (number) \#
%   Dollar        \$     Percent       \%     Ampersand     \&
%   Acute accent  \'     Left paren    \(     Right paren   \)
%   Asterisk      \*     Plus          \+     Comma         \,
%   Minus         \-     Point         \.     Solidus       \/
%   Colon         \:     Semicolon     \;     Less than     \<
%   Equals        \=     Greater than  \>     Question mark \?
%   Commercial at \@     Left bracket  \[     Backslash     \\
%   Right bracket \]     Circumflex    \^     Underscore    \_
%   Grave accent  \`     Left brace    \{     Vertical bar  \|
%   Right brace   \}     Tilde         \~}
%
% \GetFileInfo{ifluatex.drv}
%
% \title{The \xpackage{ifluatex} package}
% \date{2016/05/16 v1.4}
% \author{Heiko Oberdiek\thanks
% {Please report any issues at https://github.com/ho-tex/oberdiek/issues}\\
% \xemail{heiko.oberdiek at googlemail.com}}
%
% \maketitle
%
% \begin{abstract}
% This package looks for \LuaTeX\ regardless of its mode
% and provides the switch \cs{ifluatex}. Also it makes
% \cs{luatexversion} available if it is not present.
% It works with \plainTeX\ or \LaTeX.
% \end{abstract}
%
% \tableofcontents
%
% \section{Documentation}
%
% The package \xpackage{ifluatex} can be used with both \plainTeX\
% and \LaTeX:
% \begin{description}
% \item[\plainTeX:] |\input ifluatex.sty|
% \item[\LaTeXe:]   |\usepackage{ifluatex}|
% \end{description}
%
% \DescribeMacro{\ifluatex}
% The package provides the switch \cs{ifluatex}:
% \begin{quote}
%   |\ifluatex|\\
%   \hspace{1.5em}\LuaTeX\ is running\\
%   |\else|\\
%   \hspace{1.5em}Without \LuaTeX\\
%   |\fi|
% \end{quote}
%
% Since version 0.39 \LuaTeX\ only provides \cs{directlua} at startup
% time. Also the syntax of \cs{directlua} changed in version 0.36.
% Thus the user might want to check the LuaTeX version.
% Therefore this package also makes \cs{luatexversion} and
% \cs{luatexrevision} available, if it is not yet done.
%
% If you want to detect the mode (DVI or PDF), then use package
% \xpackage{ifpdf}. \LuaTeX\ has inherited \cs{pdfoutput} from \pdfTeX.
%
% \StopEventually{
% }
%
% \section{Implementation}
%
%    \begin{macrocode}
%<*package>
%    \end{macrocode}
%
% \subsection{Reload check and package identification}
%    Reload check, especially if the package is not used with \LaTeX.
%    \begin{macrocode}
\begingroup\catcode61\catcode48\catcode32=10\relax%
  \catcode13=5 % ^^M
  \endlinechar=13 %
  \catcode35=6 % #
  \catcode39=12 % '
  \catcode44=12 % ,
  \catcode45=12 % -
  \catcode46=12 % .
  \catcode58=12 % :
  \catcode64=11 % @
  \catcode123=1 % {
  \catcode125=2 % }
  \expandafter\let\expandafter\x\csname ver@ifluatex.sty\endcsname
  \ifx\x\relax % plain-TeX, first loading
  \else
    \def\empty{}%
    \ifx\x\empty % LaTeX, first loading,
      % variable is initialized, but \ProvidesPackage not yet seen
    \else
      \expandafter\ifx\csname PackageInfo\endcsname\relax
        \def\x#1#2{%
          \immediate\write-1{Package #1 Info: #2.}%
        }%
      \else
        \def\x#1#2{\PackageInfo{#1}{#2, stopped}}%
      \fi
      \x{ifluatex}{The package is already loaded}%
      \aftergroup\endinput
    \fi
  \fi
\endgroup%
%    \end{macrocode}
%    Package identification:
%    \begin{macrocode}
\begingroup\catcode61\catcode48\catcode32=10\relax%
  \catcode13=5 % ^^M
  \endlinechar=13 %
  \catcode35=6 % #
  \catcode39=12 % '
  \catcode40=12 % (
  \catcode41=12 % )
  \catcode44=12 % ,
  \catcode45=12 % -
  \catcode46=12 % .
  \catcode47=12 % /
  \catcode58=12 % :
  \catcode64=11 % @
  \catcode91=12 % [
  \catcode93=12 % ]
  \catcode123=1 % {
  \catcode125=2 % }
  \expandafter\ifx\csname ProvidesPackage\endcsname\relax
    \def\x#1#2#3[#4]{\endgroup
      \immediate\write-1{Package: #3 #4}%
      \xdef#1{#4}%
    }%
  \else
    \def\x#1#2[#3]{\endgroup
      #2[{#3}]%
      \ifx#1\@undefined
        \xdef#1{#3}%
      \fi
      \ifx#1\relax
        \xdef#1{#3}%
      \fi
    }%
  \fi
\expandafter\x\csname ver@ifluatex.sty\endcsname
\ProvidesPackage{ifluatex}%
  [2016/05/16 v1.4 Provides the ifluatex switch (HO)]%
%    \end{macrocode}
%
% \subsection{Catcodes}
%
%    \begin{macrocode}
\begingroup\catcode61\catcode48\catcode32=10\relax%
  \catcode13=5 % ^^M
  \endlinechar=13 %
  \catcode123=1 % {
  \catcode125=2 % }
  \catcode64=11 % @
  \def\x{\endgroup
    \expandafter\edef\csname ifluatex@AtEnd\endcsname{%
      \endlinechar=\the\endlinechar\relax
      \catcode13=\the\catcode13\relax
      \catcode32=\the\catcode32\relax
      \catcode35=\the\catcode35\relax
      \catcode61=\the\catcode61\relax
      \catcode64=\the\catcode64\relax
      \catcode123=\the\catcode123\relax
      \catcode125=\the\catcode125\relax
    }%
  }%
\x\catcode61\catcode48\catcode32=10\relax%
\catcode13=5 % ^^M
\endlinechar=13 %
\catcode35=6 % #
\catcode64=11 % @
\catcode123=1 % {
\catcode125=2 % }
\def\TMP@EnsureCode#1#2{%
  \edef\ifluatex@AtEnd{%
    \ifluatex@AtEnd
    \catcode#1=\the\catcode#1\relax
  }%
  \catcode#1=#2\relax
}
\TMP@EnsureCode{10}{12}% ^^J
\TMP@EnsureCode{39}{12}% '
\TMP@EnsureCode{40}{12}% (
\TMP@EnsureCode{41}{12}% )
\TMP@EnsureCode{44}{12}% ,
\TMP@EnsureCode{45}{12}% -
\TMP@EnsureCode{46}{12}% .
\TMP@EnsureCode{47}{12}% /
\TMP@EnsureCode{58}{12}% :
\TMP@EnsureCode{60}{12}% <
\TMP@EnsureCode{94}{7}% ^
\TMP@EnsureCode{96}{12}% `
\edef\ifluatex@AtEnd{\ifluatex@AtEnd\noexpand\endinput}
%    \end{macrocode}
%
% \subsection{Macro for error messages}
%
%    \begin{macro}{\ifluatex@Error}
%    \begin{macrocode}
\begingroup\expandafter\expandafter\expandafter\endgroup
\expandafter\ifx\csname PackageError\endcsname\relax
  \def\ifluatex@Error#1#2{%
    \begingroup
      \newlinechar=10 %
      \def\MessageBreak{^^J}%
      \edef\x{\errhelp{#2}}%
      \x
      \errmessage{Package ifluatex Error: #1}%
    \endgroup
  }%
\else
  \def\ifluatex@Error{%
    \PackageError{ifluatex}%
  }%
\fi
%    \end{macrocode}
%    \end{macro}
%
% \subsection{Check for previously defined \cs{ifluatex}}
%
%    \begin{macrocode}
\begingroup
  \expandafter\ifx\csname ifluatex\endcsname\relax
  \else
    \edef\i/{\expandafter\string\csname ifluatex\endcsname}%
    \ifluatex@Error{Name clash, \i/ is already defined}{%
      Incompatible versions of \i/ can cause problems,\MessageBreak
      therefore package loading is aborted.%
    }%
    \endgroup
    \expandafter\ifluatex@AtEnd
  \fi%
\endgroup
%    \end{macrocode}
%
% \subsection{\cs{ifluatex}}
%
%    \begin{macro}{\ifluatex}
%    \begin{macrocode}
\let\ifluatex\iffalse
%    \end{macrocode}
%    \end{macro}
%
%    Test \cs{luatexversion}. Is it  defined and different from
%    \cs{relax}? Someone could have used \LaTeX\ internal
%    \cs{@ifundefined}, or something else involving.
%    Notice, \cs{csname} is executed inside a group for the test
%    to cancel the side effect of \cs{csname}.
%    \begin{macrocode}
\begingroup\expandafter\expandafter\expandafter\endgroup
\expandafter\ifx\csname luatexversion\endcsname\relax
\else
  \expandafter\let\csname ifluatex\expandafter\endcsname
                  \csname iftrue\endcsname
\fi
%    \end{macrocode}
%
% \subsection{Lua\TeX\ v0.39}
%
%     Starting with version 0.39 \LuaTeX\ wants to provide \cs{directlua}
%     as only primitive at startup time beyond vanilla \TeX's primitives.
%     Then \cs{directlua} exists, but \cs{luatexversion} cannot be found.
%     Unhappily also the syntax of \cs{directlua} changed in v0.36,
%     thus the user would want to check \cs{luatexversion}.
%     Therefore we make \cs{luatexversion} available using
%     \LuaTeX's Lua function |tex.enableprimitives|.
%
%    \begin{macrocode}
\ifluatex
\else
  \begingroup\expandafter\expandafter\expandafter\endgroup
  \expandafter\ifx\csname directlua\endcsname\relax
  \else
    \expandafter\let\csname ifluatex\expandafter\endcsname
                    \csname iftrue\endcsname
    \begingroup
      \newlinechar=10 %
      \endlinechar=\newlinechar%
      \ifnum0%
          \directlua{%
            if tex.enableprimitives then
              tex.enableprimitives('ifluatex', {'luatexversion'})
              tex.print('1')
            end
          }%
          \ifx\ifluatexluatexversion\@undefined\else 1\fi %
          =11 %
        \global\let\luatexversion\ifluatexluatexversion%
      \else%
        \ifluatex@Error{%
          Missing \string\luatexversion%
        }{%
          Update LuaTeX.%
        }%
      \fi%
    \endgroup%
  \fi
\fi
%    \end{macrocode}
%    \begin{macrocode}
\ifluatex
  \begingroup\expandafter\expandafter\expandafter\endgroup
  \expandafter\ifx\csname luatexrevision\endcsname\relax
    \ifnum\luatexversion<36 %
    \else
      \begingroup
        \ifx\luatexrevision\relax
          \let\luatexrevision\@undefined
        \fi
        \newlinechar=10 %
        \endlinechar=\newlinechar%
        \ifcase0%
            \directlua{%
              if tex.enableprimitives then
                tex.enableprimitives('ifluatex', {'luatexrevision'})
              else
                tex.print('1')
              end
            }%
            \ifx\ifluatexluatexrevision\@undefined 1\fi%
            \relax%
          \global\let\luatexrevision\ifluatexluatexrevision%
        \fi%
      \endgroup%
    \fi
    \begingroup\expandafter\expandafter\expandafter\endgroup
    \expandafter\ifx\csname luatexrevision\endcsname\relax
      \ifluatex@Error{%
        Missing \string\luatexrevision%
      }{%
        Update LuaTeX.%
      }%
    \fi
  \fi
\fi
%    \end{macrocode}
%
% \subsection{Protocol entry}
%
%     Log comment:
%    \begin{macrocode}
\begingroup
  \expandafter\ifx\csname PackageInfo\endcsname\relax
    \def\x#1#2{%
      \immediate\write-1{Package #1 Info: #2.}%
    }%
  \else
    \let\x\PackageInfo
    \expandafter\let\csname on@line\endcsname\empty
  \fi
  \x{ifluatex}{LuaTeX \ifluatex\else not \fi detected}%
\endgroup
%    \end{macrocode}
%    \begin{macrocode}
\ifluatex@AtEnd%
%    \end{macrocode}
%    \begin{macrocode}
%</package>
%    \end{macrocode}
%
% \section{Test}
%
% \subsection{Catcode checks for loading}
%
%    \begin{macrocode}
%<*test1>
%    \end{macrocode}
%    \begin{macrocode}
\catcode`\{=1 %
\catcode`\}=2 %
\catcode`\#=6 %
\catcode`\@=11 %
\expandafter\ifx\csname count@\endcsname\relax
  \countdef\count@=255 %
\fi
\expandafter\ifx\csname @gobble\endcsname\relax
  \long\def\@gobble#1{}%
\fi
\expandafter\ifx\csname @firstofone\endcsname\relax
  \long\def\@firstofone#1{#1}%
\fi
\expandafter\ifx\csname loop\endcsname\relax
  \expandafter\@firstofone
\else
  \expandafter\@gobble
\fi
{%
  \def\loop#1\repeat{%
    \def\body{#1}%
    \iterate
  }%
  \def\iterate{%
    \body
      \let\next\iterate
    \else
      \let\next\relax
    \fi
    \next
  }%
  \let\repeat=\fi
}%
\def\RestoreCatcodes{}
\count@=0 %
\loop
  \edef\RestoreCatcodes{%
    \RestoreCatcodes
    \catcode\the\count@=\the\catcode\count@\relax
  }%
\ifnum\count@<255 %
  \advance\count@ 1 %
\repeat

\def\RangeCatcodeInvalid#1#2{%
  \count@=#1\relax
  \loop
    \catcode\count@=15 %
  \ifnum\count@<#2\relax
    \advance\count@ 1 %
  \repeat
}
\def\RangeCatcodeCheck#1#2#3{%
  \count@=#1\relax
  \loop
    \ifnum#3=\catcode\count@
    \else
      \errmessage{%
        Character \the\count@\space
        with wrong catcode \the\catcode\count@\space
        instead of \number#3%
      }%
    \fi
  \ifnum\count@<#2\relax
    \advance\count@ 1 %
  \repeat
}
\def\space{ }
\expandafter\ifx\csname LoadCommand\endcsname\relax
  \def\LoadCommand{\input ifluatex.sty\relax}%
\fi
\def\Test{%
  \RangeCatcodeInvalid{0}{47}%
  \RangeCatcodeInvalid{58}{64}%
  \RangeCatcodeInvalid{91}{96}%
  \RangeCatcodeInvalid{123}{255}%
  \catcode`\@=12 %
  \catcode`\\=0 %
  \catcode`\%=14 %
  \LoadCommand
  \RangeCatcodeCheck{0}{36}{15}%
  \RangeCatcodeCheck{37}{37}{14}%
  \RangeCatcodeCheck{38}{47}{15}%
  \RangeCatcodeCheck{48}{57}{12}%
  \RangeCatcodeCheck{58}{63}{15}%
  \RangeCatcodeCheck{64}{64}{12}%
  \RangeCatcodeCheck{65}{90}{11}%
  \RangeCatcodeCheck{91}{91}{15}%
  \RangeCatcodeCheck{92}{92}{0}%
  \RangeCatcodeCheck{93}{96}{15}%
  \RangeCatcodeCheck{97}{122}{11}%
  \RangeCatcodeCheck{123}{255}{15}%
  \RestoreCatcodes
}
\Test
\csname @@end\endcsname
\end
%    \end{macrocode}
%    \begin{macrocode}
%</test1>
%    \end{macrocode}
%
% \section{Reload check for plain}
%
%    \begin{macrocode}
%<*test-reload1>
\input ifluatex.sty\relax
\input ifluatex.sty\relax
\csname @@end\endcsname\end
%</test-reload1>
%    \end{macrocode}
%
%    \begin{macrocode}
%<*test-reload2>
\input miniltx.tex\relax
\input ifluatex.sty\relax
\input ifluatex.sty\relax
\csname @@end\endcsname\end
%</test-reload2>
%    \end{macrocode}
%
% \section{Installation}
%
% \subsection{Download}
%
% \paragraph{Package.} This package is available on
% CTAN\footnote{\url{http://ctan.org/pkg/ifluatex}}:
% \begin{description}
% \item[\CTAN{macros/latex/contrib/oberdiek/ifluatex.dtx}] The source file.
% \item[\CTAN{macros/latex/contrib/oberdiek/ifluatex.pdf}] Documentation.
% \end{description}
%
%
% \paragraph{Bundle.} All the packages of the bundle `oberdiek'
% are also available in a TDS compliant ZIP archive. There
% the packages are already unpacked and the documentation files
% are generated. The files and directories obey the TDS standard.
% \begin{description}
% \item[\CTAN{install/macros/latex/contrib/oberdiek.tds.zip}]
% \end{description}
% \emph{TDS} refers to the standard ``A Directory Structure
% for \TeX\ Files'' (\CTAN{tds/tds.pdf}). Directories
% with \xfile{texmf} in their name are usually organized this way.
%
% \subsection{Bundle installation}
%
% \paragraph{Unpacking.} Unpack the \xfile{oberdiek.tds.zip} in the
% TDS tree (also known as \xfile{texmf} tree) of your choice.
% Example (linux):
% \begin{quote}
%   |unzip oberdiek.tds.zip -d ~/texmf|
% \end{quote}
%
% \paragraph{Script installation.}
% Check the directory \xfile{TDS:scripts/oberdiek/} for
% scripts that need further installation steps.
% Package \xpackage{attachfile2} comes with the Perl script
% \xfile{pdfatfi.pl} that should be installed in such a way
% that it can be called as \texttt{pdfatfi}.
% Example (linux):
% \begin{quote}
%   |chmod +x scripts/oberdiek/pdfatfi.pl|\\
%   |cp scripts/oberdiek/pdfatfi.pl /usr/local/bin/|
% \end{quote}
%
% \subsection{Package installation}
%
% \paragraph{Unpacking.} The \xfile{.dtx} file is a self-extracting
% \docstrip\ archive. The files are extracted by running the
% \xfile{.dtx} through \plainTeX:
% \begin{quote}
%   \verb|tex ifluatex.dtx|
% \end{quote}
%
% \paragraph{TDS.} Now the different files must be moved into
% the different directories in your installation TDS tree
% (also known as \xfile{texmf} tree):
% \begin{quote}
% \def\t{^^A
% \begin{tabular}{@{}>{\ttfamily}l@{ $\rightarrow$ }>{\ttfamily}l@{}}
%   ifluatex.sty & tex/generic/oberdiek/ifluatex.sty\\
%   ifluatex.pdf & doc/latex/oberdiek/ifluatex.pdf\\
%   test/ifluatex-test1.tex & doc/latex/oberdiek/test/ifluatex-test1.tex\\
%   test/ifluatex-test2.tex & doc/latex/oberdiek/test/ifluatex-test2.tex\\
%   test/ifluatex-test3.tex & doc/latex/oberdiek/test/ifluatex-test3.tex\\
%   ifluatex.dtx & source/latex/oberdiek/ifluatex.dtx\\
% \end{tabular}^^A
% }^^A
% \sbox0{\t}^^A
% \ifdim\wd0>\linewidth
%   \begingroup
%     \advance\linewidth by\leftmargin
%     \advance\linewidth by\rightmargin
%   \edef\x{\endgroup
%     \def\noexpand\lw{\the\linewidth}^^A
%   }\x
%   \def\lwbox{^^A
%     \leavevmode
%     \hbox to \linewidth{^^A
%       \kern-\leftmargin\relax
%       \hss
%       \usebox0
%       \hss
%       \kern-\rightmargin\relax
%     }^^A
%   }^^A
%   \ifdim\wd0>\lw
%     \sbox0{\small\t}^^A
%     \ifdim\wd0>\linewidth
%       \ifdim\wd0>\lw
%         \sbox0{\footnotesize\t}^^A
%         \ifdim\wd0>\linewidth
%           \ifdim\wd0>\lw
%             \sbox0{\scriptsize\t}^^A
%             \ifdim\wd0>\linewidth
%               \ifdim\wd0>\lw
%                 \sbox0{\tiny\t}^^A
%                 \ifdim\wd0>\linewidth
%                   \lwbox
%                 \else
%                   \usebox0
%                 \fi
%               \else
%                 \lwbox
%               \fi
%             \else
%               \usebox0
%             \fi
%           \else
%             \lwbox
%           \fi
%         \else
%           \usebox0
%         \fi
%       \else
%         \lwbox
%       \fi
%     \else
%       \usebox0
%     \fi
%   \else
%     \lwbox
%   \fi
% \else
%   \usebox0
% \fi
% \end{quote}
% If you have a \xfile{docstrip.cfg} that configures and enables \docstrip's
% TDS installing feature, then some files can already be in the right
% place, see the documentation of \docstrip.
%
% \subsection{Refresh file name databases}
%
% If your \TeX~distribution
% (\teTeX, \mikTeX, \dots) relies on file name databases, you must refresh
% these. For example, \teTeX\ users run \verb|texhash| or
% \verb|mktexlsr|.
%
% \subsection{Some details for the interested}
%
% \paragraph{Attached source.}
%
% The PDF documentation on CTAN also includes the
% \xfile{.dtx} source file. It can be extracted by
% AcrobatReader 6 or higher. Another option is \textsf{pdftk},
% e.g. unpack the file into the current directory:
% \begin{quote}
%   \verb|pdftk ifluatex.pdf unpack_files output .|
% \end{quote}
%
% \paragraph{Unpacking with \LaTeX.}
% The \xfile{.dtx} chooses its action depending on the format:
% \begin{description}
% \item[\plainTeX:] Run \docstrip\ and extract the files.
% \item[\LaTeX:] Generate the documentation.
% \end{description}
% If you insist on using \LaTeX\ for \docstrip\ (really,
% \docstrip\ does not need \LaTeX), then inform the autodetect routine
% about your intention:
% \begin{quote}
%   \verb|latex \let\install=y% \iffalse meta-comment
%
% File: ifluatex.dtx
% Version: 2016/05/16 v1.4
% Info: Provides the ifluatex switch
%
% Copyright (C) 2007, 2009, 2010 by
%    Heiko Oberdiek <heiko.oberdiek at googlemail.com>
%    2016
%    https://github.com/ho-tex/oberdiek/issues
%
% This work may be distributed and/or modified under the
% conditions of the LaTeX Project Public License, either
% version 1.3c of this license or (at your option) any later
% version. This version of this license is in
%    http://www.latex-project.org/lppl/lppl-1-3c.txt
% and the latest version of this license is in
%    http://www.latex-project.org/lppl.txt
% and version 1.3 or later is part of all distributions of
% LaTeX version 2005/12/01 or later.
%
% This work has the LPPL maintenance status "maintained".
%
% This Current Maintainer of this work is Heiko Oberdiek.
%
% The Base Interpreter refers to any `TeX-Format',
% because some files are installed in TDS:tex/generic//.
%
% This work consists of the main source file ifluatex.dtx
% and the derived files
%    ifluatex.sty, ifluatex.pdf, ifluatex.ins, ifluatex.drv,
%    ifluatex-test1.tex, ifluatex-test2.tex, ifluatex-test3.tex.
%
% Distribution:
%    CTAN:macros/latex/contrib/oberdiek/ifluatex.dtx
%    CTAN:macros/latex/contrib/oberdiek/ifluatex.pdf
%
% Unpacking:
%    (a) If ifluatex.ins is present:
%           tex ifluatex.ins
%    (b) Without ifluatex.ins:
%           tex ifluatex.dtx
%    (c) If you insist on using LaTeX
%           latex \let\install=y\input{ifluatex.dtx}
%        (quote the arguments according to the demands of your shell)
%
% Documentation:
%    (a) If ifluatex.drv is present:
%           latex ifluatex.drv
%    (b) Without ifluatex.drv:
%           latex ifluatex.dtx; ...
%    The class ltxdoc loads the configuration file ltxdoc.cfg
%    if available. Here you can specify further options, e.g.
%    use A4 as paper format:
%       \PassOptionsToClass{a4paper}{article}
%
%    Programm calls to get the documentation (example):
%       pdflatex ifluatex.dtx
%       makeindex -s gind.ist ifluatex.idx
%       pdflatex ifluatex.dtx
%       makeindex -s gind.ist ifluatex.idx
%       pdflatex ifluatex.dtx
%
% Installation:
%    TDS:tex/generic/oberdiek/ifluatex.sty
%    TDS:doc/latex/oberdiek/ifluatex.pdf
%    TDS:doc/latex/oberdiek/test/ifluatex-test1.tex
%    TDS:doc/latex/oberdiek/test/ifluatex-test2.tex
%    TDS:doc/latex/oberdiek/test/ifluatex-test3.tex
%    TDS:source/latex/oberdiek/ifluatex.dtx
%
%<*ignore>
\begingroup
  \catcode123=1 %
  \catcode125=2 %
  \def\x{LaTeX2e}%
\expandafter\endgroup
\ifcase 0\ifx\install y1\fi\expandafter
         \ifx\csname processbatchFile\endcsname\relax\else1\fi
         \ifx\fmtname\x\else 1\fi\relax
\else\csname fi\endcsname
%</ignore>
%<*install>
\input docstrip.tex
\Msg{************************************************************************}
\Msg{* Installation}
\Msg{* Package: ifluatex 2016/05/16 v1.4 Provides the ifluatex switch (HO)}
\Msg{************************************************************************}

\keepsilent
\askforoverwritefalse

\let\MetaPrefix\relax
\preamble

This is a generated file.

Project: ifluatex
Version: 2016/05/16 v1.4

Copyright (C) 2007, 2009, 2010 by
   Heiko Oberdiek <heiko.oberdiek at googlemail.com>

This work may be distributed and/or modified under the
conditions of the LaTeX Project Public License, either
version 1.3c of this license or (at your option) any later
version. This version of this license is in
   http://www.latex-project.org/lppl/lppl-1-3c.txt
and the latest version of this license is in
   http://www.latex-project.org/lppl.txt
and version 1.3 or later is part of all distributions of
LaTeX version 2005/12/01 or later.

This work has the LPPL maintenance status "maintained".

This Current Maintainer of this work is Heiko Oberdiek.

The Base Interpreter refers to any `TeX-Format',
because some files are installed in TDS:tex/generic//.

This work consists of the main source file ifluatex.dtx
and the derived files
   ifluatex.sty, ifluatex.pdf, ifluatex.ins, ifluatex.drv,
   ifluatex-test1.tex, ifluatex-test2.tex, ifluatex-test3.tex.

\endpreamble
\let\MetaPrefix\DoubleperCent

\generate{%
  \file{ifluatex.ins}{\from{ifluatex.dtx}{install}}%
  \file{ifluatex.drv}{\from{ifluatex.dtx}{driver}}%
  \usedir{tex/generic/oberdiek}%
  \file{ifluatex.sty}{\from{ifluatex.dtx}{package}}%
  \usedir{doc/latex/oberdiek/test}%
  \file{ifluatex-test1.tex}{\from{ifluatex.dtx}{test1}}%
  \file{ifluatex-test2.tex}{\from{ifluatex.dtx}{test-reload1}}%
  \file{ifluatex-test3.tex}{\from{ifluatex.dtx}{test-reload2}}%
  \nopreamble
  \nopostamble
  \usedir{source/latex/oberdiek/catalogue}%
  \file{ifluatex.xml}{\from{ifluatex.dtx}{catalogue}}%
}

\catcode32=13\relax% active space
\let =\space%
\Msg{************************************************************************}
\Msg{*}
\Msg{* To finish the installation you have to move the following}
\Msg{* file into a directory searched by TeX:}
\Msg{*}
\Msg{*     ifluatex.sty}
\Msg{*}
\Msg{* To produce the documentation run the file `ifluatex.drv'}
\Msg{* through LaTeX.}
\Msg{*}
\Msg{* Happy TeXing!}
\Msg{*}
\Msg{************************************************************************}

\endbatchfile
%</install>
%<*ignore>
\fi
%</ignore>
%<*driver>
\NeedsTeXFormat{LaTeX2e}
\ProvidesFile{ifluatex.drv}%
  [2016/05/16 v1.4 Provides the ifluatex switch (HO)]%
\documentclass{ltxdoc}
\usepackage{holtxdoc}[2011/11/22]
\begin{document}
  \DocInput{ifluatex.dtx}%
\end{document}
%</driver>
% \fi
%
%
% \CharacterTable
%  {Upper-case    \A\B\C\D\E\F\G\H\I\J\K\L\M\N\O\P\Q\R\S\T\U\V\W\X\Y\Z
%   Lower-case    \a\b\c\d\e\f\g\h\i\j\k\l\m\n\o\p\q\r\s\t\u\v\w\x\y\z
%   Digits        \0\1\2\3\4\5\6\7\8\9
%   Exclamation   \!     Double quote  \"     Hash (number) \#
%   Dollar        \$     Percent       \%     Ampersand     \&
%   Acute accent  \'     Left paren    \(     Right paren   \)
%   Asterisk      \*     Plus          \+     Comma         \,
%   Minus         \-     Point         \.     Solidus       \/
%   Colon         \:     Semicolon     \;     Less than     \<
%   Equals        \=     Greater than  \>     Question mark \?
%   Commercial at \@     Left bracket  \[     Backslash     \\
%   Right bracket \]     Circumflex    \^     Underscore    \_
%   Grave accent  \`     Left brace    \{     Vertical bar  \|
%   Right brace   \}     Tilde         \~}
%
% \GetFileInfo{ifluatex.drv}
%
% \title{The \xpackage{ifluatex} package}
% \date{2016/05/16 v1.4}
% \author{Heiko Oberdiek\thanks
% {Please report any issues at https://github.com/ho-tex/oberdiek/issues}\\
% \xemail{heiko.oberdiek at googlemail.com}}
%
% \maketitle
%
% \begin{abstract}
% This package looks for \LuaTeX\ regardless of its mode
% and provides the switch \cs{ifluatex}. Also it makes
% \cs{luatexversion} available if it is not present.
% It works with \plainTeX\ or \LaTeX.
% \end{abstract}
%
% \tableofcontents
%
% \section{Documentation}
%
% The package \xpackage{ifluatex} can be used with both \plainTeX\
% and \LaTeX:
% \begin{description}
% \item[\plainTeX:] |\input ifluatex.sty|
% \item[\LaTeXe:]   |\usepackage{ifluatex}|
% \end{description}
%
% \DescribeMacro{\ifluatex}
% The package provides the switch \cs{ifluatex}:
% \begin{quote}
%   |\ifluatex|\\
%   \hspace{1.5em}\LuaTeX\ is running\\
%   |\else|\\
%   \hspace{1.5em}Without \LuaTeX\\
%   |\fi|
% \end{quote}
%
% Since version 0.39 \LuaTeX\ only provides \cs{directlua} at startup
% time. Also the syntax of \cs{directlua} changed in version 0.36.
% Thus the user might want to check the LuaTeX version.
% Therefore this package also makes \cs{luatexversion} and
% \cs{luatexrevision} available, if it is not yet done.
%
% If you want to detect the mode (DVI or PDF), then use package
% \xpackage{ifpdf}. \LuaTeX\ has inherited \cs{pdfoutput} from \pdfTeX.
%
% \StopEventually{
% }
%
% \section{Implementation}
%
%    \begin{macrocode}
%<*package>
%    \end{macrocode}
%
% \subsection{Reload check and package identification}
%    Reload check, especially if the package is not used with \LaTeX.
%    \begin{macrocode}
\begingroup\catcode61\catcode48\catcode32=10\relax%
  \catcode13=5 % ^^M
  \endlinechar=13 %
  \catcode35=6 % #
  \catcode39=12 % '
  \catcode44=12 % ,
  \catcode45=12 % -
  \catcode46=12 % .
  \catcode58=12 % :
  \catcode64=11 % @
  \catcode123=1 % {
  \catcode125=2 % }
  \expandafter\let\expandafter\x\csname ver@ifluatex.sty\endcsname
  \ifx\x\relax % plain-TeX, first loading
  \else
    \def\empty{}%
    \ifx\x\empty % LaTeX, first loading,
      % variable is initialized, but \ProvidesPackage not yet seen
    \else
      \expandafter\ifx\csname PackageInfo\endcsname\relax
        \def\x#1#2{%
          \immediate\write-1{Package #1 Info: #2.}%
        }%
      \else
        \def\x#1#2{\PackageInfo{#1}{#2, stopped}}%
      \fi
      \x{ifluatex}{The package is already loaded}%
      \aftergroup\endinput
    \fi
  \fi
\endgroup%
%    \end{macrocode}
%    Package identification:
%    \begin{macrocode}
\begingroup\catcode61\catcode48\catcode32=10\relax%
  \catcode13=5 % ^^M
  \endlinechar=13 %
  \catcode35=6 % #
  \catcode39=12 % '
  \catcode40=12 % (
  \catcode41=12 % )
  \catcode44=12 % ,
  \catcode45=12 % -
  \catcode46=12 % .
  \catcode47=12 % /
  \catcode58=12 % :
  \catcode64=11 % @
  \catcode91=12 % [
  \catcode93=12 % ]
  \catcode123=1 % {
  \catcode125=2 % }
  \expandafter\ifx\csname ProvidesPackage\endcsname\relax
    \def\x#1#2#3[#4]{\endgroup
      \immediate\write-1{Package: #3 #4}%
      \xdef#1{#4}%
    }%
  \else
    \def\x#1#2[#3]{\endgroup
      #2[{#3}]%
      \ifx#1\@undefined
        \xdef#1{#3}%
      \fi
      \ifx#1\relax
        \xdef#1{#3}%
      \fi
    }%
  \fi
\expandafter\x\csname ver@ifluatex.sty\endcsname
\ProvidesPackage{ifluatex}%
  [2016/05/16 v1.4 Provides the ifluatex switch (HO)]%
%    \end{macrocode}
%
% \subsection{Catcodes}
%
%    \begin{macrocode}
\begingroup\catcode61\catcode48\catcode32=10\relax%
  \catcode13=5 % ^^M
  \endlinechar=13 %
  \catcode123=1 % {
  \catcode125=2 % }
  \catcode64=11 % @
  \def\x{\endgroup
    \expandafter\edef\csname ifluatex@AtEnd\endcsname{%
      \endlinechar=\the\endlinechar\relax
      \catcode13=\the\catcode13\relax
      \catcode32=\the\catcode32\relax
      \catcode35=\the\catcode35\relax
      \catcode61=\the\catcode61\relax
      \catcode64=\the\catcode64\relax
      \catcode123=\the\catcode123\relax
      \catcode125=\the\catcode125\relax
    }%
  }%
\x\catcode61\catcode48\catcode32=10\relax%
\catcode13=5 % ^^M
\endlinechar=13 %
\catcode35=6 % #
\catcode64=11 % @
\catcode123=1 % {
\catcode125=2 % }
\def\TMP@EnsureCode#1#2{%
  \edef\ifluatex@AtEnd{%
    \ifluatex@AtEnd
    \catcode#1=\the\catcode#1\relax
  }%
  \catcode#1=#2\relax
}
\TMP@EnsureCode{10}{12}% ^^J
\TMP@EnsureCode{39}{12}% '
\TMP@EnsureCode{40}{12}% (
\TMP@EnsureCode{41}{12}% )
\TMP@EnsureCode{44}{12}% ,
\TMP@EnsureCode{45}{12}% -
\TMP@EnsureCode{46}{12}% .
\TMP@EnsureCode{47}{12}% /
\TMP@EnsureCode{58}{12}% :
\TMP@EnsureCode{60}{12}% <
\TMP@EnsureCode{94}{7}% ^
\TMP@EnsureCode{96}{12}% `
\edef\ifluatex@AtEnd{\ifluatex@AtEnd\noexpand\endinput}
%    \end{macrocode}
%
% \subsection{Macro for error messages}
%
%    \begin{macro}{\ifluatex@Error}
%    \begin{macrocode}
\begingroup\expandafter\expandafter\expandafter\endgroup
\expandafter\ifx\csname PackageError\endcsname\relax
  \def\ifluatex@Error#1#2{%
    \begingroup
      \newlinechar=10 %
      \def\MessageBreak{^^J}%
      \edef\x{\errhelp{#2}}%
      \x
      \errmessage{Package ifluatex Error: #1}%
    \endgroup
  }%
\else
  \def\ifluatex@Error{%
    \PackageError{ifluatex}%
  }%
\fi
%    \end{macrocode}
%    \end{macro}
%
% \subsection{Check for previously defined \cs{ifluatex}}
%
%    \begin{macrocode}
\begingroup
  \expandafter\ifx\csname ifluatex\endcsname\relax
  \else
    \edef\i/{\expandafter\string\csname ifluatex\endcsname}%
    \ifluatex@Error{Name clash, \i/ is already defined}{%
      Incompatible versions of \i/ can cause problems,\MessageBreak
      therefore package loading is aborted.%
    }%
    \endgroup
    \expandafter\ifluatex@AtEnd
  \fi%
\endgroup
%    \end{macrocode}
%
% \subsection{\cs{ifluatex}}
%
%    \begin{macro}{\ifluatex}
%    \begin{macrocode}
\let\ifluatex\iffalse
%    \end{macrocode}
%    \end{macro}
%
%    Test \cs{luatexversion}. Is it  defined and different from
%    \cs{relax}? Someone could have used \LaTeX\ internal
%    \cs{@ifundefined}, or something else involving.
%    Notice, \cs{csname} is executed inside a group for the test
%    to cancel the side effect of \cs{csname}.
%    \begin{macrocode}
\begingroup\expandafter\expandafter\expandafter\endgroup
\expandafter\ifx\csname luatexversion\endcsname\relax
\else
  \expandafter\let\csname ifluatex\expandafter\endcsname
                  \csname iftrue\endcsname
\fi
%    \end{macrocode}
%
% \subsection{Lua\TeX\ v0.39}
%
%     Starting with version 0.39 \LuaTeX\ wants to provide \cs{directlua}
%     as only primitive at startup time beyond vanilla \TeX's primitives.
%     Then \cs{directlua} exists, but \cs{luatexversion} cannot be found.
%     Unhappily also the syntax of \cs{directlua} changed in v0.36,
%     thus the user would want to check \cs{luatexversion}.
%     Therefore we make \cs{luatexversion} available using
%     \LuaTeX's Lua function |tex.enableprimitives|.
%
%    \begin{macrocode}
\ifluatex
\else
  \begingroup\expandafter\expandafter\expandafter\endgroup
  \expandafter\ifx\csname directlua\endcsname\relax
  \else
    \expandafter\let\csname ifluatex\expandafter\endcsname
                    \csname iftrue\endcsname
    \begingroup
      \newlinechar=10 %
      \endlinechar=\newlinechar%
      \ifnum0%
          \directlua{%
            if tex.enableprimitives then
              tex.enableprimitives('ifluatex', {'luatexversion'})
              tex.print('1')
            end
          }%
          \ifx\ifluatexluatexversion\@undefined\else 1\fi %
          =11 %
        \global\let\luatexversion\ifluatexluatexversion%
      \else%
        \ifluatex@Error{%
          Missing \string\luatexversion%
        }{%
          Update LuaTeX.%
        }%
      \fi%
    \endgroup%
  \fi
\fi
%    \end{macrocode}
%    \begin{macrocode}
\ifluatex
  \begingroup\expandafter\expandafter\expandafter\endgroup
  \expandafter\ifx\csname luatexrevision\endcsname\relax
    \ifnum\luatexversion<36 %
    \else
      \begingroup
        \ifx\luatexrevision\relax
          \let\luatexrevision\@undefined
        \fi
        \newlinechar=10 %
        \endlinechar=\newlinechar%
        \ifcase0%
            \directlua{%
              if tex.enableprimitives then
                tex.enableprimitives('ifluatex', {'luatexrevision'})
              else
                tex.print('1')
              end
            }%
            \ifx\ifluatexluatexrevision\@undefined 1\fi%
            \relax%
          \global\let\luatexrevision\ifluatexluatexrevision%
        \fi%
      \endgroup%
    \fi
    \begingroup\expandafter\expandafter\expandafter\endgroup
    \expandafter\ifx\csname luatexrevision\endcsname\relax
      \ifluatex@Error{%
        Missing \string\luatexrevision%
      }{%
        Update LuaTeX.%
      }%
    \fi
  \fi
\fi
%    \end{macrocode}
%
% \subsection{Protocol entry}
%
%     Log comment:
%    \begin{macrocode}
\begingroup
  \expandafter\ifx\csname PackageInfo\endcsname\relax
    \def\x#1#2{%
      \immediate\write-1{Package #1 Info: #2.}%
    }%
  \else
    \let\x\PackageInfo
    \expandafter\let\csname on@line\endcsname\empty
  \fi
  \x{ifluatex}{LuaTeX \ifluatex\else not \fi detected}%
\endgroup
%    \end{macrocode}
%    \begin{macrocode}
\ifluatex@AtEnd%
%    \end{macrocode}
%    \begin{macrocode}
%</package>
%    \end{macrocode}
%
% \section{Test}
%
% \subsection{Catcode checks for loading}
%
%    \begin{macrocode}
%<*test1>
%    \end{macrocode}
%    \begin{macrocode}
\catcode`\{=1 %
\catcode`\}=2 %
\catcode`\#=6 %
\catcode`\@=11 %
\expandafter\ifx\csname count@\endcsname\relax
  \countdef\count@=255 %
\fi
\expandafter\ifx\csname @gobble\endcsname\relax
  \long\def\@gobble#1{}%
\fi
\expandafter\ifx\csname @firstofone\endcsname\relax
  \long\def\@firstofone#1{#1}%
\fi
\expandafter\ifx\csname loop\endcsname\relax
  \expandafter\@firstofone
\else
  \expandafter\@gobble
\fi
{%
  \def\loop#1\repeat{%
    \def\body{#1}%
    \iterate
  }%
  \def\iterate{%
    \body
      \let\next\iterate
    \else
      \let\next\relax
    \fi
    \next
  }%
  \let\repeat=\fi
}%
\def\RestoreCatcodes{}
\count@=0 %
\loop
  \edef\RestoreCatcodes{%
    \RestoreCatcodes
    \catcode\the\count@=\the\catcode\count@\relax
  }%
\ifnum\count@<255 %
  \advance\count@ 1 %
\repeat

\def\RangeCatcodeInvalid#1#2{%
  \count@=#1\relax
  \loop
    \catcode\count@=15 %
  \ifnum\count@<#2\relax
    \advance\count@ 1 %
  \repeat
}
\def\RangeCatcodeCheck#1#2#3{%
  \count@=#1\relax
  \loop
    \ifnum#3=\catcode\count@
    \else
      \errmessage{%
        Character \the\count@\space
        with wrong catcode \the\catcode\count@\space
        instead of \number#3%
      }%
    \fi
  \ifnum\count@<#2\relax
    \advance\count@ 1 %
  \repeat
}
\def\space{ }
\expandafter\ifx\csname LoadCommand\endcsname\relax
  \def\LoadCommand{\input ifluatex.sty\relax}%
\fi
\def\Test{%
  \RangeCatcodeInvalid{0}{47}%
  \RangeCatcodeInvalid{58}{64}%
  \RangeCatcodeInvalid{91}{96}%
  \RangeCatcodeInvalid{123}{255}%
  \catcode`\@=12 %
  \catcode`\\=0 %
  \catcode`\%=14 %
  \LoadCommand
  \RangeCatcodeCheck{0}{36}{15}%
  \RangeCatcodeCheck{37}{37}{14}%
  \RangeCatcodeCheck{38}{47}{15}%
  \RangeCatcodeCheck{48}{57}{12}%
  \RangeCatcodeCheck{58}{63}{15}%
  \RangeCatcodeCheck{64}{64}{12}%
  \RangeCatcodeCheck{65}{90}{11}%
  \RangeCatcodeCheck{91}{91}{15}%
  \RangeCatcodeCheck{92}{92}{0}%
  \RangeCatcodeCheck{93}{96}{15}%
  \RangeCatcodeCheck{97}{122}{11}%
  \RangeCatcodeCheck{123}{255}{15}%
  \RestoreCatcodes
}
\Test
\csname @@end\endcsname
\end
%    \end{macrocode}
%    \begin{macrocode}
%</test1>
%    \end{macrocode}
%
% \section{Reload check for plain}
%
%    \begin{macrocode}
%<*test-reload1>
\input ifluatex.sty\relax
\input ifluatex.sty\relax
\csname @@end\endcsname\end
%</test-reload1>
%    \end{macrocode}
%
%    \begin{macrocode}
%<*test-reload2>
\input miniltx.tex\relax
\input ifluatex.sty\relax
\input ifluatex.sty\relax
\csname @@end\endcsname\end
%</test-reload2>
%    \end{macrocode}
%
% \section{Installation}
%
% \subsection{Download}
%
% \paragraph{Package.} This package is available on
% CTAN\footnote{\url{http://ctan.org/pkg/ifluatex}}:
% \begin{description}
% \item[\CTAN{macros/latex/contrib/oberdiek/ifluatex.dtx}] The source file.
% \item[\CTAN{macros/latex/contrib/oberdiek/ifluatex.pdf}] Documentation.
% \end{description}
%
%
% \paragraph{Bundle.} All the packages of the bundle `oberdiek'
% are also available in a TDS compliant ZIP archive. There
% the packages are already unpacked and the documentation files
% are generated. The files and directories obey the TDS standard.
% \begin{description}
% \item[\CTAN{install/macros/latex/contrib/oberdiek.tds.zip}]
% \end{description}
% \emph{TDS} refers to the standard ``A Directory Structure
% for \TeX\ Files'' (\CTAN{tds/tds.pdf}). Directories
% with \xfile{texmf} in their name are usually organized this way.
%
% \subsection{Bundle installation}
%
% \paragraph{Unpacking.} Unpack the \xfile{oberdiek.tds.zip} in the
% TDS tree (also known as \xfile{texmf} tree) of your choice.
% Example (linux):
% \begin{quote}
%   |unzip oberdiek.tds.zip -d ~/texmf|
% \end{quote}
%
% \paragraph{Script installation.}
% Check the directory \xfile{TDS:scripts/oberdiek/} for
% scripts that need further installation steps.
% Package \xpackage{attachfile2} comes with the Perl script
% \xfile{pdfatfi.pl} that should be installed in such a way
% that it can be called as \texttt{pdfatfi}.
% Example (linux):
% \begin{quote}
%   |chmod +x scripts/oberdiek/pdfatfi.pl|\\
%   |cp scripts/oberdiek/pdfatfi.pl /usr/local/bin/|
% \end{quote}
%
% \subsection{Package installation}
%
% \paragraph{Unpacking.} The \xfile{.dtx} file is a self-extracting
% \docstrip\ archive. The files are extracted by running the
% \xfile{.dtx} through \plainTeX:
% \begin{quote}
%   \verb|tex ifluatex.dtx|
% \end{quote}
%
% \paragraph{TDS.} Now the different files must be moved into
% the different directories in your installation TDS tree
% (also known as \xfile{texmf} tree):
% \begin{quote}
% \def\t{^^A
% \begin{tabular}{@{}>{\ttfamily}l@{ $\rightarrow$ }>{\ttfamily}l@{}}
%   ifluatex.sty & tex/generic/oberdiek/ifluatex.sty\\
%   ifluatex.pdf & doc/latex/oberdiek/ifluatex.pdf\\
%   test/ifluatex-test1.tex & doc/latex/oberdiek/test/ifluatex-test1.tex\\
%   test/ifluatex-test2.tex & doc/latex/oberdiek/test/ifluatex-test2.tex\\
%   test/ifluatex-test3.tex & doc/latex/oberdiek/test/ifluatex-test3.tex\\
%   ifluatex.dtx & source/latex/oberdiek/ifluatex.dtx\\
% \end{tabular}^^A
% }^^A
% \sbox0{\t}^^A
% \ifdim\wd0>\linewidth
%   \begingroup
%     \advance\linewidth by\leftmargin
%     \advance\linewidth by\rightmargin
%   \edef\x{\endgroup
%     \def\noexpand\lw{\the\linewidth}^^A
%   }\x
%   \def\lwbox{^^A
%     \leavevmode
%     \hbox to \linewidth{^^A
%       \kern-\leftmargin\relax
%       \hss
%       \usebox0
%       \hss
%       \kern-\rightmargin\relax
%     }^^A
%   }^^A
%   \ifdim\wd0>\lw
%     \sbox0{\small\t}^^A
%     \ifdim\wd0>\linewidth
%       \ifdim\wd0>\lw
%         \sbox0{\footnotesize\t}^^A
%         \ifdim\wd0>\linewidth
%           \ifdim\wd0>\lw
%             \sbox0{\scriptsize\t}^^A
%             \ifdim\wd0>\linewidth
%               \ifdim\wd0>\lw
%                 \sbox0{\tiny\t}^^A
%                 \ifdim\wd0>\linewidth
%                   \lwbox
%                 \else
%                   \usebox0
%                 \fi
%               \else
%                 \lwbox
%               \fi
%             \else
%               \usebox0
%             \fi
%           \else
%             \lwbox
%           \fi
%         \else
%           \usebox0
%         \fi
%       \else
%         \lwbox
%       \fi
%     \else
%       \usebox0
%     \fi
%   \else
%     \lwbox
%   \fi
% \else
%   \usebox0
% \fi
% \end{quote}
% If you have a \xfile{docstrip.cfg} that configures and enables \docstrip's
% TDS installing feature, then some files can already be in the right
% place, see the documentation of \docstrip.
%
% \subsection{Refresh file name databases}
%
% If your \TeX~distribution
% (\teTeX, \mikTeX, \dots) relies on file name databases, you must refresh
% these. For example, \teTeX\ users run \verb|texhash| or
% \verb|mktexlsr|.
%
% \subsection{Some details for the interested}
%
% \paragraph{Attached source.}
%
% The PDF documentation on CTAN also includes the
% \xfile{.dtx} source file. It can be extracted by
% AcrobatReader 6 or higher. Another option is \textsf{pdftk},
% e.g. unpack the file into the current directory:
% \begin{quote}
%   \verb|pdftk ifluatex.pdf unpack_files output .|
% \end{quote}
%
% \paragraph{Unpacking with \LaTeX.}
% The \xfile{.dtx} chooses its action depending on the format:
% \begin{description}
% \item[\plainTeX:] Run \docstrip\ and extract the files.
% \item[\LaTeX:] Generate the documentation.
% \end{description}
% If you insist on using \LaTeX\ for \docstrip\ (really,
% \docstrip\ does not need \LaTeX), then inform the autodetect routine
% about your intention:
% \begin{quote}
%   \verb|latex \let\install=y\input{ifluatex.dtx}|
% \end{quote}
% Do not forget to quote the argument according to the demands
% of your shell.
%
% \paragraph{Generating the documentation.}
% You can use both the \xfile{.dtx} or the \xfile{.drv} to generate
% the documentation. The process can be configured by the
% configuration file \xfile{ltxdoc.cfg}. For instance, put this
% line into this file, if you want to have A4 as paper format:
% \begin{quote}
%   \verb|\PassOptionsToClass{a4paper}{article}|
% \end{quote}
% An example follows how to generate the
% documentation with pdf\LaTeX:
% \begin{quote}
%\begin{verbatim}
%pdflatex ifluatex.dtx
%makeindex -s gind.ist ifluatex.idx
%pdflatex ifluatex.dtx
%makeindex -s gind.ist ifluatex.idx
%pdflatex ifluatex.dtx
%\end{verbatim}
% \end{quote}
%
% \section{Catalogue}
%
% The following XML file can be used as source for the
% \href{http://mirror.ctan.org/help/Catalogue/catalogue.html}{\TeX\ Catalogue}.
% The elements \texttt{caption} and \texttt{description} are imported
% from the original XML file from the Catalogue.
% The name of the XML file in the Catalogue is \xfile{ifluatex.xml}.
%    \begin{macrocode}
%<*catalogue>
<?xml version='1.0' encoding='us-ascii'?>
<!DOCTYPE entry SYSTEM 'catalogue.dtd'>
<entry datestamp='$Date$' modifier='$Author$' id='ifluatex'>
  <name>ifluatex</name>
  <caption>Provides the \ifluatex switch.</caption>
  <authorref id='auth:oberdiek'/>
  <copyright owner='Heiko Oberdiek' year='2007,2009,2010'/>
  <license type='lppl1.3'/>
  <version number='1.4'/>
  <description>
    The package looks for  LuaTeX regardless of its mode and provides
    the switch <tt>\ifluatex</tt>; it works with Plain TeX or LaTeX.
    <p/>
    The package is part of the <xref refid='oberdiek'>oberdiek</xref>
    bundle.
  </description>
  <documentation details='Package documentation'
      href='ctan:/macros/latex/contrib/oberdiek/ifluatex.pdf'/>
  <ctan file='true' path='/macros/latex/contrib/oberdiek/ifluatex.dtx'/>
  <miktex location='oberdiek'/>
  <texlive location='ifluatex'/>
  <install path='/macros/latex/contrib/oberdiek/oberdiek.tds.zip'/>
</entry>
%</catalogue>
%    \end{macrocode}
%
% \begin{History}
%   \begin{Version}{2007/12/12 v1.0}
%   \item
%     First public version.
%   \end{Version}
%   \begin{Version}{2009/04/10 v1.1}
%   \item
%     Test adopted for \LuaTeX\ 0.39.
%   \item
%     Makes \cs{luatexversion} available.
%   \end{Version}
%   \begin{Version}{2009/04/17 v1.2}
%   \item
%     Fixes (Manuel P\'egouri\'e-Gonnard).
%   \item
%     \cs{luatextrue} and \cs{luatexfalse} are no longer defined.
%   \item
%     Makes \cs{luatexrevision} available, too.
%   \end{Version}
%   \begin{Version}{2010/03/01 v1.3}
%   \item
%     Line ends fixed in case \cs{endlinechar} = \cs{newlinechar}.
%   \end{Version}
%   \begin{Version}{2016/05/16 v1.4}
%   \item
%     Documentation updates.
%   \end{Version}
% \end{History}
%
% \PrintIndex
%
% \Finale
\endinput
|
% \end{quote}
% Do not forget to quote the argument according to the demands
% of your shell.
%
% \paragraph{Generating the documentation.}
% You can use both the \xfile{.dtx} or the \xfile{.drv} to generate
% the documentation. The process can be configured by the
% configuration file \xfile{ltxdoc.cfg}. For instance, put this
% line into this file, if you want to have A4 as paper format:
% \begin{quote}
%   \verb|\PassOptionsToClass{a4paper}{article}|
% \end{quote}
% An example follows how to generate the
% documentation with pdf\LaTeX:
% \begin{quote}
%\begin{verbatim}
%pdflatex ifluatex.dtx
%makeindex -s gind.ist ifluatex.idx
%pdflatex ifluatex.dtx
%makeindex -s gind.ist ifluatex.idx
%pdflatex ifluatex.dtx
%\end{verbatim}
% \end{quote}
%
% \section{Catalogue}
%
% The following XML file can be used as source for the
% \href{http://mirror.ctan.org/help/Catalogue/catalogue.html}{\TeX\ Catalogue}.
% The elements \texttt{caption} and \texttt{description} are imported
% from the original XML file from the Catalogue.
% The name of the XML file in the Catalogue is \xfile{ifluatex.xml}.
%    \begin{macrocode}
%<*catalogue>
<?xml version='1.0' encoding='us-ascii'?>
<!DOCTYPE entry SYSTEM 'catalogue.dtd'>
<entry datestamp='$Date$' modifier='$Author$' id='ifluatex'>
  <name>ifluatex</name>
  <caption>Provides the \ifluatex switch.</caption>
  <authorref id='auth:oberdiek'/>
  <copyright owner='Heiko Oberdiek' year='2007,2009,2010'/>
  <license type='lppl1.3'/>
  <version number='1.4'/>
  <description>
    The package looks for  LuaTeX regardless of its mode and provides
    the switch <tt>\ifluatex</tt>; it works with Plain TeX or LaTeX.
    <p/>
    The package is part of the <xref refid='oberdiek'>oberdiek</xref>
    bundle.
  </description>
  <documentation details='Package documentation'
      href='ctan:/macros/latex/contrib/oberdiek/ifluatex.pdf'/>
  <ctan file='true' path='/macros/latex/contrib/oberdiek/ifluatex.dtx'/>
  <miktex location='oberdiek'/>
  <texlive location='ifluatex'/>
  <install path='/macros/latex/contrib/oberdiek/oberdiek.tds.zip'/>
</entry>
%</catalogue>
%    \end{macrocode}
%
% \begin{History}
%   \begin{Version}{2007/12/12 v1.0}
%   \item
%     First public version.
%   \end{Version}
%   \begin{Version}{2009/04/10 v1.1}
%   \item
%     Test adopted for \LuaTeX\ 0.39.
%   \item
%     Makes \cs{luatexversion} available.
%   \end{Version}
%   \begin{Version}{2009/04/17 v1.2}
%   \item
%     Fixes (Manuel P\'egouri\'e-Gonnard).
%   \item
%     \cs{luatextrue} and \cs{luatexfalse} are no longer defined.
%   \item
%     Makes \cs{luatexrevision} available, too.
%   \end{Version}
%   \begin{Version}{2010/03/01 v1.3}
%   \item
%     Line ends fixed in case \cs{endlinechar} = \cs{newlinechar}.
%   \end{Version}
%   \begin{Version}{2016/05/16 v1.4}
%   \item
%     Documentation updates.
%   \end{Version}
% \end{History}
%
% \PrintIndex
%
% \Finale
\endinput

%        (quote the arguments according to the demands of your shell)
%
% Documentation:
%    (a) If ifluatex.drv is present:
%           latex ifluatex.drv
%    (b) Without ifluatex.drv:
%           latex ifluatex.dtx; ...
%    The class ltxdoc loads the configuration file ltxdoc.cfg
%    if available. Here you can specify further options, e.g.
%    use A4 as paper format:
%       \PassOptionsToClass{a4paper}{article}
%
%    Programm calls to get the documentation (example):
%       pdflatex ifluatex.dtx
%       makeindex -s gind.ist ifluatex.idx
%       pdflatex ifluatex.dtx
%       makeindex -s gind.ist ifluatex.idx
%       pdflatex ifluatex.dtx
%
% Installation:
%    TDS:tex/generic/oberdiek/ifluatex.sty
%    TDS:doc/latex/oberdiek/ifluatex.pdf
%    TDS:doc/latex/oberdiek/test/ifluatex-test1.tex
%    TDS:doc/latex/oberdiek/test/ifluatex-test2.tex
%    TDS:doc/latex/oberdiek/test/ifluatex-test3.tex
%    TDS:source/latex/oberdiek/ifluatex.dtx
%
%<*ignore>
\begingroup
  \catcode123=1 %
  \catcode125=2 %
  \def\x{LaTeX2e}%
\expandafter\endgroup
\ifcase 0\ifx\install y1\fi\expandafter
         \ifx\csname processbatchFile\endcsname\relax\else1\fi
         \ifx\fmtname\x\else 1\fi\relax
\else\csname fi\endcsname
%</ignore>
%<*install>
\input docstrip.tex
\Msg{************************************************************************}
\Msg{* Installation}
\Msg{* Package: ifluatex 2016/05/16 v1.4 Provides the ifluatex switch (HO)}
\Msg{************************************************************************}

\keepsilent
\askforoverwritefalse

\let\MetaPrefix\relax
\preamble

This is a generated file.

Project: ifluatex
Version: 2016/05/16 v1.4

Copyright (C) 2007, 2009, 2010 by
   Heiko Oberdiek <heiko.oberdiek at googlemail.com>

This work may be distributed and/or modified under the
conditions of the LaTeX Project Public License, either
version 1.3c of this license or (at your option) any later
version. This version of this license is in
   http://www.latex-project.org/lppl/lppl-1-3c.txt
and the latest version of this license is in
   http://www.latex-project.org/lppl.txt
and version 1.3 or later is part of all distributions of
LaTeX version 2005/12/01 or later.

This work has the LPPL maintenance status "maintained".

This Current Maintainer of this work is Heiko Oberdiek.

The Base Interpreter refers to any `TeX-Format',
because some files are installed in TDS:tex/generic//.

This work consists of the main source file ifluatex.dtx
and the derived files
   ifluatex.sty, ifluatex.pdf, ifluatex.ins, ifluatex.drv,
   ifluatex-test1.tex, ifluatex-test2.tex, ifluatex-test3.tex.

\endpreamble
\let\MetaPrefix\DoubleperCent

\generate{%
  \file{ifluatex.ins}{\from{ifluatex.dtx}{install}}%
  \file{ifluatex.drv}{\from{ifluatex.dtx}{driver}}%
  \usedir{tex/generic/oberdiek}%
  \file{ifluatex.sty}{\from{ifluatex.dtx}{package}}%
  \usedir{doc/latex/oberdiek/test}%
  \file{ifluatex-test1.tex}{\from{ifluatex.dtx}{test1}}%
  \file{ifluatex-test2.tex}{\from{ifluatex.dtx}{test-reload1}}%
  \file{ifluatex-test3.tex}{\from{ifluatex.dtx}{test-reload2}}%
  \nopreamble
  \nopostamble
  \usedir{source/latex/oberdiek/catalogue}%
  \file{ifluatex.xml}{\from{ifluatex.dtx}{catalogue}}%
}

\catcode32=13\relax% active space
\let =\space%
\Msg{************************************************************************}
\Msg{*}
\Msg{* To finish the installation you have to move the following}
\Msg{* file into a directory searched by TeX:}
\Msg{*}
\Msg{*     ifluatex.sty}
\Msg{*}
\Msg{* To produce the documentation run the file `ifluatex.drv'}
\Msg{* through LaTeX.}
\Msg{*}
\Msg{* Happy TeXing!}
\Msg{*}
\Msg{************************************************************************}

\endbatchfile
%</install>
%<*ignore>
\fi
%</ignore>
%<*driver>
\NeedsTeXFormat{LaTeX2e}
\ProvidesFile{ifluatex.drv}%
  [2016/05/16 v1.4 Provides the ifluatex switch (HO)]%
\documentclass{ltxdoc}
\usepackage{holtxdoc}[2011/11/22]
\begin{document}
  \DocInput{ifluatex.dtx}%
\end{document}
%</driver>
% \fi
%
%
% \CharacterTable
%  {Upper-case    \A\B\C\D\E\F\G\H\I\J\K\L\M\N\O\P\Q\R\S\T\U\V\W\X\Y\Z
%   Lower-case    \a\b\c\d\e\f\g\h\i\j\k\l\m\n\o\p\q\r\s\t\u\v\w\x\y\z
%   Digits        \0\1\2\3\4\5\6\7\8\9
%   Exclamation   \!     Double quote  \"     Hash (number) \#
%   Dollar        \$     Percent       \%     Ampersand     \&
%   Acute accent  \'     Left paren    \(     Right paren   \)
%   Asterisk      \*     Plus          \+     Comma         \,
%   Minus         \-     Point         \.     Solidus       \/
%   Colon         \:     Semicolon     \;     Less than     \<
%   Equals        \=     Greater than  \>     Question mark \?
%   Commercial at \@     Left bracket  \[     Backslash     \\
%   Right bracket \]     Circumflex    \^     Underscore    \_
%   Grave accent  \`     Left brace    \{     Vertical bar  \|
%   Right brace   \}     Tilde         \~}
%
% \GetFileInfo{ifluatex.drv}
%
% \title{The \xpackage{ifluatex} package}
% \date{2016/05/16 v1.4}
% \author{Heiko Oberdiek\thanks
% {Please report any issues at https://github.com/ho-tex/oberdiek/issues}\\
% \xemail{heiko.oberdiek at googlemail.com}}
%
% \maketitle
%
% \begin{abstract}
% This package looks for \LuaTeX\ regardless of its mode
% and provides the switch \cs{ifluatex}. Also it makes
% \cs{luatexversion} available if it is not present.
% It works with \plainTeX\ or \LaTeX.
% \end{abstract}
%
% \tableofcontents
%
% \section{Documentation}
%
% The package \xpackage{ifluatex} can be used with both \plainTeX\
% and \LaTeX:
% \begin{description}
% \item[\plainTeX:] |\input ifluatex.sty|
% \item[\LaTeXe:]   |\usepackage{ifluatex}|
% \end{description}
%
% \DescribeMacro{\ifluatex}
% The package provides the switch \cs{ifluatex}:
% \begin{quote}
%   |\ifluatex|\\
%   \hspace{1.5em}\LuaTeX\ is running\\
%   |\else|\\
%   \hspace{1.5em}Without \LuaTeX\\
%   |\fi|
% \end{quote}
%
% Since version 0.39 \LuaTeX\ only provides \cs{directlua} at startup
% time. Also the syntax of \cs{directlua} changed in version 0.36.
% Thus the user might want to check the LuaTeX version.
% Therefore this package also makes \cs{luatexversion} and
% \cs{luatexrevision} available, if it is not yet done.
%
% If you want to detect the mode (DVI or PDF), then use package
% \xpackage{ifpdf}. \LuaTeX\ has inherited \cs{pdfoutput} from \pdfTeX.
%
% \StopEventually{
% }
%
% \section{Implementation}
%
%    \begin{macrocode}
%<*package>
%    \end{macrocode}
%
% \subsection{Reload check and package identification}
%    Reload check, especially if the package is not used with \LaTeX.
%    \begin{macrocode}
\begingroup\catcode61\catcode48\catcode32=10\relax%
  \catcode13=5 % ^^M
  \endlinechar=13 %
  \catcode35=6 % #
  \catcode39=12 % '
  \catcode44=12 % ,
  \catcode45=12 % -
  \catcode46=12 % .
  \catcode58=12 % :
  \catcode64=11 % @
  \catcode123=1 % {
  \catcode125=2 % }
  \expandafter\let\expandafter\x\csname ver@ifluatex.sty\endcsname
  \ifx\x\relax % plain-TeX, first loading
  \else
    \def\empty{}%
    \ifx\x\empty % LaTeX, first loading,
      % variable is initialized, but \ProvidesPackage not yet seen
    \else
      \expandafter\ifx\csname PackageInfo\endcsname\relax
        \def\x#1#2{%
          \immediate\write-1{Package #1 Info: #2.}%
        }%
      \else
        \def\x#1#2{\PackageInfo{#1}{#2, stopped}}%
      \fi
      \x{ifluatex}{The package is already loaded}%
      \aftergroup\endinput
    \fi
  \fi
\endgroup%
%    \end{macrocode}
%    Package identification:
%    \begin{macrocode}
\begingroup\catcode61\catcode48\catcode32=10\relax%
  \catcode13=5 % ^^M
  \endlinechar=13 %
  \catcode35=6 % #
  \catcode39=12 % '
  \catcode40=12 % (
  \catcode41=12 % )
  \catcode44=12 % ,
  \catcode45=12 % -
  \catcode46=12 % .
  \catcode47=12 % /
  \catcode58=12 % :
  \catcode64=11 % @
  \catcode91=12 % [
  \catcode93=12 % ]
  \catcode123=1 % {
  \catcode125=2 % }
  \expandafter\ifx\csname ProvidesPackage\endcsname\relax
    \def\x#1#2#3[#4]{\endgroup
      \immediate\write-1{Package: #3 #4}%
      \xdef#1{#4}%
    }%
  \else
    \def\x#1#2[#3]{\endgroup
      #2[{#3}]%
      \ifx#1\@undefined
        \xdef#1{#3}%
      \fi
      \ifx#1\relax
        \xdef#1{#3}%
      \fi
    }%
  \fi
\expandafter\x\csname ver@ifluatex.sty\endcsname
\ProvidesPackage{ifluatex}%
  [2016/05/16 v1.4 Provides the ifluatex switch (HO)]%
%    \end{macrocode}
%
% \subsection{Catcodes}
%
%    \begin{macrocode}
\begingroup\catcode61\catcode48\catcode32=10\relax%
  \catcode13=5 % ^^M
  \endlinechar=13 %
  \catcode123=1 % {
  \catcode125=2 % }
  \catcode64=11 % @
  \def\x{\endgroup
    \expandafter\edef\csname ifluatex@AtEnd\endcsname{%
      \endlinechar=\the\endlinechar\relax
      \catcode13=\the\catcode13\relax
      \catcode32=\the\catcode32\relax
      \catcode35=\the\catcode35\relax
      \catcode61=\the\catcode61\relax
      \catcode64=\the\catcode64\relax
      \catcode123=\the\catcode123\relax
      \catcode125=\the\catcode125\relax
    }%
  }%
\x\catcode61\catcode48\catcode32=10\relax%
\catcode13=5 % ^^M
\endlinechar=13 %
\catcode35=6 % #
\catcode64=11 % @
\catcode123=1 % {
\catcode125=2 % }
\def\TMP@EnsureCode#1#2{%
  \edef\ifluatex@AtEnd{%
    \ifluatex@AtEnd
    \catcode#1=\the\catcode#1\relax
  }%
  \catcode#1=#2\relax
}
\TMP@EnsureCode{10}{12}% ^^J
\TMP@EnsureCode{39}{12}% '
\TMP@EnsureCode{40}{12}% (
\TMP@EnsureCode{41}{12}% )
\TMP@EnsureCode{44}{12}% ,
\TMP@EnsureCode{45}{12}% -
\TMP@EnsureCode{46}{12}% .
\TMP@EnsureCode{47}{12}% /
\TMP@EnsureCode{58}{12}% :
\TMP@EnsureCode{60}{12}% <
\TMP@EnsureCode{94}{7}% ^
\TMP@EnsureCode{96}{12}% `
\edef\ifluatex@AtEnd{\ifluatex@AtEnd\noexpand\endinput}
%    \end{macrocode}
%
% \subsection{Macro for error messages}
%
%    \begin{macro}{\ifluatex@Error}
%    \begin{macrocode}
\begingroup\expandafter\expandafter\expandafter\endgroup
\expandafter\ifx\csname PackageError\endcsname\relax
  \def\ifluatex@Error#1#2{%
    \begingroup
      \newlinechar=10 %
      \def\MessageBreak{^^J}%
      \edef\x{\errhelp{#2}}%
      \x
      \errmessage{Package ifluatex Error: #1}%
    \endgroup
  }%
\else
  \def\ifluatex@Error{%
    \PackageError{ifluatex}%
  }%
\fi
%    \end{macrocode}
%    \end{macro}
%
% \subsection{Check for previously defined \cs{ifluatex}}
%
%    \begin{macrocode}
\begingroup
  \expandafter\ifx\csname ifluatex\endcsname\relax
  \else
    \edef\i/{\expandafter\string\csname ifluatex\endcsname}%
    \ifluatex@Error{Name clash, \i/ is already defined}{%
      Incompatible versions of \i/ can cause problems,\MessageBreak
      therefore package loading is aborted.%
    }%
    \endgroup
    \expandafter\ifluatex@AtEnd
  \fi%
\endgroup
%    \end{macrocode}
%
% \subsection{\cs{ifluatex}}
%
%    \begin{macro}{\ifluatex}
%    \begin{macrocode}
\let\ifluatex\iffalse
%    \end{macrocode}
%    \end{macro}
%
%    Test \cs{luatexversion}. Is it  defined and different from
%    \cs{relax}? Someone could have used \LaTeX\ internal
%    \cs{@ifundefined}, or something else involving.
%    Notice, \cs{csname} is executed inside a group for the test
%    to cancel the side effect of \cs{csname}.
%    \begin{macrocode}
\begingroup\expandafter\expandafter\expandafter\endgroup
\expandafter\ifx\csname luatexversion\endcsname\relax
\else
  \expandafter\let\csname ifluatex\expandafter\endcsname
                  \csname iftrue\endcsname
\fi
%    \end{macrocode}
%
% \subsection{Lua\TeX\ v0.39}
%
%     Starting with version 0.39 \LuaTeX\ wants to provide \cs{directlua}
%     as only primitive at startup time beyond vanilla \TeX's primitives.
%     Then \cs{directlua} exists, but \cs{luatexversion} cannot be found.
%     Unhappily also the syntax of \cs{directlua} changed in v0.36,
%     thus the user would want to check \cs{luatexversion}.
%     Therefore we make \cs{luatexversion} available using
%     \LuaTeX's Lua function |tex.enableprimitives|.
%
%    \begin{macrocode}
\ifluatex
\else
  \begingroup\expandafter\expandafter\expandafter\endgroup
  \expandafter\ifx\csname directlua\endcsname\relax
  \else
    \expandafter\let\csname ifluatex\expandafter\endcsname
                    \csname iftrue\endcsname
    \begingroup
      \newlinechar=10 %
      \endlinechar=\newlinechar%
      \ifnum0%
          \directlua{%
            if tex.enableprimitives then
              tex.enableprimitives('ifluatex', {'luatexversion'})
              tex.print('1')
            end
          }%
          \ifx\ifluatexluatexversion\@undefined\else 1\fi %
          =11 %
        \global\let\luatexversion\ifluatexluatexversion%
      \else%
        \ifluatex@Error{%
          Missing \string\luatexversion%
        }{%
          Update LuaTeX.%
        }%
      \fi%
    \endgroup%
  \fi
\fi
%    \end{macrocode}
%    \begin{macrocode}
\ifluatex
  \begingroup\expandafter\expandafter\expandafter\endgroup
  \expandafter\ifx\csname luatexrevision\endcsname\relax
    \ifnum\luatexversion<36 %
    \else
      \begingroup
        \ifx\luatexrevision\relax
          \let\luatexrevision\@undefined
        \fi
        \newlinechar=10 %
        \endlinechar=\newlinechar%
        \ifcase0%
            \directlua{%
              if tex.enableprimitives then
                tex.enableprimitives('ifluatex', {'luatexrevision'})
              else
                tex.print('1')
              end
            }%
            \ifx\ifluatexluatexrevision\@undefined 1\fi%
            \relax%
          \global\let\luatexrevision\ifluatexluatexrevision%
        \fi%
      \endgroup%
    \fi
    \begingroup\expandafter\expandafter\expandafter\endgroup
    \expandafter\ifx\csname luatexrevision\endcsname\relax
      \ifluatex@Error{%
        Missing \string\luatexrevision%
      }{%
        Update LuaTeX.%
      }%
    \fi
  \fi
\fi
%    \end{macrocode}
%
% \subsection{Protocol entry}
%
%     Log comment:
%    \begin{macrocode}
\begingroup
  \expandafter\ifx\csname PackageInfo\endcsname\relax
    \def\x#1#2{%
      \immediate\write-1{Package #1 Info: #2.}%
    }%
  \else
    \let\x\PackageInfo
    \expandafter\let\csname on@line\endcsname\empty
  \fi
  \x{ifluatex}{LuaTeX \ifluatex\else not \fi detected}%
\endgroup
%    \end{macrocode}
%    \begin{macrocode}
\ifluatex@AtEnd%
%    \end{macrocode}
%    \begin{macrocode}
%</package>
%    \end{macrocode}
%
% \section{Test}
%
% \subsection{Catcode checks for loading}
%
%    \begin{macrocode}
%<*test1>
%    \end{macrocode}
%    \begin{macrocode}
\catcode`\{=1 %
\catcode`\}=2 %
\catcode`\#=6 %
\catcode`\@=11 %
\expandafter\ifx\csname count@\endcsname\relax
  \countdef\count@=255 %
\fi
\expandafter\ifx\csname @gobble\endcsname\relax
  \long\def\@gobble#1{}%
\fi
\expandafter\ifx\csname @firstofone\endcsname\relax
  \long\def\@firstofone#1{#1}%
\fi
\expandafter\ifx\csname loop\endcsname\relax
  \expandafter\@firstofone
\else
  \expandafter\@gobble
\fi
{%
  \def\loop#1\repeat{%
    \def\body{#1}%
    \iterate
  }%
  \def\iterate{%
    \body
      \let\next\iterate
    \else
      \let\next\relax
    \fi
    \next
  }%
  \let\repeat=\fi
}%
\def\RestoreCatcodes{}
\count@=0 %
\loop
  \edef\RestoreCatcodes{%
    \RestoreCatcodes
    \catcode\the\count@=\the\catcode\count@\relax
  }%
\ifnum\count@<255 %
  \advance\count@ 1 %
\repeat

\def\RangeCatcodeInvalid#1#2{%
  \count@=#1\relax
  \loop
    \catcode\count@=15 %
  \ifnum\count@<#2\relax
    \advance\count@ 1 %
  \repeat
}
\def\RangeCatcodeCheck#1#2#3{%
  \count@=#1\relax
  \loop
    \ifnum#3=\catcode\count@
    \else
      \errmessage{%
        Character \the\count@\space
        with wrong catcode \the\catcode\count@\space
        instead of \number#3%
      }%
    \fi
  \ifnum\count@<#2\relax
    \advance\count@ 1 %
  \repeat
}
\def\space{ }
\expandafter\ifx\csname LoadCommand\endcsname\relax
  \def\LoadCommand{\input ifluatex.sty\relax}%
\fi
\def\Test{%
  \RangeCatcodeInvalid{0}{47}%
  \RangeCatcodeInvalid{58}{64}%
  \RangeCatcodeInvalid{91}{96}%
  \RangeCatcodeInvalid{123}{255}%
  \catcode`\@=12 %
  \catcode`\\=0 %
  \catcode`\%=14 %
  \LoadCommand
  \RangeCatcodeCheck{0}{36}{15}%
  \RangeCatcodeCheck{37}{37}{14}%
  \RangeCatcodeCheck{38}{47}{15}%
  \RangeCatcodeCheck{48}{57}{12}%
  \RangeCatcodeCheck{58}{63}{15}%
  \RangeCatcodeCheck{64}{64}{12}%
  \RangeCatcodeCheck{65}{90}{11}%
  \RangeCatcodeCheck{91}{91}{15}%
  \RangeCatcodeCheck{92}{92}{0}%
  \RangeCatcodeCheck{93}{96}{15}%
  \RangeCatcodeCheck{97}{122}{11}%
  \RangeCatcodeCheck{123}{255}{15}%
  \RestoreCatcodes
}
\Test
\csname @@end\endcsname
\end
%    \end{macrocode}
%    \begin{macrocode}
%</test1>
%    \end{macrocode}
%
% \section{Reload check for plain}
%
%    \begin{macrocode}
%<*test-reload1>
\input ifluatex.sty\relax
\input ifluatex.sty\relax
\csname @@end\endcsname\end
%</test-reload1>
%    \end{macrocode}
%
%    \begin{macrocode}
%<*test-reload2>
\input miniltx.tex\relax
\input ifluatex.sty\relax
\input ifluatex.sty\relax
\csname @@end\endcsname\end
%</test-reload2>
%    \end{macrocode}
%
% \section{Installation}
%
% \subsection{Download}
%
% \paragraph{Package.} This package is available on
% CTAN\footnote{\url{http://ctan.org/pkg/ifluatex}}:
% \begin{description}
% \item[\CTAN{macros/latex/contrib/oberdiek/ifluatex.dtx}] The source file.
% \item[\CTAN{macros/latex/contrib/oberdiek/ifluatex.pdf}] Documentation.
% \end{description}
%
%
% \paragraph{Bundle.} All the packages of the bundle `oberdiek'
% are also available in a TDS compliant ZIP archive. There
% the packages are already unpacked and the documentation files
% are generated. The files and directories obey the TDS standard.
% \begin{description}
% \item[\CTAN{install/macros/latex/contrib/oberdiek.tds.zip}]
% \end{description}
% \emph{TDS} refers to the standard ``A Directory Structure
% for \TeX\ Files'' (\CTAN{tds/tds.pdf}). Directories
% with \xfile{texmf} in their name are usually organized this way.
%
% \subsection{Bundle installation}
%
% \paragraph{Unpacking.} Unpack the \xfile{oberdiek.tds.zip} in the
% TDS tree (also known as \xfile{texmf} tree) of your choice.
% Example (linux):
% \begin{quote}
%   |unzip oberdiek.tds.zip -d ~/texmf|
% \end{quote}
%
% \paragraph{Script installation.}
% Check the directory \xfile{TDS:scripts/oberdiek/} for
% scripts that need further installation steps.
% Package \xpackage{attachfile2} comes with the Perl script
% \xfile{pdfatfi.pl} that should be installed in such a way
% that it can be called as \texttt{pdfatfi}.
% Example (linux):
% \begin{quote}
%   |chmod +x scripts/oberdiek/pdfatfi.pl|\\
%   |cp scripts/oberdiek/pdfatfi.pl /usr/local/bin/|
% \end{quote}
%
% \subsection{Package installation}
%
% \paragraph{Unpacking.} The \xfile{.dtx} file is a self-extracting
% \docstrip\ archive. The files are extracted by running the
% \xfile{.dtx} through \plainTeX:
% \begin{quote}
%   \verb|tex ifluatex.dtx|
% \end{quote}
%
% \paragraph{TDS.} Now the different files must be moved into
% the different directories in your installation TDS tree
% (also known as \xfile{texmf} tree):
% \begin{quote}
% \def\t{^^A
% \begin{tabular}{@{}>{\ttfamily}l@{ $\rightarrow$ }>{\ttfamily}l@{}}
%   ifluatex.sty & tex/generic/oberdiek/ifluatex.sty\\
%   ifluatex.pdf & doc/latex/oberdiek/ifluatex.pdf\\
%   test/ifluatex-test1.tex & doc/latex/oberdiek/test/ifluatex-test1.tex\\
%   test/ifluatex-test2.tex & doc/latex/oberdiek/test/ifluatex-test2.tex\\
%   test/ifluatex-test3.tex & doc/latex/oberdiek/test/ifluatex-test3.tex\\
%   ifluatex.dtx & source/latex/oberdiek/ifluatex.dtx\\
% \end{tabular}^^A
% }^^A
% \sbox0{\t}^^A
% \ifdim\wd0>\linewidth
%   \begingroup
%     \advance\linewidth by\leftmargin
%     \advance\linewidth by\rightmargin
%   \edef\x{\endgroup
%     \def\noexpand\lw{\the\linewidth}^^A
%   }\x
%   \def\lwbox{^^A
%     \leavevmode
%     \hbox to \linewidth{^^A
%       \kern-\leftmargin\relax
%       \hss
%       \usebox0
%       \hss
%       \kern-\rightmargin\relax
%     }^^A
%   }^^A
%   \ifdim\wd0>\lw
%     \sbox0{\small\t}^^A
%     \ifdim\wd0>\linewidth
%       \ifdim\wd0>\lw
%         \sbox0{\footnotesize\t}^^A
%         \ifdim\wd0>\linewidth
%           \ifdim\wd0>\lw
%             \sbox0{\scriptsize\t}^^A
%             \ifdim\wd0>\linewidth
%               \ifdim\wd0>\lw
%                 \sbox0{\tiny\t}^^A
%                 \ifdim\wd0>\linewidth
%                   \lwbox
%                 \else
%                   \usebox0
%                 \fi
%               \else
%                 \lwbox
%               \fi
%             \else
%               \usebox0
%             \fi
%           \else
%             \lwbox
%           \fi
%         \else
%           \usebox0
%         \fi
%       \else
%         \lwbox
%       \fi
%     \else
%       \usebox0
%     \fi
%   \else
%     \lwbox
%   \fi
% \else
%   \usebox0
% \fi
% \end{quote}
% If you have a \xfile{docstrip.cfg} that configures and enables \docstrip's
% TDS installing feature, then some files can already be in the right
% place, see the documentation of \docstrip.
%
% \subsection{Refresh file name databases}
%
% If your \TeX~distribution
% (\teTeX, \mikTeX, \dots) relies on file name databases, you must refresh
% these. For example, \teTeX\ users run \verb|texhash| or
% \verb|mktexlsr|.
%
% \subsection{Some details for the interested}
%
% \paragraph{Attached source.}
%
% The PDF documentation on CTAN also includes the
% \xfile{.dtx} source file. It can be extracted by
% AcrobatReader 6 or higher. Another option is \textsf{pdftk},
% e.g. unpack the file into the current directory:
% \begin{quote}
%   \verb|pdftk ifluatex.pdf unpack_files output .|
% \end{quote}
%
% \paragraph{Unpacking with \LaTeX.}
% The \xfile{.dtx} chooses its action depending on the format:
% \begin{description}
% \item[\plainTeX:] Run \docstrip\ and extract the files.
% \item[\LaTeX:] Generate the documentation.
% \end{description}
% If you insist on using \LaTeX\ for \docstrip\ (really,
% \docstrip\ does not need \LaTeX), then inform the autodetect routine
% about your intention:
% \begin{quote}
%   \verb|latex \let\install=y% \iffalse meta-comment
%
% File: ifluatex.dtx
% Version: 2016/05/16 v1.4
% Info: Provides the ifluatex switch
%
% Copyright (C) 2007, 2009, 2010 by
%    Heiko Oberdiek <heiko.oberdiek at googlemail.com>
%    2016
%    https://github.com/ho-tex/oberdiek/issues
%
% This work may be distributed and/or modified under the
% conditions of the LaTeX Project Public License, either
% version 1.3c of this license or (at your option) any later
% version. This version of this license is in
%    http://www.latex-project.org/lppl/lppl-1-3c.txt
% and the latest version of this license is in
%    http://www.latex-project.org/lppl.txt
% and version 1.3 or later is part of all distributions of
% LaTeX version 2005/12/01 or later.
%
% This work has the LPPL maintenance status "maintained".
%
% This Current Maintainer of this work is Heiko Oberdiek.
%
% The Base Interpreter refers to any `TeX-Format',
% because some files are installed in TDS:tex/generic//.
%
% This work consists of the main source file ifluatex.dtx
% and the derived files
%    ifluatex.sty, ifluatex.pdf, ifluatex.ins, ifluatex.drv,
%    ifluatex-test1.tex, ifluatex-test2.tex, ifluatex-test3.tex.
%
% Distribution:
%    CTAN:macros/latex/contrib/oberdiek/ifluatex.dtx
%    CTAN:macros/latex/contrib/oberdiek/ifluatex.pdf
%
% Unpacking:
%    (a) If ifluatex.ins is present:
%           tex ifluatex.ins
%    (b) Without ifluatex.ins:
%           tex ifluatex.dtx
%    (c) If you insist on using LaTeX
%           latex \let\install=y% \iffalse meta-comment
%
% File: ifluatex.dtx
% Version: 2016/05/16 v1.4
% Info: Provides the ifluatex switch
%
% Copyright (C) 2007, 2009, 2010 by
%    Heiko Oberdiek <heiko.oberdiek at googlemail.com>
%    2016
%    https://github.com/ho-tex/oberdiek/issues
%
% This work may be distributed and/or modified under the
% conditions of the LaTeX Project Public License, either
% version 1.3c of this license or (at your option) any later
% version. This version of this license is in
%    http://www.latex-project.org/lppl/lppl-1-3c.txt
% and the latest version of this license is in
%    http://www.latex-project.org/lppl.txt
% and version 1.3 or later is part of all distributions of
% LaTeX version 2005/12/01 or later.
%
% This work has the LPPL maintenance status "maintained".
%
% This Current Maintainer of this work is Heiko Oberdiek.
%
% The Base Interpreter refers to any `TeX-Format',
% because some files are installed in TDS:tex/generic//.
%
% This work consists of the main source file ifluatex.dtx
% and the derived files
%    ifluatex.sty, ifluatex.pdf, ifluatex.ins, ifluatex.drv,
%    ifluatex-test1.tex, ifluatex-test2.tex, ifluatex-test3.tex.
%
% Distribution:
%    CTAN:macros/latex/contrib/oberdiek/ifluatex.dtx
%    CTAN:macros/latex/contrib/oberdiek/ifluatex.pdf
%
% Unpacking:
%    (a) If ifluatex.ins is present:
%           tex ifluatex.ins
%    (b) Without ifluatex.ins:
%           tex ifluatex.dtx
%    (c) If you insist on using LaTeX
%           latex \let\install=y\input{ifluatex.dtx}
%        (quote the arguments according to the demands of your shell)
%
% Documentation:
%    (a) If ifluatex.drv is present:
%           latex ifluatex.drv
%    (b) Without ifluatex.drv:
%           latex ifluatex.dtx; ...
%    The class ltxdoc loads the configuration file ltxdoc.cfg
%    if available. Here you can specify further options, e.g.
%    use A4 as paper format:
%       \PassOptionsToClass{a4paper}{article}
%
%    Programm calls to get the documentation (example):
%       pdflatex ifluatex.dtx
%       makeindex -s gind.ist ifluatex.idx
%       pdflatex ifluatex.dtx
%       makeindex -s gind.ist ifluatex.idx
%       pdflatex ifluatex.dtx
%
% Installation:
%    TDS:tex/generic/oberdiek/ifluatex.sty
%    TDS:doc/latex/oberdiek/ifluatex.pdf
%    TDS:doc/latex/oberdiek/test/ifluatex-test1.tex
%    TDS:doc/latex/oberdiek/test/ifluatex-test2.tex
%    TDS:doc/latex/oberdiek/test/ifluatex-test3.tex
%    TDS:source/latex/oberdiek/ifluatex.dtx
%
%<*ignore>
\begingroup
  \catcode123=1 %
  \catcode125=2 %
  \def\x{LaTeX2e}%
\expandafter\endgroup
\ifcase 0\ifx\install y1\fi\expandafter
         \ifx\csname processbatchFile\endcsname\relax\else1\fi
         \ifx\fmtname\x\else 1\fi\relax
\else\csname fi\endcsname
%</ignore>
%<*install>
\input docstrip.tex
\Msg{************************************************************************}
\Msg{* Installation}
\Msg{* Package: ifluatex 2016/05/16 v1.4 Provides the ifluatex switch (HO)}
\Msg{************************************************************************}

\keepsilent
\askforoverwritefalse

\let\MetaPrefix\relax
\preamble

This is a generated file.

Project: ifluatex
Version: 2016/05/16 v1.4

Copyright (C) 2007, 2009, 2010 by
   Heiko Oberdiek <heiko.oberdiek at googlemail.com>

This work may be distributed and/or modified under the
conditions of the LaTeX Project Public License, either
version 1.3c of this license or (at your option) any later
version. This version of this license is in
   http://www.latex-project.org/lppl/lppl-1-3c.txt
and the latest version of this license is in
   http://www.latex-project.org/lppl.txt
and version 1.3 or later is part of all distributions of
LaTeX version 2005/12/01 or later.

This work has the LPPL maintenance status "maintained".

This Current Maintainer of this work is Heiko Oberdiek.

The Base Interpreter refers to any `TeX-Format',
because some files are installed in TDS:tex/generic//.

This work consists of the main source file ifluatex.dtx
and the derived files
   ifluatex.sty, ifluatex.pdf, ifluatex.ins, ifluatex.drv,
   ifluatex-test1.tex, ifluatex-test2.tex, ifluatex-test3.tex.

\endpreamble
\let\MetaPrefix\DoubleperCent

\generate{%
  \file{ifluatex.ins}{\from{ifluatex.dtx}{install}}%
  \file{ifluatex.drv}{\from{ifluatex.dtx}{driver}}%
  \usedir{tex/generic/oberdiek}%
  \file{ifluatex.sty}{\from{ifluatex.dtx}{package}}%
  \usedir{doc/latex/oberdiek/test}%
  \file{ifluatex-test1.tex}{\from{ifluatex.dtx}{test1}}%
  \file{ifluatex-test2.tex}{\from{ifluatex.dtx}{test-reload1}}%
  \file{ifluatex-test3.tex}{\from{ifluatex.dtx}{test-reload2}}%
  \nopreamble
  \nopostamble
  \usedir{source/latex/oberdiek/catalogue}%
  \file{ifluatex.xml}{\from{ifluatex.dtx}{catalogue}}%
}

\catcode32=13\relax% active space
\let =\space%
\Msg{************************************************************************}
\Msg{*}
\Msg{* To finish the installation you have to move the following}
\Msg{* file into a directory searched by TeX:}
\Msg{*}
\Msg{*     ifluatex.sty}
\Msg{*}
\Msg{* To produce the documentation run the file `ifluatex.drv'}
\Msg{* through LaTeX.}
\Msg{*}
\Msg{* Happy TeXing!}
\Msg{*}
\Msg{************************************************************************}

\endbatchfile
%</install>
%<*ignore>
\fi
%</ignore>
%<*driver>
\NeedsTeXFormat{LaTeX2e}
\ProvidesFile{ifluatex.drv}%
  [2016/05/16 v1.4 Provides the ifluatex switch (HO)]%
\documentclass{ltxdoc}
\usepackage{holtxdoc}[2011/11/22]
\begin{document}
  \DocInput{ifluatex.dtx}%
\end{document}
%</driver>
% \fi
%
%
% \CharacterTable
%  {Upper-case    \A\B\C\D\E\F\G\H\I\J\K\L\M\N\O\P\Q\R\S\T\U\V\W\X\Y\Z
%   Lower-case    \a\b\c\d\e\f\g\h\i\j\k\l\m\n\o\p\q\r\s\t\u\v\w\x\y\z
%   Digits        \0\1\2\3\4\5\6\7\8\9
%   Exclamation   \!     Double quote  \"     Hash (number) \#
%   Dollar        \$     Percent       \%     Ampersand     \&
%   Acute accent  \'     Left paren    \(     Right paren   \)
%   Asterisk      \*     Plus          \+     Comma         \,
%   Minus         \-     Point         \.     Solidus       \/
%   Colon         \:     Semicolon     \;     Less than     \<
%   Equals        \=     Greater than  \>     Question mark \?
%   Commercial at \@     Left bracket  \[     Backslash     \\
%   Right bracket \]     Circumflex    \^     Underscore    \_
%   Grave accent  \`     Left brace    \{     Vertical bar  \|
%   Right brace   \}     Tilde         \~}
%
% \GetFileInfo{ifluatex.drv}
%
% \title{The \xpackage{ifluatex} package}
% \date{2016/05/16 v1.4}
% \author{Heiko Oberdiek\thanks
% {Please report any issues at https://github.com/ho-tex/oberdiek/issues}\\
% \xemail{heiko.oberdiek at googlemail.com}}
%
% \maketitle
%
% \begin{abstract}
% This package looks for \LuaTeX\ regardless of its mode
% and provides the switch \cs{ifluatex}. Also it makes
% \cs{luatexversion} available if it is not present.
% It works with \plainTeX\ or \LaTeX.
% \end{abstract}
%
% \tableofcontents
%
% \section{Documentation}
%
% The package \xpackage{ifluatex} can be used with both \plainTeX\
% and \LaTeX:
% \begin{description}
% \item[\plainTeX:] |\input ifluatex.sty|
% \item[\LaTeXe:]   |\usepackage{ifluatex}|
% \end{description}
%
% \DescribeMacro{\ifluatex}
% The package provides the switch \cs{ifluatex}:
% \begin{quote}
%   |\ifluatex|\\
%   \hspace{1.5em}\LuaTeX\ is running\\
%   |\else|\\
%   \hspace{1.5em}Without \LuaTeX\\
%   |\fi|
% \end{quote}
%
% Since version 0.39 \LuaTeX\ only provides \cs{directlua} at startup
% time. Also the syntax of \cs{directlua} changed in version 0.36.
% Thus the user might want to check the LuaTeX version.
% Therefore this package also makes \cs{luatexversion} and
% \cs{luatexrevision} available, if it is not yet done.
%
% If you want to detect the mode (DVI or PDF), then use package
% \xpackage{ifpdf}. \LuaTeX\ has inherited \cs{pdfoutput} from \pdfTeX.
%
% \StopEventually{
% }
%
% \section{Implementation}
%
%    \begin{macrocode}
%<*package>
%    \end{macrocode}
%
% \subsection{Reload check and package identification}
%    Reload check, especially if the package is not used with \LaTeX.
%    \begin{macrocode}
\begingroup\catcode61\catcode48\catcode32=10\relax%
  \catcode13=5 % ^^M
  \endlinechar=13 %
  \catcode35=6 % #
  \catcode39=12 % '
  \catcode44=12 % ,
  \catcode45=12 % -
  \catcode46=12 % .
  \catcode58=12 % :
  \catcode64=11 % @
  \catcode123=1 % {
  \catcode125=2 % }
  \expandafter\let\expandafter\x\csname ver@ifluatex.sty\endcsname
  \ifx\x\relax % plain-TeX, first loading
  \else
    \def\empty{}%
    \ifx\x\empty % LaTeX, first loading,
      % variable is initialized, but \ProvidesPackage not yet seen
    \else
      \expandafter\ifx\csname PackageInfo\endcsname\relax
        \def\x#1#2{%
          \immediate\write-1{Package #1 Info: #2.}%
        }%
      \else
        \def\x#1#2{\PackageInfo{#1}{#2, stopped}}%
      \fi
      \x{ifluatex}{The package is already loaded}%
      \aftergroup\endinput
    \fi
  \fi
\endgroup%
%    \end{macrocode}
%    Package identification:
%    \begin{macrocode}
\begingroup\catcode61\catcode48\catcode32=10\relax%
  \catcode13=5 % ^^M
  \endlinechar=13 %
  \catcode35=6 % #
  \catcode39=12 % '
  \catcode40=12 % (
  \catcode41=12 % )
  \catcode44=12 % ,
  \catcode45=12 % -
  \catcode46=12 % .
  \catcode47=12 % /
  \catcode58=12 % :
  \catcode64=11 % @
  \catcode91=12 % [
  \catcode93=12 % ]
  \catcode123=1 % {
  \catcode125=2 % }
  \expandafter\ifx\csname ProvidesPackage\endcsname\relax
    \def\x#1#2#3[#4]{\endgroup
      \immediate\write-1{Package: #3 #4}%
      \xdef#1{#4}%
    }%
  \else
    \def\x#1#2[#3]{\endgroup
      #2[{#3}]%
      \ifx#1\@undefined
        \xdef#1{#3}%
      \fi
      \ifx#1\relax
        \xdef#1{#3}%
      \fi
    }%
  \fi
\expandafter\x\csname ver@ifluatex.sty\endcsname
\ProvidesPackage{ifluatex}%
  [2016/05/16 v1.4 Provides the ifluatex switch (HO)]%
%    \end{macrocode}
%
% \subsection{Catcodes}
%
%    \begin{macrocode}
\begingroup\catcode61\catcode48\catcode32=10\relax%
  \catcode13=5 % ^^M
  \endlinechar=13 %
  \catcode123=1 % {
  \catcode125=2 % }
  \catcode64=11 % @
  \def\x{\endgroup
    \expandafter\edef\csname ifluatex@AtEnd\endcsname{%
      \endlinechar=\the\endlinechar\relax
      \catcode13=\the\catcode13\relax
      \catcode32=\the\catcode32\relax
      \catcode35=\the\catcode35\relax
      \catcode61=\the\catcode61\relax
      \catcode64=\the\catcode64\relax
      \catcode123=\the\catcode123\relax
      \catcode125=\the\catcode125\relax
    }%
  }%
\x\catcode61\catcode48\catcode32=10\relax%
\catcode13=5 % ^^M
\endlinechar=13 %
\catcode35=6 % #
\catcode64=11 % @
\catcode123=1 % {
\catcode125=2 % }
\def\TMP@EnsureCode#1#2{%
  \edef\ifluatex@AtEnd{%
    \ifluatex@AtEnd
    \catcode#1=\the\catcode#1\relax
  }%
  \catcode#1=#2\relax
}
\TMP@EnsureCode{10}{12}% ^^J
\TMP@EnsureCode{39}{12}% '
\TMP@EnsureCode{40}{12}% (
\TMP@EnsureCode{41}{12}% )
\TMP@EnsureCode{44}{12}% ,
\TMP@EnsureCode{45}{12}% -
\TMP@EnsureCode{46}{12}% .
\TMP@EnsureCode{47}{12}% /
\TMP@EnsureCode{58}{12}% :
\TMP@EnsureCode{60}{12}% <
\TMP@EnsureCode{94}{7}% ^
\TMP@EnsureCode{96}{12}% `
\edef\ifluatex@AtEnd{\ifluatex@AtEnd\noexpand\endinput}
%    \end{macrocode}
%
% \subsection{Macro for error messages}
%
%    \begin{macro}{\ifluatex@Error}
%    \begin{macrocode}
\begingroup\expandafter\expandafter\expandafter\endgroup
\expandafter\ifx\csname PackageError\endcsname\relax
  \def\ifluatex@Error#1#2{%
    \begingroup
      \newlinechar=10 %
      \def\MessageBreak{^^J}%
      \edef\x{\errhelp{#2}}%
      \x
      \errmessage{Package ifluatex Error: #1}%
    \endgroup
  }%
\else
  \def\ifluatex@Error{%
    \PackageError{ifluatex}%
  }%
\fi
%    \end{macrocode}
%    \end{macro}
%
% \subsection{Check for previously defined \cs{ifluatex}}
%
%    \begin{macrocode}
\begingroup
  \expandafter\ifx\csname ifluatex\endcsname\relax
  \else
    \edef\i/{\expandafter\string\csname ifluatex\endcsname}%
    \ifluatex@Error{Name clash, \i/ is already defined}{%
      Incompatible versions of \i/ can cause problems,\MessageBreak
      therefore package loading is aborted.%
    }%
    \endgroup
    \expandafter\ifluatex@AtEnd
  \fi%
\endgroup
%    \end{macrocode}
%
% \subsection{\cs{ifluatex}}
%
%    \begin{macro}{\ifluatex}
%    \begin{macrocode}
\let\ifluatex\iffalse
%    \end{macrocode}
%    \end{macro}
%
%    Test \cs{luatexversion}. Is it  defined and different from
%    \cs{relax}? Someone could have used \LaTeX\ internal
%    \cs{@ifundefined}, or something else involving.
%    Notice, \cs{csname} is executed inside a group for the test
%    to cancel the side effect of \cs{csname}.
%    \begin{macrocode}
\begingroup\expandafter\expandafter\expandafter\endgroup
\expandafter\ifx\csname luatexversion\endcsname\relax
\else
  \expandafter\let\csname ifluatex\expandafter\endcsname
                  \csname iftrue\endcsname
\fi
%    \end{macrocode}
%
% \subsection{Lua\TeX\ v0.39}
%
%     Starting with version 0.39 \LuaTeX\ wants to provide \cs{directlua}
%     as only primitive at startup time beyond vanilla \TeX's primitives.
%     Then \cs{directlua} exists, but \cs{luatexversion} cannot be found.
%     Unhappily also the syntax of \cs{directlua} changed in v0.36,
%     thus the user would want to check \cs{luatexversion}.
%     Therefore we make \cs{luatexversion} available using
%     \LuaTeX's Lua function |tex.enableprimitives|.
%
%    \begin{macrocode}
\ifluatex
\else
  \begingroup\expandafter\expandafter\expandafter\endgroup
  \expandafter\ifx\csname directlua\endcsname\relax
  \else
    \expandafter\let\csname ifluatex\expandafter\endcsname
                    \csname iftrue\endcsname
    \begingroup
      \newlinechar=10 %
      \endlinechar=\newlinechar%
      \ifnum0%
          \directlua{%
            if tex.enableprimitives then
              tex.enableprimitives('ifluatex', {'luatexversion'})
              tex.print('1')
            end
          }%
          \ifx\ifluatexluatexversion\@undefined\else 1\fi %
          =11 %
        \global\let\luatexversion\ifluatexluatexversion%
      \else%
        \ifluatex@Error{%
          Missing \string\luatexversion%
        }{%
          Update LuaTeX.%
        }%
      \fi%
    \endgroup%
  \fi
\fi
%    \end{macrocode}
%    \begin{macrocode}
\ifluatex
  \begingroup\expandafter\expandafter\expandafter\endgroup
  \expandafter\ifx\csname luatexrevision\endcsname\relax
    \ifnum\luatexversion<36 %
    \else
      \begingroup
        \ifx\luatexrevision\relax
          \let\luatexrevision\@undefined
        \fi
        \newlinechar=10 %
        \endlinechar=\newlinechar%
        \ifcase0%
            \directlua{%
              if tex.enableprimitives then
                tex.enableprimitives('ifluatex', {'luatexrevision'})
              else
                tex.print('1')
              end
            }%
            \ifx\ifluatexluatexrevision\@undefined 1\fi%
            \relax%
          \global\let\luatexrevision\ifluatexluatexrevision%
        \fi%
      \endgroup%
    \fi
    \begingroup\expandafter\expandafter\expandafter\endgroup
    \expandafter\ifx\csname luatexrevision\endcsname\relax
      \ifluatex@Error{%
        Missing \string\luatexrevision%
      }{%
        Update LuaTeX.%
      }%
    \fi
  \fi
\fi
%    \end{macrocode}
%
% \subsection{Protocol entry}
%
%     Log comment:
%    \begin{macrocode}
\begingroup
  \expandafter\ifx\csname PackageInfo\endcsname\relax
    \def\x#1#2{%
      \immediate\write-1{Package #1 Info: #2.}%
    }%
  \else
    \let\x\PackageInfo
    \expandafter\let\csname on@line\endcsname\empty
  \fi
  \x{ifluatex}{LuaTeX \ifluatex\else not \fi detected}%
\endgroup
%    \end{macrocode}
%    \begin{macrocode}
\ifluatex@AtEnd%
%    \end{macrocode}
%    \begin{macrocode}
%</package>
%    \end{macrocode}
%
% \section{Test}
%
% \subsection{Catcode checks for loading}
%
%    \begin{macrocode}
%<*test1>
%    \end{macrocode}
%    \begin{macrocode}
\catcode`\{=1 %
\catcode`\}=2 %
\catcode`\#=6 %
\catcode`\@=11 %
\expandafter\ifx\csname count@\endcsname\relax
  \countdef\count@=255 %
\fi
\expandafter\ifx\csname @gobble\endcsname\relax
  \long\def\@gobble#1{}%
\fi
\expandafter\ifx\csname @firstofone\endcsname\relax
  \long\def\@firstofone#1{#1}%
\fi
\expandafter\ifx\csname loop\endcsname\relax
  \expandafter\@firstofone
\else
  \expandafter\@gobble
\fi
{%
  \def\loop#1\repeat{%
    \def\body{#1}%
    \iterate
  }%
  \def\iterate{%
    \body
      \let\next\iterate
    \else
      \let\next\relax
    \fi
    \next
  }%
  \let\repeat=\fi
}%
\def\RestoreCatcodes{}
\count@=0 %
\loop
  \edef\RestoreCatcodes{%
    \RestoreCatcodes
    \catcode\the\count@=\the\catcode\count@\relax
  }%
\ifnum\count@<255 %
  \advance\count@ 1 %
\repeat

\def\RangeCatcodeInvalid#1#2{%
  \count@=#1\relax
  \loop
    \catcode\count@=15 %
  \ifnum\count@<#2\relax
    \advance\count@ 1 %
  \repeat
}
\def\RangeCatcodeCheck#1#2#3{%
  \count@=#1\relax
  \loop
    \ifnum#3=\catcode\count@
    \else
      \errmessage{%
        Character \the\count@\space
        with wrong catcode \the\catcode\count@\space
        instead of \number#3%
      }%
    \fi
  \ifnum\count@<#2\relax
    \advance\count@ 1 %
  \repeat
}
\def\space{ }
\expandafter\ifx\csname LoadCommand\endcsname\relax
  \def\LoadCommand{\input ifluatex.sty\relax}%
\fi
\def\Test{%
  \RangeCatcodeInvalid{0}{47}%
  \RangeCatcodeInvalid{58}{64}%
  \RangeCatcodeInvalid{91}{96}%
  \RangeCatcodeInvalid{123}{255}%
  \catcode`\@=12 %
  \catcode`\\=0 %
  \catcode`\%=14 %
  \LoadCommand
  \RangeCatcodeCheck{0}{36}{15}%
  \RangeCatcodeCheck{37}{37}{14}%
  \RangeCatcodeCheck{38}{47}{15}%
  \RangeCatcodeCheck{48}{57}{12}%
  \RangeCatcodeCheck{58}{63}{15}%
  \RangeCatcodeCheck{64}{64}{12}%
  \RangeCatcodeCheck{65}{90}{11}%
  \RangeCatcodeCheck{91}{91}{15}%
  \RangeCatcodeCheck{92}{92}{0}%
  \RangeCatcodeCheck{93}{96}{15}%
  \RangeCatcodeCheck{97}{122}{11}%
  \RangeCatcodeCheck{123}{255}{15}%
  \RestoreCatcodes
}
\Test
\csname @@end\endcsname
\end
%    \end{macrocode}
%    \begin{macrocode}
%</test1>
%    \end{macrocode}
%
% \section{Reload check for plain}
%
%    \begin{macrocode}
%<*test-reload1>
\input ifluatex.sty\relax
\input ifluatex.sty\relax
\csname @@end\endcsname\end
%</test-reload1>
%    \end{macrocode}
%
%    \begin{macrocode}
%<*test-reload2>
\input miniltx.tex\relax
\input ifluatex.sty\relax
\input ifluatex.sty\relax
\csname @@end\endcsname\end
%</test-reload2>
%    \end{macrocode}
%
% \section{Installation}
%
% \subsection{Download}
%
% \paragraph{Package.} This package is available on
% CTAN\footnote{\url{http://ctan.org/pkg/ifluatex}}:
% \begin{description}
% \item[\CTAN{macros/latex/contrib/oberdiek/ifluatex.dtx}] The source file.
% \item[\CTAN{macros/latex/contrib/oberdiek/ifluatex.pdf}] Documentation.
% \end{description}
%
%
% \paragraph{Bundle.} All the packages of the bundle `oberdiek'
% are also available in a TDS compliant ZIP archive. There
% the packages are already unpacked and the documentation files
% are generated. The files and directories obey the TDS standard.
% \begin{description}
% \item[\CTAN{install/macros/latex/contrib/oberdiek.tds.zip}]
% \end{description}
% \emph{TDS} refers to the standard ``A Directory Structure
% for \TeX\ Files'' (\CTAN{tds/tds.pdf}). Directories
% with \xfile{texmf} in their name are usually organized this way.
%
% \subsection{Bundle installation}
%
% \paragraph{Unpacking.} Unpack the \xfile{oberdiek.tds.zip} in the
% TDS tree (also known as \xfile{texmf} tree) of your choice.
% Example (linux):
% \begin{quote}
%   |unzip oberdiek.tds.zip -d ~/texmf|
% \end{quote}
%
% \paragraph{Script installation.}
% Check the directory \xfile{TDS:scripts/oberdiek/} for
% scripts that need further installation steps.
% Package \xpackage{attachfile2} comes with the Perl script
% \xfile{pdfatfi.pl} that should be installed in such a way
% that it can be called as \texttt{pdfatfi}.
% Example (linux):
% \begin{quote}
%   |chmod +x scripts/oberdiek/pdfatfi.pl|\\
%   |cp scripts/oberdiek/pdfatfi.pl /usr/local/bin/|
% \end{quote}
%
% \subsection{Package installation}
%
% \paragraph{Unpacking.} The \xfile{.dtx} file is a self-extracting
% \docstrip\ archive. The files are extracted by running the
% \xfile{.dtx} through \plainTeX:
% \begin{quote}
%   \verb|tex ifluatex.dtx|
% \end{quote}
%
% \paragraph{TDS.} Now the different files must be moved into
% the different directories in your installation TDS tree
% (also known as \xfile{texmf} tree):
% \begin{quote}
% \def\t{^^A
% \begin{tabular}{@{}>{\ttfamily}l@{ $\rightarrow$ }>{\ttfamily}l@{}}
%   ifluatex.sty & tex/generic/oberdiek/ifluatex.sty\\
%   ifluatex.pdf & doc/latex/oberdiek/ifluatex.pdf\\
%   test/ifluatex-test1.tex & doc/latex/oberdiek/test/ifluatex-test1.tex\\
%   test/ifluatex-test2.tex & doc/latex/oberdiek/test/ifluatex-test2.tex\\
%   test/ifluatex-test3.tex & doc/latex/oberdiek/test/ifluatex-test3.tex\\
%   ifluatex.dtx & source/latex/oberdiek/ifluatex.dtx\\
% \end{tabular}^^A
% }^^A
% \sbox0{\t}^^A
% \ifdim\wd0>\linewidth
%   \begingroup
%     \advance\linewidth by\leftmargin
%     \advance\linewidth by\rightmargin
%   \edef\x{\endgroup
%     \def\noexpand\lw{\the\linewidth}^^A
%   }\x
%   \def\lwbox{^^A
%     \leavevmode
%     \hbox to \linewidth{^^A
%       \kern-\leftmargin\relax
%       \hss
%       \usebox0
%       \hss
%       \kern-\rightmargin\relax
%     }^^A
%   }^^A
%   \ifdim\wd0>\lw
%     \sbox0{\small\t}^^A
%     \ifdim\wd0>\linewidth
%       \ifdim\wd0>\lw
%         \sbox0{\footnotesize\t}^^A
%         \ifdim\wd0>\linewidth
%           \ifdim\wd0>\lw
%             \sbox0{\scriptsize\t}^^A
%             \ifdim\wd0>\linewidth
%               \ifdim\wd0>\lw
%                 \sbox0{\tiny\t}^^A
%                 \ifdim\wd0>\linewidth
%                   \lwbox
%                 \else
%                   \usebox0
%                 \fi
%               \else
%                 \lwbox
%               \fi
%             \else
%               \usebox0
%             \fi
%           \else
%             \lwbox
%           \fi
%         \else
%           \usebox0
%         \fi
%       \else
%         \lwbox
%       \fi
%     \else
%       \usebox0
%     \fi
%   \else
%     \lwbox
%   \fi
% \else
%   \usebox0
% \fi
% \end{quote}
% If you have a \xfile{docstrip.cfg} that configures and enables \docstrip's
% TDS installing feature, then some files can already be in the right
% place, see the documentation of \docstrip.
%
% \subsection{Refresh file name databases}
%
% If your \TeX~distribution
% (\teTeX, \mikTeX, \dots) relies on file name databases, you must refresh
% these. For example, \teTeX\ users run \verb|texhash| or
% \verb|mktexlsr|.
%
% \subsection{Some details for the interested}
%
% \paragraph{Attached source.}
%
% The PDF documentation on CTAN also includes the
% \xfile{.dtx} source file. It can be extracted by
% AcrobatReader 6 or higher. Another option is \textsf{pdftk},
% e.g. unpack the file into the current directory:
% \begin{quote}
%   \verb|pdftk ifluatex.pdf unpack_files output .|
% \end{quote}
%
% \paragraph{Unpacking with \LaTeX.}
% The \xfile{.dtx} chooses its action depending on the format:
% \begin{description}
% \item[\plainTeX:] Run \docstrip\ and extract the files.
% \item[\LaTeX:] Generate the documentation.
% \end{description}
% If you insist on using \LaTeX\ for \docstrip\ (really,
% \docstrip\ does not need \LaTeX), then inform the autodetect routine
% about your intention:
% \begin{quote}
%   \verb|latex \let\install=y\input{ifluatex.dtx}|
% \end{quote}
% Do not forget to quote the argument according to the demands
% of your shell.
%
% \paragraph{Generating the documentation.}
% You can use both the \xfile{.dtx} or the \xfile{.drv} to generate
% the documentation. The process can be configured by the
% configuration file \xfile{ltxdoc.cfg}. For instance, put this
% line into this file, if you want to have A4 as paper format:
% \begin{quote}
%   \verb|\PassOptionsToClass{a4paper}{article}|
% \end{quote}
% An example follows how to generate the
% documentation with pdf\LaTeX:
% \begin{quote}
%\begin{verbatim}
%pdflatex ifluatex.dtx
%makeindex -s gind.ist ifluatex.idx
%pdflatex ifluatex.dtx
%makeindex -s gind.ist ifluatex.idx
%pdflatex ifluatex.dtx
%\end{verbatim}
% \end{quote}
%
% \section{Catalogue}
%
% The following XML file can be used as source for the
% \href{http://mirror.ctan.org/help/Catalogue/catalogue.html}{\TeX\ Catalogue}.
% The elements \texttt{caption} and \texttt{description} are imported
% from the original XML file from the Catalogue.
% The name of the XML file in the Catalogue is \xfile{ifluatex.xml}.
%    \begin{macrocode}
%<*catalogue>
<?xml version='1.0' encoding='us-ascii'?>
<!DOCTYPE entry SYSTEM 'catalogue.dtd'>
<entry datestamp='$Date$' modifier='$Author$' id='ifluatex'>
  <name>ifluatex</name>
  <caption>Provides the \ifluatex switch.</caption>
  <authorref id='auth:oberdiek'/>
  <copyright owner='Heiko Oberdiek' year='2007,2009,2010'/>
  <license type='lppl1.3'/>
  <version number='1.4'/>
  <description>
    The package looks for  LuaTeX regardless of its mode and provides
    the switch <tt>\ifluatex</tt>; it works with Plain TeX or LaTeX.
    <p/>
    The package is part of the <xref refid='oberdiek'>oberdiek</xref>
    bundle.
  </description>
  <documentation details='Package documentation'
      href='ctan:/macros/latex/contrib/oberdiek/ifluatex.pdf'/>
  <ctan file='true' path='/macros/latex/contrib/oberdiek/ifluatex.dtx'/>
  <miktex location='oberdiek'/>
  <texlive location='ifluatex'/>
  <install path='/macros/latex/contrib/oberdiek/oberdiek.tds.zip'/>
</entry>
%</catalogue>
%    \end{macrocode}
%
% \begin{History}
%   \begin{Version}{2007/12/12 v1.0}
%   \item
%     First public version.
%   \end{Version}
%   \begin{Version}{2009/04/10 v1.1}
%   \item
%     Test adopted for \LuaTeX\ 0.39.
%   \item
%     Makes \cs{luatexversion} available.
%   \end{Version}
%   \begin{Version}{2009/04/17 v1.2}
%   \item
%     Fixes (Manuel P\'egouri\'e-Gonnard).
%   \item
%     \cs{luatextrue} and \cs{luatexfalse} are no longer defined.
%   \item
%     Makes \cs{luatexrevision} available, too.
%   \end{Version}
%   \begin{Version}{2010/03/01 v1.3}
%   \item
%     Line ends fixed in case \cs{endlinechar} = \cs{newlinechar}.
%   \end{Version}
%   \begin{Version}{2016/05/16 v1.4}
%   \item
%     Documentation updates.
%   \end{Version}
% \end{History}
%
% \PrintIndex
%
% \Finale
\endinput

%        (quote the arguments according to the demands of your shell)
%
% Documentation:
%    (a) If ifluatex.drv is present:
%           latex ifluatex.drv
%    (b) Without ifluatex.drv:
%           latex ifluatex.dtx; ...
%    The class ltxdoc loads the configuration file ltxdoc.cfg
%    if available. Here you can specify further options, e.g.
%    use A4 as paper format:
%       \PassOptionsToClass{a4paper}{article}
%
%    Programm calls to get the documentation (example):
%       pdflatex ifluatex.dtx
%       makeindex -s gind.ist ifluatex.idx
%       pdflatex ifluatex.dtx
%       makeindex -s gind.ist ifluatex.idx
%       pdflatex ifluatex.dtx
%
% Installation:
%    TDS:tex/generic/oberdiek/ifluatex.sty
%    TDS:doc/latex/oberdiek/ifluatex.pdf
%    TDS:doc/latex/oberdiek/test/ifluatex-test1.tex
%    TDS:doc/latex/oberdiek/test/ifluatex-test2.tex
%    TDS:doc/latex/oberdiek/test/ifluatex-test3.tex
%    TDS:source/latex/oberdiek/ifluatex.dtx
%
%<*ignore>
\begingroup
  \catcode123=1 %
  \catcode125=2 %
  \def\x{LaTeX2e}%
\expandafter\endgroup
\ifcase 0\ifx\install y1\fi\expandafter
         \ifx\csname processbatchFile\endcsname\relax\else1\fi
         \ifx\fmtname\x\else 1\fi\relax
\else\csname fi\endcsname
%</ignore>
%<*install>
\input docstrip.tex
\Msg{************************************************************************}
\Msg{* Installation}
\Msg{* Package: ifluatex 2016/05/16 v1.4 Provides the ifluatex switch (HO)}
\Msg{************************************************************************}

\keepsilent
\askforoverwritefalse

\let\MetaPrefix\relax
\preamble

This is a generated file.

Project: ifluatex
Version: 2016/05/16 v1.4

Copyright (C) 2007, 2009, 2010 by
   Heiko Oberdiek <heiko.oberdiek at googlemail.com>

This work may be distributed and/or modified under the
conditions of the LaTeX Project Public License, either
version 1.3c of this license or (at your option) any later
version. This version of this license is in
   http://www.latex-project.org/lppl/lppl-1-3c.txt
and the latest version of this license is in
   http://www.latex-project.org/lppl.txt
and version 1.3 or later is part of all distributions of
LaTeX version 2005/12/01 or later.

This work has the LPPL maintenance status "maintained".

This Current Maintainer of this work is Heiko Oberdiek.

The Base Interpreter refers to any `TeX-Format',
because some files are installed in TDS:tex/generic//.

This work consists of the main source file ifluatex.dtx
and the derived files
   ifluatex.sty, ifluatex.pdf, ifluatex.ins, ifluatex.drv,
   ifluatex-test1.tex, ifluatex-test2.tex, ifluatex-test3.tex.

\endpreamble
\let\MetaPrefix\DoubleperCent

\generate{%
  \file{ifluatex.ins}{\from{ifluatex.dtx}{install}}%
  \file{ifluatex.drv}{\from{ifluatex.dtx}{driver}}%
  \usedir{tex/generic/oberdiek}%
  \file{ifluatex.sty}{\from{ifluatex.dtx}{package}}%
  \usedir{doc/latex/oberdiek/test}%
  \file{ifluatex-test1.tex}{\from{ifluatex.dtx}{test1}}%
  \file{ifluatex-test2.tex}{\from{ifluatex.dtx}{test-reload1}}%
  \file{ifluatex-test3.tex}{\from{ifluatex.dtx}{test-reload2}}%
  \nopreamble
  \nopostamble
  \usedir{source/latex/oberdiek/catalogue}%
  \file{ifluatex.xml}{\from{ifluatex.dtx}{catalogue}}%
}

\catcode32=13\relax% active space
\let =\space%
\Msg{************************************************************************}
\Msg{*}
\Msg{* To finish the installation you have to move the following}
\Msg{* file into a directory searched by TeX:}
\Msg{*}
\Msg{*     ifluatex.sty}
\Msg{*}
\Msg{* To produce the documentation run the file `ifluatex.drv'}
\Msg{* through LaTeX.}
\Msg{*}
\Msg{* Happy TeXing!}
\Msg{*}
\Msg{************************************************************************}

\endbatchfile
%</install>
%<*ignore>
\fi
%</ignore>
%<*driver>
\NeedsTeXFormat{LaTeX2e}
\ProvidesFile{ifluatex.drv}%
  [2016/05/16 v1.4 Provides the ifluatex switch (HO)]%
\documentclass{ltxdoc}
\usepackage{holtxdoc}[2011/11/22]
\begin{document}
  \DocInput{ifluatex.dtx}%
\end{document}
%</driver>
% \fi
%
%
% \CharacterTable
%  {Upper-case    \A\B\C\D\E\F\G\H\I\J\K\L\M\N\O\P\Q\R\S\T\U\V\W\X\Y\Z
%   Lower-case    \a\b\c\d\e\f\g\h\i\j\k\l\m\n\o\p\q\r\s\t\u\v\w\x\y\z
%   Digits        \0\1\2\3\4\5\6\7\8\9
%   Exclamation   \!     Double quote  \"     Hash (number) \#
%   Dollar        \$     Percent       \%     Ampersand     \&
%   Acute accent  \'     Left paren    \(     Right paren   \)
%   Asterisk      \*     Plus          \+     Comma         \,
%   Minus         \-     Point         \.     Solidus       \/
%   Colon         \:     Semicolon     \;     Less than     \<
%   Equals        \=     Greater than  \>     Question mark \?
%   Commercial at \@     Left bracket  \[     Backslash     \\
%   Right bracket \]     Circumflex    \^     Underscore    \_
%   Grave accent  \`     Left brace    \{     Vertical bar  \|
%   Right brace   \}     Tilde         \~}
%
% \GetFileInfo{ifluatex.drv}
%
% \title{The \xpackage{ifluatex} package}
% \date{2016/05/16 v1.4}
% \author{Heiko Oberdiek\thanks
% {Please report any issues at https://github.com/ho-tex/oberdiek/issues}\\
% \xemail{heiko.oberdiek at googlemail.com}}
%
% \maketitle
%
% \begin{abstract}
% This package looks for \LuaTeX\ regardless of its mode
% and provides the switch \cs{ifluatex}. Also it makes
% \cs{luatexversion} available if it is not present.
% It works with \plainTeX\ or \LaTeX.
% \end{abstract}
%
% \tableofcontents
%
% \section{Documentation}
%
% The package \xpackage{ifluatex} can be used with both \plainTeX\
% and \LaTeX:
% \begin{description}
% \item[\plainTeX:] |\input ifluatex.sty|
% \item[\LaTeXe:]   |\usepackage{ifluatex}|
% \end{description}
%
% \DescribeMacro{\ifluatex}
% The package provides the switch \cs{ifluatex}:
% \begin{quote}
%   |\ifluatex|\\
%   \hspace{1.5em}\LuaTeX\ is running\\
%   |\else|\\
%   \hspace{1.5em}Without \LuaTeX\\
%   |\fi|
% \end{quote}
%
% Since version 0.39 \LuaTeX\ only provides \cs{directlua} at startup
% time. Also the syntax of \cs{directlua} changed in version 0.36.
% Thus the user might want to check the LuaTeX version.
% Therefore this package also makes \cs{luatexversion} and
% \cs{luatexrevision} available, if it is not yet done.
%
% If you want to detect the mode (DVI or PDF), then use package
% \xpackage{ifpdf}. \LuaTeX\ has inherited \cs{pdfoutput} from \pdfTeX.
%
% \StopEventually{
% }
%
% \section{Implementation}
%
%    \begin{macrocode}
%<*package>
%    \end{macrocode}
%
% \subsection{Reload check and package identification}
%    Reload check, especially if the package is not used with \LaTeX.
%    \begin{macrocode}
\begingroup\catcode61\catcode48\catcode32=10\relax%
  \catcode13=5 % ^^M
  \endlinechar=13 %
  \catcode35=6 % #
  \catcode39=12 % '
  \catcode44=12 % ,
  \catcode45=12 % -
  \catcode46=12 % .
  \catcode58=12 % :
  \catcode64=11 % @
  \catcode123=1 % {
  \catcode125=2 % }
  \expandafter\let\expandafter\x\csname ver@ifluatex.sty\endcsname
  \ifx\x\relax % plain-TeX, first loading
  \else
    \def\empty{}%
    \ifx\x\empty % LaTeX, first loading,
      % variable is initialized, but \ProvidesPackage not yet seen
    \else
      \expandafter\ifx\csname PackageInfo\endcsname\relax
        \def\x#1#2{%
          \immediate\write-1{Package #1 Info: #2.}%
        }%
      \else
        \def\x#1#2{\PackageInfo{#1}{#2, stopped}}%
      \fi
      \x{ifluatex}{The package is already loaded}%
      \aftergroup\endinput
    \fi
  \fi
\endgroup%
%    \end{macrocode}
%    Package identification:
%    \begin{macrocode}
\begingroup\catcode61\catcode48\catcode32=10\relax%
  \catcode13=5 % ^^M
  \endlinechar=13 %
  \catcode35=6 % #
  \catcode39=12 % '
  \catcode40=12 % (
  \catcode41=12 % )
  \catcode44=12 % ,
  \catcode45=12 % -
  \catcode46=12 % .
  \catcode47=12 % /
  \catcode58=12 % :
  \catcode64=11 % @
  \catcode91=12 % [
  \catcode93=12 % ]
  \catcode123=1 % {
  \catcode125=2 % }
  \expandafter\ifx\csname ProvidesPackage\endcsname\relax
    \def\x#1#2#3[#4]{\endgroup
      \immediate\write-1{Package: #3 #4}%
      \xdef#1{#4}%
    }%
  \else
    \def\x#1#2[#3]{\endgroup
      #2[{#3}]%
      \ifx#1\@undefined
        \xdef#1{#3}%
      \fi
      \ifx#1\relax
        \xdef#1{#3}%
      \fi
    }%
  \fi
\expandafter\x\csname ver@ifluatex.sty\endcsname
\ProvidesPackage{ifluatex}%
  [2016/05/16 v1.4 Provides the ifluatex switch (HO)]%
%    \end{macrocode}
%
% \subsection{Catcodes}
%
%    \begin{macrocode}
\begingroup\catcode61\catcode48\catcode32=10\relax%
  \catcode13=5 % ^^M
  \endlinechar=13 %
  \catcode123=1 % {
  \catcode125=2 % }
  \catcode64=11 % @
  \def\x{\endgroup
    \expandafter\edef\csname ifluatex@AtEnd\endcsname{%
      \endlinechar=\the\endlinechar\relax
      \catcode13=\the\catcode13\relax
      \catcode32=\the\catcode32\relax
      \catcode35=\the\catcode35\relax
      \catcode61=\the\catcode61\relax
      \catcode64=\the\catcode64\relax
      \catcode123=\the\catcode123\relax
      \catcode125=\the\catcode125\relax
    }%
  }%
\x\catcode61\catcode48\catcode32=10\relax%
\catcode13=5 % ^^M
\endlinechar=13 %
\catcode35=6 % #
\catcode64=11 % @
\catcode123=1 % {
\catcode125=2 % }
\def\TMP@EnsureCode#1#2{%
  \edef\ifluatex@AtEnd{%
    \ifluatex@AtEnd
    \catcode#1=\the\catcode#1\relax
  }%
  \catcode#1=#2\relax
}
\TMP@EnsureCode{10}{12}% ^^J
\TMP@EnsureCode{39}{12}% '
\TMP@EnsureCode{40}{12}% (
\TMP@EnsureCode{41}{12}% )
\TMP@EnsureCode{44}{12}% ,
\TMP@EnsureCode{45}{12}% -
\TMP@EnsureCode{46}{12}% .
\TMP@EnsureCode{47}{12}% /
\TMP@EnsureCode{58}{12}% :
\TMP@EnsureCode{60}{12}% <
\TMP@EnsureCode{94}{7}% ^
\TMP@EnsureCode{96}{12}% `
\edef\ifluatex@AtEnd{\ifluatex@AtEnd\noexpand\endinput}
%    \end{macrocode}
%
% \subsection{Macro for error messages}
%
%    \begin{macro}{\ifluatex@Error}
%    \begin{macrocode}
\begingroup\expandafter\expandafter\expandafter\endgroup
\expandafter\ifx\csname PackageError\endcsname\relax
  \def\ifluatex@Error#1#2{%
    \begingroup
      \newlinechar=10 %
      \def\MessageBreak{^^J}%
      \edef\x{\errhelp{#2}}%
      \x
      \errmessage{Package ifluatex Error: #1}%
    \endgroup
  }%
\else
  \def\ifluatex@Error{%
    \PackageError{ifluatex}%
  }%
\fi
%    \end{macrocode}
%    \end{macro}
%
% \subsection{Check for previously defined \cs{ifluatex}}
%
%    \begin{macrocode}
\begingroup
  \expandafter\ifx\csname ifluatex\endcsname\relax
  \else
    \edef\i/{\expandafter\string\csname ifluatex\endcsname}%
    \ifluatex@Error{Name clash, \i/ is already defined}{%
      Incompatible versions of \i/ can cause problems,\MessageBreak
      therefore package loading is aborted.%
    }%
    \endgroup
    \expandafter\ifluatex@AtEnd
  \fi%
\endgroup
%    \end{macrocode}
%
% \subsection{\cs{ifluatex}}
%
%    \begin{macro}{\ifluatex}
%    \begin{macrocode}
\let\ifluatex\iffalse
%    \end{macrocode}
%    \end{macro}
%
%    Test \cs{luatexversion}. Is it  defined and different from
%    \cs{relax}? Someone could have used \LaTeX\ internal
%    \cs{@ifundefined}, or something else involving.
%    Notice, \cs{csname} is executed inside a group for the test
%    to cancel the side effect of \cs{csname}.
%    \begin{macrocode}
\begingroup\expandafter\expandafter\expandafter\endgroup
\expandafter\ifx\csname luatexversion\endcsname\relax
\else
  \expandafter\let\csname ifluatex\expandafter\endcsname
                  \csname iftrue\endcsname
\fi
%    \end{macrocode}
%
% \subsection{Lua\TeX\ v0.39}
%
%     Starting with version 0.39 \LuaTeX\ wants to provide \cs{directlua}
%     as only primitive at startup time beyond vanilla \TeX's primitives.
%     Then \cs{directlua} exists, but \cs{luatexversion} cannot be found.
%     Unhappily also the syntax of \cs{directlua} changed in v0.36,
%     thus the user would want to check \cs{luatexversion}.
%     Therefore we make \cs{luatexversion} available using
%     \LuaTeX's Lua function |tex.enableprimitives|.
%
%    \begin{macrocode}
\ifluatex
\else
  \begingroup\expandafter\expandafter\expandafter\endgroup
  \expandafter\ifx\csname directlua\endcsname\relax
  \else
    \expandafter\let\csname ifluatex\expandafter\endcsname
                    \csname iftrue\endcsname
    \begingroup
      \newlinechar=10 %
      \endlinechar=\newlinechar%
      \ifnum0%
          \directlua{%
            if tex.enableprimitives then
              tex.enableprimitives('ifluatex', {'luatexversion'})
              tex.print('1')
            end
          }%
          \ifx\ifluatexluatexversion\@undefined\else 1\fi %
          =11 %
        \global\let\luatexversion\ifluatexluatexversion%
      \else%
        \ifluatex@Error{%
          Missing \string\luatexversion%
        }{%
          Update LuaTeX.%
        }%
      \fi%
    \endgroup%
  \fi
\fi
%    \end{macrocode}
%    \begin{macrocode}
\ifluatex
  \begingroup\expandafter\expandafter\expandafter\endgroup
  \expandafter\ifx\csname luatexrevision\endcsname\relax
    \ifnum\luatexversion<36 %
    \else
      \begingroup
        \ifx\luatexrevision\relax
          \let\luatexrevision\@undefined
        \fi
        \newlinechar=10 %
        \endlinechar=\newlinechar%
        \ifcase0%
            \directlua{%
              if tex.enableprimitives then
                tex.enableprimitives('ifluatex', {'luatexrevision'})
              else
                tex.print('1')
              end
            }%
            \ifx\ifluatexluatexrevision\@undefined 1\fi%
            \relax%
          \global\let\luatexrevision\ifluatexluatexrevision%
        \fi%
      \endgroup%
    \fi
    \begingroup\expandafter\expandafter\expandafter\endgroup
    \expandafter\ifx\csname luatexrevision\endcsname\relax
      \ifluatex@Error{%
        Missing \string\luatexrevision%
      }{%
        Update LuaTeX.%
      }%
    \fi
  \fi
\fi
%    \end{macrocode}
%
% \subsection{Protocol entry}
%
%     Log comment:
%    \begin{macrocode}
\begingroup
  \expandafter\ifx\csname PackageInfo\endcsname\relax
    \def\x#1#2{%
      \immediate\write-1{Package #1 Info: #2.}%
    }%
  \else
    \let\x\PackageInfo
    \expandafter\let\csname on@line\endcsname\empty
  \fi
  \x{ifluatex}{LuaTeX \ifluatex\else not \fi detected}%
\endgroup
%    \end{macrocode}
%    \begin{macrocode}
\ifluatex@AtEnd%
%    \end{macrocode}
%    \begin{macrocode}
%</package>
%    \end{macrocode}
%
% \section{Test}
%
% \subsection{Catcode checks for loading}
%
%    \begin{macrocode}
%<*test1>
%    \end{macrocode}
%    \begin{macrocode}
\catcode`\{=1 %
\catcode`\}=2 %
\catcode`\#=6 %
\catcode`\@=11 %
\expandafter\ifx\csname count@\endcsname\relax
  \countdef\count@=255 %
\fi
\expandafter\ifx\csname @gobble\endcsname\relax
  \long\def\@gobble#1{}%
\fi
\expandafter\ifx\csname @firstofone\endcsname\relax
  \long\def\@firstofone#1{#1}%
\fi
\expandafter\ifx\csname loop\endcsname\relax
  \expandafter\@firstofone
\else
  \expandafter\@gobble
\fi
{%
  \def\loop#1\repeat{%
    \def\body{#1}%
    \iterate
  }%
  \def\iterate{%
    \body
      \let\next\iterate
    \else
      \let\next\relax
    \fi
    \next
  }%
  \let\repeat=\fi
}%
\def\RestoreCatcodes{}
\count@=0 %
\loop
  \edef\RestoreCatcodes{%
    \RestoreCatcodes
    \catcode\the\count@=\the\catcode\count@\relax
  }%
\ifnum\count@<255 %
  \advance\count@ 1 %
\repeat

\def\RangeCatcodeInvalid#1#2{%
  \count@=#1\relax
  \loop
    \catcode\count@=15 %
  \ifnum\count@<#2\relax
    \advance\count@ 1 %
  \repeat
}
\def\RangeCatcodeCheck#1#2#3{%
  \count@=#1\relax
  \loop
    \ifnum#3=\catcode\count@
    \else
      \errmessage{%
        Character \the\count@\space
        with wrong catcode \the\catcode\count@\space
        instead of \number#3%
      }%
    \fi
  \ifnum\count@<#2\relax
    \advance\count@ 1 %
  \repeat
}
\def\space{ }
\expandafter\ifx\csname LoadCommand\endcsname\relax
  \def\LoadCommand{\input ifluatex.sty\relax}%
\fi
\def\Test{%
  \RangeCatcodeInvalid{0}{47}%
  \RangeCatcodeInvalid{58}{64}%
  \RangeCatcodeInvalid{91}{96}%
  \RangeCatcodeInvalid{123}{255}%
  \catcode`\@=12 %
  \catcode`\\=0 %
  \catcode`\%=14 %
  \LoadCommand
  \RangeCatcodeCheck{0}{36}{15}%
  \RangeCatcodeCheck{37}{37}{14}%
  \RangeCatcodeCheck{38}{47}{15}%
  \RangeCatcodeCheck{48}{57}{12}%
  \RangeCatcodeCheck{58}{63}{15}%
  \RangeCatcodeCheck{64}{64}{12}%
  \RangeCatcodeCheck{65}{90}{11}%
  \RangeCatcodeCheck{91}{91}{15}%
  \RangeCatcodeCheck{92}{92}{0}%
  \RangeCatcodeCheck{93}{96}{15}%
  \RangeCatcodeCheck{97}{122}{11}%
  \RangeCatcodeCheck{123}{255}{15}%
  \RestoreCatcodes
}
\Test
\csname @@end\endcsname
\end
%    \end{macrocode}
%    \begin{macrocode}
%</test1>
%    \end{macrocode}
%
% \section{Reload check for plain}
%
%    \begin{macrocode}
%<*test-reload1>
\input ifluatex.sty\relax
\input ifluatex.sty\relax
\csname @@end\endcsname\end
%</test-reload1>
%    \end{macrocode}
%
%    \begin{macrocode}
%<*test-reload2>
\input miniltx.tex\relax
\input ifluatex.sty\relax
\input ifluatex.sty\relax
\csname @@end\endcsname\end
%</test-reload2>
%    \end{macrocode}
%
% \section{Installation}
%
% \subsection{Download}
%
% \paragraph{Package.} This package is available on
% CTAN\footnote{\url{http://ctan.org/pkg/ifluatex}}:
% \begin{description}
% \item[\CTAN{macros/latex/contrib/oberdiek/ifluatex.dtx}] The source file.
% \item[\CTAN{macros/latex/contrib/oberdiek/ifluatex.pdf}] Documentation.
% \end{description}
%
%
% \paragraph{Bundle.} All the packages of the bundle `oberdiek'
% are also available in a TDS compliant ZIP archive. There
% the packages are already unpacked and the documentation files
% are generated. The files and directories obey the TDS standard.
% \begin{description}
% \item[\CTAN{install/macros/latex/contrib/oberdiek.tds.zip}]
% \end{description}
% \emph{TDS} refers to the standard ``A Directory Structure
% for \TeX\ Files'' (\CTAN{tds/tds.pdf}). Directories
% with \xfile{texmf} in their name are usually organized this way.
%
% \subsection{Bundle installation}
%
% \paragraph{Unpacking.} Unpack the \xfile{oberdiek.tds.zip} in the
% TDS tree (also known as \xfile{texmf} tree) of your choice.
% Example (linux):
% \begin{quote}
%   |unzip oberdiek.tds.zip -d ~/texmf|
% \end{quote}
%
% \paragraph{Script installation.}
% Check the directory \xfile{TDS:scripts/oberdiek/} for
% scripts that need further installation steps.
% Package \xpackage{attachfile2} comes with the Perl script
% \xfile{pdfatfi.pl} that should be installed in such a way
% that it can be called as \texttt{pdfatfi}.
% Example (linux):
% \begin{quote}
%   |chmod +x scripts/oberdiek/pdfatfi.pl|\\
%   |cp scripts/oberdiek/pdfatfi.pl /usr/local/bin/|
% \end{quote}
%
% \subsection{Package installation}
%
% \paragraph{Unpacking.} The \xfile{.dtx} file is a self-extracting
% \docstrip\ archive. The files are extracted by running the
% \xfile{.dtx} through \plainTeX:
% \begin{quote}
%   \verb|tex ifluatex.dtx|
% \end{quote}
%
% \paragraph{TDS.} Now the different files must be moved into
% the different directories in your installation TDS tree
% (also known as \xfile{texmf} tree):
% \begin{quote}
% \def\t{^^A
% \begin{tabular}{@{}>{\ttfamily}l@{ $\rightarrow$ }>{\ttfamily}l@{}}
%   ifluatex.sty & tex/generic/oberdiek/ifluatex.sty\\
%   ifluatex.pdf & doc/latex/oberdiek/ifluatex.pdf\\
%   test/ifluatex-test1.tex & doc/latex/oberdiek/test/ifluatex-test1.tex\\
%   test/ifluatex-test2.tex & doc/latex/oberdiek/test/ifluatex-test2.tex\\
%   test/ifluatex-test3.tex & doc/latex/oberdiek/test/ifluatex-test3.tex\\
%   ifluatex.dtx & source/latex/oberdiek/ifluatex.dtx\\
% \end{tabular}^^A
% }^^A
% \sbox0{\t}^^A
% \ifdim\wd0>\linewidth
%   \begingroup
%     \advance\linewidth by\leftmargin
%     \advance\linewidth by\rightmargin
%   \edef\x{\endgroup
%     \def\noexpand\lw{\the\linewidth}^^A
%   }\x
%   \def\lwbox{^^A
%     \leavevmode
%     \hbox to \linewidth{^^A
%       \kern-\leftmargin\relax
%       \hss
%       \usebox0
%       \hss
%       \kern-\rightmargin\relax
%     }^^A
%   }^^A
%   \ifdim\wd0>\lw
%     \sbox0{\small\t}^^A
%     \ifdim\wd0>\linewidth
%       \ifdim\wd0>\lw
%         \sbox0{\footnotesize\t}^^A
%         \ifdim\wd0>\linewidth
%           \ifdim\wd0>\lw
%             \sbox0{\scriptsize\t}^^A
%             \ifdim\wd0>\linewidth
%               \ifdim\wd0>\lw
%                 \sbox0{\tiny\t}^^A
%                 \ifdim\wd0>\linewidth
%                   \lwbox
%                 \else
%                   \usebox0
%                 \fi
%               \else
%                 \lwbox
%               \fi
%             \else
%               \usebox0
%             \fi
%           \else
%             \lwbox
%           \fi
%         \else
%           \usebox0
%         \fi
%       \else
%         \lwbox
%       \fi
%     \else
%       \usebox0
%     \fi
%   \else
%     \lwbox
%   \fi
% \else
%   \usebox0
% \fi
% \end{quote}
% If you have a \xfile{docstrip.cfg} that configures and enables \docstrip's
% TDS installing feature, then some files can already be in the right
% place, see the documentation of \docstrip.
%
% \subsection{Refresh file name databases}
%
% If your \TeX~distribution
% (\teTeX, \mikTeX, \dots) relies on file name databases, you must refresh
% these. For example, \teTeX\ users run \verb|texhash| or
% \verb|mktexlsr|.
%
% \subsection{Some details for the interested}
%
% \paragraph{Attached source.}
%
% The PDF documentation on CTAN also includes the
% \xfile{.dtx} source file. It can be extracted by
% AcrobatReader 6 or higher. Another option is \textsf{pdftk},
% e.g. unpack the file into the current directory:
% \begin{quote}
%   \verb|pdftk ifluatex.pdf unpack_files output .|
% \end{quote}
%
% \paragraph{Unpacking with \LaTeX.}
% The \xfile{.dtx} chooses its action depending on the format:
% \begin{description}
% \item[\plainTeX:] Run \docstrip\ and extract the files.
% \item[\LaTeX:] Generate the documentation.
% \end{description}
% If you insist on using \LaTeX\ for \docstrip\ (really,
% \docstrip\ does not need \LaTeX), then inform the autodetect routine
% about your intention:
% \begin{quote}
%   \verb|latex \let\install=y% \iffalse meta-comment
%
% File: ifluatex.dtx
% Version: 2016/05/16 v1.4
% Info: Provides the ifluatex switch
%
% Copyright (C) 2007, 2009, 2010 by
%    Heiko Oberdiek <heiko.oberdiek at googlemail.com>
%    2016
%    https://github.com/ho-tex/oberdiek/issues
%
% This work may be distributed and/or modified under the
% conditions of the LaTeX Project Public License, either
% version 1.3c of this license or (at your option) any later
% version. This version of this license is in
%    http://www.latex-project.org/lppl/lppl-1-3c.txt
% and the latest version of this license is in
%    http://www.latex-project.org/lppl.txt
% and version 1.3 or later is part of all distributions of
% LaTeX version 2005/12/01 or later.
%
% This work has the LPPL maintenance status "maintained".
%
% This Current Maintainer of this work is Heiko Oberdiek.
%
% The Base Interpreter refers to any `TeX-Format',
% because some files are installed in TDS:tex/generic//.
%
% This work consists of the main source file ifluatex.dtx
% and the derived files
%    ifluatex.sty, ifluatex.pdf, ifluatex.ins, ifluatex.drv,
%    ifluatex-test1.tex, ifluatex-test2.tex, ifluatex-test3.tex.
%
% Distribution:
%    CTAN:macros/latex/contrib/oberdiek/ifluatex.dtx
%    CTAN:macros/latex/contrib/oberdiek/ifluatex.pdf
%
% Unpacking:
%    (a) If ifluatex.ins is present:
%           tex ifluatex.ins
%    (b) Without ifluatex.ins:
%           tex ifluatex.dtx
%    (c) If you insist on using LaTeX
%           latex \let\install=y\input{ifluatex.dtx}
%        (quote the arguments according to the demands of your shell)
%
% Documentation:
%    (a) If ifluatex.drv is present:
%           latex ifluatex.drv
%    (b) Without ifluatex.drv:
%           latex ifluatex.dtx; ...
%    The class ltxdoc loads the configuration file ltxdoc.cfg
%    if available. Here you can specify further options, e.g.
%    use A4 as paper format:
%       \PassOptionsToClass{a4paper}{article}
%
%    Programm calls to get the documentation (example):
%       pdflatex ifluatex.dtx
%       makeindex -s gind.ist ifluatex.idx
%       pdflatex ifluatex.dtx
%       makeindex -s gind.ist ifluatex.idx
%       pdflatex ifluatex.dtx
%
% Installation:
%    TDS:tex/generic/oberdiek/ifluatex.sty
%    TDS:doc/latex/oberdiek/ifluatex.pdf
%    TDS:doc/latex/oberdiek/test/ifluatex-test1.tex
%    TDS:doc/latex/oberdiek/test/ifluatex-test2.tex
%    TDS:doc/latex/oberdiek/test/ifluatex-test3.tex
%    TDS:source/latex/oberdiek/ifluatex.dtx
%
%<*ignore>
\begingroup
  \catcode123=1 %
  \catcode125=2 %
  \def\x{LaTeX2e}%
\expandafter\endgroup
\ifcase 0\ifx\install y1\fi\expandafter
         \ifx\csname processbatchFile\endcsname\relax\else1\fi
         \ifx\fmtname\x\else 1\fi\relax
\else\csname fi\endcsname
%</ignore>
%<*install>
\input docstrip.tex
\Msg{************************************************************************}
\Msg{* Installation}
\Msg{* Package: ifluatex 2016/05/16 v1.4 Provides the ifluatex switch (HO)}
\Msg{************************************************************************}

\keepsilent
\askforoverwritefalse

\let\MetaPrefix\relax
\preamble

This is a generated file.

Project: ifluatex
Version: 2016/05/16 v1.4

Copyright (C) 2007, 2009, 2010 by
   Heiko Oberdiek <heiko.oberdiek at googlemail.com>

This work may be distributed and/or modified under the
conditions of the LaTeX Project Public License, either
version 1.3c of this license or (at your option) any later
version. This version of this license is in
   http://www.latex-project.org/lppl/lppl-1-3c.txt
and the latest version of this license is in
   http://www.latex-project.org/lppl.txt
and version 1.3 or later is part of all distributions of
LaTeX version 2005/12/01 or later.

This work has the LPPL maintenance status "maintained".

This Current Maintainer of this work is Heiko Oberdiek.

The Base Interpreter refers to any `TeX-Format',
because some files are installed in TDS:tex/generic//.

This work consists of the main source file ifluatex.dtx
and the derived files
   ifluatex.sty, ifluatex.pdf, ifluatex.ins, ifluatex.drv,
   ifluatex-test1.tex, ifluatex-test2.tex, ifluatex-test3.tex.

\endpreamble
\let\MetaPrefix\DoubleperCent

\generate{%
  \file{ifluatex.ins}{\from{ifluatex.dtx}{install}}%
  \file{ifluatex.drv}{\from{ifluatex.dtx}{driver}}%
  \usedir{tex/generic/oberdiek}%
  \file{ifluatex.sty}{\from{ifluatex.dtx}{package}}%
  \usedir{doc/latex/oberdiek/test}%
  \file{ifluatex-test1.tex}{\from{ifluatex.dtx}{test1}}%
  \file{ifluatex-test2.tex}{\from{ifluatex.dtx}{test-reload1}}%
  \file{ifluatex-test3.tex}{\from{ifluatex.dtx}{test-reload2}}%
  \nopreamble
  \nopostamble
  \usedir{source/latex/oberdiek/catalogue}%
  \file{ifluatex.xml}{\from{ifluatex.dtx}{catalogue}}%
}

\catcode32=13\relax% active space
\let =\space%
\Msg{************************************************************************}
\Msg{*}
\Msg{* To finish the installation you have to move the following}
\Msg{* file into a directory searched by TeX:}
\Msg{*}
\Msg{*     ifluatex.sty}
\Msg{*}
\Msg{* To produce the documentation run the file `ifluatex.drv'}
\Msg{* through LaTeX.}
\Msg{*}
\Msg{* Happy TeXing!}
\Msg{*}
\Msg{************************************************************************}

\endbatchfile
%</install>
%<*ignore>
\fi
%</ignore>
%<*driver>
\NeedsTeXFormat{LaTeX2e}
\ProvidesFile{ifluatex.drv}%
  [2016/05/16 v1.4 Provides the ifluatex switch (HO)]%
\documentclass{ltxdoc}
\usepackage{holtxdoc}[2011/11/22]
\begin{document}
  \DocInput{ifluatex.dtx}%
\end{document}
%</driver>
% \fi
%
%
% \CharacterTable
%  {Upper-case    \A\B\C\D\E\F\G\H\I\J\K\L\M\N\O\P\Q\R\S\T\U\V\W\X\Y\Z
%   Lower-case    \a\b\c\d\e\f\g\h\i\j\k\l\m\n\o\p\q\r\s\t\u\v\w\x\y\z
%   Digits        \0\1\2\3\4\5\6\7\8\9
%   Exclamation   \!     Double quote  \"     Hash (number) \#
%   Dollar        \$     Percent       \%     Ampersand     \&
%   Acute accent  \'     Left paren    \(     Right paren   \)
%   Asterisk      \*     Plus          \+     Comma         \,
%   Minus         \-     Point         \.     Solidus       \/
%   Colon         \:     Semicolon     \;     Less than     \<
%   Equals        \=     Greater than  \>     Question mark \?
%   Commercial at \@     Left bracket  \[     Backslash     \\
%   Right bracket \]     Circumflex    \^     Underscore    \_
%   Grave accent  \`     Left brace    \{     Vertical bar  \|
%   Right brace   \}     Tilde         \~}
%
% \GetFileInfo{ifluatex.drv}
%
% \title{The \xpackage{ifluatex} package}
% \date{2016/05/16 v1.4}
% \author{Heiko Oberdiek\thanks
% {Please report any issues at https://github.com/ho-tex/oberdiek/issues}\\
% \xemail{heiko.oberdiek at googlemail.com}}
%
% \maketitle
%
% \begin{abstract}
% This package looks for \LuaTeX\ regardless of its mode
% and provides the switch \cs{ifluatex}. Also it makes
% \cs{luatexversion} available if it is not present.
% It works with \plainTeX\ or \LaTeX.
% \end{abstract}
%
% \tableofcontents
%
% \section{Documentation}
%
% The package \xpackage{ifluatex} can be used with both \plainTeX\
% and \LaTeX:
% \begin{description}
% \item[\plainTeX:] |\input ifluatex.sty|
% \item[\LaTeXe:]   |\usepackage{ifluatex}|
% \end{description}
%
% \DescribeMacro{\ifluatex}
% The package provides the switch \cs{ifluatex}:
% \begin{quote}
%   |\ifluatex|\\
%   \hspace{1.5em}\LuaTeX\ is running\\
%   |\else|\\
%   \hspace{1.5em}Without \LuaTeX\\
%   |\fi|
% \end{quote}
%
% Since version 0.39 \LuaTeX\ only provides \cs{directlua} at startup
% time. Also the syntax of \cs{directlua} changed in version 0.36.
% Thus the user might want to check the LuaTeX version.
% Therefore this package also makes \cs{luatexversion} and
% \cs{luatexrevision} available, if it is not yet done.
%
% If you want to detect the mode (DVI or PDF), then use package
% \xpackage{ifpdf}. \LuaTeX\ has inherited \cs{pdfoutput} from \pdfTeX.
%
% \StopEventually{
% }
%
% \section{Implementation}
%
%    \begin{macrocode}
%<*package>
%    \end{macrocode}
%
% \subsection{Reload check and package identification}
%    Reload check, especially if the package is not used with \LaTeX.
%    \begin{macrocode}
\begingroup\catcode61\catcode48\catcode32=10\relax%
  \catcode13=5 % ^^M
  \endlinechar=13 %
  \catcode35=6 % #
  \catcode39=12 % '
  \catcode44=12 % ,
  \catcode45=12 % -
  \catcode46=12 % .
  \catcode58=12 % :
  \catcode64=11 % @
  \catcode123=1 % {
  \catcode125=2 % }
  \expandafter\let\expandafter\x\csname ver@ifluatex.sty\endcsname
  \ifx\x\relax % plain-TeX, first loading
  \else
    \def\empty{}%
    \ifx\x\empty % LaTeX, first loading,
      % variable is initialized, but \ProvidesPackage not yet seen
    \else
      \expandafter\ifx\csname PackageInfo\endcsname\relax
        \def\x#1#2{%
          \immediate\write-1{Package #1 Info: #2.}%
        }%
      \else
        \def\x#1#2{\PackageInfo{#1}{#2, stopped}}%
      \fi
      \x{ifluatex}{The package is already loaded}%
      \aftergroup\endinput
    \fi
  \fi
\endgroup%
%    \end{macrocode}
%    Package identification:
%    \begin{macrocode}
\begingroup\catcode61\catcode48\catcode32=10\relax%
  \catcode13=5 % ^^M
  \endlinechar=13 %
  \catcode35=6 % #
  \catcode39=12 % '
  \catcode40=12 % (
  \catcode41=12 % )
  \catcode44=12 % ,
  \catcode45=12 % -
  \catcode46=12 % .
  \catcode47=12 % /
  \catcode58=12 % :
  \catcode64=11 % @
  \catcode91=12 % [
  \catcode93=12 % ]
  \catcode123=1 % {
  \catcode125=2 % }
  \expandafter\ifx\csname ProvidesPackage\endcsname\relax
    \def\x#1#2#3[#4]{\endgroup
      \immediate\write-1{Package: #3 #4}%
      \xdef#1{#4}%
    }%
  \else
    \def\x#1#2[#3]{\endgroup
      #2[{#3}]%
      \ifx#1\@undefined
        \xdef#1{#3}%
      \fi
      \ifx#1\relax
        \xdef#1{#3}%
      \fi
    }%
  \fi
\expandafter\x\csname ver@ifluatex.sty\endcsname
\ProvidesPackage{ifluatex}%
  [2016/05/16 v1.4 Provides the ifluatex switch (HO)]%
%    \end{macrocode}
%
% \subsection{Catcodes}
%
%    \begin{macrocode}
\begingroup\catcode61\catcode48\catcode32=10\relax%
  \catcode13=5 % ^^M
  \endlinechar=13 %
  \catcode123=1 % {
  \catcode125=2 % }
  \catcode64=11 % @
  \def\x{\endgroup
    \expandafter\edef\csname ifluatex@AtEnd\endcsname{%
      \endlinechar=\the\endlinechar\relax
      \catcode13=\the\catcode13\relax
      \catcode32=\the\catcode32\relax
      \catcode35=\the\catcode35\relax
      \catcode61=\the\catcode61\relax
      \catcode64=\the\catcode64\relax
      \catcode123=\the\catcode123\relax
      \catcode125=\the\catcode125\relax
    }%
  }%
\x\catcode61\catcode48\catcode32=10\relax%
\catcode13=5 % ^^M
\endlinechar=13 %
\catcode35=6 % #
\catcode64=11 % @
\catcode123=1 % {
\catcode125=2 % }
\def\TMP@EnsureCode#1#2{%
  \edef\ifluatex@AtEnd{%
    \ifluatex@AtEnd
    \catcode#1=\the\catcode#1\relax
  }%
  \catcode#1=#2\relax
}
\TMP@EnsureCode{10}{12}% ^^J
\TMP@EnsureCode{39}{12}% '
\TMP@EnsureCode{40}{12}% (
\TMP@EnsureCode{41}{12}% )
\TMP@EnsureCode{44}{12}% ,
\TMP@EnsureCode{45}{12}% -
\TMP@EnsureCode{46}{12}% .
\TMP@EnsureCode{47}{12}% /
\TMP@EnsureCode{58}{12}% :
\TMP@EnsureCode{60}{12}% <
\TMP@EnsureCode{94}{7}% ^
\TMP@EnsureCode{96}{12}% `
\edef\ifluatex@AtEnd{\ifluatex@AtEnd\noexpand\endinput}
%    \end{macrocode}
%
% \subsection{Macro for error messages}
%
%    \begin{macro}{\ifluatex@Error}
%    \begin{macrocode}
\begingroup\expandafter\expandafter\expandafter\endgroup
\expandafter\ifx\csname PackageError\endcsname\relax
  \def\ifluatex@Error#1#2{%
    \begingroup
      \newlinechar=10 %
      \def\MessageBreak{^^J}%
      \edef\x{\errhelp{#2}}%
      \x
      \errmessage{Package ifluatex Error: #1}%
    \endgroup
  }%
\else
  \def\ifluatex@Error{%
    \PackageError{ifluatex}%
  }%
\fi
%    \end{macrocode}
%    \end{macro}
%
% \subsection{Check for previously defined \cs{ifluatex}}
%
%    \begin{macrocode}
\begingroup
  \expandafter\ifx\csname ifluatex\endcsname\relax
  \else
    \edef\i/{\expandafter\string\csname ifluatex\endcsname}%
    \ifluatex@Error{Name clash, \i/ is already defined}{%
      Incompatible versions of \i/ can cause problems,\MessageBreak
      therefore package loading is aborted.%
    }%
    \endgroup
    \expandafter\ifluatex@AtEnd
  \fi%
\endgroup
%    \end{macrocode}
%
% \subsection{\cs{ifluatex}}
%
%    \begin{macro}{\ifluatex}
%    \begin{macrocode}
\let\ifluatex\iffalse
%    \end{macrocode}
%    \end{macro}
%
%    Test \cs{luatexversion}. Is it  defined and different from
%    \cs{relax}? Someone could have used \LaTeX\ internal
%    \cs{@ifundefined}, or something else involving.
%    Notice, \cs{csname} is executed inside a group for the test
%    to cancel the side effect of \cs{csname}.
%    \begin{macrocode}
\begingroup\expandafter\expandafter\expandafter\endgroup
\expandafter\ifx\csname luatexversion\endcsname\relax
\else
  \expandafter\let\csname ifluatex\expandafter\endcsname
                  \csname iftrue\endcsname
\fi
%    \end{macrocode}
%
% \subsection{Lua\TeX\ v0.39}
%
%     Starting with version 0.39 \LuaTeX\ wants to provide \cs{directlua}
%     as only primitive at startup time beyond vanilla \TeX's primitives.
%     Then \cs{directlua} exists, but \cs{luatexversion} cannot be found.
%     Unhappily also the syntax of \cs{directlua} changed in v0.36,
%     thus the user would want to check \cs{luatexversion}.
%     Therefore we make \cs{luatexversion} available using
%     \LuaTeX's Lua function |tex.enableprimitives|.
%
%    \begin{macrocode}
\ifluatex
\else
  \begingroup\expandafter\expandafter\expandafter\endgroup
  \expandafter\ifx\csname directlua\endcsname\relax
  \else
    \expandafter\let\csname ifluatex\expandafter\endcsname
                    \csname iftrue\endcsname
    \begingroup
      \newlinechar=10 %
      \endlinechar=\newlinechar%
      \ifnum0%
          \directlua{%
            if tex.enableprimitives then
              tex.enableprimitives('ifluatex', {'luatexversion'})
              tex.print('1')
            end
          }%
          \ifx\ifluatexluatexversion\@undefined\else 1\fi %
          =11 %
        \global\let\luatexversion\ifluatexluatexversion%
      \else%
        \ifluatex@Error{%
          Missing \string\luatexversion%
        }{%
          Update LuaTeX.%
        }%
      \fi%
    \endgroup%
  \fi
\fi
%    \end{macrocode}
%    \begin{macrocode}
\ifluatex
  \begingroup\expandafter\expandafter\expandafter\endgroup
  \expandafter\ifx\csname luatexrevision\endcsname\relax
    \ifnum\luatexversion<36 %
    \else
      \begingroup
        \ifx\luatexrevision\relax
          \let\luatexrevision\@undefined
        \fi
        \newlinechar=10 %
        \endlinechar=\newlinechar%
        \ifcase0%
            \directlua{%
              if tex.enableprimitives then
                tex.enableprimitives('ifluatex', {'luatexrevision'})
              else
                tex.print('1')
              end
            }%
            \ifx\ifluatexluatexrevision\@undefined 1\fi%
            \relax%
          \global\let\luatexrevision\ifluatexluatexrevision%
        \fi%
      \endgroup%
    \fi
    \begingroup\expandafter\expandafter\expandafter\endgroup
    \expandafter\ifx\csname luatexrevision\endcsname\relax
      \ifluatex@Error{%
        Missing \string\luatexrevision%
      }{%
        Update LuaTeX.%
      }%
    \fi
  \fi
\fi
%    \end{macrocode}
%
% \subsection{Protocol entry}
%
%     Log comment:
%    \begin{macrocode}
\begingroup
  \expandafter\ifx\csname PackageInfo\endcsname\relax
    \def\x#1#2{%
      \immediate\write-1{Package #1 Info: #2.}%
    }%
  \else
    \let\x\PackageInfo
    \expandafter\let\csname on@line\endcsname\empty
  \fi
  \x{ifluatex}{LuaTeX \ifluatex\else not \fi detected}%
\endgroup
%    \end{macrocode}
%    \begin{macrocode}
\ifluatex@AtEnd%
%    \end{macrocode}
%    \begin{macrocode}
%</package>
%    \end{macrocode}
%
% \section{Test}
%
% \subsection{Catcode checks for loading}
%
%    \begin{macrocode}
%<*test1>
%    \end{macrocode}
%    \begin{macrocode}
\catcode`\{=1 %
\catcode`\}=2 %
\catcode`\#=6 %
\catcode`\@=11 %
\expandafter\ifx\csname count@\endcsname\relax
  \countdef\count@=255 %
\fi
\expandafter\ifx\csname @gobble\endcsname\relax
  \long\def\@gobble#1{}%
\fi
\expandafter\ifx\csname @firstofone\endcsname\relax
  \long\def\@firstofone#1{#1}%
\fi
\expandafter\ifx\csname loop\endcsname\relax
  \expandafter\@firstofone
\else
  \expandafter\@gobble
\fi
{%
  \def\loop#1\repeat{%
    \def\body{#1}%
    \iterate
  }%
  \def\iterate{%
    \body
      \let\next\iterate
    \else
      \let\next\relax
    \fi
    \next
  }%
  \let\repeat=\fi
}%
\def\RestoreCatcodes{}
\count@=0 %
\loop
  \edef\RestoreCatcodes{%
    \RestoreCatcodes
    \catcode\the\count@=\the\catcode\count@\relax
  }%
\ifnum\count@<255 %
  \advance\count@ 1 %
\repeat

\def\RangeCatcodeInvalid#1#2{%
  \count@=#1\relax
  \loop
    \catcode\count@=15 %
  \ifnum\count@<#2\relax
    \advance\count@ 1 %
  \repeat
}
\def\RangeCatcodeCheck#1#2#3{%
  \count@=#1\relax
  \loop
    \ifnum#3=\catcode\count@
    \else
      \errmessage{%
        Character \the\count@\space
        with wrong catcode \the\catcode\count@\space
        instead of \number#3%
      }%
    \fi
  \ifnum\count@<#2\relax
    \advance\count@ 1 %
  \repeat
}
\def\space{ }
\expandafter\ifx\csname LoadCommand\endcsname\relax
  \def\LoadCommand{\input ifluatex.sty\relax}%
\fi
\def\Test{%
  \RangeCatcodeInvalid{0}{47}%
  \RangeCatcodeInvalid{58}{64}%
  \RangeCatcodeInvalid{91}{96}%
  \RangeCatcodeInvalid{123}{255}%
  \catcode`\@=12 %
  \catcode`\\=0 %
  \catcode`\%=14 %
  \LoadCommand
  \RangeCatcodeCheck{0}{36}{15}%
  \RangeCatcodeCheck{37}{37}{14}%
  \RangeCatcodeCheck{38}{47}{15}%
  \RangeCatcodeCheck{48}{57}{12}%
  \RangeCatcodeCheck{58}{63}{15}%
  \RangeCatcodeCheck{64}{64}{12}%
  \RangeCatcodeCheck{65}{90}{11}%
  \RangeCatcodeCheck{91}{91}{15}%
  \RangeCatcodeCheck{92}{92}{0}%
  \RangeCatcodeCheck{93}{96}{15}%
  \RangeCatcodeCheck{97}{122}{11}%
  \RangeCatcodeCheck{123}{255}{15}%
  \RestoreCatcodes
}
\Test
\csname @@end\endcsname
\end
%    \end{macrocode}
%    \begin{macrocode}
%</test1>
%    \end{macrocode}
%
% \section{Reload check for plain}
%
%    \begin{macrocode}
%<*test-reload1>
\input ifluatex.sty\relax
\input ifluatex.sty\relax
\csname @@end\endcsname\end
%</test-reload1>
%    \end{macrocode}
%
%    \begin{macrocode}
%<*test-reload2>
\input miniltx.tex\relax
\input ifluatex.sty\relax
\input ifluatex.sty\relax
\csname @@end\endcsname\end
%</test-reload2>
%    \end{macrocode}
%
% \section{Installation}
%
% \subsection{Download}
%
% \paragraph{Package.} This package is available on
% CTAN\footnote{\url{http://ctan.org/pkg/ifluatex}}:
% \begin{description}
% \item[\CTAN{macros/latex/contrib/oberdiek/ifluatex.dtx}] The source file.
% \item[\CTAN{macros/latex/contrib/oberdiek/ifluatex.pdf}] Documentation.
% \end{description}
%
%
% \paragraph{Bundle.} All the packages of the bundle `oberdiek'
% are also available in a TDS compliant ZIP archive. There
% the packages are already unpacked and the documentation files
% are generated. The files and directories obey the TDS standard.
% \begin{description}
% \item[\CTAN{install/macros/latex/contrib/oberdiek.tds.zip}]
% \end{description}
% \emph{TDS} refers to the standard ``A Directory Structure
% for \TeX\ Files'' (\CTAN{tds/tds.pdf}). Directories
% with \xfile{texmf} in their name are usually organized this way.
%
% \subsection{Bundle installation}
%
% \paragraph{Unpacking.} Unpack the \xfile{oberdiek.tds.zip} in the
% TDS tree (also known as \xfile{texmf} tree) of your choice.
% Example (linux):
% \begin{quote}
%   |unzip oberdiek.tds.zip -d ~/texmf|
% \end{quote}
%
% \paragraph{Script installation.}
% Check the directory \xfile{TDS:scripts/oberdiek/} for
% scripts that need further installation steps.
% Package \xpackage{attachfile2} comes with the Perl script
% \xfile{pdfatfi.pl} that should be installed in such a way
% that it can be called as \texttt{pdfatfi}.
% Example (linux):
% \begin{quote}
%   |chmod +x scripts/oberdiek/pdfatfi.pl|\\
%   |cp scripts/oberdiek/pdfatfi.pl /usr/local/bin/|
% \end{quote}
%
% \subsection{Package installation}
%
% \paragraph{Unpacking.} The \xfile{.dtx} file is a self-extracting
% \docstrip\ archive. The files are extracted by running the
% \xfile{.dtx} through \plainTeX:
% \begin{quote}
%   \verb|tex ifluatex.dtx|
% \end{quote}
%
% \paragraph{TDS.} Now the different files must be moved into
% the different directories in your installation TDS tree
% (also known as \xfile{texmf} tree):
% \begin{quote}
% \def\t{^^A
% \begin{tabular}{@{}>{\ttfamily}l@{ $\rightarrow$ }>{\ttfamily}l@{}}
%   ifluatex.sty & tex/generic/oberdiek/ifluatex.sty\\
%   ifluatex.pdf & doc/latex/oberdiek/ifluatex.pdf\\
%   test/ifluatex-test1.tex & doc/latex/oberdiek/test/ifluatex-test1.tex\\
%   test/ifluatex-test2.tex & doc/latex/oberdiek/test/ifluatex-test2.tex\\
%   test/ifluatex-test3.tex & doc/latex/oberdiek/test/ifluatex-test3.tex\\
%   ifluatex.dtx & source/latex/oberdiek/ifluatex.dtx\\
% \end{tabular}^^A
% }^^A
% \sbox0{\t}^^A
% \ifdim\wd0>\linewidth
%   \begingroup
%     \advance\linewidth by\leftmargin
%     \advance\linewidth by\rightmargin
%   \edef\x{\endgroup
%     \def\noexpand\lw{\the\linewidth}^^A
%   }\x
%   \def\lwbox{^^A
%     \leavevmode
%     \hbox to \linewidth{^^A
%       \kern-\leftmargin\relax
%       \hss
%       \usebox0
%       \hss
%       \kern-\rightmargin\relax
%     }^^A
%   }^^A
%   \ifdim\wd0>\lw
%     \sbox0{\small\t}^^A
%     \ifdim\wd0>\linewidth
%       \ifdim\wd0>\lw
%         \sbox0{\footnotesize\t}^^A
%         \ifdim\wd0>\linewidth
%           \ifdim\wd0>\lw
%             \sbox0{\scriptsize\t}^^A
%             \ifdim\wd0>\linewidth
%               \ifdim\wd0>\lw
%                 \sbox0{\tiny\t}^^A
%                 \ifdim\wd0>\linewidth
%                   \lwbox
%                 \else
%                   \usebox0
%                 \fi
%               \else
%                 \lwbox
%               \fi
%             \else
%               \usebox0
%             \fi
%           \else
%             \lwbox
%           \fi
%         \else
%           \usebox0
%         \fi
%       \else
%         \lwbox
%       \fi
%     \else
%       \usebox0
%     \fi
%   \else
%     \lwbox
%   \fi
% \else
%   \usebox0
% \fi
% \end{quote}
% If you have a \xfile{docstrip.cfg} that configures and enables \docstrip's
% TDS installing feature, then some files can already be in the right
% place, see the documentation of \docstrip.
%
% \subsection{Refresh file name databases}
%
% If your \TeX~distribution
% (\teTeX, \mikTeX, \dots) relies on file name databases, you must refresh
% these. For example, \teTeX\ users run \verb|texhash| or
% \verb|mktexlsr|.
%
% \subsection{Some details for the interested}
%
% \paragraph{Attached source.}
%
% The PDF documentation on CTAN also includes the
% \xfile{.dtx} source file. It can be extracted by
% AcrobatReader 6 or higher. Another option is \textsf{pdftk},
% e.g. unpack the file into the current directory:
% \begin{quote}
%   \verb|pdftk ifluatex.pdf unpack_files output .|
% \end{quote}
%
% \paragraph{Unpacking with \LaTeX.}
% The \xfile{.dtx} chooses its action depending on the format:
% \begin{description}
% \item[\plainTeX:] Run \docstrip\ and extract the files.
% \item[\LaTeX:] Generate the documentation.
% \end{description}
% If you insist on using \LaTeX\ for \docstrip\ (really,
% \docstrip\ does not need \LaTeX), then inform the autodetect routine
% about your intention:
% \begin{quote}
%   \verb|latex \let\install=y\input{ifluatex.dtx}|
% \end{quote}
% Do not forget to quote the argument according to the demands
% of your shell.
%
% \paragraph{Generating the documentation.}
% You can use both the \xfile{.dtx} or the \xfile{.drv} to generate
% the documentation. The process can be configured by the
% configuration file \xfile{ltxdoc.cfg}. For instance, put this
% line into this file, if you want to have A4 as paper format:
% \begin{quote}
%   \verb|\PassOptionsToClass{a4paper}{article}|
% \end{quote}
% An example follows how to generate the
% documentation with pdf\LaTeX:
% \begin{quote}
%\begin{verbatim}
%pdflatex ifluatex.dtx
%makeindex -s gind.ist ifluatex.idx
%pdflatex ifluatex.dtx
%makeindex -s gind.ist ifluatex.idx
%pdflatex ifluatex.dtx
%\end{verbatim}
% \end{quote}
%
% \section{Catalogue}
%
% The following XML file can be used as source for the
% \href{http://mirror.ctan.org/help/Catalogue/catalogue.html}{\TeX\ Catalogue}.
% The elements \texttt{caption} and \texttt{description} are imported
% from the original XML file from the Catalogue.
% The name of the XML file in the Catalogue is \xfile{ifluatex.xml}.
%    \begin{macrocode}
%<*catalogue>
<?xml version='1.0' encoding='us-ascii'?>
<!DOCTYPE entry SYSTEM 'catalogue.dtd'>
<entry datestamp='$Date$' modifier='$Author$' id='ifluatex'>
  <name>ifluatex</name>
  <caption>Provides the \ifluatex switch.</caption>
  <authorref id='auth:oberdiek'/>
  <copyright owner='Heiko Oberdiek' year='2007,2009,2010'/>
  <license type='lppl1.3'/>
  <version number='1.4'/>
  <description>
    The package looks for  LuaTeX regardless of its mode and provides
    the switch <tt>\ifluatex</tt>; it works with Plain TeX or LaTeX.
    <p/>
    The package is part of the <xref refid='oberdiek'>oberdiek</xref>
    bundle.
  </description>
  <documentation details='Package documentation'
      href='ctan:/macros/latex/contrib/oberdiek/ifluatex.pdf'/>
  <ctan file='true' path='/macros/latex/contrib/oberdiek/ifluatex.dtx'/>
  <miktex location='oberdiek'/>
  <texlive location='ifluatex'/>
  <install path='/macros/latex/contrib/oberdiek/oberdiek.tds.zip'/>
</entry>
%</catalogue>
%    \end{macrocode}
%
% \begin{History}
%   \begin{Version}{2007/12/12 v1.0}
%   \item
%     First public version.
%   \end{Version}
%   \begin{Version}{2009/04/10 v1.1}
%   \item
%     Test adopted for \LuaTeX\ 0.39.
%   \item
%     Makes \cs{luatexversion} available.
%   \end{Version}
%   \begin{Version}{2009/04/17 v1.2}
%   \item
%     Fixes (Manuel P\'egouri\'e-Gonnard).
%   \item
%     \cs{luatextrue} and \cs{luatexfalse} are no longer defined.
%   \item
%     Makes \cs{luatexrevision} available, too.
%   \end{Version}
%   \begin{Version}{2010/03/01 v1.3}
%   \item
%     Line ends fixed in case \cs{endlinechar} = \cs{newlinechar}.
%   \end{Version}
%   \begin{Version}{2016/05/16 v1.4}
%   \item
%     Documentation updates.
%   \end{Version}
% \end{History}
%
% \PrintIndex
%
% \Finale
\endinput
|
% \end{quote}
% Do not forget to quote the argument according to the demands
% of your shell.
%
% \paragraph{Generating the documentation.}
% You can use both the \xfile{.dtx} or the \xfile{.drv} to generate
% the documentation. The process can be configured by the
% configuration file \xfile{ltxdoc.cfg}. For instance, put this
% line into this file, if you want to have A4 as paper format:
% \begin{quote}
%   \verb|\PassOptionsToClass{a4paper}{article}|
% \end{quote}
% An example follows how to generate the
% documentation with pdf\LaTeX:
% \begin{quote}
%\begin{verbatim}
%pdflatex ifluatex.dtx
%makeindex -s gind.ist ifluatex.idx
%pdflatex ifluatex.dtx
%makeindex -s gind.ist ifluatex.idx
%pdflatex ifluatex.dtx
%\end{verbatim}
% \end{quote}
%
% \section{Catalogue}
%
% The following XML file can be used as source for the
% \href{http://mirror.ctan.org/help/Catalogue/catalogue.html}{\TeX\ Catalogue}.
% The elements \texttt{caption} and \texttt{description} are imported
% from the original XML file from the Catalogue.
% The name of the XML file in the Catalogue is \xfile{ifluatex.xml}.
%    \begin{macrocode}
%<*catalogue>
<?xml version='1.0' encoding='us-ascii'?>
<!DOCTYPE entry SYSTEM 'catalogue.dtd'>
<entry datestamp='$Date$' modifier='$Author$' id='ifluatex'>
  <name>ifluatex</name>
  <caption>Provides the \ifluatex switch.</caption>
  <authorref id='auth:oberdiek'/>
  <copyright owner='Heiko Oberdiek' year='2007,2009,2010'/>
  <license type='lppl1.3'/>
  <version number='1.4'/>
  <description>
    The package looks for  LuaTeX regardless of its mode and provides
    the switch <tt>\ifluatex</tt>; it works with Plain TeX or LaTeX.
    <p/>
    The package is part of the <xref refid='oberdiek'>oberdiek</xref>
    bundle.
  </description>
  <documentation details='Package documentation'
      href='ctan:/macros/latex/contrib/oberdiek/ifluatex.pdf'/>
  <ctan file='true' path='/macros/latex/contrib/oberdiek/ifluatex.dtx'/>
  <miktex location='oberdiek'/>
  <texlive location='ifluatex'/>
  <install path='/macros/latex/contrib/oberdiek/oberdiek.tds.zip'/>
</entry>
%</catalogue>
%    \end{macrocode}
%
% \begin{History}
%   \begin{Version}{2007/12/12 v1.0}
%   \item
%     First public version.
%   \end{Version}
%   \begin{Version}{2009/04/10 v1.1}
%   \item
%     Test adopted for \LuaTeX\ 0.39.
%   \item
%     Makes \cs{luatexversion} available.
%   \end{Version}
%   \begin{Version}{2009/04/17 v1.2}
%   \item
%     Fixes (Manuel P\'egouri\'e-Gonnard).
%   \item
%     \cs{luatextrue} and \cs{luatexfalse} are no longer defined.
%   \item
%     Makes \cs{luatexrevision} available, too.
%   \end{Version}
%   \begin{Version}{2010/03/01 v1.3}
%   \item
%     Line ends fixed in case \cs{endlinechar} = \cs{newlinechar}.
%   \end{Version}
%   \begin{Version}{2016/05/16 v1.4}
%   \item
%     Documentation updates.
%   \end{Version}
% \end{History}
%
% \PrintIndex
%
% \Finale
\endinput
|
% \end{quote}
% Do not forget to quote the argument according to the demands
% of your shell.
%
% \paragraph{Generating the documentation.}
% You can use both the \xfile{.dtx} or the \xfile{.drv} to generate
% the documentation. The process can be configured by the
% configuration file \xfile{ltxdoc.cfg}. For instance, put this
% line into this file, if you want to have A4 as paper format:
% \begin{quote}
%   \verb|\PassOptionsToClass{a4paper}{article}|
% \end{quote}
% An example follows how to generate the
% documentation with pdf\LaTeX:
% \begin{quote}
%\begin{verbatim}
%pdflatex ifluatex.dtx
%makeindex -s gind.ist ifluatex.idx
%pdflatex ifluatex.dtx
%makeindex -s gind.ist ifluatex.idx
%pdflatex ifluatex.dtx
%\end{verbatim}
% \end{quote}
%
% \section{Catalogue}
%
% The following XML file can be used as source for the
% \href{http://mirror.ctan.org/help/Catalogue/catalogue.html}{\TeX\ Catalogue}.
% The elements \texttt{caption} and \texttt{description} are imported
% from the original XML file from the Catalogue.
% The name of the XML file in the Catalogue is \xfile{ifluatex.xml}.
%    \begin{macrocode}
%<*catalogue>
<?xml version='1.0' encoding='us-ascii'?>
<!DOCTYPE entry SYSTEM 'catalogue.dtd'>
<entry datestamp='$Date$' modifier='$Author$' id='ifluatex'>
  <name>ifluatex</name>
  <caption>Provides the \ifluatex switch.</caption>
  <authorref id='auth:oberdiek'/>
  <copyright owner='Heiko Oberdiek' year='2007,2009,2010'/>
  <license type='lppl1.3'/>
  <version number='1.4'/>
  <description>
    The package looks for  LuaTeX regardless of its mode and provides
    the switch <tt>\ifluatex</tt>; it works with Plain TeX or LaTeX.
    <p/>
    The package is part of the <xref refid='oberdiek'>oberdiek</xref>
    bundle.
  </description>
  <documentation details='Package documentation'
      href='ctan:/macros/latex/contrib/oberdiek/ifluatex.pdf'/>
  <ctan file='true' path='/macros/latex/contrib/oberdiek/ifluatex.dtx'/>
  <miktex location='oberdiek'/>
  <texlive location='ifluatex'/>
  <install path='/macros/latex/contrib/oberdiek/oberdiek.tds.zip'/>
</entry>
%</catalogue>
%    \end{macrocode}
%
% \begin{History}
%   \begin{Version}{2007/12/12 v1.0}
%   \item
%     First public version.
%   \end{Version}
%   \begin{Version}{2009/04/10 v1.1}
%   \item
%     Test adopted for \LuaTeX\ 0.39.
%   \item
%     Makes \cs{luatexversion} available.
%   \end{Version}
%   \begin{Version}{2009/04/17 v1.2}
%   \item
%     Fixes (Manuel P\'egouri\'e-Gonnard).
%   \item
%     \cs{luatextrue} and \cs{luatexfalse} are no longer defined.
%   \item
%     Makes \cs{luatexrevision} available, too.
%   \end{Version}
%   \begin{Version}{2010/03/01 v1.3}
%   \item
%     Line ends fixed in case \cs{endlinechar} = \cs{newlinechar}.
%   \end{Version}
%   \begin{Version}{2016/05/16 v1.4}
%   \item
%     Documentation updates.
%   \end{Version}
% \end{History}
%
% \PrintIndex
%
% \Finale
\endinput

%        (quote the arguments according to the demands of your shell)
%
% Documentation:
%    (a) If ifluatex.drv is present:
%           latex ifluatex.drv
%    (b) Without ifluatex.drv:
%           latex ifluatex.dtx; ...
%    The class ltxdoc loads the configuration file ltxdoc.cfg
%    if available. Here you can specify further options, e.g.
%    use A4 as paper format:
%       \PassOptionsToClass{a4paper}{article}
%
%    Programm calls to get the documentation (example):
%       pdflatex ifluatex.dtx
%       makeindex -s gind.ist ifluatex.idx
%       pdflatex ifluatex.dtx
%       makeindex -s gind.ist ifluatex.idx
%       pdflatex ifluatex.dtx
%
% Installation:
%    TDS:tex/generic/oberdiek/ifluatex.sty
%    TDS:doc/latex/oberdiek/ifluatex.pdf
%    TDS:doc/latex/oberdiek/test/ifluatex-test1.tex
%    TDS:doc/latex/oberdiek/test/ifluatex-test2.tex
%    TDS:doc/latex/oberdiek/test/ifluatex-test3.tex
%    TDS:source/latex/oberdiek/ifluatex.dtx
%
%<*ignore>
\begingroup
  \catcode123=1 %
  \catcode125=2 %
  \def\x{LaTeX2e}%
\expandafter\endgroup
\ifcase 0\ifx\install y1\fi\expandafter
         \ifx\csname processbatchFile\endcsname\relax\else1\fi
         \ifx\fmtname\x\else 1\fi\relax
\else\csname fi\endcsname
%</ignore>
%<*install>
\input docstrip.tex
\Msg{************************************************************************}
\Msg{* Installation}
\Msg{* Package: ifluatex 2016/05/16 v1.4 Provides the ifluatex switch (HO)}
\Msg{************************************************************************}

\keepsilent
\askforoverwritefalse

\let\MetaPrefix\relax
\preamble

This is a generated file.

Project: ifluatex
Version: 2016/05/16 v1.4

Copyright (C) 2007, 2009, 2010 by
   Heiko Oberdiek <heiko.oberdiek at googlemail.com>

This work may be distributed and/or modified under the
conditions of the LaTeX Project Public License, either
version 1.3c of this license or (at your option) any later
version. This version of this license is in
   http://www.latex-project.org/lppl/lppl-1-3c.txt
and the latest version of this license is in
   http://www.latex-project.org/lppl.txt
and version 1.3 or later is part of all distributions of
LaTeX version 2005/12/01 or later.

This work has the LPPL maintenance status "maintained".

This Current Maintainer of this work is Heiko Oberdiek.

The Base Interpreter refers to any `TeX-Format',
because some files are installed in TDS:tex/generic//.

This work consists of the main source file ifluatex.dtx
and the derived files
   ifluatex.sty, ifluatex.pdf, ifluatex.ins, ifluatex.drv,
   ifluatex-test1.tex, ifluatex-test2.tex, ifluatex-test3.tex.

\endpreamble
\let\MetaPrefix\DoubleperCent

\generate{%
  \file{ifluatex.ins}{\from{ifluatex.dtx}{install}}%
  \file{ifluatex.drv}{\from{ifluatex.dtx}{driver}}%
  \usedir{tex/generic/oberdiek}%
  \file{ifluatex.sty}{\from{ifluatex.dtx}{package}}%
  \usedir{doc/latex/oberdiek/test}%
  \file{ifluatex-test1.tex}{\from{ifluatex.dtx}{test1}}%
  \file{ifluatex-test2.tex}{\from{ifluatex.dtx}{test-reload1}}%
  \file{ifluatex-test3.tex}{\from{ifluatex.dtx}{test-reload2}}%
  \nopreamble
  \nopostamble
  \usedir{source/latex/oberdiek/catalogue}%
  \file{ifluatex.xml}{\from{ifluatex.dtx}{catalogue}}%
}

\catcode32=13\relax% active space
\let =\space%
\Msg{************************************************************************}
\Msg{*}
\Msg{* To finish the installation you have to move the following}
\Msg{* file into a directory searched by TeX:}
\Msg{*}
\Msg{*     ifluatex.sty}
\Msg{*}
\Msg{* To produce the documentation run the file `ifluatex.drv'}
\Msg{* through LaTeX.}
\Msg{*}
\Msg{* Happy TeXing!}
\Msg{*}
\Msg{************************************************************************}

\endbatchfile
%</install>
%<*ignore>
\fi
%</ignore>
%<*driver>
\NeedsTeXFormat{LaTeX2e}
\ProvidesFile{ifluatex.drv}%
  [2016/05/16 v1.4 Provides the ifluatex switch (HO)]%
\documentclass{ltxdoc}
\usepackage{holtxdoc}[2011/11/22]
\begin{document}
  \DocInput{ifluatex.dtx}%
\end{document}
%</driver>
% \fi
%
%
% \CharacterTable
%  {Upper-case    \A\B\C\D\E\F\G\H\I\J\K\L\M\N\O\P\Q\R\S\T\U\V\W\X\Y\Z
%   Lower-case    \a\b\c\d\e\f\g\h\i\j\k\l\m\n\o\p\q\r\s\t\u\v\w\x\y\z
%   Digits        \0\1\2\3\4\5\6\7\8\9
%   Exclamation   \!     Double quote  \"     Hash (number) \#
%   Dollar        \$     Percent       \%     Ampersand     \&
%   Acute accent  \'     Left paren    \(     Right paren   \)
%   Asterisk      \*     Plus          \+     Comma         \,
%   Minus         \-     Point         \.     Solidus       \/
%   Colon         \:     Semicolon     \;     Less than     \<
%   Equals        \=     Greater than  \>     Question mark \?
%   Commercial at \@     Left bracket  \[     Backslash     \\
%   Right bracket \]     Circumflex    \^     Underscore    \_
%   Grave accent  \`     Left brace    \{     Vertical bar  \|
%   Right brace   \}     Tilde         \~}
%
% \GetFileInfo{ifluatex.drv}
%
% \title{The \xpackage{ifluatex} package}
% \date{2016/05/16 v1.4}
% \author{Heiko Oberdiek\thanks
% {Please report any issues at https://github.com/ho-tex/oberdiek/issues}\\
% \xemail{heiko.oberdiek at googlemail.com}}
%
% \maketitle
%
% \begin{abstract}
% This package looks for \LuaTeX\ regardless of its mode
% and provides the switch \cs{ifluatex}. Also it makes
% \cs{luatexversion} available if it is not present.
% It works with \plainTeX\ or \LaTeX.
% \end{abstract}
%
% \tableofcontents
%
% \section{Documentation}
%
% The package \xpackage{ifluatex} can be used with both \plainTeX\
% and \LaTeX:
% \begin{description}
% \item[\plainTeX:] |\input ifluatex.sty|
% \item[\LaTeXe:]   |\usepackage{ifluatex}|
% \end{description}
%
% \DescribeMacro{\ifluatex}
% The package provides the switch \cs{ifluatex}:
% \begin{quote}
%   |\ifluatex|\\
%   \hspace{1.5em}\LuaTeX\ is running\\
%   |\else|\\
%   \hspace{1.5em}Without \LuaTeX\\
%   |\fi|
% \end{quote}
%
% Since version 0.39 \LuaTeX\ only provides \cs{directlua} at startup
% time. Also the syntax of \cs{directlua} changed in version 0.36.
% Thus the user might want to check the LuaTeX version.
% Therefore this package also makes \cs{luatexversion} and
% \cs{luatexrevision} available, if it is not yet done.
%
% If you want to detect the mode (DVI or PDF), then use package
% \xpackage{ifpdf}. \LuaTeX\ has inherited \cs{pdfoutput} from \pdfTeX.
%
% \StopEventually{
% }
%
% \section{Implementation}
%
%    \begin{macrocode}
%<*package>
%    \end{macrocode}
%
% \subsection{Reload check and package identification}
%    Reload check, especially if the package is not used with \LaTeX.
%    \begin{macrocode}
\begingroup\catcode61\catcode48\catcode32=10\relax%
  \catcode13=5 % ^^M
  \endlinechar=13 %
  \catcode35=6 % #
  \catcode39=12 % '
  \catcode44=12 % ,
  \catcode45=12 % -
  \catcode46=12 % .
  \catcode58=12 % :
  \catcode64=11 % @
  \catcode123=1 % {
  \catcode125=2 % }
  \expandafter\let\expandafter\x\csname ver@ifluatex.sty\endcsname
  \ifx\x\relax % plain-TeX, first loading
  \else
    \def\empty{}%
    \ifx\x\empty % LaTeX, first loading,
      % variable is initialized, but \ProvidesPackage not yet seen
    \else
      \expandafter\ifx\csname PackageInfo\endcsname\relax
        \def\x#1#2{%
          \immediate\write-1{Package #1 Info: #2.}%
        }%
      \else
        \def\x#1#2{\PackageInfo{#1}{#2, stopped}}%
      \fi
      \x{ifluatex}{The package is already loaded}%
      \aftergroup\endinput
    \fi
  \fi
\endgroup%
%    \end{macrocode}
%    Package identification:
%    \begin{macrocode}
\begingroup\catcode61\catcode48\catcode32=10\relax%
  \catcode13=5 % ^^M
  \endlinechar=13 %
  \catcode35=6 % #
  \catcode39=12 % '
  \catcode40=12 % (
  \catcode41=12 % )
  \catcode44=12 % ,
  \catcode45=12 % -
  \catcode46=12 % .
  \catcode47=12 % /
  \catcode58=12 % :
  \catcode64=11 % @
  \catcode91=12 % [
  \catcode93=12 % ]
  \catcode123=1 % {
  \catcode125=2 % }
  \expandafter\ifx\csname ProvidesPackage\endcsname\relax
    \def\x#1#2#3[#4]{\endgroup
      \immediate\write-1{Package: #3 #4}%
      \xdef#1{#4}%
    }%
  \else
    \def\x#1#2[#3]{\endgroup
      #2[{#3}]%
      \ifx#1\@undefined
        \xdef#1{#3}%
      \fi
      \ifx#1\relax
        \xdef#1{#3}%
      \fi
    }%
  \fi
\expandafter\x\csname ver@ifluatex.sty\endcsname
\ProvidesPackage{ifluatex}%
  [2016/05/16 v1.4 Provides the ifluatex switch (HO)]%
%    \end{macrocode}
%
% \subsection{Catcodes}
%
%    \begin{macrocode}
\begingroup\catcode61\catcode48\catcode32=10\relax%
  \catcode13=5 % ^^M
  \endlinechar=13 %
  \catcode123=1 % {
  \catcode125=2 % }
  \catcode64=11 % @
  \def\x{\endgroup
    \expandafter\edef\csname ifluatex@AtEnd\endcsname{%
      \endlinechar=\the\endlinechar\relax
      \catcode13=\the\catcode13\relax
      \catcode32=\the\catcode32\relax
      \catcode35=\the\catcode35\relax
      \catcode61=\the\catcode61\relax
      \catcode64=\the\catcode64\relax
      \catcode123=\the\catcode123\relax
      \catcode125=\the\catcode125\relax
    }%
  }%
\x\catcode61\catcode48\catcode32=10\relax%
\catcode13=5 % ^^M
\endlinechar=13 %
\catcode35=6 % #
\catcode64=11 % @
\catcode123=1 % {
\catcode125=2 % }
\def\TMP@EnsureCode#1#2{%
  \edef\ifluatex@AtEnd{%
    \ifluatex@AtEnd
    \catcode#1=\the\catcode#1\relax
  }%
  \catcode#1=#2\relax
}
\TMP@EnsureCode{10}{12}% ^^J
\TMP@EnsureCode{39}{12}% '
\TMP@EnsureCode{40}{12}% (
\TMP@EnsureCode{41}{12}% )
\TMP@EnsureCode{44}{12}% ,
\TMP@EnsureCode{45}{12}% -
\TMP@EnsureCode{46}{12}% .
\TMP@EnsureCode{47}{12}% /
\TMP@EnsureCode{58}{12}% :
\TMP@EnsureCode{60}{12}% <
\TMP@EnsureCode{94}{7}% ^
\TMP@EnsureCode{96}{12}% `
\edef\ifluatex@AtEnd{\ifluatex@AtEnd\noexpand\endinput}
%    \end{macrocode}
%
% \subsection{Macro for error messages}
%
%    \begin{macro}{\ifluatex@Error}
%    \begin{macrocode}
\begingroup\expandafter\expandafter\expandafter\endgroup
\expandafter\ifx\csname PackageError\endcsname\relax
  \def\ifluatex@Error#1#2{%
    \begingroup
      \newlinechar=10 %
      \def\MessageBreak{^^J}%
      \edef\x{\errhelp{#2}}%
      \x
      \errmessage{Package ifluatex Error: #1}%
    \endgroup
  }%
\else
  \def\ifluatex@Error{%
    \PackageError{ifluatex}%
  }%
\fi
%    \end{macrocode}
%    \end{macro}
%
% \subsection{Check for previously defined \cs{ifluatex}}
%
%    \begin{macrocode}
\begingroup
  \expandafter\ifx\csname ifluatex\endcsname\relax
  \else
    \edef\i/{\expandafter\string\csname ifluatex\endcsname}%
    \ifluatex@Error{Name clash, \i/ is already defined}{%
      Incompatible versions of \i/ can cause problems,\MessageBreak
      therefore package loading is aborted.%
    }%
    \endgroup
    \expandafter\ifluatex@AtEnd
  \fi%
\endgroup
%    \end{macrocode}
%
% \subsection{\cs{ifluatex}}
%
%    \begin{macro}{\ifluatex}
%    \begin{macrocode}
\let\ifluatex\iffalse
%    \end{macrocode}
%    \end{macro}
%
%    Test \cs{luatexversion}. Is it  defined and different from
%    \cs{relax}? Someone could have used \LaTeX\ internal
%    \cs{@ifundefined}, or something else involving.
%    Notice, \cs{csname} is executed inside a group for the test
%    to cancel the side effect of \cs{csname}.
%    \begin{macrocode}
\begingroup\expandafter\expandafter\expandafter\endgroup
\expandafter\ifx\csname luatexversion\endcsname\relax
\else
  \expandafter\let\csname ifluatex\expandafter\endcsname
                  \csname iftrue\endcsname
\fi
%    \end{macrocode}
%
% \subsection{Lua\TeX\ v0.39}
%
%     Starting with version 0.39 \LuaTeX\ wants to provide \cs{directlua}
%     as only primitive at startup time beyond vanilla \TeX's primitives.
%     Then \cs{directlua} exists, but \cs{luatexversion} cannot be found.
%     Unhappily also the syntax of \cs{directlua} changed in v0.36,
%     thus the user would want to check \cs{luatexversion}.
%     Therefore we make \cs{luatexversion} available using
%     \LuaTeX's Lua function |tex.enableprimitives|.
%
%    \begin{macrocode}
\ifluatex
\else
  \begingroup\expandafter\expandafter\expandafter\endgroup
  \expandafter\ifx\csname directlua\endcsname\relax
  \else
    \expandafter\let\csname ifluatex\expandafter\endcsname
                    \csname iftrue\endcsname
    \begingroup
      \newlinechar=10 %
      \endlinechar=\newlinechar%
      \ifnum0%
          \directlua{%
            if tex.enableprimitives then
              tex.enableprimitives('ifluatex', {'luatexversion'})
              tex.print('1')
            end
          }%
          \ifx\ifluatexluatexversion\@undefined\else 1\fi %
          =11 %
        \global\let\luatexversion\ifluatexluatexversion%
      \else%
        \ifluatex@Error{%
          Missing \string\luatexversion%
        }{%
          Update LuaTeX.%
        }%
      \fi%
    \endgroup%
  \fi
\fi
%    \end{macrocode}
%    \begin{macrocode}
\ifluatex
  \begingroup\expandafter\expandafter\expandafter\endgroup
  \expandafter\ifx\csname luatexrevision\endcsname\relax
    \ifnum\luatexversion<36 %
    \else
      \begingroup
        \ifx\luatexrevision\relax
          \let\luatexrevision\@undefined
        \fi
        \newlinechar=10 %
        \endlinechar=\newlinechar%
        \ifcase0%
            \directlua{%
              if tex.enableprimitives then
                tex.enableprimitives('ifluatex', {'luatexrevision'})
              else
                tex.print('1')
              end
            }%
            \ifx\ifluatexluatexrevision\@undefined 1\fi%
            \relax%
          \global\let\luatexrevision\ifluatexluatexrevision%
        \fi%
      \endgroup%
    \fi
    \begingroup\expandafter\expandafter\expandafter\endgroup
    \expandafter\ifx\csname luatexrevision\endcsname\relax
      \ifluatex@Error{%
        Missing \string\luatexrevision%
      }{%
        Update LuaTeX.%
      }%
    \fi
  \fi
\fi
%    \end{macrocode}
%
% \subsection{Protocol entry}
%
%     Log comment:
%    \begin{macrocode}
\begingroup
  \expandafter\ifx\csname PackageInfo\endcsname\relax
    \def\x#1#2{%
      \immediate\write-1{Package #1 Info: #2.}%
    }%
  \else
    \let\x\PackageInfo
    \expandafter\let\csname on@line\endcsname\empty
  \fi
  \x{ifluatex}{LuaTeX \ifluatex\else not \fi detected}%
\endgroup
%    \end{macrocode}
%    \begin{macrocode}
\ifluatex@AtEnd%
%    \end{macrocode}
%    \begin{macrocode}
%</package>
%    \end{macrocode}
%
% \section{Test}
%
% \subsection{Catcode checks for loading}
%
%    \begin{macrocode}
%<*test1>
%    \end{macrocode}
%    \begin{macrocode}
\catcode`\{=1 %
\catcode`\}=2 %
\catcode`\#=6 %
\catcode`\@=11 %
\expandafter\ifx\csname count@\endcsname\relax
  \countdef\count@=255 %
\fi
\expandafter\ifx\csname @gobble\endcsname\relax
  \long\def\@gobble#1{}%
\fi
\expandafter\ifx\csname @firstofone\endcsname\relax
  \long\def\@firstofone#1{#1}%
\fi
\expandafter\ifx\csname loop\endcsname\relax
  \expandafter\@firstofone
\else
  \expandafter\@gobble
\fi
{%
  \def\loop#1\repeat{%
    \def\body{#1}%
    \iterate
  }%
  \def\iterate{%
    \body
      \let\next\iterate
    \else
      \let\next\relax
    \fi
    \next
  }%
  \let\repeat=\fi
}%
\def\RestoreCatcodes{}
\count@=0 %
\loop
  \edef\RestoreCatcodes{%
    \RestoreCatcodes
    \catcode\the\count@=\the\catcode\count@\relax
  }%
\ifnum\count@<255 %
  \advance\count@ 1 %
\repeat

\def\RangeCatcodeInvalid#1#2{%
  \count@=#1\relax
  \loop
    \catcode\count@=15 %
  \ifnum\count@<#2\relax
    \advance\count@ 1 %
  \repeat
}
\def\RangeCatcodeCheck#1#2#3{%
  \count@=#1\relax
  \loop
    \ifnum#3=\catcode\count@
    \else
      \errmessage{%
        Character \the\count@\space
        with wrong catcode \the\catcode\count@\space
        instead of \number#3%
      }%
    \fi
  \ifnum\count@<#2\relax
    \advance\count@ 1 %
  \repeat
}
\def\space{ }
\expandafter\ifx\csname LoadCommand\endcsname\relax
  \def\LoadCommand{\input ifluatex.sty\relax}%
\fi
\def\Test{%
  \RangeCatcodeInvalid{0}{47}%
  \RangeCatcodeInvalid{58}{64}%
  \RangeCatcodeInvalid{91}{96}%
  \RangeCatcodeInvalid{123}{255}%
  \catcode`\@=12 %
  \catcode`\\=0 %
  \catcode`\%=14 %
  \LoadCommand
  \RangeCatcodeCheck{0}{36}{15}%
  \RangeCatcodeCheck{37}{37}{14}%
  \RangeCatcodeCheck{38}{47}{15}%
  \RangeCatcodeCheck{48}{57}{12}%
  \RangeCatcodeCheck{58}{63}{15}%
  \RangeCatcodeCheck{64}{64}{12}%
  \RangeCatcodeCheck{65}{90}{11}%
  \RangeCatcodeCheck{91}{91}{15}%
  \RangeCatcodeCheck{92}{92}{0}%
  \RangeCatcodeCheck{93}{96}{15}%
  \RangeCatcodeCheck{97}{122}{11}%
  \RangeCatcodeCheck{123}{255}{15}%
  \RestoreCatcodes
}
\Test
\csname @@end\endcsname
\end
%    \end{macrocode}
%    \begin{macrocode}
%</test1>
%    \end{macrocode}
%
% \section{Reload check for plain}
%
%    \begin{macrocode}
%<*test-reload1>
\input ifluatex.sty\relax
\input ifluatex.sty\relax
\csname @@end\endcsname\end
%</test-reload1>
%    \end{macrocode}
%
%    \begin{macrocode}
%<*test-reload2>
\input miniltx.tex\relax
\input ifluatex.sty\relax
\input ifluatex.sty\relax
\csname @@end\endcsname\end
%</test-reload2>
%    \end{macrocode}
%
% \section{Installation}
%
% \subsection{Download}
%
% \paragraph{Package.} This package is available on
% CTAN\footnote{\url{http://ctan.org/pkg/ifluatex}}:
% \begin{description}
% \item[\CTAN{macros/latex/contrib/oberdiek/ifluatex.dtx}] The source file.
% \item[\CTAN{macros/latex/contrib/oberdiek/ifluatex.pdf}] Documentation.
% \end{description}
%
%
% \paragraph{Bundle.} All the packages of the bundle `oberdiek'
% are also available in a TDS compliant ZIP archive. There
% the packages are already unpacked and the documentation files
% are generated. The files and directories obey the TDS standard.
% \begin{description}
% \item[\CTAN{install/macros/latex/contrib/oberdiek.tds.zip}]
% \end{description}
% \emph{TDS} refers to the standard ``A Directory Structure
% for \TeX\ Files'' (\CTAN{tds/tds.pdf}). Directories
% with \xfile{texmf} in their name are usually organized this way.
%
% \subsection{Bundle installation}
%
% \paragraph{Unpacking.} Unpack the \xfile{oberdiek.tds.zip} in the
% TDS tree (also known as \xfile{texmf} tree) of your choice.
% Example (linux):
% \begin{quote}
%   |unzip oberdiek.tds.zip -d ~/texmf|
% \end{quote}
%
% \paragraph{Script installation.}
% Check the directory \xfile{TDS:scripts/oberdiek/} for
% scripts that need further installation steps.
% Package \xpackage{attachfile2} comes with the Perl script
% \xfile{pdfatfi.pl} that should be installed in such a way
% that it can be called as \texttt{pdfatfi}.
% Example (linux):
% \begin{quote}
%   |chmod +x scripts/oberdiek/pdfatfi.pl|\\
%   |cp scripts/oberdiek/pdfatfi.pl /usr/local/bin/|
% \end{quote}
%
% \subsection{Package installation}
%
% \paragraph{Unpacking.} The \xfile{.dtx} file is a self-extracting
% \docstrip\ archive. The files are extracted by running the
% \xfile{.dtx} through \plainTeX:
% \begin{quote}
%   \verb|tex ifluatex.dtx|
% \end{quote}
%
% \paragraph{TDS.} Now the different files must be moved into
% the different directories in your installation TDS tree
% (also known as \xfile{texmf} tree):
% \begin{quote}
% \def\t{^^A
% \begin{tabular}{@{}>{\ttfamily}l@{ $\rightarrow$ }>{\ttfamily}l@{}}
%   ifluatex.sty & tex/generic/oberdiek/ifluatex.sty\\
%   ifluatex.pdf & doc/latex/oberdiek/ifluatex.pdf\\
%   test/ifluatex-test1.tex & doc/latex/oberdiek/test/ifluatex-test1.tex\\
%   test/ifluatex-test2.tex & doc/latex/oberdiek/test/ifluatex-test2.tex\\
%   test/ifluatex-test3.tex & doc/latex/oberdiek/test/ifluatex-test3.tex\\
%   ifluatex.dtx & source/latex/oberdiek/ifluatex.dtx\\
% \end{tabular}^^A
% }^^A
% \sbox0{\t}^^A
% \ifdim\wd0>\linewidth
%   \begingroup
%     \advance\linewidth by\leftmargin
%     \advance\linewidth by\rightmargin
%   \edef\x{\endgroup
%     \def\noexpand\lw{\the\linewidth}^^A
%   }\x
%   \def\lwbox{^^A
%     \leavevmode
%     \hbox to \linewidth{^^A
%       \kern-\leftmargin\relax
%       \hss
%       \usebox0
%       \hss
%       \kern-\rightmargin\relax
%     }^^A
%   }^^A
%   \ifdim\wd0>\lw
%     \sbox0{\small\t}^^A
%     \ifdim\wd0>\linewidth
%       \ifdim\wd0>\lw
%         \sbox0{\footnotesize\t}^^A
%         \ifdim\wd0>\linewidth
%           \ifdim\wd0>\lw
%             \sbox0{\scriptsize\t}^^A
%             \ifdim\wd0>\linewidth
%               \ifdim\wd0>\lw
%                 \sbox0{\tiny\t}^^A
%                 \ifdim\wd0>\linewidth
%                   \lwbox
%                 \else
%                   \usebox0
%                 \fi
%               \else
%                 \lwbox
%               \fi
%             \else
%               \usebox0
%             \fi
%           \else
%             \lwbox
%           \fi
%         \else
%           \usebox0
%         \fi
%       \else
%         \lwbox
%       \fi
%     \else
%       \usebox0
%     \fi
%   \else
%     \lwbox
%   \fi
% \else
%   \usebox0
% \fi
% \end{quote}
% If you have a \xfile{docstrip.cfg} that configures and enables \docstrip's
% TDS installing feature, then some files can already be in the right
% place, see the documentation of \docstrip.
%
% \subsection{Refresh file name databases}
%
% If your \TeX~distribution
% (\teTeX, \mikTeX, \dots) relies on file name databases, you must refresh
% these. For example, \teTeX\ users run \verb|texhash| or
% \verb|mktexlsr|.
%
% \subsection{Some details for the interested}
%
% \paragraph{Attached source.}
%
% The PDF documentation on CTAN also includes the
% \xfile{.dtx} source file. It can be extracted by
% AcrobatReader 6 or higher. Another option is \textsf{pdftk},
% e.g. unpack the file into the current directory:
% \begin{quote}
%   \verb|pdftk ifluatex.pdf unpack_files output .|
% \end{quote}
%
% \paragraph{Unpacking with \LaTeX.}
% The \xfile{.dtx} chooses its action depending on the format:
% \begin{description}
% \item[\plainTeX:] Run \docstrip\ and extract the files.
% \item[\LaTeX:] Generate the documentation.
% \end{description}
% If you insist on using \LaTeX\ for \docstrip\ (really,
% \docstrip\ does not need \LaTeX), then inform the autodetect routine
% about your intention:
% \begin{quote}
%   \verb|latex \let\install=y% \iffalse meta-comment
%
% File: ifluatex.dtx
% Version: 2016/05/16 v1.4
% Info: Provides the ifluatex switch
%
% Copyright (C) 2007, 2009, 2010 by
%    Heiko Oberdiek <heiko.oberdiek at googlemail.com>
%    2016
%    https://github.com/ho-tex/oberdiek/issues
%
% This work may be distributed and/or modified under the
% conditions of the LaTeX Project Public License, either
% version 1.3c of this license or (at your option) any later
% version. This version of this license is in
%    http://www.latex-project.org/lppl/lppl-1-3c.txt
% and the latest version of this license is in
%    http://www.latex-project.org/lppl.txt
% and version 1.3 or later is part of all distributions of
% LaTeX version 2005/12/01 or later.
%
% This work has the LPPL maintenance status "maintained".
%
% This Current Maintainer of this work is Heiko Oberdiek.
%
% The Base Interpreter refers to any `TeX-Format',
% because some files are installed in TDS:tex/generic//.
%
% This work consists of the main source file ifluatex.dtx
% and the derived files
%    ifluatex.sty, ifluatex.pdf, ifluatex.ins, ifluatex.drv,
%    ifluatex-test1.tex, ifluatex-test2.tex, ifluatex-test3.tex.
%
% Distribution:
%    CTAN:macros/latex/contrib/oberdiek/ifluatex.dtx
%    CTAN:macros/latex/contrib/oberdiek/ifluatex.pdf
%
% Unpacking:
%    (a) If ifluatex.ins is present:
%           tex ifluatex.ins
%    (b) Without ifluatex.ins:
%           tex ifluatex.dtx
%    (c) If you insist on using LaTeX
%           latex \let\install=y% \iffalse meta-comment
%
% File: ifluatex.dtx
% Version: 2016/05/16 v1.4
% Info: Provides the ifluatex switch
%
% Copyright (C) 2007, 2009, 2010 by
%    Heiko Oberdiek <heiko.oberdiek at googlemail.com>
%    2016
%    https://github.com/ho-tex/oberdiek/issues
%
% This work may be distributed and/or modified under the
% conditions of the LaTeX Project Public License, either
% version 1.3c of this license or (at your option) any later
% version. This version of this license is in
%    http://www.latex-project.org/lppl/lppl-1-3c.txt
% and the latest version of this license is in
%    http://www.latex-project.org/lppl.txt
% and version 1.3 or later is part of all distributions of
% LaTeX version 2005/12/01 or later.
%
% This work has the LPPL maintenance status "maintained".
%
% This Current Maintainer of this work is Heiko Oberdiek.
%
% The Base Interpreter refers to any `TeX-Format',
% because some files are installed in TDS:tex/generic//.
%
% This work consists of the main source file ifluatex.dtx
% and the derived files
%    ifluatex.sty, ifluatex.pdf, ifluatex.ins, ifluatex.drv,
%    ifluatex-test1.tex, ifluatex-test2.tex, ifluatex-test3.tex.
%
% Distribution:
%    CTAN:macros/latex/contrib/oberdiek/ifluatex.dtx
%    CTAN:macros/latex/contrib/oberdiek/ifluatex.pdf
%
% Unpacking:
%    (a) If ifluatex.ins is present:
%           tex ifluatex.ins
%    (b) Without ifluatex.ins:
%           tex ifluatex.dtx
%    (c) If you insist on using LaTeX
%           latex \let\install=y% \iffalse meta-comment
%
% File: ifluatex.dtx
% Version: 2016/05/16 v1.4
% Info: Provides the ifluatex switch
%
% Copyright (C) 2007, 2009, 2010 by
%    Heiko Oberdiek <heiko.oberdiek at googlemail.com>
%    2016
%    https://github.com/ho-tex/oberdiek/issues
%
% This work may be distributed and/or modified under the
% conditions of the LaTeX Project Public License, either
% version 1.3c of this license or (at your option) any later
% version. This version of this license is in
%    http://www.latex-project.org/lppl/lppl-1-3c.txt
% and the latest version of this license is in
%    http://www.latex-project.org/lppl.txt
% and version 1.3 or later is part of all distributions of
% LaTeX version 2005/12/01 or later.
%
% This work has the LPPL maintenance status "maintained".
%
% This Current Maintainer of this work is Heiko Oberdiek.
%
% The Base Interpreter refers to any `TeX-Format',
% because some files are installed in TDS:tex/generic//.
%
% This work consists of the main source file ifluatex.dtx
% and the derived files
%    ifluatex.sty, ifluatex.pdf, ifluatex.ins, ifluatex.drv,
%    ifluatex-test1.tex, ifluatex-test2.tex, ifluatex-test3.tex.
%
% Distribution:
%    CTAN:macros/latex/contrib/oberdiek/ifluatex.dtx
%    CTAN:macros/latex/contrib/oberdiek/ifluatex.pdf
%
% Unpacking:
%    (a) If ifluatex.ins is present:
%           tex ifluatex.ins
%    (b) Without ifluatex.ins:
%           tex ifluatex.dtx
%    (c) If you insist on using LaTeX
%           latex \let\install=y\input{ifluatex.dtx}
%        (quote the arguments according to the demands of your shell)
%
% Documentation:
%    (a) If ifluatex.drv is present:
%           latex ifluatex.drv
%    (b) Without ifluatex.drv:
%           latex ifluatex.dtx; ...
%    The class ltxdoc loads the configuration file ltxdoc.cfg
%    if available. Here you can specify further options, e.g.
%    use A4 as paper format:
%       \PassOptionsToClass{a4paper}{article}
%
%    Programm calls to get the documentation (example):
%       pdflatex ifluatex.dtx
%       makeindex -s gind.ist ifluatex.idx
%       pdflatex ifluatex.dtx
%       makeindex -s gind.ist ifluatex.idx
%       pdflatex ifluatex.dtx
%
% Installation:
%    TDS:tex/generic/oberdiek/ifluatex.sty
%    TDS:doc/latex/oberdiek/ifluatex.pdf
%    TDS:doc/latex/oberdiek/test/ifluatex-test1.tex
%    TDS:doc/latex/oberdiek/test/ifluatex-test2.tex
%    TDS:doc/latex/oberdiek/test/ifluatex-test3.tex
%    TDS:source/latex/oberdiek/ifluatex.dtx
%
%<*ignore>
\begingroup
  \catcode123=1 %
  \catcode125=2 %
  \def\x{LaTeX2e}%
\expandafter\endgroup
\ifcase 0\ifx\install y1\fi\expandafter
         \ifx\csname processbatchFile\endcsname\relax\else1\fi
         \ifx\fmtname\x\else 1\fi\relax
\else\csname fi\endcsname
%</ignore>
%<*install>
\input docstrip.tex
\Msg{************************************************************************}
\Msg{* Installation}
\Msg{* Package: ifluatex 2016/05/16 v1.4 Provides the ifluatex switch (HO)}
\Msg{************************************************************************}

\keepsilent
\askforoverwritefalse

\let\MetaPrefix\relax
\preamble

This is a generated file.

Project: ifluatex
Version: 2016/05/16 v1.4

Copyright (C) 2007, 2009, 2010 by
   Heiko Oberdiek <heiko.oberdiek at googlemail.com>

This work may be distributed and/or modified under the
conditions of the LaTeX Project Public License, either
version 1.3c of this license or (at your option) any later
version. This version of this license is in
   http://www.latex-project.org/lppl/lppl-1-3c.txt
and the latest version of this license is in
   http://www.latex-project.org/lppl.txt
and version 1.3 or later is part of all distributions of
LaTeX version 2005/12/01 or later.

This work has the LPPL maintenance status "maintained".

This Current Maintainer of this work is Heiko Oberdiek.

The Base Interpreter refers to any `TeX-Format',
because some files are installed in TDS:tex/generic//.

This work consists of the main source file ifluatex.dtx
and the derived files
   ifluatex.sty, ifluatex.pdf, ifluatex.ins, ifluatex.drv,
   ifluatex-test1.tex, ifluatex-test2.tex, ifluatex-test3.tex.

\endpreamble
\let\MetaPrefix\DoubleperCent

\generate{%
  \file{ifluatex.ins}{\from{ifluatex.dtx}{install}}%
  \file{ifluatex.drv}{\from{ifluatex.dtx}{driver}}%
  \usedir{tex/generic/oberdiek}%
  \file{ifluatex.sty}{\from{ifluatex.dtx}{package}}%
  \usedir{doc/latex/oberdiek/test}%
  \file{ifluatex-test1.tex}{\from{ifluatex.dtx}{test1}}%
  \file{ifluatex-test2.tex}{\from{ifluatex.dtx}{test-reload1}}%
  \file{ifluatex-test3.tex}{\from{ifluatex.dtx}{test-reload2}}%
  \nopreamble
  \nopostamble
  \usedir{source/latex/oberdiek/catalogue}%
  \file{ifluatex.xml}{\from{ifluatex.dtx}{catalogue}}%
}

\catcode32=13\relax% active space
\let =\space%
\Msg{************************************************************************}
\Msg{*}
\Msg{* To finish the installation you have to move the following}
\Msg{* file into a directory searched by TeX:}
\Msg{*}
\Msg{*     ifluatex.sty}
\Msg{*}
\Msg{* To produce the documentation run the file `ifluatex.drv'}
\Msg{* through LaTeX.}
\Msg{*}
\Msg{* Happy TeXing!}
\Msg{*}
\Msg{************************************************************************}

\endbatchfile
%</install>
%<*ignore>
\fi
%</ignore>
%<*driver>
\NeedsTeXFormat{LaTeX2e}
\ProvidesFile{ifluatex.drv}%
  [2016/05/16 v1.4 Provides the ifluatex switch (HO)]%
\documentclass{ltxdoc}
\usepackage{holtxdoc}[2011/11/22]
\begin{document}
  \DocInput{ifluatex.dtx}%
\end{document}
%</driver>
% \fi
%
%
% \CharacterTable
%  {Upper-case    \A\B\C\D\E\F\G\H\I\J\K\L\M\N\O\P\Q\R\S\T\U\V\W\X\Y\Z
%   Lower-case    \a\b\c\d\e\f\g\h\i\j\k\l\m\n\o\p\q\r\s\t\u\v\w\x\y\z
%   Digits        \0\1\2\3\4\5\6\7\8\9
%   Exclamation   \!     Double quote  \"     Hash (number) \#
%   Dollar        \$     Percent       \%     Ampersand     \&
%   Acute accent  \'     Left paren    \(     Right paren   \)
%   Asterisk      \*     Plus          \+     Comma         \,
%   Minus         \-     Point         \.     Solidus       \/
%   Colon         \:     Semicolon     \;     Less than     \<
%   Equals        \=     Greater than  \>     Question mark \?
%   Commercial at \@     Left bracket  \[     Backslash     \\
%   Right bracket \]     Circumflex    \^     Underscore    \_
%   Grave accent  \`     Left brace    \{     Vertical bar  \|
%   Right brace   \}     Tilde         \~}
%
% \GetFileInfo{ifluatex.drv}
%
% \title{The \xpackage{ifluatex} package}
% \date{2016/05/16 v1.4}
% \author{Heiko Oberdiek\thanks
% {Please report any issues at https://github.com/ho-tex/oberdiek/issues}\\
% \xemail{heiko.oberdiek at googlemail.com}}
%
% \maketitle
%
% \begin{abstract}
% This package looks for \LuaTeX\ regardless of its mode
% and provides the switch \cs{ifluatex}. Also it makes
% \cs{luatexversion} available if it is not present.
% It works with \plainTeX\ or \LaTeX.
% \end{abstract}
%
% \tableofcontents
%
% \section{Documentation}
%
% The package \xpackage{ifluatex} can be used with both \plainTeX\
% and \LaTeX:
% \begin{description}
% \item[\plainTeX:] |\input ifluatex.sty|
% \item[\LaTeXe:]   |\usepackage{ifluatex}|
% \end{description}
%
% \DescribeMacro{\ifluatex}
% The package provides the switch \cs{ifluatex}:
% \begin{quote}
%   |\ifluatex|\\
%   \hspace{1.5em}\LuaTeX\ is running\\
%   |\else|\\
%   \hspace{1.5em}Without \LuaTeX\\
%   |\fi|
% \end{quote}
%
% Since version 0.39 \LuaTeX\ only provides \cs{directlua} at startup
% time. Also the syntax of \cs{directlua} changed in version 0.36.
% Thus the user might want to check the LuaTeX version.
% Therefore this package also makes \cs{luatexversion} and
% \cs{luatexrevision} available, if it is not yet done.
%
% If you want to detect the mode (DVI or PDF), then use package
% \xpackage{ifpdf}. \LuaTeX\ has inherited \cs{pdfoutput} from \pdfTeX.
%
% \StopEventually{
% }
%
% \section{Implementation}
%
%    \begin{macrocode}
%<*package>
%    \end{macrocode}
%
% \subsection{Reload check and package identification}
%    Reload check, especially if the package is not used with \LaTeX.
%    \begin{macrocode}
\begingroup\catcode61\catcode48\catcode32=10\relax%
  \catcode13=5 % ^^M
  \endlinechar=13 %
  \catcode35=6 % #
  \catcode39=12 % '
  \catcode44=12 % ,
  \catcode45=12 % -
  \catcode46=12 % .
  \catcode58=12 % :
  \catcode64=11 % @
  \catcode123=1 % {
  \catcode125=2 % }
  \expandafter\let\expandafter\x\csname ver@ifluatex.sty\endcsname
  \ifx\x\relax % plain-TeX, first loading
  \else
    \def\empty{}%
    \ifx\x\empty % LaTeX, first loading,
      % variable is initialized, but \ProvidesPackage not yet seen
    \else
      \expandafter\ifx\csname PackageInfo\endcsname\relax
        \def\x#1#2{%
          \immediate\write-1{Package #1 Info: #2.}%
        }%
      \else
        \def\x#1#2{\PackageInfo{#1}{#2, stopped}}%
      \fi
      \x{ifluatex}{The package is already loaded}%
      \aftergroup\endinput
    \fi
  \fi
\endgroup%
%    \end{macrocode}
%    Package identification:
%    \begin{macrocode}
\begingroup\catcode61\catcode48\catcode32=10\relax%
  \catcode13=5 % ^^M
  \endlinechar=13 %
  \catcode35=6 % #
  \catcode39=12 % '
  \catcode40=12 % (
  \catcode41=12 % )
  \catcode44=12 % ,
  \catcode45=12 % -
  \catcode46=12 % .
  \catcode47=12 % /
  \catcode58=12 % :
  \catcode64=11 % @
  \catcode91=12 % [
  \catcode93=12 % ]
  \catcode123=1 % {
  \catcode125=2 % }
  \expandafter\ifx\csname ProvidesPackage\endcsname\relax
    \def\x#1#2#3[#4]{\endgroup
      \immediate\write-1{Package: #3 #4}%
      \xdef#1{#4}%
    }%
  \else
    \def\x#1#2[#3]{\endgroup
      #2[{#3}]%
      \ifx#1\@undefined
        \xdef#1{#3}%
      \fi
      \ifx#1\relax
        \xdef#1{#3}%
      \fi
    }%
  \fi
\expandafter\x\csname ver@ifluatex.sty\endcsname
\ProvidesPackage{ifluatex}%
  [2016/05/16 v1.4 Provides the ifluatex switch (HO)]%
%    \end{macrocode}
%
% \subsection{Catcodes}
%
%    \begin{macrocode}
\begingroup\catcode61\catcode48\catcode32=10\relax%
  \catcode13=5 % ^^M
  \endlinechar=13 %
  \catcode123=1 % {
  \catcode125=2 % }
  \catcode64=11 % @
  \def\x{\endgroup
    \expandafter\edef\csname ifluatex@AtEnd\endcsname{%
      \endlinechar=\the\endlinechar\relax
      \catcode13=\the\catcode13\relax
      \catcode32=\the\catcode32\relax
      \catcode35=\the\catcode35\relax
      \catcode61=\the\catcode61\relax
      \catcode64=\the\catcode64\relax
      \catcode123=\the\catcode123\relax
      \catcode125=\the\catcode125\relax
    }%
  }%
\x\catcode61\catcode48\catcode32=10\relax%
\catcode13=5 % ^^M
\endlinechar=13 %
\catcode35=6 % #
\catcode64=11 % @
\catcode123=1 % {
\catcode125=2 % }
\def\TMP@EnsureCode#1#2{%
  \edef\ifluatex@AtEnd{%
    \ifluatex@AtEnd
    \catcode#1=\the\catcode#1\relax
  }%
  \catcode#1=#2\relax
}
\TMP@EnsureCode{10}{12}% ^^J
\TMP@EnsureCode{39}{12}% '
\TMP@EnsureCode{40}{12}% (
\TMP@EnsureCode{41}{12}% )
\TMP@EnsureCode{44}{12}% ,
\TMP@EnsureCode{45}{12}% -
\TMP@EnsureCode{46}{12}% .
\TMP@EnsureCode{47}{12}% /
\TMP@EnsureCode{58}{12}% :
\TMP@EnsureCode{60}{12}% <
\TMP@EnsureCode{94}{7}% ^
\TMP@EnsureCode{96}{12}% `
\edef\ifluatex@AtEnd{\ifluatex@AtEnd\noexpand\endinput}
%    \end{macrocode}
%
% \subsection{Macro for error messages}
%
%    \begin{macro}{\ifluatex@Error}
%    \begin{macrocode}
\begingroup\expandafter\expandafter\expandafter\endgroup
\expandafter\ifx\csname PackageError\endcsname\relax
  \def\ifluatex@Error#1#2{%
    \begingroup
      \newlinechar=10 %
      \def\MessageBreak{^^J}%
      \edef\x{\errhelp{#2}}%
      \x
      \errmessage{Package ifluatex Error: #1}%
    \endgroup
  }%
\else
  \def\ifluatex@Error{%
    \PackageError{ifluatex}%
  }%
\fi
%    \end{macrocode}
%    \end{macro}
%
% \subsection{Check for previously defined \cs{ifluatex}}
%
%    \begin{macrocode}
\begingroup
  \expandafter\ifx\csname ifluatex\endcsname\relax
  \else
    \edef\i/{\expandafter\string\csname ifluatex\endcsname}%
    \ifluatex@Error{Name clash, \i/ is already defined}{%
      Incompatible versions of \i/ can cause problems,\MessageBreak
      therefore package loading is aborted.%
    }%
    \endgroup
    \expandafter\ifluatex@AtEnd
  \fi%
\endgroup
%    \end{macrocode}
%
% \subsection{\cs{ifluatex}}
%
%    \begin{macro}{\ifluatex}
%    \begin{macrocode}
\let\ifluatex\iffalse
%    \end{macrocode}
%    \end{macro}
%
%    Test \cs{luatexversion}. Is it  defined and different from
%    \cs{relax}? Someone could have used \LaTeX\ internal
%    \cs{@ifundefined}, or something else involving.
%    Notice, \cs{csname} is executed inside a group for the test
%    to cancel the side effect of \cs{csname}.
%    \begin{macrocode}
\begingroup\expandafter\expandafter\expandafter\endgroup
\expandafter\ifx\csname luatexversion\endcsname\relax
\else
  \expandafter\let\csname ifluatex\expandafter\endcsname
                  \csname iftrue\endcsname
\fi
%    \end{macrocode}
%
% \subsection{Lua\TeX\ v0.39}
%
%     Starting with version 0.39 \LuaTeX\ wants to provide \cs{directlua}
%     as only primitive at startup time beyond vanilla \TeX's primitives.
%     Then \cs{directlua} exists, but \cs{luatexversion} cannot be found.
%     Unhappily also the syntax of \cs{directlua} changed in v0.36,
%     thus the user would want to check \cs{luatexversion}.
%     Therefore we make \cs{luatexversion} available using
%     \LuaTeX's Lua function |tex.enableprimitives|.
%
%    \begin{macrocode}
\ifluatex
\else
  \begingroup\expandafter\expandafter\expandafter\endgroup
  \expandafter\ifx\csname directlua\endcsname\relax
  \else
    \expandafter\let\csname ifluatex\expandafter\endcsname
                    \csname iftrue\endcsname
    \begingroup
      \newlinechar=10 %
      \endlinechar=\newlinechar%
      \ifnum0%
          \directlua{%
            if tex.enableprimitives then
              tex.enableprimitives('ifluatex', {'luatexversion'})
              tex.print('1')
            end
          }%
          \ifx\ifluatexluatexversion\@undefined\else 1\fi %
          =11 %
        \global\let\luatexversion\ifluatexluatexversion%
      \else%
        \ifluatex@Error{%
          Missing \string\luatexversion%
        }{%
          Update LuaTeX.%
        }%
      \fi%
    \endgroup%
  \fi
\fi
%    \end{macrocode}
%    \begin{macrocode}
\ifluatex
  \begingroup\expandafter\expandafter\expandafter\endgroup
  \expandafter\ifx\csname luatexrevision\endcsname\relax
    \ifnum\luatexversion<36 %
    \else
      \begingroup
        \ifx\luatexrevision\relax
          \let\luatexrevision\@undefined
        \fi
        \newlinechar=10 %
        \endlinechar=\newlinechar%
        \ifcase0%
            \directlua{%
              if tex.enableprimitives then
                tex.enableprimitives('ifluatex', {'luatexrevision'})
              else
                tex.print('1')
              end
            }%
            \ifx\ifluatexluatexrevision\@undefined 1\fi%
            \relax%
          \global\let\luatexrevision\ifluatexluatexrevision%
        \fi%
      \endgroup%
    \fi
    \begingroup\expandafter\expandafter\expandafter\endgroup
    \expandafter\ifx\csname luatexrevision\endcsname\relax
      \ifluatex@Error{%
        Missing \string\luatexrevision%
      }{%
        Update LuaTeX.%
      }%
    \fi
  \fi
\fi
%    \end{macrocode}
%
% \subsection{Protocol entry}
%
%     Log comment:
%    \begin{macrocode}
\begingroup
  \expandafter\ifx\csname PackageInfo\endcsname\relax
    \def\x#1#2{%
      \immediate\write-1{Package #1 Info: #2.}%
    }%
  \else
    \let\x\PackageInfo
    \expandafter\let\csname on@line\endcsname\empty
  \fi
  \x{ifluatex}{LuaTeX \ifluatex\else not \fi detected}%
\endgroup
%    \end{macrocode}
%    \begin{macrocode}
\ifluatex@AtEnd%
%    \end{macrocode}
%    \begin{macrocode}
%</package>
%    \end{macrocode}
%
% \section{Test}
%
% \subsection{Catcode checks for loading}
%
%    \begin{macrocode}
%<*test1>
%    \end{macrocode}
%    \begin{macrocode}
\catcode`\{=1 %
\catcode`\}=2 %
\catcode`\#=6 %
\catcode`\@=11 %
\expandafter\ifx\csname count@\endcsname\relax
  \countdef\count@=255 %
\fi
\expandafter\ifx\csname @gobble\endcsname\relax
  \long\def\@gobble#1{}%
\fi
\expandafter\ifx\csname @firstofone\endcsname\relax
  \long\def\@firstofone#1{#1}%
\fi
\expandafter\ifx\csname loop\endcsname\relax
  \expandafter\@firstofone
\else
  \expandafter\@gobble
\fi
{%
  \def\loop#1\repeat{%
    \def\body{#1}%
    \iterate
  }%
  \def\iterate{%
    \body
      \let\next\iterate
    \else
      \let\next\relax
    \fi
    \next
  }%
  \let\repeat=\fi
}%
\def\RestoreCatcodes{}
\count@=0 %
\loop
  \edef\RestoreCatcodes{%
    \RestoreCatcodes
    \catcode\the\count@=\the\catcode\count@\relax
  }%
\ifnum\count@<255 %
  \advance\count@ 1 %
\repeat

\def\RangeCatcodeInvalid#1#2{%
  \count@=#1\relax
  \loop
    \catcode\count@=15 %
  \ifnum\count@<#2\relax
    \advance\count@ 1 %
  \repeat
}
\def\RangeCatcodeCheck#1#2#3{%
  \count@=#1\relax
  \loop
    \ifnum#3=\catcode\count@
    \else
      \errmessage{%
        Character \the\count@\space
        with wrong catcode \the\catcode\count@\space
        instead of \number#3%
      }%
    \fi
  \ifnum\count@<#2\relax
    \advance\count@ 1 %
  \repeat
}
\def\space{ }
\expandafter\ifx\csname LoadCommand\endcsname\relax
  \def\LoadCommand{\input ifluatex.sty\relax}%
\fi
\def\Test{%
  \RangeCatcodeInvalid{0}{47}%
  \RangeCatcodeInvalid{58}{64}%
  \RangeCatcodeInvalid{91}{96}%
  \RangeCatcodeInvalid{123}{255}%
  \catcode`\@=12 %
  \catcode`\\=0 %
  \catcode`\%=14 %
  \LoadCommand
  \RangeCatcodeCheck{0}{36}{15}%
  \RangeCatcodeCheck{37}{37}{14}%
  \RangeCatcodeCheck{38}{47}{15}%
  \RangeCatcodeCheck{48}{57}{12}%
  \RangeCatcodeCheck{58}{63}{15}%
  \RangeCatcodeCheck{64}{64}{12}%
  \RangeCatcodeCheck{65}{90}{11}%
  \RangeCatcodeCheck{91}{91}{15}%
  \RangeCatcodeCheck{92}{92}{0}%
  \RangeCatcodeCheck{93}{96}{15}%
  \RangeCatcodeCheck{97}{122}{11}%
  \RangeCatcodeCheck{123}{255}{15}%
  \RestoreCatcodes
}
\Test
\csname @@end\endcsname
\end
%    \end{macrocode}
%    \begin{macrocode}
%</test1>
%    \end{macrocode}
%
% \section{Reload check for plain}
%
%    \begin{macrocode}
%<*test-reload1>
\input ifluatex.sty\relax
\input ifluatex.sty\relax
\csname @@end\endcsname\end
%</test-reload1>
%    \end{macrocode}
%
%    \begin{macrocode}
%<*test-reload2>
\input miniltx.tex\relax
\input ifluatex.sty\relax
\input ifluatex.sty\relax
\csname @@end\endcsname\end
%</test-reload2>
%    \end{macrocode}
%
% \section{Installation}
%
% \subsection{Download}
%
% \paragraph{Package.} This package is available on
% CTAN\footnote{\url{http://ctan.org/pkg/ifluatex}}:
% \begin{description}
% \item[\CTAN{macros/latex/contrib/oberdiek/ifluatex.dtx}] The source file.
% \item[\CTAN{macros/latex/contrib/oberdiek/ifluatex.pdf}] Documentation.
% \end{description}
%
%
% \paragraph{Bundle.} All the packages of the bundle `oberdiek'
% are also available in a TDS compliant ZIP archive. There
% the packages are already unpacked and the documentation files
% are generated. The files and directories obey the TDS standard.
% \begin{description}
% \item[\CTAN{install/macros/latex/contrib/oberdiek.tds.zip}]
% \end{description}
% \emph{TDS} refers to the standard ``A Directory Structure
% for \TeX\ Files'' (\CTAN{tds/tds.pdf}). Directories
% with \xfile{texmf} in their name are usually organized this way.
%
% \subsection{Bundle installation}
%
% \paragraph{Unpacking.} Unpack the \xfile{oberdiek.tds.zip} in the
% TDS tree (also known as \xfile{texmf} tree) of your choice.
% Example (linux):
% \begin{quote}
%   |unzip oberdiek.tds.zip -d ~/texmf|
% \end{quote}
%
% \paragraph{Script installation.}
% Check the directory \xfile{TDS:scripts/oberdiek/} for
% scripts that need further installation steps.
% Package \xpackage{attachfile2} comes with the Perl script
% \xfile{pdfatfi.pl} that should be installed in such a way
% that it can be called as \texttt{pdfatfi}.
% Example (linux):
% \begin{quote}
%   |chmod +x scripts/oberdiek/pdfatfi.pl|\\
%   |cp scripts/oberdiek/pdfatfi.pl /usr/local/bin/|
% \end{quote}
%
% \subsection{Package installation}
%
% \paragraph{Unpacking.} The \xfile{.dtx} file is a self-extracting
% \docstrip\ archive. The files are extracted by running the
% \xfile{.dtx} through \plainTeX:
% \begin{quote}
%   \verb|tex ifluatex.dtx|
% \end{quote}
%
% \paragraph{TDS.} Now the different files must be moved into
% the different directories in your installation TDS tree
% (also known as \xfile{texmf} tree):
% \begin{quote}
% \def\t{^^A
% \begin{tabular}{@{}>{\ttfamily}l@{ $\rightarrow$ }>{\ttfamily}l@{}}
%   ifluatex.sty & tex/generic/oberdiek/ifluatex.sty\\
%   ifluatex.pdf & doc/latex/oberdiek/ifluatex.pdf\\
%   test/ifluatex-test1.tex & doc/latex/oberdiek/test/ifluatex-test1.tex\\
%   test/ifluatex-test2.tex & doc/latex/oberdiek/test/ifluatex-test2.tex\\
%   test/ifluatex-test3.tex & doc/latex/oberdiek/test/ifluatex-test3.tex\\
%   ifluatex.dtx & source/latex/oberdiek/ifluatex.dtx\\
% \end{tabular}^^A
% }^^A
% \sbox0{\t}^^A
% \ifdim\wd0>\linewidth
%   \begingroup
%     \advance\linewidth by\leftmargin
%     \advance\linewidth by\rightmargin
%   \edef\x{\endgroup
%     \def\noexpand\lw{\the\linewidth}^^A
%   }\x
%   \def\lwbox{^^A
%     \leavevmode
%     \hbox to \linewidth{^^A
%       \kern-\leftmargin\relax
%       \hss
%       \usebox0
%       \hss
%       \kern-\rightmargin\relax
%     }^^A
%   }^^A
%   \ifdim\wd0>\lw
%     \sbox0{\small\t}^^A
%     \ifdim\wd0>\linewidth
%       \ifdim\wd0>\lw
%         \sbox0{\footnotesize\t}^^A
%         \ifdim\wd0>\linewidth
%           \ifdim\wd0>\lw
%             \sbox0{\scriptsize\t}^^A
%             \ifdim\wd0>\linewidth
%               \ifdim\wd0>\lw
%                 \sbox0{\tiny\t}^^A
%                 \ifdim\wd0>\linewidth
%                   \lwbox
%                 \else
%                   \usebox0
%                 \fi
%               \else
%                 \lwbox
%               \fi
%             \else
%               \usebox0
%             \fi
%           \else
%             \lwbox
%           \fi
%         \else
%           \usebox0
%         \fi
%       \else
%         \lwbox
%       \fi
%     \else
%       \usebox0
%     \fi
%   \else
%     \lwbox
%   \fi
% \else
%   \usebox0
% \fi
% \end{quote}
% If you have a \xfile{docstrip.cfg} that configures and enables \docstrip's
% TDS installing feature, then some files can already be in the right
% place, see the documentation of \docstrip.
%
% \subsection{Refresh file name databases}
%
% If your \TeX~distribution
% (\teTeX, \mikTeX, \dots) relies on file name databases, you must refresh
% these. For example, \teTeX\ users run \verb|texhash| or
% \verb|mktexlsr|.
%
% \subsection{Some details for the interested}
%
% \paragraph{Attached source.}
%
% The PDF documentation on CTAN also includes the
% \xfile{.dtx} source file. It can be extracted by
% AcrobatReader 6 or higher. Another option is \textsf{pdftk},
% e.g. unpack the file into the current directory:
% \begin{quote}
%   \verb|pdftk ifluatex.pdf unpack_files output .|
% \end{quote}
%
% \paragraph{Unpacking with \LaTeX.}
% The \xfile{.dtx} chooses its action depending on the format:
% \begin{description}
% \item[\plainTeX:] Run \docstrip\ and extract the files.
% \item[\LaTeX:] Generate the documentation.
% \end{description}
% If you insist on using \LaTeX\ for \docstrip\ (really,
% \docstrip\ does not need \LaTeX), then inform the autodetect routine
% about your intention:
% \begin{quote}
%   \verb|latex \let\install=y\input{ifluatex.dtx}|
% \end{quote}
% Do not forget to quote the argument according to the demands
% of your shell.
%
% \paragraph{Generating the documentation.}
% You can use both the \xfile{.dtx} or the \xfile{.drv} to generate
% the documentation. The process can be configured by the
% configuration file \xfile{ltxdoc.cfg}. For instance, put this
% line into this file, if you want to have A4 as paper format:
% \begin{quote}
%   \verb|\PassOptionsToClass{a4paper}{article}|
% \end{quote}
% An example follows how to generate the
% documentation with pdf\LaTeX:
% \begin{quote}
%\begin{verbatim}
%pdflatex ifluatex.dtx
%makeindex -s gind.ist ifluatex.idx
%pdflatex ifluatex.dtx
%makeindex -s gind.ist ifluatex.idx
%pdflatex ifluatex.dtx
%\end{verbatim}
% \end{quote}
%
% \section{Catalogue}
%
% The following XML file can be used as source for the
% \href{http://mirror.ctan.org/help/Catalogue/catalogue.html}{\TeX\ Catalogue}.
% The elements \texttt{caption} and \texttt{description} are imported
% from the original XML file from the Catalogue.
% The name of the XML file in the Catalogue is \xfile{ifluatex.xml}.
%    \begin{macrocode}
%<*catalogue>
<?xml version='1.0' encoding='us-ascii'?>
<!DOCTYPE entry SYSTEM 'catalogue.dtd'>
<entry datestamp='$Date$' modifier='$Author$' id='ifluatex'>
  <name>ifluatex</name>
  <caption>Provides the \ifluatex switch.</caption>
  <authorref id='auth:oberdiek'/>
  <copyright owner='Heiko Oberdiek' year='2007,2009,2010'/>
  <license type='lppl1.3'/>
  <version number='1.4'/>
  <description>
    The package looks for  LuaTeX regardless of its mode and provides
    the switch <tt>\ifluatex</tt>; it works with Plain TeX or LaTeX.
    <p/>
    The package is part of the <xref refid='oberdiek'>oberdiek</xref>
    bundle.
  </description>
  <documentation details='Package documentation'
      href='ctan:/macros/latex/contrib/oberdiek/ifluatex.pdf'/>
  <ctan file='true' path='/macros/latex/contrib/oberdiek/ifluatex.dtx'/>
  <miktex location='oberdiek'/>
  <texlive location='ifluatex'/>
  <install path='/macros/latex/contrib/oberdiek/oberdiek.tds.zip'/>
</entry>
%</catalogue>
%    \end{macrocode}
%
% \begin{History}
%   \begin{Version}{2007/12/12 v1.0}
%   \item
%     First public version.
%   \end{Version}
%   \begin{Version}{2009/04/10 v1.1}
%   \item
%     Test adopted for \LuaTeX\ 0.39.
%   \item
%     Makes \cs{luatexversion} available.
%   \end{Version}
%   \begin{Version}{2009/04/17 v1.2}
%   \item
%     Fixes (Manuel P\'egouri\'e-Gonnard).
%   \item
%     \cs{luatextrue} and \cs{luatexfalse} are no longer defined.
%   \item
%     Makes \cs{luatexrevision} available, too.
%   \end{Version}
%   \begin{Version}{2010/03/01 v1.3}
%   \item
%     Line ends fixed in case \cs{endlinechar} = \cs{newlinechar}.
%   \end{Version}
%   \begin{Version}{2016/05/16 v1.4}
%   \item
%     Documentation updates.
%   \end{Version}
% \end{History}
%
% \PrintIndex
%
% \Finale
\endinput

%        (quote the arguments according to the demands of your shell)
%
% Documentation:
%    (a) If ifluatex.drv is present:
%           latex ifluatex.drv
%    (b) Without ifluatex.drv:
%           latex ifluatex.dtx; ...
%    The class ltxdoc loads the configuration file ltxdoc.cfg
%    if available. Here you can specify further options, e.g.
%    use A4 as paper format:
%       \PassOptionsToClass{a4paper}{article}
%
%    Programm calls to get the documentation (example):
%       pdflatex ifluatex.dtx
%       makeindex -s gind.ist ifluatex.idx
%       pdflatex ifluatex.dtx
%       makeindex -s gind.ist ifluatex.idx
%       pdflatex ifluatex.dtx
%
% Installation:
%    TDS:tex/generic/oberdiek/ifluatex.sty
%    TDS:doc/latex/oberdiek/ifluatex.pdf
%    TDS:doc/latex/oberdiek/test/ifluatex-test1.tex
%    TDS:doc/latex/oberdiek/test/ifluatex-test2.tex
%    TDS:doc/latex/oberdiek/test/ifluatex-test3.tex
%    TDS:source/latex/oberdiek/ifluatex.dtx
%
%<*ignore>
\begingroup
  \catcode123=1 %
  \catcode125=2 %
  \def\x{LaTeX2e}%
\expandafter\endgroup
\ifcase 0\ifx\install y1\fi\expandafter
         \ifx\csname processbatchFile\endcsname\relax\else1\fi
         \ifx\fmtname\x\else 1\fi\relax
\else\csname fi\endcsname
%</ignore>
%<*install>
\input docstrip.tex
\Msg{************************************************************************}
\Msg{* Installation}
\Msg{* Package: ifluatex 2016/05/16 v1.4 Provides the ifluatex switch (HO)}
\Msg{************************************************************************}

\keepsilent
\askforoverwritefalse

\let\MetaPrefix\relax
\preamble

This is a generated file.

Project: ifluatex
Version: 2016/05/16 v1.4

Copyright (C) 2007, 2009, 2010 by
   Heiko Oberdiek <heiko.oberdiek at googlemail.com>

This work may be distributed and/or modified under the
conditions of the LaTeX Project Public License, either
version 1.3c of this license or (at your option) any later
version. This version of this license is in
   http://www.latex-project.org/lppl/lppl-1-3c.txt
and the latest version of this license is in
   http://www.latex-project.org/lppl.txt
and version 1.3 or later is part of all distributions of
LaTeX version 2005/12/01 or later.

This work has the LPPL maintenance status "maintained".

This Current Maintainer of this work is Heiko Oberdiek.

The Base Interpreter refers to any `TeX-Format',
because some files are installed in TDS:tex/generic//.

This work consists of the main source file ifluatex.dtx
and the derived files
   ifluatex.sty, ifluatex.pdf, ifluatex.ins, ifluatex.drv,
   ifluatex-test1.tex, ifluatex-test2.tex, ifluatex-test3.tex.

\endpreamble
\let\MetaPrefix\DoubleperCent

\generate{%
  \file{ifluatex.ins}{\from{ifluatex.dtx}{install}}%
  \file{ifluatex.drv}{\from{ifluatex.dtx}{driver}}%
  \usedir{tex/generic/oberdiek}%
  \file{ifluatex.sty}{\from{ifluatex.dtx}{package}}%
  \usedir{doc/latex/oberdiek/test}%
  \file{ifluatex-test1.tex}{\from{ifluatex.dtx}{test1}}%
  \file{ifluatex-test2.tex}{\from{ifluatex.dtx}{test-reload1}}%
  \file{ifluatex-test3.tex}{\from{ifluatex.dtx}{test-reload2}}%
  \nopreamble
  \nopostamble
  \usedir{source/latex/oberdiek/catalogue}%
  \file{ifluatex.xml}{\from{ifluatex.dtx}{catalogue}}%
}

\catcode32=13\relax% active space
\let =\space%
\Msg{************************************************************************}
\Msg{*}
\Msg{* To finish the installation you have to move the following}
\Msg{* file into a directory searched by TeX:}
\Msg{*}
\Msg{*     ifluatex.sty}
\Msg{*}
\Msg{* To produce the documentation run the file `ifluatex.drv'}
\Msg{* through LaTeX.}
\Msg{*}
\Msg{* Happy TeXing!}
\Msg{*}
\Msg{************************************************************************}

\endbatchfile
%</install>
%<*ignore>
\fi
%</ignore>
%<*driver>
\NeedsTeXFormat{LaTeX2e}
\ProvidesFile{ifluatex.drv}%
  [2016/05/16 v1.4 Provides the ifluatex switch (HO)]%
\documentclass{ltxdoc}
\usepackage{holtxdoc}[2011/11/22]
\begin{document}
  \DocInput{ifluatex.dtx}%
\end{document}
%</driver>
% \fi
%
%
% \CharacterTable
%  {Upper-case    \A\B\C\D\E\F\G\H\I\J\K\L\M\N\O\P\Q\R\S\T\U\V\W\X\Y\Z
%   Lower-case    \a\b\c\d\e\f\g\h\i\j\k\l\m\n\o\p\q\r\s\t\u\v\w\x\y\z
%   Digits        \0\1\2\3\4\5\6\7\8\9
%   Exclamation   \!     Double quote  \"     Hash (number) \#
%   Dollar        \$     Percent       \%     Ampersand     \&
%   Acute accent  \'     Left paren    \(     Right paren   \)
%   Asterisk      \*     Plus          \+     Comma         \,
%   Minus         \-     Point         \.     Solidus       \/
%   Colon         \:     Semicolon     \;     Less than     \<
%   Equals        \=     Greater than  \>     Question mark \?
%   Commercial at \@     Left bracket  \[     Backslash     \\
%   Right bracket \]     Circumflex    \^     Underscore    \_
%   Grave accent  \`     Left brace    \{     Vertical bar  \|
%   Right brace   \}     Tilde         \~}
%
% \GetFileInfo{ifluatex.drv}
%
% \title{The \xpackage{ifluatex} package}
% \date{2016/05/16 v1.4}
% \author{Heiko Oberdiek\thanks
% {Please report any issues at https://github.com/ho-tex/oberdiek/issues}\\
% \xemail{heiko.oberdiek at googlemail.com}}
%
% \maketitle
%
% \begin{abstract}
% This package looks for \LuaTeX\ regardless of its mode
% and provides the switch \cs{ifluatex}. Also it makes
% \cs{luatexversion} available if it is not present.
% It works with \plainTeX\ or \LaTeX.
% \end{abstract}
%
% \tableofcontents
%
% \section{Documentation}
%
% The package \xpackage{ifluatex} can be used with both \plainTeX\
% and \LaTeX:
% \begin{description}
% \item[\plainTeX:] |\input ifluatex.sty|
% \item[\LaTeXe:]   |\usepackage{ifluatex}|
% \end{description}
%
% \DescribeMacro{\ifluatex}
% The package provides the switch \cs{ifluatex}:
% \begin{quote}
%   |\ifluatex|\\
%   \hspace{1.5em}\LuaTeX\ is running\\
%   |\else|\\
%   \hspace{1.5em}Without \LuaTeX\\
%   |\fi|
% \end{quote}
%
% Since version 0.39 \LuaTeX\ only provides \cs{directlua} at startup
% time. Also the syntax of \cs{directlua} changed in version 0.36.
% Thus the user might want to check the LuaTeX version.
% Therefore this package also makes \cs{luatexversion} and
% \cs{luatexrevision} available, if it is not yet done.
%
% If you want to detect the mode (DVI or PDF), then use package
% \xpackage{ifpdf}. \LuaTeX\ has inherited \cs{pdfoutput} from \pdfTeX.
%
% \StopEventually{
% }
%
% \section{Implementation}
%
%    \begin{macrocode}
%<*package>
%    \end{macrocode}
%
% \subsection{Reload check and package identification}
%    Reload check, especially if the package is not used with \LaTeX.
%    \begin{macrocode}
\begingroup\catcode61\catcode48\catcode32=10\relax%
  \catcode13=5 % ^^M
  \endlinechar=13 %
  \catcode35=6 % #
  \catcode39=12 % '
  \catcode44=12 % ,
  \catcode45=12 % -
  \catcode46=12 % .
  \catcode58=12 % :
  \catcode64=11 % @
  \catcode123=1 % {
  \catcode125=2 % }
  \expandafter\let\expandafter\x\csname ver@ifluatex.sty\endcsname
  \ifx\x\relax % plain-TeX, first loading
  \else
    \def\empty{}%
    \ifx\x\empty % LaTeX, first loading,
      % variable is initialized, but \ProvidesPackage not yet seen
    \else
      \expandafter\ifx\csname PackageInfo\endcsname\relax
        \def\x#1#2{%
          \immediate\write-1{Package #1 Info: #2.}%
        }%
      \else
        \def\x#1#2{\PackageInfo{#1}{#2, stopped}}%
      \fi
      \x{ifluatex}{The package is already loaded}%
      \aftergroup\endinput
    \fi
  \fi
\endgroup%
%    \end{macrocode}
%    Package identification:
%    \begin{macrocode}
\begingroup\catcode61\catcode48\catcode32=10\relax%
  \catcode13=5 % ^^M
  \endlinechar=13 %
  \catcode35=6 % #
  \catcode39=12 % '
  \catcode40=12 % (
  \catcode41=12 % )
  \catcode44=12 % ,
  \catcode45=12 % -
  \catcode46=12 % .
  \catcode47=12 % /
  \catcode58=12 % :
  \catcode64=11 % @
  \catcode91=12 % [
  \catcode93=12 % ]
  \catcode123=1 % {
  \catcode125=2 % }
  \expandafter\ifx\csname ProvidesPackage\endcsname\relax
    \def\x#1#2#3[#4]{\endgroup
      \immediate\write-1{Package: #3 #4}%
      \xdef#1{#4}%
    }%
  \else
    \def\x#1#2[#3]{\endgroup
      #2[{#3}]%
      \ifx#1\@undefined
        \xdef#1{#3}%
      \fi
      \ifx#1\relax
        \xdef#1{#3}%
      \fi
    }%
  \fi
\expandafter\x\csname ver@ifluatex.sty\endcsname
\ProvidesPackage{ifluatex}%
  [2016/05/16 v1.4 Provides the ifluatex switch (HO)]%
%    \end{macrocode}
%
% \subsection{Catcodes}
%
%    \begin{macrocode}
\begingroup\catcode61\catcode48\catcode32=10\relax%
  \catcode13=5 % ^^M
  \endlinechar=13 %
  \catcode123=1 % {
  \catcode125=2 % }
  \catcode64=11 % @
  \def\x{\endgroup
    \expandafter\edef\csname ifluatex@AtEnd\endcsname{%
      \endlinechar=\the\endlinechar\relax
      \catcode13=\the\catcode13\relax
      \catcode32=\the\catcode32\relax
      \catcode35=\the\catcode35\relax
      \catcode61=\the\catcode61\relax
      \catcode64=\the\catcode64\relax
      \catcode123=\the\catcode123\relax
      \catcode125=\the\catcode125\relax
    }%
  }%
\x\catcode61\catcode48\catcode32=10\relax%
\catcode13=5 % ^^M
\endlinechar=13 %
\catcode35=6 % #
\catcode64=11 % @
\catcode123=1 % {
\catcode125=2 % }
\def\TMP@EnsureCode#1#2{%
  \edef\ifluatex@AtEnd{%
    \ifluatex@AtEnd
    \catcode#1=\the\catcode#1\relax
  }%
  \catcode#1=#2\relax
}
\TMP@EnsureCode{10}{12}% ^^J
\TMP@EnsureCode{39}{12}% '
\TMP@EnsureCode{40}{12}% (
\TMP@EnsureCode{41}{12}% )
\TMP@EnsureCode{44}{12}% ,
\TMP@EnsureCode{45}{12}% -
\TMP@EnsureCode{46}{12}% .
\TMP@EnsureCode{47}{12}% /
\TMP@EnsureCode{58}{12}% :
\TMP@EnsureCode{60}{12}% <
\TMP@EnsureCode{94}{7}% ^
\TMP@EnsureCode{96}{12}% `
\edef\ifluatex@AtEnd{\ifluatex@AtEnd\noexpand\endinput}
%    \end{macrocode}
%
% \subsection{Macro for error messages}
%
%    \begin{macro}{\ifluatex@Error}
%    \begin{macrocode}
\begingroup\expandafter\expandafter\expandafter\endgroup
\expandafter\ifx\csname PackageError\endcsname\relax
  \def\ifluatex@Error#1#2{%
    \begingroup
      \newlinechar=10 %
      \def\MessageBreak{^^J}%
      \edef\x{\errhelp{#2}}%
      \x
      \errmessage{Package ifluatex Error: #1}%
    \endgroup
  }%
\else
  \def\ifluatex@Error{%
    \PackageError{ifluatex}%
  }%
\fi
%    \end{macrocode}
%    \end{macro}
%
% \subsection{Check for previously defined \cs{ifluatex}}
%
%    \begin{macrocode}
\begingroup
  \expandafter\ifx\csname ifluatex\endcsname\relax
  \else
    \edef\i/{\expandafter\string\csname ifluatex\endcsname}%
    \ifluatex@Error{Name clash, \i/ is already defined}{%
      Incompatible versions of \i/ can cause problems,\MessageBreak
      therefore package loading is aborted.%
    }%
    \endgroup
    \expandafter\ifluatex@AtEnd
  \fi%
\endgroup
%    \end{macrocode}
%
% \subsection{\cs{ifluatex}}
%
%    \begin{macro}{\ifluatex}
%    \begin{macrocode}
\let\ifluatex\iffalse
%    \end{macrocode}
%    \end{macro}
%
%    Test \cs{luatexversion}. Is it  defined and different from
%    \cs{relax}? Someone could have used \LaTeX\ internal
%    \cs{@ifundefined}, or something else involving.
%    Notice, \cs{csname} is executed inside a group for the test
%    to cancel the side effect of \cs{csname}.
%    \begin{macrocode}
\begingroup\expandafter\expandafter\expandafter\endgroup
\expandafter\ifx\csname luatexversion\endcsname\relax
\else
  \expandafter\let\csname ifluatex\expandafter\endcsname
                  \csname iftrue\endcsname
\fi
%    \end{macrocode}
%
% \subsection{Lua\TeX\ v0.39}
%
%     Starting with version 0.39 \LuaTeX\ wants to provide \cs{directlua}
%     as only primitive at startup time beyond vanilla \TeX's primitives.
%     Then \cs{directlua} exists, but \cs{luatexversion} cannot be found.
%     Unhappily also the syntax of \cs{directlua} changed in v0.36,
%     thus the user would want to check \cs{luatexversion}.
%     Therefore we make \cs{luatexversion} available using
%     \LuaTeX's Lua function |tex.enableprimitives|.
%
%    \begin{macrocode}
\ifluatex
\else
  \begingroup\expandafter\expandafter\expandafter\endgroup
  \expandafter\ifx\csname directlua\endcsname\relax
  \else
    \expandafter\let\csname ifluatex\expandafter\endcsname
                    \csname iftrue\endcsname
    \begingroup
      \newlinechar=10 %
      \endlinechar=\newlinechar%
      \ifnum0%
          \directlua{%
            if tex.enableprimitives then
              tex.enableprimitives('ifluatex', {'luatexversion'})
              tex.print('1')
            end
          }%
          \ifx\ifluatexluatexversion\@undefined\else 1\fi %
          =11 %
        \global\let\luatexversion\ifluatexluatexversion%
      \else%
        \ifluatex@Error{%
          Missing \string\luatexversion%
        }{%
          Update LuaTeX.%
        }%
      \fi%
    \endgroup%
  \fi
\fi
%    \end{macrocode}
%    \begin{macrocode}
\ifluatex
  \begingroup\expandafter\expandafter\expandafter\endgroup
  \expandafter\ifx\csname luatexrevision\endcsname\relax
    \ifnum\luatexversion<36 %
    \else
      \begingroup
        \ifx\luatexrevision\relax
          \let\luatexrevision\@undefined
        \fi
        \newlinechar=10 %
        \endlinechar=\newlinechar%
        \ifcase0%
            \directlua{%
              if tex.enableprimitives then
                tex.enableprimitives('ifluatex', {'luatexrevision'})
              else
                tex.print('1')
              end
            }%
            \ifx\ifluatexluatexrevision\@undefined 1\fi%
            \relax%
          \global\let\luatexrevision\ifluatexluatexrevision%
        \fi%
      \endgroup%
    \fi
    \begingroup\expandafter\expandafter\expandafter\endgroup
    \expandafter\ifx\csname luatexrevision\endcsname\relax
      \ifluatex@Error{%
        Missing \string\luatexrevision%
      }{%
        Update LuaTeX.%
      }%
    \fi
  \fi
\fi
%    \end{macrocode}
%
% \subsection{Protocol entry}
%
%     Log comment:
%    \begin{macrocode}
\begingroup
  \expandafter\ifx\csname PackageInfo\endcsname\relax
    \def\x#1#2{%
      \immediate\write-1{Package #1 Info: #2.}%
    }%
  \else
    \let\x\PackageInfo
    \expandafter\let\csname on@line\endcsname\empty
  \fi
  \x{ifluatex}{LuaTeX \ifluatex\else not \fi detected}%
\endgroup
%    \end{macrocode}
%    \begin{macrocode}
\ifluatex@AtEnd%
%    \end{macrocode}
%    \begin{macrocode}
%</package>
%    \end{macrocode}
%
% \section{Test}
%
% \subsection{Catcode checks for loading}
%
%    \begin{macrocode}
%<*test1>
%    \end{macrocode}
%    \begin{macrocode}
\catcode`\{=1 %
\catcode`\}=2 %
\catcode`\#=6 %
\catcode`\@=11 %
\expandafter\ifx\csname count@\endcsname\relax
  \countdef\count@=255 %
\fi
\expandafter\ifx\csname @gobble\endcsname\relax
  \long\def\@gobble#1{}%
\fi
\expandafter\ifx\csname @firstofone\endcsname\relax
  \long\def\@firstofone#1{#1}%
\fi
\expandafter\ifx\csname loop\endcsname\relax
  \expandafter\@firstofone
\else
  \expandafter\@gobble
\fi
{%
  \def\loop#1\repeat{%
    \def\body{#1}%
    \iterate
  }%
  \def\iterate{%
    \body
      \let\next\iterate
    \else
      \let\next\relax
    \fi
    \next
  }%
  \let\repeat=\fi
}%
\def\RestoreCatcodes{}
\count@=0 %
\loop
  \edef\RestoreCatcodes{%
    \RestoreCatcodes
    \catcode\the\count@=\the\catcode\count@\relax
  }%
\ifnum\count@<255 %
  \advance\count@ 1 %
\repeat

\def\RangeCatcodeInvalid#1#2{%
  \count@=#1\relax
  \loop
    \catcode\count@=15 %
  \ifnum\count@<#2\relax
    \advance\count@ 1 %
  \repeat
}
\def\RangeCatcodeCheck#1#2#3{%
  \count@=#1\relax
  \loop
    \ifnum#3=\catcode\count@
    \else
      \errmessage{%
        Character \the\count@\space
        with wrong catcode \the\catcode\count@\space
        instead of \number#3%
      }%
    \fi
  \ifnum\count@<#2\relax
    \advance\count@ 1 %
  \repeat
}
\def\space{ }
\expandafter\ifx\csname LoadCommand\endcsname\relax
  \def\LoadCommand{\input ifluatex.sty\relax}%
\fi
\def\Test{%
  \RangeCatcodeInvalid{0}{47}%
  \RangeCatcodeInvalid{58}{64}%
  \RangeCatcodeInvalid{91}{96}%
  \RangeCatcodeInvalid{123}{255}%
  \catcode`\@=12 %
  \catcode`\\=0 %
  \catcode`\%=14 %
  \LoadCommand
  \RangeCatcodeCheck{0}{36}{15}%
  \RangeCatcodeCheck{37}{37}{14}%
  \RangeCatcodeCheck{38}{47}{15}%
  \RangeCatcodeCheck{48}{57}{12}%
  \RangeCatcodeCheck{58}{63}{15}%
  \RangeCatcodeCheck{64}{64}{12}%
  \RangeCatcodeCheck{65}{90}{11}%
  \RangeCatcodeCheck{91}{91}{15}%
  \RangeCatcodeCheck{92}{92}{0}%
  \RangeCatcodeCheck{93}{96}{15}%
  \RangeCatcodeCheck{97}{122}{11}%
  \RangeCatcodeCheck{123}{255}{15}%
  \RestoreCatcodes
}
\Test
\csname @@end\endcsname
\end
%    \end{macrocode}
%    \begin{macrocode}
%</test1>
%    \end{macrocode}
%
% \section{Reload check for plain}
%
%    \begin{macrocode}
%<*test-reload1>
\input ifluatex.sty\relax
\input ifluatex.sty\relax
\csname @@end\endcsname\end
%</test-reload1>
%    \end{macrocode}
%
%    \begin{macrocode}
%<*test-reload2>
\input miniltx.tex\relax
\input ifluatex.sty\relax
\input ifluatex.sty\relax
\csname @@end\endcsname\end
%</test-reload2>
%    \end{macrocode}
%
% \section{Installation}
%
% \subsection{Download}
%
% \paragraph{Package.} This package is available on
% CTAN\footnote{\url{http://ctan.org/pkg/ifluatex}}:
% \begin{description}
% \item[\CTAN{macros/latex/contrib/oberdiek/ifluatex.dtx}] The source file.
% \item[\CTAN{macros/latex/contrib/oberdiek/ifluatex.pdf}] Documentation.
% \end{description}
%
%
% \paragraph{Bundle.} All the packages of the bundle `oberdiek'
% are also available in a TDS compliant ZIP archive. There
% the packages are already unpacked and the documentation files
% are generated. The files and directories obey the TDS standard.
% \begin{description}
% \item[\CTAN{install/macros/latex/contrib/oberdiek.tds.zip}]
% \end{description}
% \emph{TDS} refers to the standard ``A Directory Structure
% for \TeX\ Files'' (\CTAN{tds/tds.pdf}). Directories
% with \xfile{texmf} in their name are usually organized this way.
%
% \subsection{Bundle installation}
%
% \paragraph{Unpacking.} Unpack the \xfile{oberdiek.tds.zip} in the
% TDS tree (also known as \xfile{texmf} tree) of your choice.
% Example (linux):
% \begin{quote}
%   |unzip oberdiek.tds.zip -d ~/texmf|
% \end{quote}
%
% \paragraph{Script installation.}
% Check the directory \xfile{TDS:scripts/oberdiek/} for
% scripts that need further installation steps.
% Package \xpackage{attachfile2} comes with the Perl script
% \xfile{pdfatfi.pl} that should be installed in such a way
% that it can be called as \texttt{pdfatfi}.
% Example (linux):
% \begin{quote}
%   |chmod +x scripts/oberdiek/pdfatfi.pl|\\
%   |cp scripts/oberdiek/pdfatfi.pl /usr/local/bin/|
% \end{quote}
%
% \subsection{Package installation}
%
% \paragraph{Unpacking.} The \xfile{.dtx} file is a self-extracting
% \docstrip\ archive. The files are extracted by running the
% \xfile{.dtx} through \plainTeX:
% \begin{quote}
%   \verb|tex ifluatex.dtx|
% \end{quote}
%
% \paragraph{TDS.} Now the different files must be moved into
% the different directories in your installation TDS tree
% (also known as \xfile{texmf} tree):
% \begin{quote}
% \def\t{^^A
% \begin{tabular}{@{}>{\ttfamily}l@{ $\rightarrow$ }>{\ttfamily}l@{}}
%   ifluatex.sty & tex/generic/oberdiek/ifluatex.sty\\
%   ifluatex.pdf & doc/latex/oberdiek/ifluatex.pdf\\
%   test/ifluatex-test1.tex & doc/latex/oberdiek/test/ifluatex-test1.tex\\
%   test/ifluatex-test2.tex & doc/latex/oberdiek/test/ifluatex-test2.tex\\
%   test/ifluatex-test3.tex & doc/latex/oberdiek/test/ifluatex-test3.tex\\
%   ifluatex.dtx & source/latex/oberdiek/ifluatex.dtx\\
% \end{tabular}^^A
% }^^A
% \sbox0{\t}^^A
% \ifdim\wd0>\linewidth
%   \begingroup
%     \advance\linewidth by\leftmargin
%     \advance\linewidth by\rightmargin
%   \edef\x{\endgroup
%     \def\noexpand\lw{\the\linewidth}^^A
%   }\x
%   \def\lwbox{^^A
%     \leavevmode
%     \hbox to \linewidth{^^A
%       \kern-\leftmargin\relax
%       \hss
%       \usebox0
%       \hss
%       \kern-\rightmargin\relax
%     }^^A
%   }^^A
%   \ifdim\wd0>\lw
%     \sbox0{\small\t}^^A
%     \ifdim\wd0>\linewidth
%       \ifdim\wd0>\lw
%         \sbox0{\footnotesize\t}^^A
%         \ifdim\wd0>\linewidth
%           \ifdim\wd0>\lw
%             \sbox0{\scriptsize\t}^^A
%             \ifdim\wd0>\linewidth
%               \ifdim\wd0>\lw
%                 \sbox0{\tiny\t}^^A
%                 \ifdim\wd0>\linewidth
%                   \lwbox
%                 \else
%                   \usebox0
%                 \fi
%               \else
%                 \lwbox
%               \fi
%             \else
%               \usebox0
%             \fi
%           \else
%             \lwbox
%           \fi
%         \else
%           \usebox0
%         \fi
%       \else
%         \lwbox
%       \fi
%     \else
%       \usebox0
%     \fi
%   \else
%     \lwbox
%   \fi
% \else
%   \usebox0
% \fi
% \end{quote}
% If you have a \xfile{docstrip.cfg} that configures and enables \docstrip's
% TDS installing feature, then some files can already be in the right
% place, see the documentation of \docstrip.
%
% \subsection{Refresh file name databases}
%
% If your \TeX~distribution
% (\teTeX, \mikTeX, \dots) relies on file name databases, you must refresh
% these. For example, \teTeX\ users run \verb|texhash| or
% \verb|mktexlsr|.
%
% \subsection{Some details for the interested}
%
% \paragraph{Attached source.}
%
% The PDF documentation on CTAN also includes the
% \xfile{.dtx} source file. It can be extracted by
% AcrobatReader 6 or higher. Another option is \textsf{pdftk},
% e.g. unpack the file into the current directory:
% \begin{quote}
%   \verb|pdftk ifluatex.pdf unpack_files output .|
% \end{quote}
%
% \paragraph{Unpacking with \LaTeX.}
% The \xfile{.dtx} chooses its action depending on the format:
% \begin{description}
% \item[\plainTeX:] Run \docstrip\ and extract the files.
% \item[\LaTeX:] Generate the documentation.
% \end{description}
% If you insist on using \LaTeX\ for \docstrip\ (really,
% \docstrip\ does not need \LaTeX), then inform the autodetect routine
% about your intention:
% \begin{quote}
%   \verb|latex \let\install=y% \iffalse meta-comment
%
% File: ifluatex.dtx
% Version: 2016/05/16 v1.4
% Info: Provides the ifluatex switch
%
% Copyright (C) 2007, 2009, 2010 by
%    Heiko Oberdiek <heiko.oberdiek at googlemail.com>
%    2016
%    https://github.com/ho-tex/oberdiek/issues
%
% This work may be distributed and/or modified under the
% conditions of the LaTeX Project Public License, either
% version 1.3c of this license or (at your option) any later
% version. This version of this license is in
%    http://www.latex-project.org/lppl/lppl-1-3c.txt
% and the latest version of this license is in
%    http://www.latex-project.org/lppl.txt
% and version 1.3 or later is part of all distributions of
% LaTeX version 2005/12/01 or later.
%
% This work has the LPPL maintenance status "maintained".
%
% This Current Maintainer of this work is Heiko Oberdiek.
%
% The Base Interpreter refers to any `TeX-Format',
% because some files are installed in TDS:tex/generic//.
%
% This work consists of the main source file ifluatex.dtx
% and the derived files
%    ifluatex.sty, ifluatex.pdf, ifluatex.ins, ifluatex.drv,
%    ifluatex-test1.tex, ifluatex-test2.tex, ifluatex-test3.tex.
%
% Distribution:
%    CTAN:macros/latex/contrib/oberdiek/ifluatex.dtx
%    CTAN:macros/latex/contrib/oberdiek/ifluatex.pdf
%
% Unpacking:
%    (a) If ifluatex.ins is present:
%           tex ifluatex.ins
%    (b) Without ifluatex.ins:
%           tex ifluatex.dtx
%    (c) If you insist on using LaTeX
%           latex \let\install=y\input{ifluatex.dtx}
%        (quote the arguments according to the demands of your shell)
%
% Documentation:
%    (a) If ifluatex.drv is present:
%           latex ifluatex.drv
%    (b) Without ifluatex.drv:
%           latex ifluatex.dtx; ...
%    The class ltxdoc loads the configuration file ltxdoc.cfg
%    if available. Here you can specify further options, e.g.
%    use A4 as paper format:
%       \PassOptionsToClass{a4paper}{article}
%
%    Programm calls to get the documentation (example):
%       pdflatex ifluatex.dtx
%       makeindex -s gind.ist ifluatex.idx
%       pdflatex ifluatex.dtx
%       makeindex -s gind.ist ifluatex.idx
%       pdflatex ifluatex.dtx
%
% Installation:
%    TDS:tex/generic/oberdiek/ifluatex.sty
%    TDS:doc/latex/oberdiek/ifluatex.pdf
%    TDS:doc/latex/oberdiek/test/ifluatex-test1.tex
%    TDS:doc/latex/oberdiek/test/ifluatex-test2.tex
%    TDS:doc/latex/oberdiek/test/ifluatex-test3.tex
%    TDS:source/latex/oberdiek/ifluatex.dtx
%
%<*ignore>
\begingroup
  \catcode123=1 %
  \catcode125=2 %
  \def\x{LaTeX2e}%
\expandafter\endgroup
\ifcase 0\ifx\install y1\fi\expandafter
         \ifx\csname processbatchFile\endcsname\relax\else1\fi
         \ifx\fmtname\x\else 1\fi\relax
\else\csname fi\endcsname
%</ignore>
%<*install>
\input docstrip.tex
\Msg{************************************************************************}
\Msg{* Installation}
\Msg{* Package: ifluatex 2016/05/16 v1.4 Provides the ifluatex switch (HO)}
\Msg{************************************************************************}

\keepsilent
\askforoverwritefalse

\let\MetaPrefix\relax
\preamble

This is a generated file.

Project: ifluatex
Version: 2016/05/16 v1.4

Copyright (C) 2007, 2009, 2010 by
   Heiko Oberdiek <heiko.oberdiek at googlemail.com>

This work may be distributed and/or modified under the
conditions of the LaTeX Project Public License, either
version 1.3c of this license or (at your option) any later
version. This version of this license is in
   http://www.latex-project.org/lppl/lppl-1-3c.txt
and the latest version of this license is in
   http://www.latex-project.org/lppl.txt
and version 1.3 or later is part of all distributions of
LaTeX version 2005/12/01 or later.

This work has the LPPL maintenance status "maintained".

This Current Maintainer of this work is Heiko Oberdiek.

The Base Interpreter refers to any `TeX-Format',
because some files are installed in TDS:tex/generic//.

This work consists of the main source file ifluatex.dtx
and the derived files
   ifluatex.sty, ifluatex.pdf, ifluatex.ins, ifluatex.drv,
   ifluatex-test1.tex, ifluatex-test2.tex, ifluatex-test3.tex.

\endpreamble
\let\MetaPrefix\DoubleperCent

\generate{%
  \file{ifluatex.ins}{\from{ifluatex.dtx}{install}}%
  \file{ifluatex.drv}{\from{ifluatex.dtx}{driver}}%
  \usedir{tex/generic/oberdiek}%
  \file{ifluatex.sty}{\from{ifluatex.dtx}{package}}%
  \usedir{doc/latex/oberdiek/test}%
  \file{ifluatex-test1.tex}{\from{ifluatex.dtx}{test1}}%
  \file{ifluatex-test2.tex}{\from{ifluatex.dtx}{test-reload1}}%
  \file{ifluatex-test3.tex}{\from{ifluatex.dtx}{test-reload2}}%
  \nopreamble
  \nopostamble
  \usedir{source/latex/oberdiek/catalogue}%
  \file{ifluatex.xml}{\from{ifluatex.dtx}{catalogue}}%
}

\catcode32=13\relax% active space
\let =\space%
\Msg{************************************************************************}
\Msg{*}
\Msg{* To finish the installation you have to move the following}
\Msg{* file into a directory searched by TeX:}
\Msg{*}
\Msg{*     ifluatex.sty}
\Msg{*}
\Msg{* To produce the documentation run the file `ifluatex.drv'}
\Msg{* through LaTeX.}
\Msg{*}
\Msg{* Happy TeXing!}
\Msg{*}
\Msg{************************************************************************}

\endbatchfile
%</install>
%<*ignore>
\fi
%</ignore>
%<*driver>
\NeedsTeXFormat{LaTeX2e}
\ProvidesFile{ifluatex.drv}%
  [2016/05/16 v1.4 Provides the ifluatex switch (HO)]%
\documentclass{ltxdoc}
\usepackage{holtxdoc}[2011/11/22]
\begin{document}
  \DocInput{ifluatex.dtx}%
\end{document}
%</driver>
% \fi
%
%
% \CharacterTable
%  {Upper-case    \A\B\C\D\E\F\G\H\I\J\K\L\M\N\O\P\Q\R\S\T\U\V\W\X\Y\Z
%   Lower-case    \a\b\c\d\e\f\g\h\i\j\k\l\m\n\o\p\q\r\s\t\u\v\w\x\y\z
%   Digits        \0\1\2\3\4\5\6\7\8\9
%   Exclamation   \!     Double quote  \"     Hash (number) \#
%   Dollar        \$     Percent       \%     Ampersand     \&
%   Acute accent  \'     Left paren    \(     Right paren   \)
%   Asterisk      \*     Plus          \+     Comma         \,
%   Minus         \-     Point         \.     Solidus       \/
%   Colon         \:     Semicolon     \;     Less than     \<
%   Equals        \=     Greater than  \>     Question mark \?
%   Commercial at \@     Left bracket  \[     Backslash     \\
%   Right bracket \]     Circumflex    \^     Underscore    \_
%   Grave accent  \`     Left brace    \{     Vertical bar  \|
%   Right brace   \}     Tilde         \~}
%
% \GetFileInfo{ifluatex.drv}
%
% \title{The \xpackage{ifluatex} package}
% \date{2016/05/16 v1.4}
% \author{Heiko Oberdiek\thanks
% {Please report any issues at https://github.com/ho-tex/oberdiek/issues}\\
% \xemail{heiko.oberdiek at googlemail.com}}
%
% \maketitle
%
% \begin{abstract}
% This package looks for \LuaTeX\ regardless of its mode
% and provides the switch \cs{ifluatex}. Also it makes
% \cs{luatexversion} available if it is not present.
% It works with \plainTeX\ or \LaTeX.
% \end{abstract}
%
% \tableofcontents
%
% \section{Documentation}
%
% The package \xpackage{ifluatex} can be used with both \plainTeX\
% and \LaTeX:
% \begin{description}
% \item[\plainTeX:] |\input ifluatex.sty|
% \item[\LaTeXe:]   |\usepackage{ifluatex}|
% \end{description}
%
% \DescribeMacro{\ifluatex}
% The package provides the switch \cs{ifluatex}:
% \begin{quote}
%   |\ifluatex|\\
%   \hspace{1.5em}\LuaTeX\ is running\\
%   |\else|\\
%   \hspace{1.5em}Without \LuaTeX\\
%   |\fi|
% \end{quote}
%
% Since version 0.39 \LuaTeX\ only provides \cs{directlua} at startup
% time. Also the syntax of \cs{directlua} changed in version 0.36.
% Thus the user might want to check the LuaTeX version.
% Therefore this package also makes \cs{luatexversion} and
% \cs{luatexrevision} available, if it is not yet done.
%
% If you want to detect the mode (DVI or PDF), then use package
% \xpackage{ifpdf}. \LuaTeX\ has inherited \cs{pdfoutput} from \pdfTeX.
%
% \StopEventually{
% }
%
% \section{Implementation}
%
%    \begin{macrocode}
%<*package>
%    \end{macrocode}
%
% \subsection{Reload check and package identification}
%    Reload check, especially if the package is not used with \LaTeX.
%    \begin{macrocode}
\begingroup\catcode61\catcode48\catcode32=10\relax%
  \catcode13=5 % ^^M
  \endlinechar=13 %
  \catcode35=6 % #
  \catcode39=12 % '
  \catcode44=12 % ,
  \catcode45=12 % -
  \catcode46=12 % .
  \catcode58=12 % :
  \catcode64=11 % @
  \catcode123=1 % {
  \catcode125=2 % }
  \expandafter\let\expandafter\x\csname ver@ifluatex.sty\endcsname
  \ifx\x\relax % plain-TeX, first loading
  \else
    \def\empty{}%
    \ifx\x\empty % LaTeX, first loading,
      % variable is initialized, but \ProvidesPackage not yet seen
    \else
      \expandafter\ifx\csname PackageInfo\endcsname\relax
        \def\x#1#2{%
          \immediate\write-1{Package #1 Info: #2.}%
        }%
      \else
        \def\x#1#2{\PackageInfo{#1}{#2, stopped}}%
      \fi
      \x{ifluatex}{The package is already loaded}%
      \aftergroup\endinput
    \fi
  \fi
\endgroup%
%    \end{macrocode}
%    Package identification:
%    \begin{macrocode}
\begingroup\catcode61\catcode48\catcode32=10\relax%
  \catcode13=5 % ^^M
  \endlinechar=13 %
  \catcode35=6 % #
  \catcode39=12 % '
  \catcode40=12 % (
  \catcode41=12 % )
  \catcode44=12 % ,
  \catcode45=12 % -
  \catcode46=12 % .
  \catcode47=12 % /
  \catcode58=12 % :
  \catcode64=11 % @
  \catcode91=12 % [
  \catcode93=12 % ]
  \catcode123=1 % {
  \catcode125=2 % }
  \expandafter\ifx\csname ProvidesPackage\endcsname\relax
    \def\x#1#2#3[#4]{\endgroup
      \immediate\write-1{Package: #3 #4}%
      \xdef#1{#4}%
    }%
  \else
    \def\x#1#2[#3]{\endgroup
      #2[{#3}]%
      \ifx#1\@undefined
        \xdef#1{#3}%
      \fi
      \ifx#1\relax
        \xdef#1{#3}%
      \fi
    }%
  \fi
\expandafter\x\csname ver@ifluatex.sty\endcsname
\ProvidesPackage{ifluatex}%
  [2016/05/16 v1.4 Provides the ifluatex switch (HO)]%
%    \end{macrocode}
%
% \subsection{Catcodes}
%
%    \begin{macrocode}
\begingroup\catcode61\catcode48\catcode32=10\relax%
  \catcode13=5 % ^^M
  \endlinechar=13 %
  \catcode123=1 % {
  \catcode125=2 % }
  \catcode64=11 % @
  \def\x{\endgroup
    \expandafter\edef\csname ifluatex@AtEnd\endcsname{%
      \endlinechar=\the\endlinechar\relax
      \catcode13=\the\catcode13\relax
      \catcode32=\the\catcode32\relax
      \catcode35=\the\catcode35\relax
      \catcode61=\the\catcode61\relax
      \catcode64=\the\catcode64\relax
      \catcode123=\the\catcode123\relax
      \catcode125=\the\catcode125\relax
    }%
  }%
\x\catcode61\catcode48\catcode32=10\relax%
\catcode13=5 % ^^M
\endlinechar=13 %
\catcode35=6 % #
\catcode64=11 % @
\catcode123=1 % {
\catcode125=2 % }
\def\TMP@EnsureCode#1#2{%
  \edef\ifluatex@AtEnd{%
    \ifluatex@AtEnd
    \catcode#1=\the\catcode#1\relax
  }%
  \catcode#1=#2\relax
}
\TMP@EnsureCode{10}{12}% ^^J
\TMP@EnsureCode{39}{12}% '
\TMP@EnsureCode{40}{12}% (
\TMP@EnsureCode{41}{12}% )
\TMP@EnsureCode{44}{12}% ,
\TMP@EnsureCode{45}{12}% -
\TMP@EnsureCode{46}{12}% .
\TMP@EnsureCode{47}{12}% /
\TMP@EnsureCode{58}{12}% :
\TMP@EnsureCode{60}{12}% <
\TMP@EnsureCode{94}{7}% ^
\TMP@EnsureCode{96}{12}% `
\edef\ifluatex@AtEnd{\ifluatex@AtEnd\noexpand\endinput}
%    \end{macrocode}
%
% \subsection{Macro for error messages}
%
%    \begin{macro}{\ifluatex@Error}
%    \begin{macrocode}
\begingroup\expandafter\expandafter\expandafter\endgroup
\expandafter\ifx\csname PackageError\endcsname\relax
  \def\ifluatex@Error#1#2{%
    \begingroup
      \newlinechar=10 %
      \def\MessageBreak{^^J}%
      \edef\x{\errhelp{#2}}%
      \x
      \errmessage{Package ifluatex Error: #1}%
    \endgroup
  }%
\else
  \def\ifluatex@Error{%
    \PackageError{ifluatex}%
  }%
\fi
%    \end{macrocode}
%    \end{macro}
%
% \subsection{Check for previously defined \cs{ifluatex}}
%
%    \begin{macrocode}
\begingroup
  \expandafter\ifx\csname ifluatex\endcsname\relax
  \else
    \edef\i/{\expandafter\string\csname ifluatex\endcsname}%
    \ifluatex@Error{Name clash, \i/ is already defined}{%
      Incompatible versions of \i/ can cause problems,\MessageBreak
      therefore package loading is aborted.%
    }%
    \endgroup
    \expandafter\ifluatex@AtEnd
  \fi%
\endgroup
%    \end{macrocode}
%
% \subsection{\cs{ifluatex}}
%
%    \begin{macro}{\ifluatex}
%    \begin{macrocode}
\let\ifluatex\iffalse
%    \end{macrocode}
%    \end{macro}
%
%    Test \cs{luatexversion}. Is it  defined and different from
%    \cs{relax}? Someone could have used \LaTeX\ internal
%    \cs{@ifundefined}, or something else involving.
%    Notice, \cs{csname} is executed inside a group for the test
%    to cancel the side effect of \cs{csname}.
%    \begin{macrocode}
\begingroup\expandafter\expandafter\expandafter\endgroup
\expandafter\ifx\csname luatexversion\endcsname\relax
\else
  \expandafter\let\csname ifluatex\expandafter\endcsname
                  \csname iftrue\endcsname
\fi
%    \end{macrocode}
%
% \subsection{Lua\TeX\ v0.39}
%
%     Starting with version 0.39 \LuaTeX\ wants to provide \cs{directlua}
%     as only primitive at startup time beyond vanilla \TeX's primitives.
%     Then \cs{directlua} exists, but \cs{luatexversion} cannot be found.
%     Unhappily also the syntax of \cs{directlua} changed in v0.36,
%     thus the user would want to check \cs{luatexversion}.
%     Therefore we make \cs{luatexversion} available using
%     \LuaTeX's Lua function |tex.enableprimitives|.
%
%    \begin{macrocode}
\ifluatex
\else
  \begingroup\expandafter\expandafter\expandafter\endgroup
  \expandafter\ifx\csname directlua\endcsname\relax
  \else
    \expandafter\let\csname ifluatex\expandafter\endcsname
                    \csname iftrue\endcsname
    \begingroup
      \newlinechar=10 %
      \endlinechar=\newlinechar%
      \ifnum0%
          \directlua{%
            if tex.enableprimitives then
              tex.enableprimitives('ifluatex', {'luatexversion'})
              tex.print('1')
            end
          }%
          \ifx\ifluatexluatexversion\@undefined\else 1\fi %
          =11 %
        \global\let\luatexversion\ifluatexluatexversion%
      \else%
        \ifluatex@Error{%
          Missing \string\luatexversion%
        }{%
          Update LuaTeX.%
        }%
      \fi%
    \endgroup%
  \fi
\fi
%    \end{macrocode}
%    \begin{macrocode}
\ifluatex
  \begingroup\expandafter\expandafter\expandafter\endgroup
  \expandafter\ifx\csname luatexrevision\endcsname\relax
    \ifnum\luatexversion<36 %
    \else
      \begingroup
        \ifx\luatexrevision\relax
          \let\luatexrevision\@undefined
        \fi
        \newlinechar=10 %
        \endlinechar=\newlinechar%
        \ifcase0%
            \directlua{%
              if tex.enableprimitives then
                tex.enableprimitives('ifluatex', {'luatexrevision'})
              else
                tex.print('1')
              end
            }%
            \ifx\ifluatexluatexrevision\@undefined 1\fi%
            \relax%
          \global\let\luatexrevision\ifluatexluatexrevision%
        \fi%
      \endgroup%
    \fi
    \begingroup\expandafter\expandafter\expandafter\endgroup
    \expandafter\ifx\csname luatexrevision\endcsname\relax
      \ifluatex@Error{%
        Missing \string\luatexrevision%
      }{%
        Update LuaTeX.%
      }%
    \fi
  \fi
\fi
%    \end{macrocode}
%
% \subsection{Protocol entry}
%
%     Log comment:
%    \begin{macrocode}
\begingroup
  \expandafter\ifx\csname PackageInfo\endcsname\relax
    \def\x#1#2{%
      \immediate\write-1{Package #1 Info: #2.}%
    }%
  \else
    \let\x\PackageInfo
    \expandafter\let\csname on@line\endcsname\empty
  \fi
  \x{ifluatex}{LuaTeX \ifluatex\else not \fi detected}%
\endgroup
%    \end{macrocode}
%    \begin{macrocode}
\ifluatex@AtEnd%
%    \end{macrocode}
%    \begin{macrocode}
%</package>
%    \end{macrocode}
%
% \section{Test}
%
% \subsection{Catcode checks for loading}
%
%    \begin{macrocode}
%<*test1>
%    \end{macrocode}
%    \begin{macrocode}
\catcode`\{=1 %
\catcode`\}=2 %
\catcode`\#=6 %
\catcode`\@=11 %
\expandafter\ifx\csname count@\endcsname\relax
  \countdef\count@=255 %
\fi
\expandafter\ifx\csname @gobble\endcsname\relax
  \long\def\@gobble#1{}%
\fi
\expandafter\ifx\csname @firstofone\endcsname\relax
  \long\def\@firstofone#1{#1}%
\fi
\expandafter\ifx\csname loop\endcsname\relax
  \expandafter\@firstofone
\else
  \expandafter\@gobble
\fi
{%
  \def\loop#1\repeat{%
    \def\body{#1}%
    \iterate
  }%
  \def\iterate{%
    \body
      \let\next\iterate
    \else
      \let\next\relax
    \fi
    \next
  }%
  \let\repeat=\fi
}%
\def\RestoreCatcodes{}
\count@=0 %
\loop
  \edef\RestoreCatcodes{%
    \RestoreCatcodes
    \catcode\the\count@=\the\catcode\count@\relax
  }%
\ifnum\count@<255 %
  \advance\count@ 1 %
\repeat

\def\RangeCatcodeInvalid#1#2{%
  \count@=#1\relax
  \loop
    \catcode\count@=15 %
  \ifnum\count@<#2\relax
    \advance\count@ 1 %
  \repeat
}
\def\RangeCatcodeCheck#1#2#3{%
  \count@=#1\relax
  \loop
    \ifnum#3=\catcode\count@
    \else
      \errmessage{%
        Character \the\count@\space
        with wrong catcode \the\catcode\count@\space
        instead of \number#3%
      }%
    \fi
  \ifnum\count@<#2\relax
    \advance\count@ 1 %
  \repeat
}
\def\space{ }
\expandafter\ifx\csname LoadCommand\endcsname\relax
  \def\LoadCommand{\input ifluatex.sty\relax}%
\fi
\def\Test{%
  \RangeCatcodeInvalid{0}{47}%
  \RangeCatcodeInvalid{58}{64}%
  \RangeCatcodeInvalid{91}{96}%
  \RangeCatcodeInvalid{123}{255}%
  \catcode`\@=12 %
  \catcode`\\=0 %
  \catcode`\%=14 %
  \LoadCommand
  \RangeCatcodeCheck{0}{36}{15}%
  \RangeCatcodeCheck{37}{37}{14}%
  \RangeCatcodeCheck{38}{47}{15}%
  \RangeCatcodeCheck{48}{57}{12}%
  \RangeCatcodeCheck{58}{63}{15}%
  \RangeCatcodeCheck{64}{64}{12}%
  \RangeCatcodeCheck{65}{90}{11}%
  \RangeCatcodeCheck{91}{91}{15}%
  \RangeCatcodeCheck{92}{92}{0}%
  \RangeCatcodeCheck{93}{96}{15}%
  \RangeCatcodeCheck{97}{122}{11}%
  \RangeCatcodeCheck{123}{255}{15}%
  \RestoreCatcodes
}
\Test
\csname @@end\endcsname
\end
%    \end{macrocode}
%    \begin{macrocode}
%</test1>
%    \end{macrocode}
%
% \section{Reload check for plain}
%
%    \begin{macrocode}
%<*test-reload1>
\input ifluatex.sty\relax
\input ifluatex.sty\relax
\csname @@end\endcsname\end
%</test-reload1>
%    \end{macrocode}
%
%    \begin{macrocode}
%<*test-reload2>
\input miniltx.tex\relax
\input ifluatex.sty\relax
\input ifluatex.sty\relax
\csname @@end\endcsname\end
%</test-reload2>
%    \end{macrocode}
%
% \section{Installation}
%
% \subsection{Download}
%
% \paragraph{Package.} This package is available on
% CTAN\footnote{\url{http://ctan.org/pkg/ifluatex}}:
% \begin{description}
% \item[\CTAN{macros/latex/contrib/oberdiek/ifluatex.dtx}] The source file.
% \item[\CTAN{macros/latex/contrib/oberdiek/ifluatex.pdf}] Documentation.
% \end{description}
%
%
% \paragraph{Bundle.} All the packages of the bundle `oberdiek'
% are also available in a TDS compliant ZIP archive. There
% the packages are already unpacked and the documentation files
% are generated. The files and directories obey the TDS standard.
% \begin{description}
% \item[\CTAN{install/macros/latex/contrib/oberdiek.tds.zip}]
% \end{description}
% \emph{TDS} refers to the standard ``A Directory Structure
% for \TeX\ Files'' (\CTAN{tds/tds.pdf}). Directories
% with \xfile{texmf} in their name are usually organized this way.
%
% \subsection{Bundle installation}
%
% \paragraph{Unpacking.} Unpack the \xfile{oberdiek.tds.zip} in the
% TDS tree (also known as \xfile{texmf} tree) of your choice.
% Example (linux):
% \begin{quote}
%   |unzip oberdiek.tds.zip -d ~/texmf|
% \end{quote}
%
% \paragraph{Script installation.}
% Check the directory \xfile{TDS:scripts/oberdiek/} for
% scripts that need further installation steps.
% Package \xpackage{attachfile2} comes with the Perl script
% \xfile{pdfatfi.pl} that should be installed in such a way
% that it can be called as \texttt{pdfatfi}.
% Example (linux):
% \begin{quote}
%   |chmod +x scripts/oberdiek/pdfatfi.pl|\\
%   |cp scripts/oberdiek/pdfatfi.pl /usr/local/bin/|
% \end{quote}
%
% \subsection{Package installation}
%
% \paragraph{Unpacking.} The \xfile{.dtx} file is a self-extracting
% \docstrip\ archive. The files are extracted by running the
% \xfile{.dtx} through \plainTeX:
% \begin{quote}
%   \verb|tex ifluatex.dtx|
% \end{quote}
%
% \paragraph{TDS.} Now the different files must be moved into
% the different directories in your installation TDS tree
% (also known as \xfile{texmf} tree):
% \begin{quote}
% \def\t{^^A
% \begin{tabular}{@{}>{\ttfamily}l@{ $\rightarrow$ }>{\ttfamily}l@{}}
%   ifluatex.sty & tex/generic/oberdiek/ifluatex.sty\\
%   ifluatex.pdf & doc/latex/oberdiek/ifluatex.pdf\\
%   test/ifluatex-test1.tex & doc/latex/oberdiek/test/ifluatex-test1.tex\\
%   test/ifluatex-test2.tex & doc/latex/oberdiek/test/ifluatex-test2.tex\\
%   test/ifluatex-test3.tex & doc/latex/oberdiek/test/ifluatex-test3.tex\\
%   ifluatex.dtx & source/latex/oberdiek/ifluatex.dtx\\
% \end{tabular}^^A
% }^^A
% \sbox0{\t}^^A
% \ifdim\wd0>\linewidth
%   \begingroup
%     \advance\linewidth by\leftmargin
%     \advance\linewidth by\rightmargin
%   \edef\x{\endgroup
%     \def\noexpand\lw{\the\linewidth}^^A
%   }\x
%   \def\lwbox{^^A
%     \leavevmode
%     \hbox to \linewidth{^^A
%       \kern-\leftmargin\relax
%       \hss
%       \usebox0
%       \hss
%       \kern-\rightmargin\relax
%     }^^A
%   }^^A
%   \ifdim\wd0>\lw
%     \sbox0{\small\t}^^A
%     \ifdim\wd0>\linewidth
%       \ifdim\wd0>\lw
%         \sbox0{\footnotesize\t}^^A
%         \ifdim\wd0>\linewidth
%           \ifdim\wd0>\lw
%             \sbox0{\scriptsize\t}^^A
%             \ifdim\wd0>\linewidth
%               \ifdim\wd0>\lw
%                 \sbox0{\tiny\t}^^A
%                 \ifdim\wd0>\linewidth
%                   \lwbox
%                 \else
%                   \usebox0
%                 \fi
%               \else
%                 \lwbox
%               \fi
%             \else
%               \usebox0
%             \fi
%           \else
%             \lwbox
%           \fi
%         \else
%           \usebox0
%         \fi
%       \else
%         \lwbox
%       \fi
%     \else
%       \usebox0
%     \fi
%   \else
%     \lwbox
%   \fi
% \else
%   \usebox0
% \fi
% \end{quote}
% If you have a \xfile{docstrip.cfg} that configures and enables \docstrip's
% TDS installing feature, then some files can already be in the right
% place, see the documentation of \docstrip.
%
% \subsection{Refresh file name databases}
%
% If your \TeX~distribution
% (\teTeX, \mikTeX, \dots) relies on file name databases, you must refresh
% these. For example, \teTeX\ users run \verb|texhash| or
% \verb|mktexlsr|.
%
% \subsection{Some details for the interested}
%
% \paragraph{Attached source.}
%
% The PDF documentation on CTAN also includes the
% \xfile{.dtx} source file. It can be extracted by
% AcrobatReader 6 or higher. Another option is \textsf{pdftk},
% e.g. unpack the file into the current directory:
% \begin{quote}
%   \verb|pdftk ifluatex.pdf unpack_files output .|
% \end{quote}
%
% \paragraph{Unpacking with \LaTeX.}
% The \xfile{.dtx} chooses its action depending on the format:
% \begin{description}
% \item[\plainTeX:] Run \docstrip\ and extract the files.
% \item[\LaTeX:] Generate the documentation.
% \end{description}
% If you insist on using \LaTeX\ for \docstrip\ (really,
% \docstrip\ does not need \LaTeX), then inform the autodetect routine
% about your intention:
% \begin{quote}
%   \verb|latex \let\install=y\input{ifluatex.dtx}|
% \end{quote}
% Do not forget to quote the argument according to the demands
% of your shell.
%
% \paragraph{Generating the documentation.}
% You can use both the \xfile{.dtx} or the \xfile{.drv} to generate
% the documentation. The process can be configured by the
% configuration file \xfile{ltxdoc.cfg}. For instance, put this
% line into this file, if you want to have A4 as paper format:
% \begin{quote}
%   \verb|\PassOptionsToClass{a4paper}{article}|
% \end{quote}
% An example follows how to generate the
% documentation with pdf\LaTeX:
% \begin{quote}
%\begin{verbatim}
%pdflatex ifluatex.dtx
%makeindex -s gind.ist ifluatex.idx
%pdflatex ifluatex.dtx
%makeindex -s gind.ist ifluatex.idx
%pdflatex ifluatex.dtx
%\end{verbatim}
% \end{quote}
%
% \section{Catalogue}
%
% The following XML file can be used as source for the
% \href{http://mirror.ctan.org/help/Catalogue/catalogue.html}{\TeX\ Catalogue}.
% The elements \texttt{caption} and \texttt{description} are imported
% from the original XML file from the Catalogue.
% The name of the XML file in the Catalogue is \xfile{ifluatex.xml}.
%    \begin{macrocode}
%<*catalogue>
<?xml version='1.0' encoding='us-ascii'?>
<!DOCTYPE entry SYSTEM 'catalogue.dtd'>
<entry datestamp='$Date$' modifier='$Author$' id='ifluatex'>
  <name>ifluatex</name>
  <caption>Provides the \ifluatex switch.</caption>
  <authorref id='auth:oberdiek'/>
  <copyright owner='Heiko Oberdiek' year='2007,2009,2010'/>
  <license type='lppl1.3'/>
  <version number='1.4'/>
  <description>
    The package looks for  LuaTeX regardless of its mode and provides
    the switch <tt>\ifluatex</tt>; it works with Plain TeX or LaTeX.
    <p/>
    The package is part of the <xref refid='oberdiek'>oberdiek</xref>
    bundle.
  </description>
  <documentation details='Package documentation'
      href='ctan:/macros/latex/contrib/oberdiek/ifluatex.pdf'/>
  <ctan file='true' path='/macros/latex/contrib/oberdiek/ifluatex.dtx'/>
  <miktex location='oberdiek'/>
  <texlive location='ifluatex'/>
  <install path='/macros/latex/contrib/oberdiek/oberdiek.tds.zip'/>
</entry>
%</catalogue>
%    \end{macrocode}
%
% \begin{History}
%   \begin{Version}{2007/12/12 v1.0}
%   \item
%     First public version.
%   \end{Version}
%   \begin{Version}{2009/04/10 v1.1}
%   \item
%     Test adopted for \LuaTeX\ 0.39.
%   \item
%     Makes \cs{luatexversion} available.
%   \end{Version}
%   \begin{Version}{2009/04/17 v1.2}
%   \item
%     Fixes (Manuel P\'egouri\'e-Gonnard).
%   \item
%     \cs{luatextrue} and \cs{luatexfalse} are no longer defined.
%   \item
%     Makes \cs{luatexrevision} available, too.
%   \end{Version}
%   \begin{Version}{2010/03/01 v1.3}
%   \item
%     Line ends fixed in case \cs{endlinechar} = \cs{newlinechar}.
%   \end{Version}
%   \begin{Version}{2016/05/16 v1.4}
%   \item
%     Documentation updates.
%   \end{Version}
% \end{History}
%
% \PrintIndex
%
% \Finale
\endinput
|
% \end{quote}
% Do not forget to quote the argument according to the demands
% of your shell.
%
% \paragraph{Generating the documentation.}
% You can use both the \xfile{.dtx} or the \xfile{.drv} to generate
% the documentation. The process can be configured by the
% configuration file \xfile{ltxdoc.cfg}. For instance, put this
% line into this file, if you want to have A4 as paper format:
% \begin{quote}
%   \verb|\PassOptionsToClass{a4paper}{article}|
% \end{quote}
% An example follows how to generate the
% documentation with pdf\LaTeX:
% \begin{quote}
%\begin{verbatim}
%pdflatex ifluatex.dtx
%makeindex -s gind.ist ifluatex.idx
%pdflatex ifluatex.dtx
%makeindex -s gind.ist ifluatex.idx
%pdflatex ifluatex.dtx
%\end{verbatim}
% \end{quote}
%
% \section{Catalogue}
%
% The following XML file can be used as source for the
% \href{http://mirror.ctan.org/help/Catalogue/catalogue.html}{\TeX\ Catalogue}.
% The elements \texttt{caption} and \texttt{description} are imported
% from the original XML file from the Catalogue.
% The name of the XML file in the Catalogue is \xfile{ifluatex.xml}.
%    \begin{macrocode}
%<*catalogue>
<?xml version='1.0' encoding='us-ascii'?>
<!DOCTYPE entry SYSTEM 'catalogue.dtd'>
<entry datestamp='$Date$' modifier='$Author$' id='ifluatex'>
  <name>ifluatex</name>
  <caption>Provides the \ifluatex switch.</caption>
  <authorref id='auth:oberdiek'/>
  <copyright owner='Heiko Oberdiek' year='2007,2009,2010'/>
  <license type='lppl1.3'/>
  <version number='1.4'/>
  <description>
    The package looks for  LuaTeX regardless of its mode and provides
    the switch <tt>\ifluatex</tt>; it works with Plain TeX or LaTeX.
    <p/>
    The package is part of the <xref refid='oberdiek'>oberdiek</xref>
    bundle.
  </description>
  <documentation details='Package documentation'
      href='ctan:/macros/latex/contrib/oberdiek/ifluatex.pdf'/>
  <ctan file='true' path='/macros/latex/contrib/oberdiek/ifluatex.dtx'/>
  <miktex location='oberdiek'/>
  <texlive location='ifluatex'/>
  <install path='/macros/latex/contrib/oberdiek/oberdiek.tds.zip'/>
</entry>
%</catalogue>
%    \end{macrocode}
%
% \begin{History}
%   \begin{Version}{2007/12/12 v1.0}
%   \item
%     First public version.
%   \end{Version}
%   \begin{Version}{2009/04/10 v1.1}
%   \item
%     Test adopted for \LuaTeX\ 0.39.
%   \item
%     Makes \cs{luatexversion} available.
%   \end{Version}
%   \begin{Version}{2009/04/17 v1.2}
%   \item
%     Fixes (Manuel P\'egouri\'e-Gonnard).
%   \item
%     \cs{luatextrue} and \cs{luatexfalse} are no longer defined.
%   \item
%     Makes \cs{luatexrevision} available, too.
%   \end{Version}
%   \begin{Version}{2010/03/01 v1.3}
%   \item
%     Line ends fixed in case \cs{endlinechar} = \cs{newlinechar}.
%   \end{Version}
%   \begin{Version}{2016/05/16 v1.4}
%   \item
%     Documentation updates.
%   \end{Version}
% \end{History}
%
% \PrintIndex
%
% \Finale
\endinput

%        (quote the arguments according to the demands of your shell)
%
% Documentation:
%    (a) If ifluatex.drv is present:
%           latex ifluatex.drv
%    (b) Without ifluatex.drv:
%           latex ifluatex.dtx; ...
%    The class ltxdoc loads the configuration file ltxdoc.cfg
%    if available. Here you can specify further options, e.g.
%    use A4 as paper format:
%       \PassOptionsToClass{a4paper}{article}
%
%    Programm calls to get the documentation (example):
%       pdflatex ifluatex.dtx
%       makeindex -s gind.ist ifluatex.idx
%       pdflatex ifluatex.dtx
%       makeindex -s gind.ist ifluatex.idx
%       pdflatex ifluatex.dtx
%
% Installation:
%    TDS:tex/generic/oberdiek/ifluatex.sty
%    TDS:doc/latex/oberdiek/ifluatex.pdf
%    TDS:doc/latex/oberdiek/test/ifluatex-test1.tex
%    TDS:doc/latex/oberdiek/test/ifluatex-test2.tex
%    TDS:doc/latex/oberdiek/test/ifluatex-test3.tex
%    TDS:source/latex/oberdiek/ifluatex.dtx
%
%<*ignore>
\begingroup
  \catcode123=1 %
  \catcode125=2 %
  \def\x{LaTeX2e}%
\expandafter\endgroup
\ifcase 0\ifx\install y1\fi\expandafter
         \ifx\csname processbatchFile\endcsname\relax\else1\fi
         \ifx\fmtname\x\else 1\fi\relax
\else\csname fi\endcsname
%</ignore>
%<*install>
\input docstrip.tex
\Msg{************************************************************************}
\Msg{* Installation}
\Msg{* Package: ifluatex 2016/05/16 v1.4 Provides the ifluatex switch (HO)}
\Msg{************************************************************************}

\keepsilent
\askforoverwritefalse

\let\MetaPrefix\relax
\preamble

This is a generated file.

Project: ifluatex
Version: 2016/05/16 v1.4

Copyright (C) 2007, 2009, 2010 by
   Heiko Oberdiek <heiko.oberdiek at googlemail.com>

This work may be distributed and/or modified under the
conditions of the LaTeX Project Public License, either
version 1.3c of this license or (at your option) any later
version. This version of this license is in
   http://www.latex-project.org/lppl/lppl-1-3c.txt
and the latest version of this license is in
   http://www.latex-project.org/lppl.txt
and version 1.3 or later is part of all distributions of
LaTeX version 2005/12/01 or later.

This work has the LPPL maintenance status "maintained".

This Current Maintainer of this work is Heiko Oberdiek.

The Base Interpreter refers to any `TeX-Format',
because some files are installed in TDS:tex/generic//.

This work consists of the main source file ifluatex.dtx
and the derived files
   ifluatex.sty, ifluatex.pdf, ifluatex.ins, ifluatex.drv,
   ifluatex-test1.tex, ifluatex-test2.tex, ifluatex-test3.tex.

\endpreamble
\let\MetaPrefix\DoubleperCent

\generate{%
  \file{ifluatex.ins}{\from{ifluatex.dtx}{install}}%
  \file{ifluatex.drv}{\from{ifluatex.dtx}{driver}}%
  \usedir{tex/generic/oberdiek}%
  \file{ifluatex.sty}{\from{ifluatex.dtx}{package}}%
  \usedir{doc/latex/oberdiek/test}%
  \file{ifluatex-test1.tex}{\from{ifluatex.dtx}{test1}}%
  \file{ifluatex-test2.tex}{\from{ifluatex.dtx}{test-reload1}}%
  \file{ifluatex-test3.tex}{\from{ifluatex.dtx}{test-reload2}}%
  \nopreamble
  \nopostamble
  \usedir{source/latex/oberdiek/catalogue}%
  \file{ifluatex.xml}{\from{ifluatex.dtx}{catalogue}}%
}

\catcode32=13\relax% active space
\let =\space%
\Msg{************************************************************************}
\Msg{*}
\Msg{* To finish the installation you have to move the following}
\Msg{* file into a directory searched by TeX:}
\Msg{*}
\Msg{*     ifluatex.sty}
\Msg{*}
\Msg{* To produce the documentation run the file `ifluatex.drv'}
\Msg{* through LaTeX.}
\Msg{*}
\Msg{* Happy TeXing!}
\Msg{*}
\Msg{************************************************************************}

\endbatchfile
%</install>
%<*ignore>
\fi
%</ignore>
%<*driver>
\NeedsTeXFormat{LaTeX2e}
\ProvidesFile{ifluatex.drv}%
  [2016/05/16 v1.4 Provides the ifluatex switch (HO)]%
\documentclass{ltxdoc}
\usepackage{holtxdoc}[2011/11/22]
\begin{document}
  \DocInput{ifluatex.dtx}%
\end{document}
%</driver>
% \fi
%
%
% \CharacterTable
%  {Upper-case    \A\B\C\D\E\F\G\H\I\J\K\L\M\N\O\P\Q\R\S\T\U\V\W\X\Y\Z
%   Lower-case    \a\b\c\d\e\f\g\h\i\j\k\l\m\n\o\p\q\r\s\t\u\v\w\x\y\z
%   Digits        \0\1\2\3\4\5\6\7\8\9
%   Exclamation   \!     Double quote  \"     Hash (number) \#
%   Dollar        \$     Percent       \%     Ampersand     \&
%   Acute accent  \'     Left paren    \(     Right paren   \)
%   Asterisk      \*     Plus          \+     Comma         \,
%   Minus         \-     Point         \.     Solidus       \/
%   Colon         \:     Semicolon     \;     Less than     \<
%   Equals        \=     Greater than  \>     Question mark \?
%   Commercial at \@     Left bracket  \[     Backslash     \\
%   Right bracket \]     Circumflex    \^     Underscore    \_
%   Grave accent  \`     Left brace    \{     Vertical bar  \|
%   Right brace   \}     Tilde         \~}
%
% \GetFileInfo{ifluatex.drv}
%
% \title{The \xpackage{ifluatex} package}
% \date{2016/05/16 v1.4}
% \author{Heiko Oberdiek\thanks
% {Please report any issues at https://github.com/ho-tex/oberdiek/issues}\\
% \xemail{heiko.oberdiek at googlemail.com}}
%
% \maketitle
%
% \begin{abstract}
% This package looks for \LuaTeX\ regardless of its mode
% and provides the switch \cs{ifluatex}. Also it makes
% \cs{luatexversion} available if it is not present.
% It works with \plainTeX\ or \LaTeX.
% \end{abstract}
%
% \tableofcontents
%
% \section{Documentation}
%
% The package \xpackage{ifluatex} can be used with both \plainTeX\
% and \LaTeX:
% \begin{description}
% \item[\plainTeX:] |\input ifluatex.sty|
% \item[\LaTeXe:]   |\usepackage{ifluatex}|
% \end{description}
%
% \DescribeMacro{\ifluatex}
% The package provides the switch \cs{ifluatex}:
% \begin{quote}
%   |\ifluatex|\\
%   \hspace{1.5em}\LuaTeX\ is running\\
%   |\else|\\
%   \hspace{1.5em}Without \LuaTeX\\
%   |\fi|
% \end{quote}
%
% Since version 0.39 \LuaTeX\ only provides \cs{directlua} at startup
% time. Also the syntax of \cs{directlua} changed in version 0.36.
% Thus the user might want to check the LuaTeX version.
% Therefore this package also makes \cs{luatexversion} and
% \cs{luatexrevision} available, if it is not yet done.
%
% If you want to detect the mode (DVI or PDF), then use package
% \xpackage{ifpdf}. \LuaTeX\ has inherited \cs{pdfoutput} from \pdfTeX.
%
% \StopEventually{
% }
%
% \section{Implementation}
%
%    \begin{macrocode}
%<*package>
%    \end{macrocode}
%
% \subsection{Reload check and package identification}
%    Reload check, especially if the package is not used with \LaTeX.
%    \begin{macrocode}
\begingroup\catcode61\catcode48\catcode32=10\relax%
  \catcode13=5 % ^^M
  \endlinechar=13 %
  \catcode35=6 % #
  \catcode39=12 % '
  \catcode44=12 % ,
  \catcode45=12 % -
  \catcode46=12 % .
  \catcode58=12 % :
  \catcode64=11 % @
  \catcode123=1 % {
  \catcode125=2 % }
  \expandafter\let\expandafter\x\csname ver@ifluatex.sty\endcsname
  \ifx\x\relax % plain-TeX, first loading
  \else
    \def\empty{}%
    \ifx\x\empty % LaTeX, first loading,
      % variable is initialized, but \ProvidesPackage not yet seen
    \else
      \expandafter\ifx\csname PackageInfo\endcsname\relax
        \def\x#1#2{%
          \immediate\write-1{Package #1 Info: #2.}%
        }%
      \else
        \def\x#1#2{\PackageInfo{#1}{#2, stopped}}%
      \fi
      \x{ifluatex}{The package is already loaded}%
      \aftergroup\endinput
    \fi
  \fi
\endgroup%
%    \end{macrocode}
%    Package identification:
%    \begin{macrocode}
\begingroup\catcode61\catcode48\catcode32=10\relax%
  \catcode13=5 % ^^M
  \endlinechar=13 %
  \catcode35=6 % #
  \catcode39=12 % '
  \catcode40=12 % (
  \catcode41=12 % )
  \catcode44=12 % ,
  \catcode45=12 % -
  \catcode46=12 % .
  \catcode47=12 % /
  \catcode58=12 % :
  \catcode64=11 % @
  \catcode91=12 % [
  \catcode93=12 % ]
  \catcode123=1 % {
  \catcode125=2 % }
  \expandafter\ifx\csname ProvidesPackage\endcsname\relax
    \def\x#1#2#3[#4]{\endgroup
      \immediate\write-1{Package: #3 #4}%
      \xdef#1{#4}%
    }%
  \else
    \def\x#1#2[#3]{\endgroup
      #2[{#3}]%
      \ifx#1\@undefined
        \xdef#1{#3}%
      \fi
      \ifx#1\relax
        \xdef#1{#3}%
      \fi
    }%
  \fi
\expandafter\x\csname ver@ifluatex.sty\endcsname
\ProvidesPackage{ifluatex}%
  [2016/05/16 v1.4 Provides the ifluatex switch (HO)]%
%    \end{macrocode}
%
% \subsection{Catcodes}
%
%    \begin{macrocode}
\begingroup\catcode61\catcode48\catcode32=10\relax%
  \catcode13=5 % ^^M
  \endlinechar=13 %
  \catcode123=1 % {
  \catcode125=2 % }
  \catcode64=11 % @
  \def\x{\endgroup
    \expandafter\edef\csname ifluatex@AtEnd\endcsname{%
      \endlinechar=\the\endlinechar\relax
      \catcode13=\the\catcode13\relax
      \catcode32=\the\catcode32\relax
      \catcode35=\the\catcode35\relax
      \catcode61=\the\catcode61\relax
      \catcode64=\the\catcode64\relax
      \catcode123=\the\catcode123\relax
      \catcode125=\the\catcode125\relax
    }%
  }%
\x\catcode61\catcode48\catcode32=10\relax%
\catcode13=5 % ^^M
\endlinechar=13 %
\catcode35=6 % #
\catcode64=11 % @
\catcode123=1 % {
\catcode125=2 % }
\def\TMP@EnsureCode#1#2{%
  \edef\ifluatex@AtEnd{%
    \ifluatex@AtEnd
    \catcode#1=\the\catcode#1\relax
  }%
  \catcode#1=#2\relax
}
\TMP@EnsureCode{10}{12}% ^^J
\TMP@EnsureCode{39}{12}% '
\TMP@EnsureCode{40}{12}% (
\TMP@EnsureCode{41}{12}% )
\TMP@EnsureCode{44}{12}% ,
\TMP@EnsureCode{45}{12}% -
\TMP@EnsureCode{46}{12}% .
\TMP@EnsureCode{47}{12}% /
\TMP@EnsureCode{58}{12}% :
\TMP@EnsureCode{60}{12}% <
\TMP@EnsureCode{94}{7}% ^
\TMP@EnsureCode{96}{12}% `
\edef\ifluatex@AtEnd{\ifluatex@AtEnd\noexpand\endinput}
%    \end{macrocode}
%
% \subsection{Macro for error messages}
%
%    \begin{macro}{\ifluatex@Error}
%    \begin{macrocode}
\begingroup\expandafter\expandafter\expandafter\endgroup
\expandafter\ifx\csname PackageError\endcsname\relax
  \def\ifluatex@Error#1#2{%
    \begingroup
      \newlinechar=10 %
      \def\MessageBreak{^^J}%
      \edef\x{\errhelp{#2}}%
      \x
      \errmessage{Package ifluatex Error: #1}%
    \endgroup
  }%
\else
  \def\ifluatex@Error{%
    \PackageError{ifluatex}%
  }%
\fi
%    \end{macrocode}
%    \end{macro}
%
% \subsection{Check for previously defined \cs{ifluatex}}
%
%    \begin{macrocode}
\begingroup
  \expandafter\ifx\csname ifluatex\endcsname\relax
  \else
    \edef\i/{\expandafter\string\csname ifluatex\endcsname}%
    \ifluatex@Error{Name clash, \i/ is already defined}{%
      Incompatible versions of \i/ can cause problems,\MessageBreak
      therefore package loading is aborted.%
    }%
    \endgroup
    \expandafter\ifluatex@AtEnd
  \fi%
\endgroup
%    \end{macrocode}
%
% \subsection{\cs{ifluatex}}
%
%    \begin{macro}{\ifluatex}
%    \begin{macrocode}
\let\ifluatex\iffalse
%    \end{macrocode}
%    \end{macro}
%
%    Test \cs{luatexversion}. Is it  defined and different from
%    \cs{relax}? Someone could have used \LaTeX\ internal
%    \cs{@ifundefined}, or something else involving.
%    Notice, \cs{csname} is executed inside a group for the test
%    to cancel the side effect of \cs{csname}.
%    \begin{macrocode}
\begingroup\expandafter\expandafter\expandafter\endgroup
\expandafter\ifx\csname luatexversion\endcsname\relax
\else
  \expandafter\let\csname ifluatex\expandafter\endcsname
                  \csname iftrue\endcsname
\fi
%    \end{macrocode}
%
% \subsection{Lua\TeX\ v0.39}
%
%     Starting with version 0.39 \LuaTeX\ wants to provide \cs{directlua}
%     as only primitive at startup time beyond vanilla \TeX's primitives.
%     Then \cs{directlua} exists, but \cs{luatexversion} cannot be found.
%     Unhappily also the syntax of \cs{directlua} changed in v0.36,
%     thus the user would want to check \cs{luatexversion}.
%     Therefore we make \cs{luatexversion} available using
%     \LuaTeX's Lua function |tex.enableprimitives|.
%
%    \begin{macrocode}
\ifluatex
\else
  \begingroup\expandafter\expandafter\expandafter\endgroup
  \expandafter\ifx\csname directlua\endcsname\relax
  \else
    \expandafter\let\csname ifluatex\expandafter\endcsname
                    \csname iftrue\endcsname
    \begingroup
      \newlinechar=10 %
      \endlinechar=\newlinechar%
      \ifnum0%
          \directlua{%
            if tex.enableprimitives then
              tex.enableprimitives('ifluatex', {'luatexversion'})
              tex.print('1')
            end
          }%
          \ifx\ifluatexluatexversion\@undefined\else 1\fi %
          =11 %
        \global\let\luatexversion\ifluatexluatexversion%
      \else%
        \ifluatex@Error{%
          Missing \string\luatexversion%
        }{%
          Update LuaTeX.%
        }%
      \fi%
    \endgroup%
  \fi
\fi
%    \end{macrocode}
%    \begin{macrocode}
\ifluatex
  \begingroup\expandafter\expandafter\expandafter\endgroup
  \expandafter\ifx\csname luatexrevision\endcsname\relax
    \ifnum\luatexversion<36 %
    \else
      \begingroup
        \ifx\luatexrevision\relax
          \let\luatexrevision\@undefined
        \fi
        \newlinechar=10 %
        \endlinechar=\newlinechar%
        \ifcase0%
            \directlua{%
              if tex.enableprimitives then
                tex.enableprimitives('ifluatex', {'luatexrevision'})
              else
                tex.print('1')
              end
            }%
            \ifx\ifluatexluatexrevision\@undefined 1\fi%
            \relax%
          \global\let\luatexrevision\ifluatexluatexrevision%
        \fi%
      \endgroup%
    \fi
    \begingroup\expandafter\expandafter\expandafter\endgroup
    \expandafter\ifx\csname luatexrevision\endcsname\relax
      \ifluatex@Error{%
        Missing \string\luatexrevision%
      }{%
        Update LuaTeX.%
      }%
    \fi
  \fi
\fi
%    \end{macrocode}
%
% \subsection{Protocol entry}
%
%     Log comment:
%    \begin{macrocode}
\begingroup
  \expandafter\ifx\csname PackageInfo\endcsname\relax
    \def\x#1#2{%
      \immediate\write-1{Package #1 Info: #2.}%
    }%
  \else
    \let\x\PackageInfo
    \expandafter\let\csname on@line\endcsname\empty
  \fi
  \x{ifluatex}{LuaTeX \ifluatex\else not \fi detected}%
\endgroup
%    \end{macrocode}
%    \begin{macrocode}
\ifluatex@AtEnd%
%    \end{macrocode}
%    \begin{macrocode}
%</package>
%    \end{macrocode}
%
% \section{Test}
%
% \subsection{Catcode checks for loading}
%
%    \begin{macrocode}
%<*test1>
%    \end{macrocode}
%    \begin{macrocode}
\catcode`\{=1 %
\catcode`\}=2 %
\catcode`\#=6 %
\catcode`\@=11 %
\expandafter\ifx\csname count@\endcsname\relax
  \countdef\count@=255 %
\fi
\expandafter\ifx\csname @gobble\endcsname\relax
  \long\def\@gobble#1{}%
\fi
\expandafter\ifx\csname @firstofone\endcsname\relax
  \long\def\@firstofone#1{#1}%
\fi
\expandafter\ifx\csname loop\endcsname\relax
  \expandafter\@firstofone
\else
  \expandafter\@gobble
\fi
{%
  \def\loop#1\repeat{%
    \def\body{#1}%
    \iterate
  }%
  \def\iterate{%
    \body
      \let\next\iterate
    \else
      \let\next\relax
    \fi
    \next
  }%
  \let\repeat=\fi
}%
\def\RestoreCatcodes{}
\count@=0 %
\loop
  \edef\RestoreCatcodes{%
    \RestoreCatcodes
    \catcode\the\count@=\the\catcode\count@\relax
  }%
\ifnum\count@<255 %
  \advance\count@ 1 %
\repeat

\def\RangeCatcodeInvalid#1#2{%
  \count@=#1\relax
  \loop
    \catcode\count@=15 %
  \ifnum\count@<#2\relax
    \advance\count@ 1 %
  \repeat
}
\def\RangeCatcodeCheck#1#2#3{%
  \count@=#1\relax
  \loop
    \ifnum#3=\catcode\count@
    \else
      \errmessage{%
        Character \the\count@\space
        with wrong catcode \the\catcode\count@\space
        instead of \number#3%
      }%
    \fi
  \ifnum\count@<#2\relax
    \advance\count@ 1 %
  \repeat
}
\def\space{ }
\expandafter\ifx\csname LoadCommand\endcsname\relax
  \def\LoadCommand{\input ifluatex.sty\relax}%
\fi
\def\Test{%
  \RangeCatcodeInvalid{0}{47}%
  \RangeCatcodeInvalid{58}{64}%
  \RangeCatcodeInvalid{91}{96}%
  \RangeCatcodeInvalid{123}{255}%
  \catcode`\@=12 %
  \catcode`\\=0 %
  \catcode`\%=14 %
  \LoadCommand
  \RangeCatcodeCheck{0}{36}{15}%
  \RangeCatcodeCheck{37}{37}{14}%
  \RangeCatcodeCheck{38}{47}{15}%
  \RangeCatcodeCheck{48}{57}{12}%
  \RangeCatcodeCheck{58}{63}{15}%
  \RangeCatcodeCheck{64}{64}{12}%
  \RangeCatcodeCheck{65}{90}{11}%
  \RangeCatcodeCheck{91}{91}{15}%
  \RangeCatcodeCheck{92}{92}{0}%
  \RangeCatcodeCheck{93}{96}{15}%
  \RangeCatcodeCheck{97}{122}{11}%
  \RangeCatcodeCheck{123}{255}{15}%
  \RestoreCatcodes
}
\Test
\csname @@end\endcsname
\end
%    \end{macrocode}
%    \begin{macrocode}
%</test1>
%    \end{macrocode}
%
% \section{Reload check for plain}
%
%    \begin{macrocode}
%<*test-reload1>
\input ifluatex.sty\relax
\input ifluatex.sty\relax
\csname @@end\endcsname\end
%</test-reload1>
%    \end{macrocode}
%
%    \begin{macrocode}
%<*test-reload2>
\input miniltx.tex\relax
\input ifluatex.sty\relax
\input ifluatex.sty\relax
\csname @@end\endcsname\end
%</test-reload2>
%    \end{macrocode}
%
% \section{Installation}
%
% \subsection{Download}
%
% \paragraph{Package.} This package is available on
% CTAN\footnote{\url{http://ctan.org/pkg/ifluatex}}:
% \begin{description}
% \item[\CTAN{macros/latex/contrib/oberdiek/ifluatex.dtx}] The source file.
% \item[\CTAN{macros/latex/contrib/oberdiek/ifluatex.pdf}] Documentation.
% \end{description}
%
%
% \paragraph{Bundle.} All the packages of the bundle `oberdiek'
% are also available in a TDS compliant ZIP archive. There
% the packages are already unpacked and the documentation files
% are generated. The files and directories obey the TDS standard.
% \begin{description}
% \item[\CTAN{install/macros/latex/contrib/oberdiek.tds.zip}]
% \end{description}
% \emph{TDS} refers to the standard ``A Directory Structure
% for \TeX\ Files'' (\CTAN{tds/tds.pdf}). Directories
% with \xfile{texmf} in their name are usually organized this way.
%
% \subsection{Bundle installation}
%
% \paragraph{Unpacking.} Unpack the \xfile{oberdiek.tds.zip} in the
% TDS tree (also known as \xfile{texmf} tree) of your choice.
% Example (linux):
% \begin{quote}
%   |unzip oberdiek.tds.zip -d ~/texmf|
% \end{quote}
%
% \paragraph{Script installation.}
% Check the directory \xfile{TDS:scripts/oberdiek/} for
% scripts that need further installation steps.
% Package \xpackage{attachfile2} comes with the Perl script
% \xfile{pdfatfi.pl} that should be installed in such a way
% that it can be called as \texttt{pdfatfi}.
% Example (linux):
% \begin{quote}
%   |chmod +x scripts/oberdiek/pdfatfi.pl|\\
%   |cp scripts/oberdiek/pdfatfi.pl /usr/local/bin/|
% \end{quote}
%
% \subsection{Package installation}
%
% \paragraph{Unpacking.} The \xfile{.dtx} file is a self-extracting
% \docstrip\ archive. The files are extracted by running the
% \xfile{.dtx} through \plainTeX:
% \begin{quote}
%   \verb|tex ifluatex.dtx|
% \end{quote}
%
% \paragraph{TDS.} Now the different files must be moved into
% the different directories in your installation TDS tree
% (also known as \xfile{texmf} tree):
% \begin{quote}
% \def\t{^^A
% \begin{tabular}{@{}>{\ttfamily}l@{ $\rightarrow$ }>{\ttfamily}l@{}}
%   ifluatex.sty & tex/generic/oberdiek/ifluatex.sty\\
%   ifluatex.pdf & doc/latex/oberdiek/ifluatex.pdf\\
%   test/ifluatex-test1.tex & doc/latex/oberdiek/test/ifluatex-test1.tex\\
%   test/ifluatex-test2.tex & doc/latex/oberdiek/test/ifluatex-test2.tex\\
%   test/ifluatex-test3.tex & doc/latex/oberdiek/test/ifluatex-test3.tex\\
%   ifluatex.dtx & source/latex/oberdiek/ifluatex.dtx\\
% \end{tabular}^^A
% }^^A
% \sbox0{\t}^^A
% \ifdim\wd0>\linewidth
%   \begingroup
%     \advance\linewidth by\leftmargin
%     \advance\linewidth by\rightmargin
%   \edef\x{\endgroup
%     \def\noexpand\lw{\the\linewidth}^^A
%   }\x
%   \def\lwbox{^^A
%     \leavevmode
%     \hbox to \linewidth{^^A
%       \kern-\leftmargin\relax
%       \hss
%       \usebox0
%       \hss
%       \kern-\rightmargin\relax
%     }^^A
%   }^^A
%   \ifdim\wd0>\lw
%     \sbox0{\small\t}^^A
%     \ifdim\wd0>\linewidth
%       \ifdim\wd0>\lw
%         \sbox0{\footnotesize\t}^^A
%         \ifdim\wd0>\linewidth
%           \ifdim\wd0>\lw
%             \sbox0{\scriptsize\t}^^A
%             \ifdim\wd0>\linewidth
%               \ifdim\wd0>\lw
%                 \sbox0{\tiny\t}^^A
%                 \ifdim\wd0>\linewidth
%                   \lwbox
%                 \else
%                   \usebox0
%                 \fi
%               \else
%                 \lwbox
%               \fi
%             \else
%               \usebox0
%             \fi
%           \else
%             \lwbox
%           \fi
%         \else
%           \usebox0
%         \fi
%       \else
%         \lwbox
%       \fi
%     \else
%       \usebox0
%     \fi
%   \else
%     \lwbox
%   \fi
% \else
%   \usebox0
% \fi
% \end{quote}
% If you have a \xfile{docstrip.cfg} that configures and enables \docstrip's
% TDS installing feature, then some files can already be in the right
% place, see the documentation of \docstrip.
%
% \subsection{Refresh file name databases}
%
% If your \TeX~distribution
% (\teTeX, \mikTeX, \dots) relies on file name databases, you must refresh
% these. For example, \teTeX\ users run \verb|texhash| or
% \verb|mktexlsr|.
%
% \subsection{Some details for the interested}
%
% \paragraph{Attached source.}
%
% The PDF documentation on CTAN also includes the
% \xfile{.dtx} source file. It can be extracted by
% AcrobatReader 6 or higher. Another option is \textsf{pdftk},
% e.g. unpack the file into the current directory:
% \begin{quote}
%   \verb|pdftk ifluatex.pdf unpack_files output .|
% \end{quote}
%
% \paragraph{Unpacking with \LaTeX.}
% The \xfile{.dtx} chooses its action depending on the format:
% \begin{description}
% \item[\plainTeX:] Run \docstrip\ and extract the files.
% \item[\LaTeX:] Generate the documentation.
% \end{description}
% If you insist on using \LaTeX\ for \docstrip\ (really,
% \docstrip\ does not need \LaTeX), then inform the autodetect routine
% about your intention:
% \begin{quote}
%   \verb|latex \let\install=y% \iffalse meta-comment
%
% File: ifluatex.dtx
% Version: 2016/05/16 v1.4
% Info: Provides the ifluatex switch
%
% Copyright (C) 2007, 2009, 2010 by
%    Heiko Oberdiek <heiko.oberdiek at googlemail.com>
%    2016
%    https://github.com/ho-tex/oberdiek/issues
%
% This work may be distributed and/or modified under the
% conditions of the LaTeX Project Public License, either
% version 1.3c of this license or (at your option) any later
% version. This version of this license is in
%    http://www.latex-project.org/lppl/lppl-1-3c.txt
% and the latest version of this license is in
%    http://www.latex-project.org/lppl.txt
% and version 1.3 or later is part of all distributions of
% LaTeX version 2005/12/01 or later.
%
% This work has the LPPL maintenance status "maintained".
%
% This Current Maintainer of this work is Heiko Oberdiek.
%
% The Base Interpreter refers to any `TeX-Format',
% because some files are installed in TDS:tex/generic//.
%
% This work consists of the main source file ifluatex.dtx
% and the derived files
%    ifluatex.sty, ifluatex.pdf, ifluatex.ins, ifluatex.drv,
%    ifluatex-test1.tex, ifluatex-test2.tex, ifluatex-test3.tex.
%
% Distribution:
%    CTAN:macros/latex/contrib/oberdiek/ifluatex.dtx
%    CTAN:macros/latex/contrib/oberdiek/ifluatex.pdf
%
% Unpacking:
%    (a) If ifluatex.ins is present:
%           tex ifluatex.ins
%    (b) Without ifluatex.ins:
%           tex ifluatex.dtx
%    (c) If you insist on using LaTeX
%           latex \let\install=y% \iffalse meta-comment
%
% File: ifluatex.dtx
% Version: 2016/05/16 v1.4
% Info: Provides the ifluatex switch
%
% Copyright (C) 2007, 2009, 2010 by
%    Heiko Oberdiek <heiko.oberdiek at googlemail.com>
%    2016
%    https://github.com/ho-tex/oberdiek/issues
%
% This work may be distributed and/or modified under the
% conditions of the LaTeX Project Public License, either
% version 1.3c of this license or (at your option) any later
% version. This version of this license is in
%    http://www.latex-project.org/lppl/lppl-1-3c.txt
% and the latest version of this license is in
%    http://www.latex-project.org/lppl.txt
% and version 1.3 or later is part of all distributions of
% LaTeX version 2005/12/01 or later.
%
% This work has the LPPL maintenance status "maintained".
%
% This Current Maintainer of this work is Heiko Oberdiek.
%
% The Base Interpreter refers to any `TeX-Format',
% because some files are installed in TDS:tex/generic//.
%
% This work consists of the main source file ifluatex.dtx
% and the derived files
%    ifluatex.sty, ifluatex.pdf, ifluatex.ins, ifluatex.drv,
%    ifluatex-test1.tex, ifluatex-test2.tex, ifluatex-test3.tex.
%
% Distribution:
%    CTAN:macros/latex/contrib/oberdiek/ifluatex.dtx
%    CTAN:macros/latex/contrib/oberdiek/ifluatex.pdf
%
% Unpacking:
%    (a) If ifluatex.ins is present:
%           tex ifluatex.ins
%    (b) Without ifluatex.ins:
%           tex ifluatex.dtx
%    (c) If you insist on using LaTeX
%           latex \let\install=y\input{ifluatex.dtx}
%        (quote the arguments according to the demands of your shell)
%
% Documentation:
%    (a) If ifluatex.drv is present:
%           latex ifluatex.drv
%    (b) Without ifluatex.drv:
%           latex ifluatex.dtx; ...
%    The class ltxdoc loads the configuration file ltxdoc.cfg
%    if available. Here you can specify further options, e.g.
%    use A4 as paper format:
%       \PassOptionsToClass{a4paper}{article}
%
%    Programm calls to get the documentation (example):
%       pdflatex ifluatex.dtx
%       makeindex -s gind.ist ifluatex.idx
%       pdflatex ifluatex.dtx
%       makeindex -s gind.ist ifluatex.idx
%       pdflatex ifluatex.dtx
%
% Installation:
%    TDS:tex/generic/oberdiek/ifluatex.sty
%    TDS:doc/latex/oberdiek/ifluatex.pdf
%    TDS:doc/latex/oberdiek/test/ifluatex-test1.tex
%    TDS:doc/latex/oberdiek/test/ifluatex-test2.tex
%    TDS:doc/latex/oberdiek/test/ifluatex-test3.tex
%    TDS:source/latex/oberdiek/ifluatex.dtx
%
%<*ignore>
\begingroup
  \catcode123=1 %
  \catcode125=2 %
  \def\x{LaTeX2e}%
\expandafter\endgroup
\ifcase 0\ifx\install y1\fi\expandafter
         \ifx\csname processbatchFile\endcsname\relax\else1\fi
         \ifx\fmtname\x\else 1\fi\relax
\else\csname fi\endcsname
%</ignore>
%<*install>
\input docstrip.tex
\Msg{************************************************************************}
\Msg{* Installation}
\Msg{* Package: ifluatex 2016/05/16 v1.4 Provides the ifluatex switch (HO)}
\Msg{************************************************************************}

\keepsilent
\askforoverwritefalse

\let\MetaPrefix\relax
\preamble

This is a generated file.

Project: ifluatex
Version: 2016/05/16 v1.4

Copyright (C) 2007, 2009, 2010 by
   Heiko Oberdiek <heiko.oberdiek at googlemail.com>

This work may be distributed and/or modified under the
conditions of the LaTeX Project Public License, either
version 1.3c of this license or (at your option) any later
version. This version of this license is in
   http://www.latex-project.org/lppl/lppl-1-3c.txt
and the latest version of this license is in
   http://www.latex-project.org/lppl.txt
and version 1.3 or later is part of all distributions of
LaTeX version 2005/12/01 or later.

This work has the LPPL maintenance status "maintained".

This Current Maintainer of this work is Heiko Oberdiek.

The Base Interpreter refers to any `TeX-Format',
because some files are installed in TDS:tex/generic//.

This work consists of the main source file ifluatex.dtx
and the derived files
   ifluatex.sty, ifluatex.pdf, ifluatex.ins, ifluatex.drv,
   ifluatex-test1.tex, ifluatex-test2.tex, ifluatex-test3.tex.

\endpreamble
\let\MetaPrefix\DoubleperCent

\generate{%
  \file{ifluatex.ins}{\from{ifluatex.dtx}{install}}%
  \file{ifluatex.drv}{\from{ifluatex.dtx}{driver}}%
  \usedir{tex/generic/oberdiek}%
  \file{ifluatex.sty}{\from{ifluatex.dtx}{package}}%
  \usedir{doc/latex/oberdiek/test}%
  \file{ifluatex-test1.tex}{\from{ifluatex.dtx}{test1}}%
  \file{ifluatex-test2.tex}{\from{ifluatex.dtx}{test-reload1}}%
  \file{ifluatex-test3.tex}{\from{ifluatex.dtx}{test-reload2}}%
  \nopreamble
  \nopostamble
  \usedir{source/latex/oberdiek/catalogue}%
  \file{ifluatex.xml}{\from{ifluatex.dtx}{catalogue}}%
}

\catcode32=13\relax% active space
\let =\space%
\Msg{************************************************************************}
\Msg{*}
\Msg{* To finish the installation you have to move the following}
\Msg{* file into a directory searched by TeX:}
\Msg{*}
\Msg{*     ifluatex.sty}
\Msg{*}
\Msg{* To produce the documentation run the file `ifluatex.drv'}
\Msg{* through LaTeX.}
\Msg{*}
\Msg{* Happy TeXing!}
\Msg{*}
\Msg{************************************************************************}

\endbatchfile
%</install>
%<*ignore>
\fi
%</ignore>
%<*driver>
\NeedsTeXFormat{LaTeX2e}
\ProvidesFile{ifluatex.drv}%
  [2016/05/16 v1.4 Provides the ifluatex switch (HO)]%
\documentclass{ltxdoc}
\usepackage{holtxdoc}[2011/11/22]
\begin{document}
  \DocInput{ifluatex.dtx}%
\end{document}
%</driver>
% \fi
%
%
% \CharacterTable
%  {Upper-case    \A\B\C\D\E\F\G\H\I\J\K\L\M\N\O\P\Q\R\S\T\U\V\W\X\Y\Z
%   Lower-case    \a\b\c\d\e\f\g\h\i\j\k\l\m\n\o\p\q\r\s\t\u\v\w\x\y\z
%   Digits        \0\1\2\3\4\5\6\7\8\9
%   Exclamation   \!     Double quote  \"     Hash (number) \#
%   Dollar        \$     Percent       \%     Ampersand     \&
%   Acute accent  \'     Left paren    \(     Right paren   \)
%   Asterisk      \*     Plus          \+     Comma         \,
%   Minus         \-     Point         \.     Solidus       \/
%   Colon         \:     Semicolon     \;     Less than     \<
%   Equals        \=     Greater than  \>     Question mark \?
%   Commercial at \@     Left bracket  \[     Backslash     \\
%   Right bracket \]     Circumflex    \^     Underscore    \_
%   Grave accent  \`     Left brace    \{     Vertical bar  \|
%   Right brace   \}     Tilde         \~}
%
% \GetFileInfo{ifluatex.drv}
%
% \title{The \xpackage{ifluatex} package}
% \date{2016/05/16 v1.4}
% \author{Heiko Oberdiek\thanks
% {Please report any issues at https://github.com/ho-tex/oberdiek/issues}\\
% \xemail{heiko.oberdiek at googlemail.com}}
%
% \maketitle
%
% \begin{abstract}
% This package looks for \LuaTeX\ regardless of its mode
% and provides the switch \cs{ifluatex}. Also it makes
% \cs{luatexversion} available if it is not present.
% It works with \plainTeX\ or \LaTeX.
% \end{abstract}
%
% \tableofcontents
%
% \section{Documentation}
%
% The package \xpackage{ifluatex} can be used with both \plainTeX\
% and \LaTeX:
% \begin{description}
% \item[\plainTeX:] |\input ifluatex.sty|
% \item[\LaTeXe:]   |\usepackage{ifluatex}|
% \end{description}
%
% \DescribeMacro{\ifluatex}
% The package provides the switch \cs{ifluatex}:
% \begin{quote}
%   |\ifluatex|\\
%   \hspace{1.5em}\LuaTeX\ is running\\
%   |\else|\\
%   \hspace{1.5em}Without \LuaTeX\\
%   |\fi|
% \end{quote}
%
% Since version 0.39 \LuaTeX\ only provides \cs{directlua} at startup
% time. Also the syntax of \cs{directlua} changed in version 0.36.
% Thus the user might want to check the LuaTeX version.
% Therefore this package also makes \cs{luatexversion} and
% \cs{luatexrevision} available, if it is not yet done.
%
% If you want to detect the mode (DVI or PDF), then use package
% \xpackage{ifpdf}. \LuaTeX\ has inherited \cs{pdfoutput} from \pdfTeX.
%
% \StopEventually{
% }
%
% \section{Implementation}
%
%    \begin{macrocode}
%<*package>
%    \end{macrocode}
%
% \subsection{Reload check and package identification}
%    Reload check, especially if the package is not used with \LaTeX.
%    \begin{macrocode}
\begingroup\catcode61\catcode48\catcode32=10\relax%
  \catcode13=5 % ^^M
  \endlinechar=13 %
  \catcode35=6 % #
  \catcode39=12 % '
  \catcode44=12 % ,
  \catcode45=12 % -
  \catcode46=12 % .
  \catcode58=12 % :
  \catcode64=11 % @
  \catcode123=1 % {
  \catcode125=2 % }
  \expandafter\let\expandafter\x\csname ver@ifluatex.sty\endcsname
  \ifx\x\relax % plain-TeX, first loading
  \else
    \def\empty{}%
    \ifx\x\empty % LaTeX, first loading,
      % variable is initialized, but \ProvidesPackage not yet seen
    \else
      \expandafter\ifx\csname PackageInfo\endcsname\relax
        \def\x#1#2{%
          \immediate\write-1{Package #1 Info: #2.}%
        }%
      \else
        \def\x#1#2{\PackageInfo{#1}{#2, stopped}}%
      \fi
      \x{ifluatex}{The package is already loaded}%
      \aftergroup\endinput
    \fi
  \fi
\endgroup%
%    \end{macrocode}
%    Package identification:
%    \begin{macrocode}
\begingroup\catcode61\catcode48\catcode32=10\relax%
  \catcode13=5 % ^^M
  \endlinechar=13 %
  \catcode35=6 % #
  \catcode39=12 % '
  \catcode40=12 % (
  \catcode41=12 % )
  \catcode44=12 % ,
  \catcode45=12 % -
  \catcode46=12 % .
  \catcode47=12 % /
  \catcode58=12 % :
  \catcode64=11 % @
  \catcode91=12 % [
  \catcode93=12 % ]
  \catcode123=1 % {
  \catcode125=2 % }
  \expandafter\ifx\csname ProvidesPackage\endcsname\relax
    \def\x#1#2#3[#4]{\endgroup
      \immediate\write-1{Package: #3 #4}%
      \xdef#1{#4}%
    }%
  \else
    \def\x#1#2[#3]{\endgroup
      #2[{#3}]%
      \ifx#1\@undefined
        \xdef#1{#3}%
      \fi
      \ifx#1\relax
        \xdef#1{#3}%
      \fi
    }%
  \fi
\expandafter\x\csname ver@ifluatex.sty\endcsname
\ProvidesPackage{ifluatex}%
  [2016/05/16 v1.4 Provides the ifluatex switch (HO)]%
%    \end{macrocode}
%
% \subsection{Catcodes}
%
%    \begin{macrocode}
\begingroup\catcode61\catcode48\catcode32=10\relax%
  \catcode13=5 % ^^M
  \endlinechar=13 %
  \catcode123=1 % {
  \catcode125=2 % }
  \catcode64=11 % @
  \def\x{\endgroup
    \expandafter\edef\csname ifluatex@AtEnd\endcsname{%
      \endlinechar=\the\endlinechar\relax
      \catcode13=\the\catcode13\relax
      \catcode32=\the\catcode32\relax
      \catcode35=\the\catcode35\relax
      \catcode61=\the\catcode61\relax
      \catcode64=\the\catcode64\relax
      \catcode123=\the\catcode123\relax
      \catcode125=\the\catcode125\relax
    }%
  }%
\x\catcode61\catcode48\catcode32=10\relax%
\catcode13=5 % ^^M
\endlinechar=13 %
\catcode35=6 % #
\catcode64=11 % @
\catcode123=1 % {
\catcode125=2 % }
\def\TMP@EnsureCode#1#2{%
  \edef\ifluatex@AtEnd{%
    \ifluatex@AtEnd
    \catcode#1=\the\catcode#1\relax
  }%
  \catcode#1=#2\relax
}
\TMP@EnsureCode{10}{12}% ^^J
\TMP@EnsureCode{39}{12}% '
\TMP@EnsureCode{40}{12}% (
\TMP@EnsureCode{41}{12}% )
\TMP@EnsureCode{44}{12}% ,
\TMP@EnsureCode{45}{12}% -
\TMP@EnsureCode{46}{12}% .
\TMP@EnsureCode{47}{12}% /
\TMP@EnsureCode{58}{12}% :
\TMP@EnsureCode{60}{12}% <
\TMP@EnsureCode{94}{7}% ^
\TMP@EnsureCode{96}{12}% `
\edef\ifluatex@AtEnd{\ifluatex@AtEnd\noexpand\endinput}
%    \end{macrocode}
%
% \subsection{Macro for error messages}
%
%    \begin{macro}{\ifluatex@Error}
%    \begin{macrocode}
\begingroup\expandafter\expandafter\expandafter\endgroup
\expandafter\ifx\csname PackageError\endcsname\relax
  \def\ifluatex@Error#1#2{%
    \begingroup
      \newlinechar=10 %
      \def\MessageBreak{^^J}%
      \edef\x{\errhelp{#2}}%
      \x
      \errmessage{Package ifluatex Error: #1}%
    \endgroup
  }%
\else
  \def\ifluatex@Error{%
    \PackageError{ifluatex}%
  }%
\fi
%    \end{macrocode}
%    \end{macro}
%
% \subsection{Check for previously defined \cs{ifluatex}}
%
%    \begin{macrocode}
\begingroup
  \expandafter\ifx\csname ifluatex\endcsname\relax
  \else
    \edef\i/{\expandafter\string\csname ifluatex\endcsname}%
    \ifluatex@Error{Name clash, \i/ is already defined}{%
      Incompatible versions of \i/ can cause problems,\MessageBreak
      therefore package loading is aborted.%
    }%
    \endgroup
    \expandafter\ifluatex@AtEnd
  \fi%
\endgroup
%    \end{macrocode}
%
% \subsection{\cs{ifluatex}}
%
%    \begin{macro}{\ifluatex}
%    \begin{macrocode}
\let\ifluatex\iffalse
%    \end{macrocode}
%    \end{macro}
%
%    Test \cs{luatexversion}. Is it  defined and different from
%    \cs{relax}? Someone could have used \LaTeX\ internal
%    \cs{@ifundefined}, or something else involving.
%    Notice, \cs{csname} is executed inside a group for the test
%    to cancel the side effect of \cs{csname}.
%    \begin{macrocode}
\begingroup\expandafter\expandafter\expandafter\endgroup
\expandafter\ifx\csname luatexversion\endcsname\relax
\else
  \expandafter\let\csname ifluatex\expandafter\endcsname
                  \csname iftrue\endcsname
\fi
%    \end{macrocode}
%
% \subsection{Lua\TeX\ v0.39}
%
%     Starting with version 0.39 \LuaTeX\ wants to provide \cs{directlua}
%     as only primitive at startup time beyond vanilla \TeX's primitives.
%     Then \cs{directlua} exists, but \cs{luatexversion} cannot be found.
%     Unhappily also the syntax of \cs{directlua} changed in v0.36,
%     thus the user would want to check \cs{luatexversion}.
%     Therefore we make \cs{luatexversion} available using
%     \LuaTeX's Lua function |tex.enableprimitives|.
%
%    \begin{macrocode}
\ifluatex
\else
  \begingroup\expandafter\expandafter\expandafter\endgroup
  \expandafter\ifx\csname directlua\endcsname\relax
  \else
    \expandafter\let\csname ifluatex\expandafter\endcsname
                    \csname iftrue\endcsname
    \begingroup
      \newlinechar=10 %
      \endlinechar=\newlinechar%
      \ifnum0%
          \directlua{%
            if tex.enableprimitives then
              tex.enableprimitives('ifluatex', {'luatexversion'})
              tex.print('1')
            end
          }%
          \ifx\ifluatexluatexversion\@undefined\else 1\fi %
          =11 %
        \global\let\luatexversion\ifluatexluatexversion%
      \else%
        \ifluatex@Error{%
          Missing \string\luatexversion%
        }{%
          Update LuaTeX.%
        }%
      \fi%
    \endgroup%
  \fi
\fi
%    \end{macrocode}
%    \begin{macrocode}
\ifluatex
  \begingroup\expandafter\expandafter\expandafter\endgroup
  \expandafter\ifx\csname luatexrevision\endcsname\relax
    \ifnum\luatexversion<36 %
    \else
      \begingroup
        \ifx\luatexrevision\relax
          \let\luatexrevision\@undefined
        \fi
        \newlinechar=10 %
        \endlinechar=\newlinechar%
        \ifcase0%
            \directlua{%
              if tex.enableprimitives then
                tex.enableprimitives('ifluatex', {'luatexrevision'})
              else
                tex.print('1')
              end
            }%
            \ifx\ifluatexluatexrevision\@undefined 1\fi%
            \relax%
          \global\let\luatexrevision\ifluatexluatexrevision%
        \fi%
      \endgroup%
    \fi
    \begingroup\expandafter\expandafter\expandafter\endgroup
    \expandafter\ifx\csname luatexrevision\endcsname\relax
      \ifluatex@Error{%
        Missing \string\luatexrevision%
      }{%
        Update LuaTeX.%
      }%
    \fi
  \fi
\fi
%    \end{macrocode}
%
% \subsection{Protocol entry}
%
%     Log comment:
%    \begin{macrocode}
\begingroup
  \expandafter\ifx\csname PackageInfo\endcsname\relax
    \def\x#1#2{%
      \immediate\write-1{Package #1 Info: #2.}%
    }%
  \else
    \let\x\PackageInfo
    \expandafter\let\csname on@line\endcsname\empty
  \fi
  \x{ifluatex}{LuaTeX \ifluatex\else not \fi detected}%
\endgroup
%    \end{macrocode}
%    \begin{macrocode}
\ifluatex@AtEnd%
%    \end{macrocode}
%    \begin{macrocode}
%</package>
%    \end{macrocode}
%
% \section{Test}
%
% \subsection{Catcode checks for loading}
%
%    \begin{macrocode}
%<*test1>
%    \end{macrocode}
%    \begin{macrocode}
\catcode`\{=1 %
\catcode`\}=2 %
\catcode`\#=6 %
\catcode`\@=11 %
\expandafter\ifx\csname count@\endcsname\relax
  \countdef\count@=255 %
\fi
\expandafter\ifx\csname @gobble\endcsname\relax
  \long\def\@gobble#1{}%
\fi
\expandafter\ifx\csname @firstofone\endcsname\relax
  \long\def\@firstofone#1{#1}%
\fi
\expandafter\ifx\csname loop\endcsname\relax
  \expandafter\@firstofone
\else
  \expandafter\@gobble
\fi
{%
  \def\loop#1\repeat{%
    \def\body{#1}%
    \iterate
  }%
  \def\iterate{%
    \body
      \let\next\iterate
    \else
      \let\next\relax
    \fi
    \next
  }%
  \let\repeat=\fi
}%
\def\RestoreCatcodes{}
\count@=0 %
\loop
  \edef\RestoreCatcodes{%
    \RestoreCatcodes
    \catcode\the\count@=\the\catcode\count@\relax
  }%
\ifnum\count@<255 %
  \advance\count@ 1 %
\repeat

\def\RangeCatcodeInvalid#1#2{%
  \count@=#1\relax
  \loop
    \catcode\count@=15 %
  \ifnum\count@<#2\relax
    \advance\count@ 1 %
  \repeat
}
\def\RangeCatcodeCheck#1#2#3{%
  \count@=#1\relax
  \loop
    \ifnum#3=\catcode\count@
    \else
      \errmessage{%
        Character \the\count@\space
        with wrong catcode \the\catcode\count@\space
        instead of \number#3%
      }%
    \fi
  \ifnum\count@<#2\relax
    \advance\count@ 1 %
  \repeat
}
\def\space{ }
\expandafter\ifx\csname LoadCommand\endcsname\relax
  \def\LoadCommand{\input ifluatex.sty\relax}%
\fi
\def\Test{%
  \RangeCatcodeInvalid{0}{47}%
  \RangeCatcodeInvalid{58}{64}%
  \RangeCatcodeInvalid{91}{96}%
  \RangeCatcodeInvalid{123}{255}%
  \catcode`\@=12 %
  \catcode`\\=0 %
  \catcode`\%=14 %
  \LoadCommand
  \RangeCatcodeCheck{0}{36}{15}%
  \RangeCatcodeCheck{37}{37}{14}%
  \RangeCatcodeCheck{38}{47}{15}%
  \RangeCatcodeCheck{48}{57}{12}%
  \RangeCatcodeCheck{58}{63}{15}%
  \RangeCatcodeCheck{64}{64}{12}%
  \RangeCatcodeCheck{65}{90}{11}%
  \RangeCatcodeCheck{91}{91}{15}%
  \RangeCatcodeCheck{92}{92}{0}%
  \RangeCatcodeCheck{93}{96}{15}%
  \RangeCatcodeCheck{97}{122}{11}%
  \RangeCatcodeCheck{123}{255}{15}%
  \RestoreCatcodes
}
\Test
\csname @@end\endcsname
\end
%    \end{macrocode}
%    \begin{macrocode}
%</test1>
%    \end{macrocode}
%
% \section{Reload check for plain}
%
%    \begin{macrocode}
%<*test-reload1>
\input ifluatex.sty\relax
\input ifluatex.sty\relax
\csname @@end\endcsname\end
%</test-reload1>
%    \end{macrocode}
%
%    \begin{macrocode}
%<*test-reload2>
\input miniltx.tex\relax
\input ifluatex.sty\relax
\input ifluatex.sty\relax
\csname @@end\endcsname\end
%</test-reload2>
%    \end{macrocode}
%
% \section{Installation}
%
% \subsection{Download}
%
% \paragraph{Package.} This package is available on
% CTAN\footnote{\url{http://ctan.org/pkg/ifluatex}}:
% \begin{description}
% \item[\CTAN{macros/latex/contrib/oberdiek/ifluatex.dtx}] The source file.
% \item[\CTAN{macros/latex/contrib/oberdiek/ifluatex.pdf}] Documentation.
% \end{description}
%
%
% \paragraph{Bundle.} All the packages of the bundle `oberdiek'
% are also available in a TDS compliant ZIP archive. There
% the packages are already unpacked and the documentation files
% are generated. The files and directories obey the TDS standard.
% \begin{description}
% \item[\CTAN{install/macros/latex/contrib/oberdiek.tds.zip}]
% \end{description}
% \emph{TDS} refers to the standard ``A Directory Structure
% for \TeX\ Files'' (\CTAN{tds/tds.pdf}). Directories
% with \xfile{texmf} in their name are usually organized this way.
%
% \subsection{Bundle installation}
%
% \paragraph{Unpacking.} Unpack the \xfile{oberdiek.tds.zip} in the
% TDS tree (also known as \xfile{texmf} tree) of your choice.
% Example (linux):
% \begin{quote}
%   |unzip oberdiek.tds.zip -d ~/texmf|
% \end{quote}
%
% \paragraph{Script installation.}
% Check the directory \xfile{TDS:scripts/oberdiek/} for
% scripts that need further installation steps.
% Package \xpackage{attachfile2} comes with the Perl script
% \xfile{pdfatfi.pl} that should be installed in such a way
% that it can be called as \texttt{pdfatfi}.
% Example (linux):
% \begin{quote}
%   |chmod +x scripts/oberdiek/pdfatfi.pl|\\
%   |cp scripts/oberdiek/pdfatfi.pl /usr/local/bin/|
% \end{quote}
%
% \subsection{Package installation}
%
% \paragraph{Unpacking.} The \xfile{.dtx} file is a self-extracting
% \docstrip\ archive. The files are extracted by running the
% \xfile{.dtx} through \plainTeX:
% \begin{quote}
%   \verb|tex ifluatex.dtx|
% \end{quote}
%
% \paragraph{TDS.} Now the different files must be moved into
% the different directories in your installation TDS tree
% (also known as \xfile{texmf} tree):
% \begin{quote}
% \def\t{^^A
% \begin{tabular}{@{}>{\ttfamily}l@{ $\rightarrow$ }>{\ttfamily}l@{}}
%   ifluatex.sty & tex/generic/oberdiek/ifluatex.sty\\
%   ifluatex.pdf & doc/latex/oberdiek/ifluatex.pdf\\
%   test/ifluatex-test1.tex & doc/latex/oberdiek/test/ifluatex-test1.tex\\
%   test/ifluatex-test2.tex & doc/latex/oberdiek/test/ifluatex-test2.tex\\
%   test/ifluatex-test3.tex & doc/latex/oberdiek/test/ifluatex-test3.tex\\
%   ifluatex.dtx & source/latex/oberdiek/ifluatex.dtx\\
% \end{tabular}^^A
% }^^A
% \sbox0{\t}^^A
% \ifdim\wd0>\linewidth
%   \begingroup
%     \advance\linewidth by\leftmargin
%     \advance\linewidth by\rightmargin
%   \edef\x{\endgroup
%     \def\noexpand\lw{\the\linewidth}^^A
%   }\x
%   \def\lwbox{^^A
%     \leavevmode
%     \hbox to \linewidth{^^A
%       \kern-\leftmargin\relax
%       \hss
%       \usebox0
%       \hss
%       \kern-\rightmargin\relax
%     }^^A
%   }^^A
%   \ifdim\wd0>\lw
%     \sbox0{\small\t}^^A
%     \ifdim\wd0>\linewidth
%       \ifdim\wd0>\lw
%         \sbox0{\footnotesize\t}^^A
%         \ifdim\wd0>\linewidth
%           \ifdim\wd0>\lw
%             \sbox0{\scriptsize\t}^^A
%             \ifdim\wd0>\linewidth
%               \ifdim\wd0>\lw
%                 \sbox0{\tiny\t}^^A
%                 \ifdim\wd0>\linewidth
%                   \lwbox
%                 \else
%                   \usebox0
%                 \fi
%               \else
%                 \lwbox
%               \fi
%             \else
%               \usebox0
%             \fi
%           \else
%             \lwbox
%           \fi
%         \else
%           \usebox0
%         \fi
%       \else
%         \lwbox
%       \fi
%     \else
%       \usebox0
%     \fi
%   \else
%     \lwbox
%   \fi
% \else
%   \usebox0
% \fi
% \end{quote}
% If you have a \xfile{docstrip.cfg} that configures and enables \docstrip's
% TDS installing feature, then some files can already be in the right
% place, see the documentation of \docstrip.
%
% \subsection{Refresh file name databases}
%
% If your \TeX~distribution
% (\teTeX, \mikTeX, \dots) relies on file name databases, you must refresh
% these. For example, \teTeX\ users run \verb|texhash| or
% \verb|mktexlsr|.
%
% \subsection{Some details for the interested}
%
% \paragraph{Attached source.}
%
% The PDF documentation on CTAN also includes the
% \xfile{.dtx} source file. It can be extracted by
% AcrobatReader 6 or higher. Another option is \textsf{pdftk},
% e.g. unpack the file into the current directory:
% \begin{quote}
%   \verb|pdftk ifluatex.pdf unpack_files output .|
% \end{quote}
%
% \paragraph{Unpacking with \LaTeX.}
% The \xfile{.dtx} chooses its action depending on the format:
% \begin{description}
% \item[\plainTeX:] Run \docstrip\ and extract the files.
% \item[\LaTeX:] Generate the documentation.
% \end{description}
% If you insist on using \LaTeX\ for \docstrip\ (really,
% \docstrip\ does not need \LaTeX), then inform the autodetect routine
% about your intention:
% \begin{quote}
%   \verb|latex \let\install=y\input{ifluatex.dtx}|
% \end{quote}
% Do not forget to quote the argument according to the demands
% of your shell.
%
% \paragraph{Generating the documentation.}
% You can use both the \xfile{.dtx} or the \xfile{.drv} to generate
% the documentation. The process can be configured by the
% configuration file \xfile{ltxdoc.cfg}. For instance, put this
% line into this file, if you want to have A4 as paper format:
% \begin{quote}
%   \verb|\PassOptionsToClass{a4paper}{article}|
% \end{quote}
% An example follows how to generate the
% documentation with pdf\LaTeX:
% \begin{quote}
%\begin{verbatim}
%pdflatex ifluatex.dtx
%makeindex -s gind.ist ifluatex.idx
%pdflatex ifluatex.dtx
%makeindex -s gind.ist ifluatex.idx
%pdflatex ifluatex.dtx
%\end{verbatim}
% \end{quote}
%
% \section{Catalogue}
%
% The following XML file can be used as source for the
% \href{http://mirror.ctan.org/help/Catalogue/catalogue.html}{\TeX\ Catalogue}.
% The elements \texttt{caption} and \texttt{description} are imported
% from the original XML file from the Catalogue.
% The name of the XML file in the Catalogue is \xfile{ifluatex.xml}.
%    \begin{macrocode}
%<*catalogue>
<?xml version='1.0' encoding='us-ascii'?>
<!DOCTYPE entry SYSTEM 'catalogue.dtd'>
<entry datestamp='$Date$' modifier='$Author$' id='ifluatex'>
  <name>ifluatex</name>
  <caption>Provides the \ifluatex switch.</caption>
  <authorref id='auth:oberdiek'/>
  <copyright owner='Heiko Oberdiek' year='2007,2009,2010'/>
  <license type='lppl1.3'/>
  <version number='1.4'/>
  <description>
    The package looks for  LuaTeX regardless of its mode and provides
    the switch <tt>\ifluatex</tt>; it works with Plain TeX or LaTeX.
    <p/>
    The package is part of the <xref refid='oberdiek'>oberdiek</xref>
    bundle.
  </description>
  <documentation details='Package documentation'
      href='ctan:/macros/latex/contrib/oberdiek/ifluatex.pdf'/>
  <ctan file='true' path='/macros/latex/contrib/oberdiek/ifluatex.dtx'/>
  <miktex location='oberdiek'/>
  <texlive location='ifluatex'/>
  <install path='/macros/latex/contrib/oberdiek/oberdiek.tds.zip'/>
</entry>
%</catalogue>
%    \end{macrocode}
%
% \begin{History}
%   \begin{Version}{2007/12/12 v1.0}
%   \item
%     First public version.
%   \end{Version}
%   \begin{Version}{2009/04/10 v1.1}
%   \item
%     Test adopted for \LuaTeX\ 0.39.
%   \item
%     Makes \cs{luatexversion} available.
%   \end{Version}
%   \begin{Version}{2009/04/17 v1.2}
%   \item
%     Fixes (Manuel P\'egouri\'e-Gonnard).
%   \item
%     \cs{luatextrue} and \cs{luatexfalse} are no longer defined.
%   \item
%     Makes \cs{luatexrevision} available, too.
%   \end{Version}
%   \begin{Version}{2010/03/01 v1.3}
%   \item
%     Line ends fixed in case \cs{endlinechar} = \cs{newlinechar}.
%   \end{Version}
%   \begin{Version}{2016/05/16 v1.4}
%   \item
%     Documentation updates.
%   \end{Version}
% \end{History}
%
% \PrintIndex
%
% \Finale
\endinput

%        (quote the arguments according to the demands of your shell)
%
% Documentation:
%    (a) If ifluatex.drv is present:
%           latex ifluatex.drv
%    (b) Without ifluatex.drv:
%           latex ifluatex.dtx; ...
%    The class ltxdoc loads the configuration file ltxdoc.cfg
%    if available. Here you can specify further options, e.g.
%    use A4 as paper format:
%       \PassOptionsToClass{a4paper}{article}
%
%    Programm calls to get the documentation (example):
%       pdflatex ifluatex.dtx
%       makeindex -s gind.ist ifluatex.idx
%       pdflatex ifluatex.dtx
%       makeindex -s gind.ist ifluatex.idx
%       pdflatex ifluatex.dtx
%
% Installation:
%    TDS:tex/generic/oberdiek/ifluatex.sty
%    TDS:doc/latex/oberdiek/ifluatex.pdf
%    TDS:doc/latex/oberdiek/test/ifluatex-test1.tex
%    TDS:doc/latex/oberdiek/test/ifluatex-test2.tex
%    TDS:doc/latex/oberdiek/test/ifluatex-test3.tex
%    TDS:source/latex/oberdiek/ifluatex.dtx
%
%<*ignore>
\begingroup
  \catcode123=1 %
  \catcode125=2 %
  \def\x{LaTeX2e}%
\expandafter\endgroup
\ifcase 0\ifx\install y1\fi\expandafter
         \ifx\csname processbatchFile\endcsname\relax\else1\fi
         \ifx\fmtname\x\else 1\fi\relax
\else\csname fi\endcsname
%</ignore>
%<*install>
\input docstrip.tex
\Msg{************************************************************************}
\Msg{* Installation}
\Msg{* Package: ifluatex 2016/05/16 v1.4 Provides the ifluatex switch (HO)}
\Msg{************************************************************************}

\keepsilent
\askforoverwritefalse

\let\MetaPrefix\relax
\preamble

This is a generated file.

Project: ifluatex
Version: 2016/05/16 v1.4

Copyright (C) 2007, 2009, 2010 by
   Heiko Oberdiek <heiko.oberdiek at googlemail.com>

This work may be distributed and/or modified under the
conditions of the LaTeX Project Public License, either
version 1.3c of this license or (at your option) any later
version. This version of this license is in
   http://www.latex-project.org/lppl/lppl-1-3c.txt
and the latest version of this license is in
   http://www.latex-project.org/lppl.txt
and version 1.3 or later is part of all distributions of
LaTeX version 2005/12/01 or later.

This work has the LPPL maintenance status "maintained".

This Current Maintainer of this work is Heiko Oberdiek.

The Base Interpreter refers to any `TeX-Format',
because some files are installed in TDS:tex/generic//.

This work consists of the main source file ifluatex.dtx
and the derived files
   ifluatex.sty, ifluatex.pdf, ifluatex.ins, ifluatex.drv,
   ifluatex-test1.tex, ifluatex-test2.tex, ifluatex-test3.tex.

\endpreamble
\let\MetaPrefix\DoubleperCent

\generate{%
  \file{ifluatex.ins}{\from{ifluatex.dtx}{install}}%
  \file{ifluatex.drv}{\from{ifluatex.dtx}{driver}}%
  \usedir{tex/generic/oberdiek}%
  \file{ifluatex.sty}{\from{ifluatex.dtx}{package}}%
  \usedir{doc/latex/oberdiek/test}%
  \file{ifluatex-test1.tex}{\from{ifluatex.dtx}{test1}}%
  \file{ifluatex-test2.tex}{\from{ifluatex.dtx}{test-reload1}}%
  \file{ifluatex-test3.tex}{\from{ifluatex.dtx}{test-reload2}}%
  \nopreamble
  \nopostamble
  \usedir{source/latex/oberdiek/catalogue}%
  \file{ifluatex.xml}{\from{ifluatex.dtx}{catalogue}}%
}

\catcode32=13\relax% active space
\let =\space%
\Msg{************************************************************************}
\Msg{*}
\Msg{* To finish the installation you have to move the following}
\Msg{* file into a directory searched by TeX:}
\Msg{*}
\Msg{*     ifluatex.sty}
\Msg{*}
\Msg{* To produce the documentation run the file `ifluatex.drv'}
\Msg{* through LaTeX.}
\Msg{*}
\Msg{* Happy TeXing!}
\Msg{*}
\Msg{************************************************************************}

\endbatchfile
%</install>
%<*ignore>
\fi
%</ignore>
%<*driver>
\NeedsTeXFormat{LaTeX2e}
\ProvidesFile{ifluatex.drv}%
  [2016/05/16 v1.4 Provides the ifluatex switch (HO)]%
\documentclass{ltxdoc}
\usepackage{holtxdoc}[2011/11/22]
\begin{document}
  \DocInput{ifluatex.dtx}%
\end{document}
%</driver>
% \fi
%
%
% \CharacterTable
%  {Upper-case    \A\B\C\D\E\F\G\H\I\J\K\L\M\N\O\P\Q\R\S\T\U\V\W\X\Y\Z
%   Lower-case    \a\b\c\d\e\f\g\h\i\j\k\l\m\n\o\p\q\r\s\t\u\v\w\x\y\z
%   Digits        \0\1\2\3\4\5\6\7\8\9
%   Exclamation   \!     Double quote  \"     Hash (number) \#
%   Dollar        \$     Percent       \%     Ampersand     \&
%   Acute accent  \'     Left paren    \(     Right paren   \)
%   Asterisk      \*     Plus          \+     Comma         \,
%   Minus         \-     Point         \.     Solidus       \/
%   Colon         \:     Semicolon     \;     Less than     \<
%   Equals        \=     Greater than  \>     Question mark \?
%   Commercial at \@     Left bracket  \[     Backslash     \\
%   Right bracket \]     Circumflex    \^     Underscore    \_
%   Grave accent  \`     Left brace    \{     Vertical bar  \|
%   Right brace   \}     Tilde         \~}
%
% \GetFileInfo{ifluatex.drv}
%
% \title{The \xpackage{ifluatex} package}
% \date{2016/05/16 v1.4}
% \author{Heiko Oberdiek\thanks
% {Please report any issues at https://github.com/ho-tex/oberdiek/issues}\\
% \xemail{heiko.oberdiek at googlemail.com}}
%
% \maketitle
%
% \begin{abstract}
% This package looks for \LuaTeX\ regardless of its mode
% and provides the switch \cs{ifluatex}. Also it makes
% \cs{luatexversion} available if it is not present.
% It works with \plainTeX\ or \LaTeX.
% \end{abstract}
%
% \tableofcontents
%
% \section{Documentation}
%
% The package \xpackage{ifluatex} can be used with both \plainTeX\
% and \LaTeX:
% \begin{description}
% \item[\plainTeX:] |\input ifluatex.sty|
% \item[\LaTeXe:]   |\usepackage{ifluatex}|
% \end{description}
%
% \DescribeMacro{\ifluatex}
% The package provides the switch \cs{ifluatex}:
% \begin{quote}
%   |\ifluatex|\\
%   \hspace{1.5em}\LuaTeX\ is running\\
%   |\else|\\
%   \hspace{1.5em}Without \LuaTeX\\
%   |\fi|
% \end{quote}
%
% Since version 0.39 \LuaTeX\ only provides \cs{directlua} at startup
% time. Also the syntax of \cs{directlua} changed in version 0.36.
% Thus the user might want to check the LuaTeX version.
% Therefore this package also makes \cs{luatexversion} and
% \cs{luatexrevision} available, if it is not yet done.
%
% If you want to detect the mode (DVI or PDF), then use package
% \xpackage{ifpdf}. \LuaTeX\ has inherited \cs{pdfoutput} from \pdfTeX.
%
% \StopEventually{
% }
%
% \section{Implementation}
%
%    \begin{macrocode}
%<*package>
%    \end{macrocode}
%
% \subsection{Reload check and package identification}
%    Reload check, especially if the package is not used with \LaTeX.
%    \begin{macrocode}
\begingroup\catcode61\catcode48\catcode32=10\relax%
  \catcode13=5 % ^^M
  \endlinechar=13 %
  \catcode35=6 % #
  \catcode39=12 % '
  \catcode44=12 % ,
  \catcode45=12 % -
  \catcode46=12 % .
  \catcode58=12 % :
  \catcode64=11 % @
  \catcode123=1 % {
  \catcode125=2 % }
  \expandafter\let\expandafter\x\csname ver@ifluatex.sty\endcsname
  \ifx\x\relax % plain-TeX, first loading
  \else
    \def\empty{}%
    \ifx\x\empty % LaTeX, first loading,
      % variable is initialized, but \ProvidesPackage not yet seen
    \else
      \expandafter\ifx\csname PackageInfo\endcsname\relax
        \def\x#1#2{%
          \immediate\write-1{Package #1 Info: #2.}%
        }%
      \else
        \def\x#1#2{\PackageInfo{#1}{#2, stopped}}%
      \fi
      \x{ifluatex}{The package is already loaded}%
      \aftergroup\endinput
    \fi
  \fi
\endgroup%
%    \end{macrocode}
%    Package identification:
%    \begin{macrocode}
\begingroup\catcode61\catcode48\catcode32=10\relax%
  \catcode13=5 % ^^M
  \endlinechar=13 %
  \catcode35=6 % #
  \catcode39=12 % '
  \catcode40=12 % (
  \catcode41=12 % )
  \catcode44=12 % ,
  \catcode45=12 % -
  \catcode46=12 % .
  \catcode47=12 % /
  \catcode58=12 % :
  \catcode64=11 % @
  \catcode91=12 % [
  \catcode93=12 % ]
  \catcode123=1 % {
  \catcode125=2 % }
  \expandafter\ifx\csname ProvidesPackage\endcsname\relax
    \def\x#1#2#3[#4]{\endgroup
      \immediate\write-1{Package: #3 #4}%
      \xdef#1{#4}%
    }%
  \else
    \def\x#1#2[#3]{\endgroup
      #2[{#3}]%
      \ifx#1\@undefined
        \xdef#1{#3}%
      \fi
      \ifx#1\relax
        \xdef#1{#3}%
      \fi
    }%
  \fi
\expandafter\x\csname ver@ifluatex.sty\endcsname
\ProvidesPackage{ifluatex}%
  [2016/05/16 v1.4 Provides the ifluatex switch (HO)]%
%    \end{macrocode}
%
% \subsection{Catcodes}
%
%    \begin{macrocode}
\begingroup\catcode61\catcode48\catcode32=10\relax%
  \catcode13=5 % ^^M
  \endlinechar=13 %
  \catcode123=1 % {
  \catcode125=2 % }
  \catcode64=11 % @
  \def\x{\endgroup
    \expandafter\edef\csname ifluatex@AtEnd\endcsname{%
      \endlinechar=\the\endlinechar\relax
      \catcode13=\the\catcode13\relax
      \catcode32=\the\catcode32\relax
      \catcode35=\the\catcode35\relax
      \catcode61=\the\catcode61\relax
      \catcode64=\the\catcode64\relax
      \catcode123=\the\catcode123\relax
      \catcode125=\the\catcode125\relax
    }%
  }%
\x\catcode61\catcode48\catcode32=10\relax%
\catcode13=5 % ^^M
\endlinechar=13 %
\catcode35=6 % #
\catcode64=11 % @
\catcode123=1 % {
\catcode125=2 % }
\def\TMP@EnsureCode#1#2{%
  \edef\ifluatex@AtEnd{%
    \ifluatex@AtEnd
    \catcode#1=\the\catcode#1\relax
  }%
  \catcode#1=#2\relax
}
\TMP@EnsureCode{10}{12}% ^^J
\TMP@EnsureCode{39}{12}% '
\TMP@EnsureCode{40}{12}% (
\TMP@EnsureCode{41}{12}% )
\TMP@EnsureCode{44}{12}% ,
\TMP@EnsureCode{45}{12}% -
\TMP@EnsureCode{46}{12}% .
\TMP@EnsureCode{47}{12}% /
\TMP@EnsureCode{58}{12}% :
\TMP@EnsureCode{60}{12}% <
\TMP@EnsureCode{94}{7}% ^
\TMP@EnsureCode{96}{12}% `
\edef\ifluatex@AtEnd{\ifluatex@AtEnd\noexpand\endinput}
%    \end{macrocode}
%
% \subsection{Macro for error messages}
%
%    \begin{macro}{\ifluatex@Error}
%    \begin{macrocode}
\begingroup\expandafter\expandafter\expandafter\endgroup
\expandafter\ifx\csname PackageError\endcsname\relax
  \def\ifluatex@Error#1#2{%
    \begingroup
      \newlinechar=10 %
      \def\MessageBreak{^^J}%
      \edef\x{\errhelp{#2}}%
      \x
      \errmessage{Package ifluatex Error: #1}%
    \endgroup
  }%
\else
  \def\ifluatex@Error{%
    \PackageError{ifluatex}%
  }%
\fi
%    \end{macrocode}
%    \end{macro}
%
% \subsection{Check for previously defined \cs{ifluatex}}
%
%    \begin{macrocode}
\begingroup
  \expandafter\ifx\csname ifluatex\endcsname\relax
  \else
    \edef\i/{\expandafter\string\csname ifluatex\endcsname}%
    \ifluatex@Error{Name clash, \i/ is already defined}{%
      Incompatible versions of \i/ can cause problems,\MessageBreak
      therefore package loading is aborted.%
    }%
    \endgroup
    \expandafter\ifluatex@AtEnd
  \fi%
\endgroup
%    \end{macrocode}
%
% \subsection{\cs{ifluatex}}
%
%    \begin{macro}{\ifluatex}
%    \begin{macrocode}
\let\ifluatex\iffalse
%    \end{macrocode}
%    \end{macro}
%
%    Test \cs{luatexversion}. Is it  defined and different from
%    \cs{relax}? Someone could have used \LaTeX\ internal
%    \cs{@ifundefined}, or something else involving.
%    Notice, \cs{csname} is executed inside a group for the test
%    to cancel the side effect of \cs{csname}.
%    \begin{macrocode}
\begingroup\expandafter\expandafter\expandafter\endgroup
\expandafter\ifx\csname luatexversion\endcsname\relax
\else
  \expandafter\let\csname ifluatex\expandafter\endcsname
                  \csname iftrue\endcsname
\fi
%    \end{macrocode}
%
% \subsection{Lua\TeX\ v0.39}
%
%     Starting with version 0.39 \LuaTeX\ wants to provide \cs{directlua}
%     as only primitive at startup time beyond vanilla \TeX's primitives.
%     Then \cs{directlua} exists, but \cs{luatexversion} cannot be found.
%     Unhappily also the syntax of \cs{directlua} changed in v0.36,
%     thus the user would want to check \cs{luatexversion}.
%     Therefore we make \cs{luatexversion} available using
%     \LuaTeX's Lua function |tex.enableprimitives|.
%
%    \begin{macrocode}
\ifluatex
\else
  \begingroup\expandafter\expandafter\expandafter\endgroup
  \expandafter\ifx\csname directlua\endcsname\relax
  \else
    \expandafter\let\csname ifluatex\expandafter\endcsname
                    \csname iftrue\endcsname
    \begingroup
      \newlinechar=10 %
      \endlinechar=\newlinechar%
      \ifnum0%
          \directlua{%
            if tex.enableprimitives then
              tex.enableprimitives('ifluatex', {'luatexversion'})
              tex.print('1')
            end
          }%
          \ifx\ifluatexluatexversion\@undefined\else 1\fi %
          =11 %
        \global\let\luatexversion\ifluatexluatexversion%
      \else%
        \ifluatex@Error{%
          Missing \string\luatexversion%
        }{%
          Update LuaTeX.%
        }%
      \fi%
    \endgroup%
  \fi
\fi
%    \end{macrocode}
%    \begin{macrocode}
\ifluatex
  \begingroup\expandafter\expandafter\expandafter\endgroup
  \expandafter\ifx\csname luatexrevision\endcsname\relax
    \ifnum\luatexversion<36 %
    \else
      \begingroup
        \ifx\luatexrevision\relax
          \let\luatexrevision\@undefined
        \fi
        \newlinechar=10 %
        \endlinechar=\newlinechar%
        \ifcase0%
            \directlua{%
              if tex.enableprimitives then
                tex.enableprimitives('ifluatex', {'luatexrevision'})
              else
                tex.print('1')
              end
            }%
            \ifx\ifluatexluatexrevision\@undefined 1\fi%
            \relax%
          \global\let\luatexrevision\ifluatexluatexrevision%
        \fi%
      \endgroup%
    \fi
    \begingroup\expandafter\expandafter\expandafter\endgroup
    \expandafter\ifx\csname luatexrevision\endcsname\relax
      \ifluatex@Error{%
        Missing \string\luatexrevision%
      }{%
        Update LuaTeX.%
      }%
    \fi
  \fi
\fi
%    \end{macrocode}
%
% \subsection{Protocol entry}
%
%     Log comment:
%    \begin{macrocode}
\begingroup
  \expandafter\ifx\csname PackageInfo\endcsname\relax
    \def\x#1#2{%
      \immediate\write-1{Package #1 Info: #2.}%
    }%
  \else
    \let\x\PackageInfo
    \expandafter\let\csname on@line\endcsname\empty
  \fi
  \x{ifluatex}{LuaTeX \ifluatex\else not \fi detected}%
\endgroup
%    \end{macrocode}
%    \begin{macrocode}
\ifluatex@AtEnd%
%    \end{macrocode}
%    \begin{macrocode}
%</package>
%    \end{macrocode}
%
% \section{Test}
%
% \subsection{Catcode checks for loading}
%
%    \begin{macrocode}
%<*test1>
%    \end{macrocode}
%    \begin{macrocode}
\catcode`\{=1 %
\catcode`\}=2 %
\catcode`\#=6 %
\catcode`\@=11 %
\expandafter\ifx\csname count@\endcsname\relax
  \countdef\count@=255 %
\fi
\expandafter\ifx\csname @gobble\endcsname\relax
  \long\def\@gobble#1{}%
\fi
\expandafter\ifx\csname @firstofone\endcsname\relax
  \long\def\@firstofone#1{#1}%
\fi
\expandafter\ifx\csname loop\endcsname\relax
  \expandafter\@firstofone
\else
  \expandafter\@gobble
\fi
{%
  \def\loop#1\repeat{%
    \def\body{#1}%
    \iterate
  }%
  \def\iterate{%
    \body
      \let\next\iterate
    \else
      \let\next\relax
    \fi
    \next
  }%
  \let\repeat=\fi
}%
\def\RestoreCatcodes{}
\count@=0 %
\loop
  \edef\RestoreCatcodes{%
    \RestoreCatcodes
    \catcode\the\count@=\the\catcode\count@\relax
  }%
\ifnum\count@<255 %
  \advance\count@ 1 %
\repeat

\def\RangeCatcodeInvalid#1#2{%
  \count@=#1\relax
  \loop
    \catcode\count@=15 %
  \ifnum\count@<#2\relax
    \advance\count@ 1 %
  \repeat
}
\def\RangeCatcodeCheck#1#2#3{%
  \count@=#1\relax
  \loop
    \ifnum#3=\catcode\count@
    \else
      \errmessage{%
        Character \the\count@\space
        with wrong catcode \the\catcode\count@\space
        instead of \number#3%
      }%
    \fi
  \ifnum\count@<#2\relax
    \advance\count@ 1 %
  \repeat
}
\def\space{ }
\expandafter\ifx\csname LoadCommand\endcsname\relax
  \def\LoadCommand{\input ifluatex.sty\relax}%
\fi
\def\Test{%
  \RangeCatcodeInvalid{0}{47}%
  \RangeCatcodeInvalid{58}{64}%
  \RangeCatcodeInvalid{91}{96}%
  \RangeCatcodeInvalid{123}{255}%
  \catcode`\@=12 %
  \catcode`\\=0 %
  \catcode`\%=14 %
  \LoadCommand
  \RangeCatcodeCheck{0}{36}{15}%
  \RangeCatcodeCheck{37}{37}{14}%
  \RangeCatcodeCheck{38}{47}{15}%
  \RangeCatcodeCheck{48}{57}{12}%
  \RangeCatcodeCheck{58}{63}{15}%
  \RangeCatcodeCheck{64}{64}{12}%
  \RangeCatcodeCheck{65}{90}{11}%
  \RangeCatcodeCheck{91}{91}{15}%
  \RangeCatcodeCheck{92}{92}{0}%
  \RangeCatcodeCheck{93}{96}{15}%
  \RangeCatcodeCheck{97}{122}{11}%
  \RangeCatcodeCheck{123}{255}{15}%
  \RestoreCatcodes
}
\Test
\csname @@end\endcsname
\end
%    \end{macrocode}
%    \begin{macrocode}
%</test1>
%    \end{macrocode}
%
% \section{Reload check for plain}
%
%    \begin{macrocode}
%<*test-reload1>
\input ifluatex.sty\relax
\input ifluatex.sty\relax
\csname @@end\endcsname\end
%</test-reload1>
%    \end{macrocode}
%
%    \begin{macrocode}
%<*test-reload2>
\input miniltx.tex\relax
\input ifluatex.sty\relax
\input ifluatex.sty\relax
\csname @@end\endcsname\end
%</test-reload2>
%    \end{macrocode}
%
% \section{Installation}
%
% \subsection{Download}
%
% \paragraph{Package.} This package is available on
% CTAN\footnote{\url{http://ctan.org/pkg/ifluatex}}:
% \begin{description}
% \item[\CTAN{macros/latex/contrib/oberdiek/ifluatex.dtx}] The source file.
% \item[\CTAN{macros/latex/contrib/oberdiek/ifluatex.pdf}] Documentation.
% \end{description}
%
%
% \paragraph{Bundle.} All the packages of the bundle `oberdiek'
% are also available in a TDS compliant ZIP archive. There
% the packages are already unpacked and the documentation files
% are generated. The files and directories obey the TDS standard.
% \begin{description}
% \item[\CTAN{install/macros/latex/contrib/oberdiek.tds.zip}]
% \end{description}
% \emph{TDS} refers to the standard ``A Directory Structure
% for \TeX\ Files'' (\CTAN{tds/tds.pdf}). Directories
% with \xfile{texmf} in their name are usually organized this way.
%
% \subsection{Bundle installation}
%
% \paragraph{Unpacking.} Unpack the \xfile{oberdiek.tds.zip} in the
% TDS tree (also known as \xfile{texmf} tree) of your choice.
% Example (linux):
% \begin{quote}
%   |unzip oberdiek.tds.zip -d ~/texmf|
% \end{quote}
%
% \paragraph{Script installation.}
% Check the directory \xfile{TDS:scripts/oberdiek/} for
% scripts that need further installation steps.
% Package \xpackage{attachfile2} comes with the Perl script
% \xfile{pdfatfi.pl} that should be installed in such a way
% that it can be called as \texttt{pdfatfi}.
% Example (linux):
% \begin{quote}
%   |chmod +x scripts/oberdiek/pdfatfi.pl|\\
%   |cp scripts/oberdiek/pdfatfi.pl /usr/local/bin/|
% \end{quote}
%
% \subsection{Package installation}
%
% \paragraph{Unpacking.} The \xfile{.dtx} file is a self-extracting
% \docstrip\ archive. The files are extracted by running the
% \xfile{.dtx} through \plainTeX:
% \begin{quote}
%   \verb|tex ifluatex.dtx|
% \end{quote}
%
% \paragraph{TDS.} Now the different files must be moved into
% the different directories in your installation TDS tree
% (also known as \xfile{texmf} tree):
% \begin{quote}
% \def\t{^^A
% \begin{tabular}{@{}>{\ttfamily}l@{ $\rightarrow$ }>{\ttfamily}l@{}}
%   ifluatex.sty & tex/generic/oberdiek/ifluatex.sty\\
%   ifluatex.pdf & doc/latex/oberdiek/ifluatex.pdf\\
%   test/ifluatex-test1.tex & doc/latex/oberdiek/test/ifluatex-test1.tex\\
%   test/ifluatex-test2.tex & doc/latex/oberdiek/test/ifluatex-test2.tex\\
%   test/ifluatex-test3.tex & doc/latex/oberdiek/test/ifluatex-test3.tex\\
%   ifluatex.dtx & source/latex/oberdiek/ifluatex.dtx\\
% \end{tabular}^^A
% }^^A
% \sbox0{\t}^^A
% \ifdim\wd0>\linewidth
%   \begingroup
%     \advance\linewidth by\leftmargin
%     \advance\linewidth by\rightmargin
%   \edef\x{\endgroup
%     \def\noexpand\lw{\the\linewidth}^^A
%   }\x
%   \def\lwbox{^^A
%     \leavevmode
%     \hbox to \linewidth{^^A
%       \kern-\leftmargin\relax
%       \hss
%       \usebox0
%       \hss
%       \kern-\rightmargin\relax
%     }^^A
%   }^^A
%   \ifdim\wd0>\lw
%     \sbox0{\small\t}^^A
%     \ifdim\wd0>\linewidth
%       \ifdim\wd0>\lw
%         \sbox0{\footnotesize\t}^^A
%         \ifdim\wd0>\linewidth
%           \ifdim\wd0>\lw
%             \sbox0{\scriptsize\t}^^A
%             \ifdim\wd0>\linewidth
%               \ifdim\wd0>\lw
%                 \sbox0{\tiny\t}^^A
%                 \ifdim\wd0>\linewidth
%                   \lwbox
%                 \else
%                   \usebox0
%                 \fi
%               \else
%                 \lwbox
%               \fi
%             \else
%               \usebox0
%             \fi
%           \else
%             \lwbox
%           \fi
%         \else
%           \usebox0
%         \fi
%       \else
%         \lwbox
%       \fi
%     \else
%       \usebox0
%     \fi
%   \else
%     \lwbox
%   \fi
% \else
%   \usebox0
% \fi
% \end{quote}
% If you have a \xfile{docstrip.cfg} that configures and enables \docstrip's
% TDS installing feature, then some files can already be in the right
% place, see the documentation of \docstrip.
%
% \subsection{Refresh file name databases}
%
% If your \TeX~distribution
% (\teTeX, \mikTeX, \dots) relies on file name databases, you must refresh
% these. For example, \teTeX\ users run \verb|texhash| or
% \verb|mktexlsr|.
%
% \subsection{Some details for the interested}
%
% \paragraph{Attached source.}
%
% The PDF documentation on CTAN also includes the
% \xfile{.dtx} source file. It can be extracted by
% AcrobatReader 6 or higher. Another option is \textsf{pdftk},
% e.g. unpack the file into the current directory:
% \begin{quote}
%   \verb|pdftk ifluatex.pdf unpack_files output .|
% \end{quote}
%
% \paragraph{Unpacking with \LaTeX.}
% The \xfile{.dtx} chooses its action depending on the format:
% \begin{description}
% \item[\plainTeX:] Run \docstrip\ and extract the files.
% \item[\LaTeX:] Generate the documentation.
% \end{description}
% If you insist on using \LaTeX\ for \docstrip\ (really,
% \docstrip\ does not need \LaTeX), then inform the autodetect routine
% about your intention:
% \begin{quote}
%   \verb|latex \let\install=y% \iffalse meta-comment
%
% File: ifluatex.dtx
% Version: 2016/05/16 v1.4
% Info: Provides the ifluatex switch
%
% Copyright (C) 2007, 2009, 2010 by
%    Heiko Oberdiek <heiko.oberdiek at googlemail.com>
%    2016
%    https://github.com/ho-tex/oberdiek/issues
%
% This work may be distributed and/or modified under the
% conditions of the LaTeX Project Public License, either
% version 1.3c of this license or (at your option) any later
% version. This version of this license is in
%    http://www.latex-project.org/lppl/lppl-1-3c.txt
% and the latest version of this license is in
%    http://www.latex-project.org/lppl.txt
% and version 1.3 or later is part of all distributions of
% LaTeX version 2005/12/01 or later.
%
% This work has the LPPL maintenance status "maintained".
%
% This Current Maintainer of this work is Heiko Oberdiek.
%
% The Base Interpreter refers to any `TeX-Format',
% because some files are installed in TDS:tex/generic//.
%
% This work consists of the main source file ifluatex.dtx
% and the derived files
%    ifluatex.sty, ifluatex.pdf, ifluatex.ins, ifluatex.drv,
%    ifluatex-test1.tex, ifluatex-test2.tex, ifluatex-test3.tex.
%
% Distribution:
%    CTAN:macros/latex/contrib/oberdiek/ifluatex.dtx
%    CTAN:macros/latex/contrib/oberdiek/ifluatex.pdf
%
% Unpacking:
%    (a) If ifluatex.ins is present:
%           tex ifluatex.ins
%    (b) Without ifluatex.ins:
%           tex ifluatex.dtx
%    (c) If you insist on using LaTeX
%           latex \let\install=y\input{ifluatex.dtx}
%        (quote the arguments according to the demands of your shell)
%
% Documentation:
%    (a) If ifluatex.drv is present:
%           latex ifluatex.drv
%    (b) Without ifluatex.drv:
%           latex ifluatex.dtx; ...
%    The class ltxdoc loads the configuration file ltxdoc.cfg
%    if available. Here you can specify further options, e.g.
%    use A4 as paper format:
%       \PassOptionsToClass{a4paper}{article}
%
%    Programm calls to get the documentation (example):
%       pdflatex ifluatex.dtx
%       makeindex -s gind.ist ifluatex.idx
%       pdflatex ifluatex.dtx
%       makeindex -s gind.ist ifluatex.idx
%       pdflatex ifluatex.dtx
%
% Installation:
%    TDS:tex/generic/oberdiek/ifluatex.sty
%    TDS:doc/latex/oberdiek/ifluatex.pdf
%    TDS:doc/latex/oberdiek/test/ifluatex-test1.tex
%    TDS:doc/latex/oberdiek/test/ifluatex-test2.tex
%    TDS:doc/latex/oberdiek/test/ifluatex-test3.tex
%    TDS:source/latex/oberdiek/ifluatex.dtx
%
%<*ignore>
\begingroup
  \catcode123=1 %
  \catcode125=2 %
  \def\x{LaTeX2e}%
\expandafter\endgroup
\ifcase 0\ifx\install y1\fi\expandafter
         \ifx\csname processbatchFile\endcsname\relax\else1\fi
         \ifx\fmtname\x\else 1\fi\relax
\else\csname fi\endcsname
%</ignore>
%<*install>
\input docstrip.tex
\Msg{************************************************************************}
\Msg{* Installation}
\Msg{* Package: ifluatex 2016/05/16 v1.4 Provides the ifluatex switch (HO)}
\Msg{************************************************************************}

\keepsilent
\askforoverwritefalse

\let\MetaPrefix\relax
\preamble

This is a generated file.

Project: ifluatex
Version: 2016/05/16 v1.4

Copyright (C) 2007, 2009, 2010 by
   Heiko Oberdiek <heiko.oberdiek at googlemail.com>

This work may be distributed and/or modified under the
conditions of the LaTeX Project Public License, either
version 1.3c of this license or (at your option) any later
version. This version of this license is in
   http://www.latex-project.org/lppl/lppl-1-3c.txt
and the latest version of this license is in
   http://www.latex-project.org/lppl.txt
and version 1.3 or later is part of all distributions of
LaTeX version 2005/12/01 or later.

This work has the LPPL maintenance status "maintained".

This Current Maintainer of this work is Heiko Oberdiek.

The Base Interpreter refers to any `TeX-Format',
because some files are installed in TDS:tex/generic//.

This work consists of the main source file ifluatex.dtx
and the derived files
   ifluatex.sty, ifluatex.pdf, ifluatex.ins, ifluatex.drv,
   ifluatex-test1.tex, ifluatex-test2.tex, ifluatex-test3.tex.

\endpreamble
\let\MetaPrefix\DoubleperCent

\generate{%
  \file{ifluatex.ins}{\from{ifluatex.dtx}{install}}%
  \file{ifluatex.drv}{\from{ifluatex.dtx}{driver}}%
  \usedir{tex/generic/oberdiek}%
  \file{ifluatex.sty}{\from{ifluatex.dtx}{package}}%
  \usedir{doc/latex/oberdiek/test}%
  \file{ifluatex-test1.tex}{\from{ifluatex.dtx}{test1}}%
  \file{ifluatex-test2.tex}{\from{ifluatex.dtx}{test-reload1}}%
  \file{ifluatex-test3.tex}{\from{ifluatex.dtx}{test-reload2}}%
  \nopreamble
  \nopostamble
  \usedir{source/latex/oberdiek/catalogue}%
  \file{ifluatex.xml}{\from{ifluatex.dtx}{catalogue}}%
}

\catcode32=13\relax% active space
\let =\space%
\Msg{************************************************************************}
\Msg{*}
\Msg{* To finish the installation you have to move the following}
\Msg{* file into a directory searched by TeX:}
\Msg{*}
\Msg{*     ifluatex.sty}
\Msg{*}
\Msg{* To produce the documentation run the file `ifluatex.drv'}
\Msg{* through LaTeX.}
\Msg{*}
\Msg{* Happy TeXing!}
\Msg{*}
\Msg{************************************************************************}

\endbatchfile
%</install>
%<*ignore>
\fi
%</ignore>
%<*driver>
\NeedsTeXFormat{LaTeX2e}
\ProvidesFile{ifluatex.drv}%
  [2016/05/16 v1.4 Provides the ifluatex switch (HO)]%
\documentclass{ltxdoc}
\usepackage{holtxdoc}[2011/11/22]
\begin{document}
  \DocInput{ifluatex.dtx}%
\end{document}
%</driver>
% \fi
%
%
% \CharacterTable
%  {Upper-case    \A\B\C\D\E\F\G\H\I\J\K\L\M\N\O\P\Q\R\S\T\U\V\W\X\Y\Z
%   Lower-case    \a\b\c\d\e\f\g\h\i\j\k\l\m\n\o\p\q\r\s\t\u\v\w\x\y\z
%   Digits        \0\1\2\3\4\5\6\7\8\9
%   Exclamation   \!     Double quote  \"     Hash (number) \#
%   Dollar        \$     Percent       \%     Ampersand     \&
%   Acute accent  \'     Left paren    \(     Right paren   \)
%   Asterisk      \*     Plus          \+     Comma         \,
%   Minus         \-     Point         \.     Solidus       \/
%   Colon         \:     Semicolon     \;     Less than     \<
%   Equals        \=     Greater than  \>     Question mark \?
%   Commercial at \@     Left bracket  \[     Backslash     \\
%   Right bracket \]     Circumflex    \^     Underscore    \_
%   Grave accent  \`     Left brace    \{     Vertical bar  \|
%   Right brace   \}     Tilde         \~}
%
% \GetFileInfo{ifluatex.drv}
%
% \title{The \xpackage{ifluatex} package}
% \date{2016/05/16 v1.4}
% \author{Heiko Oberdiek\thanks
% {Please report any issues at https://github.com/ho-tex/oberdiek/issues}\\
% \xemail{heiko.oberdiek at googlemail.com}}
%
% \maketitle
%
% \begin{abstract}
% This package looks for \LuaTeX\ regardless of its mode
% and provides the switch \cs{ifluatex}. Also it makes
% \cs{luatexversion} available if it is not present.
% It works with \plainTeX\ or \LaTeX.
% \end{abstract}
%
% \tableofcontents
%
% \section{Documentation}
%
% The package \xpackage{ifluatex} can be used with both \plainTeX\
% and \LaTeX:
% \begin{description}
% \item[\plainTeX:] |\input ifluatex.sty|
% \item[\LaTeXe:]   |\usepackage{ifluatex}|
% \end{description}
%
% \DescribeMacro{\ifluatex}
% The package provides the switch \cs{ifluatex}:
% \begin{quote}
%   |\ifluatex|\\
%   \hspace{1.5em}\LuaTeX\ is running\\
%   |\else|\\
%   \hspace{1.5em}Without \LuaTeX\\
%   |\fi|
% \end{quote}
%
% Since version 0.39 \LuaTeX\ only provides \cs{directlua} at startup
% time. Also the syntax of \cs{directlua} changed in version 0.36.
% Thus the user might want to check the LuaTeX version.
% Therefore this package also makes \cs{luatexversion} and
% \cs{luatexrevision} available, if it is not yet done.
%
% If you want to detect the mode (DVI or PDF), then use package
% \xpackage{ifpdf}. \LuaTeX\ has inherited \cs{pdfoutput} from \pdfTeX.
%
% \StopEventually{
% }
%
% \section{Implementation}
%
%    \begin{macrocode}
%<*package>
%    \end{macrocode}
%
% \subsection{Reload check and package identification}
%    Reload check, especially if the package is not used with \LaTeX.
%    \begin{macrocode}
\begingroup\catcode61\catcode48\catcode32=10\relax%
  \catcode13=5 % ^^M
  \endlinechar=13 %
  \catcode35=6 % #
  \catcode39=12 % '
  \catcode44=12 % ,
  \catcode45=12 % -
  \catcode46=12 % .
  \catcode58=12 % :
  \catcode64=11 % @
  \catcode123=1 % {
  \catcode125=2 % }
  \expandafter\let\expandafter\x\csname ver@ifluatex.sty\endcsname
  \ifx\x\relax % plain-TeX, first loading
  \else
    \def\empty{}%
    \ifx\x\empty % LaTeX, first loading,
      % variable is initialized, but \ProvidesPackage not yet seen
    \else
      \expandafter\ifx\csname PackageInfo\endcsname\relax
        \def\x#1#2{%
          \immediate\write-1{Package #1 Info: #2.}%
        }%
      \else
        \def\x#1#2{\PackageInfo{#1}{#2, stopped}}%
      \fi
      \x{ifluatex}{The package is already loaded}%
      \aftergroup\endinput
    \fi
  \fi
\endgroup%
%    \end{macrocode}
%    Package identification:
%    \begin{macrocode}
\begingroup\catcode61\catcode48\catcode32=10\relax%
  \catcode13=5 % ^^M
  \endlinechar=13 %
  \catcode35=6 % #
  \catcode39=12 % '
  \catcode40=12 % (
  \catcode41=12 % )
  \catcode44=12 % ,
  \catcode45=12 % -
  \catcode46=12 % .
  \catcode47=12 % /
  \catcode58=12 % :
  \catcode64=11 % @
  \catcode91=12 % [
  \catcode93=12 % ]
  \catcode123=1 % {
  \catcode125=2 % }
  \expandafter\ifx\csname ProvidesPackage\endcsname\relax
    \def\x#1#2#3[#4]{\endgroup
      \immediate\write-1{Package: #3 #4}%
      \xdef#1{#4}%
    }%
  \else
    \def\x#1#2[#3]{\endgroup
      #2[{#3}]%
      \ifx#1\@undefined
        \xdef#1{#3}%
      \fi
      \ifx#1\relax
        \xdef#1{#3}%
      \fi
    }%
  \fi
\expandafter\x\csname ver@ifluatex.sty\endcsname
\ProvidesPackage{ifluatex}%
  [2016/05/16 v1.4 Provides the ifluatex switch (HO)]%
%    \end{macrocode}
%
% \subsection{Catcodes}
%
%    \begin{macrocode}
\begingroup\catcode61\catcode48\catcode32=10\relax%
  \catcode13=5 % ^^M
  \endlinechar=13 %
  \catcode123=1 % {
  \catcode125=2 % }
  \catcode64=11 % @
  \def\x{\endgroup
    \expandafter\edef\csname ifluatex@AtEnd\endcsname{%
      \endlinechar=\the\endlinechar\relax
      \catcode13=\the\catcode13\relax
      \catcode32=\the\catcode32\relax
      \catcode35=\the\catcode35\relax
      \catcode61=\the\catcode61\relax
      \catcode64=\the\catcode64\relax
      \catcode123=\the\catcode123\relax
      \catcode125=\the\catcode125\relax
    }%
  }%
\x\catcode61\catcode48\catcode32=10\relax%
\catcode13=5 % ^^M
\endlinechar=13 %
\catcode35=6 % #
\catcode64=11 % @
\catcode123=1 % {
\catcode125=2 % }
\def\TMP@EnsureCode#1#2{%
  \edef\ifluatex@AtEnd{%
    \ifluatex@AtEnd
    \catcode#1=\the\catcode#1\relax
  }%
  \catcode#1=#2\relax
}
\TMP@EnsureCode{10}{12}% ^^J
\TMP@EnsureCode{39}{12}% '
\TMP@EnsureCode{40}{12}% (
\TMP@EnsureCode{41}{12}% )
\TMP@EnsureCode{44}{12}% ,
\TMP@EnsureCode{45}{12}% -
\TMP@EnsureCode{46}{12}% .
\TMP@EnsureCode{47}{12}% /
\TMP@EnsureCode{58}{12}% :
\TMP@EnsureCode{60}{12}% <
\TMP@EnsureCode{94}{7}% ^
\TMP@EnsureCode{96}{12}% `
\edef\ifluatex@AtEnd{\ifluatex@AtEnd\noexpand\endinput}
%    \end{macrocode}
%
% \subsection{Macro for error messages}
%
%    \begin{macro}{\ifluatex@Error}
%    \begin{macrocode}
\begingroup\expandafter\expandafter\expandafter\endgroup
\expandafter\ifx\csname PackageError\endcsname\relax
  \def\ifluatex@Error#1#2{%
    \begingroup
      \newlinechar=10 %
      \def\MessageBreak{^^J}%
      \edef\x{\errhelp{#2}}%
      \x
      \errmessage{Package ifluatex Error: #1}%
    \endgroup
  }%
\else
  \def\ifluatex@Error{%
    \PackageError{ifluatex}%
  }%
\fi
%    \end{macrocode}
%    \end{macro}
%
% \subsection{Check for previously defined \cs{ifluatex}}
%
%    \begin{macrocode}
\begingroup
  \expandafter\ifx\csname ifluatex\endcsname\relax
  \else
    \edef\i/{\expandafter\string\csname ifluatex\endcsname}%
    \ifluatex@Error{Name clash, \i/ is already defined}{%
      Incompatible versions of \i/ can cause problems,\MessageBreak
      therefore package loading is aborted.%
    }%
    \endgroup
    \expandafter\ifluatex@AtEnd
  \fi%
\endgroup
%    \end{macrocode}
%
% \subsection{\cs{ifluatex}}
%
%    \begin{macro}{\ifluatex}
%    \begin{macrocode}
\let\ifluatex\iffalse
%    \end{macrocode}
%    \end{macro}
%
%    Test \cs{luatexversion}. Is it  defined and different from
%    \cs{relax}? Someone could have used \LaTeX\ internal
%    \cs{@ifundefined}, or something else involving.
%    Notice, \cs{csname} is executed inside a group for the test
%    to cancel the side effect of \cs{csname}.
%    \begin{macrocode}
\begingroup\expandafter\expandafter\expandafter\endgroup
\expandafter\ifx\csname luatexversion\endcsname\relax
\else
  \expandafter\let\csname ifluatex\expandafter\endcsname
                  \csname iftrue\endcsname
\fi
%    \end{macrocode}
%
% \subsection{Lua\TeX\ v0.39}
%
%     Starting with version 0.39 \LuaTeX\ wants to provide \cs{directlua}
%     as only primitive at startup time beyond vanilla \TeX's primitives.
%     Then \cs{directlua} exists, but \cs{luatexversion} cannot be found.
%     Unhappily also the syntax of \cs{directlua} changed in v0.36,
%     thus the user would want to check \cs{luatexversion}.
%     Therefore we make \cs{luatexversion} available using
%     \LuaTeX's Lua function |tex.enableprimitives|.
%
%    \begin{macrocode}
\ifluatex
\else
  \begingroup\expandafter\expandafter\expandafter\endgroup
  \expandafter\ifx\csname directlua\endcsname\relax
  \else
    \expandafter\let\csname ifluatex\expandafter\endcsname
                    \csname iftrue\endcsname
    \begingroup
      \newlinechar=10 %
      \endlinechar=\newlinechar%
      \ifnum0%
          \directlua{%
            if tex.enableprimitives then
              tex.enableprimitives('ifluatex', {'luatexversion'})
              tex.print('1')
            end
          }%
          \ifx\ifluatexluatexversion\@undefined\else 1\fi %
          =11 %
        \global\let\luatexversion\ifluatexluatexversion%
      \else%
        \ifluatex@Error{%
          Missing \string\luatexversion%
        }{%
          Update LuaTeX.%
        }%
      \fi%
    \endgroup%
  \fi
\fi
%    \end{macrocode}
%    \begin{macrocode}
\ifluatex
  \begingroup\expandafter\expandafter\expandafter\endgroup
  \expandafter\ifx\csname luatexrevision\endcsname\relax
    \ifnum\luatexversion<36 %
    \else
      \begingroup
        \ifx\luatexrevision\relax
          \let\luatexrevision\@undefined
        \fi
        \newlinechar=10 %
        \endlinechar=\newlinechar%
        \ifcase0%
            \directlua{%
              if tex.enableprimitives then
                tex.enableprimitives('ifluatex', {'luatexrevision'})
              else
                tex.print('1')
              end
            }%
            \ifx\ifluatexluatexrevision\@undefined 1\fi%
            \relax%
          \global\let\luatexrevision\ifluatexluatexrevision%
        \fi%
      \endgroup%
    \fi
    \begingroup\expandafter\expandafter\expandafter\endgroup
    \expandafter\ifx\csname luatexrevision\endcsname\relax
      \ifluatex@Error{%
        Missing \string\luatexrevision%
      }{%
        Update LuaTeX.%
      }%
    \fi
  \fi
\fi
%    \end{macrocode}
%
% \subsection{Protocol entry}
%
%     Log comment:
%    \begin{macrocode}
\begingroup
  \expandafter\ifx\csname PackageInfo\endcsname\relax
    \def\x#1#2{%
      \immediate\write-1{Package #1 Info: #2.}%
    }%
  \else
    \let\x\PackageInfo
    \expandafter\let\csname on@line\endcsname\empty
  \fi
  \x{ifluatex}{LuaTeX \ifluatex\else not \fi detected}%
\endgroup
%    \end{macrocode}
%    \begin{macrocode}
\ifluatex@AtEnd%
%    \end{macrocode}
%    \begin{macrocode}
%</package>
%    \end{macrocode}
%
% \section{Test}
%
% \subsection{Catcode checks for loading}
%
%    \begin{macrocode}
%<*test1>
%    \end{macrocode}
%    \begin{macrocode}
\catcode`\{=1 %
\catcode`\}=2 %
\catcode`\#=6 %
\catcode`\@=11 %
\expandafter\ifx\csname count@\endcsname\relax
  \countdef\count@=255 %
\fi
\expandafter\ifx\csname @gobble\endcsname\relax
  \long\def\@gobble#1{}%
\fi
\expandafter\ifx\csname @firstofone\endcsname\relax
  \long\def\@firstofone#1{#1}%
\fi
\expandafter\ifx\csname loop\endcsname\relax
  \expandafter\@firstofone
\else
  \expandafter\@gobble
\fi
{%
  \def\loop#1\repeat{%
    \def\body{#1}%
    \iterate
  }%
  \def\iterate{%
    \body
      \let\next\iterate
    \else
      \let\next\relax
    \fi
    \next
  }%
  \let\repeat=\fi
}%
\def\RestoreCatcodes{}
\count@=0 %
\loop
  \edef\RestoreCatcodes{%
    \RestoreCatcodes
    \catcode\the\count@=\the\catcode\count@\relax
  }%
\ifnum\count@<255 %
  \advance\count@ 1 %
\repeat

\def\RangeCatcodeInvalid#1#2{%
  \count@=#1\relax
  \loop
    \catcode\count@=15 %
  \ifnum\count@<#2\relax
    \advance\count@ 1 %
  \repeat
}
\def\RangeCatcodeCheck#1#2#3{%
  \count@=#1\relax
  \loop
    \ifnum#3=\catcode\count@
    \else
      \errmessage{%
        Character \the\count@\space
        with wrong catcode \the\catcode\count@\space
        instead of \number#3%
      }%
    \fi
  \ifnum\count@<#2\relax
    \advance\count@ 1 %
  \repeat
}
\def\space{ }
\expandafter\ifx\csname LoadCommand\endcsname\relax
  \def\LoadCommand{\input ifluatex.sty\relax}%
\fi
\def\Test{%
  \RangeCatcodeInvalid{0}{47}%
  \RangeCatcodeInvalid{58}{64}%
  \RangeCatcodeInvalid{91}{96}%
  \RangeCatcodeInvalid{123}{255}%
  \catcode`\@=12 %
  \catcode`\\=0 %
  \catcode`\%=14 %
  \LoadCommand
  \RangeCatcodeCheck{0}{36}{15}%
  \RangeCatcodeCheck{37}{37}{14}%
  \RangeCatcodeCheck{38}{47}{15}%
  \RangeCatcodeCheck{48}{57}{12}%
  \RangeCatcodeCheck{58}{63}{15}%
  \RangeCatcodeCheck{64}{64}{12}%
  \RangeCatcodeCheck{65}{90}{11}%
  \RangeCatcodeCheck{91}{91}{15}%
  \RangeCatcodeCheck{92}{92}{0}%
  \RangeCatcodeCheck{93}{96}{15}%
  \RangeCatcodeCheck{97}{122}{11}%
  \RangeCatcodeCheck{123}{255}{15}%
  \RestoreCatcodes
}
\Test
\csname @@end\endcsname
\end
%    \end{macrocode}
%    \begin{macrocode}
%</test1>
%    \end{macrocode}
%
% \section{Reload check for plain}
%
%    \begin{macrocode}
%<*test-reload1>
\input ifluatex.sty\relax
\input ifluatex.sty\relax
\csname @@end\endcsname\end
%</test-reload1>
%    \end{macrocode}
%
%    \begin{macrocode}
%<*test-reload2>
\input miniltx.tex\relax
\input ifluatex.sty\relax
\input ifluatex.sty\relax
\csname @@end\endcsname\end
%</test-reload2>
%    \end{macrocode}
%
% \section{Installation}
%
% \subsection{Download}
%
% \paragraph{Package.} This package is available on
% CTAN\footnote{\url{http://ctan.org/pkg/ifluatex}}:
% \begin{description}
% \item[\CTAN{macros/latex/contrib/oberdiek/ifluatex.dtx}] The source file.
% \item[\CTAN{macros/latex/contrib/oberdiek/ifluatex.pdf}] Documentation.
% \end{description}
%
%
% \paragraph{Bundle.} All the packages of the bundle `oberdiek'
% are also available in a TDS compliant ZIP archive. There
% the packages are already unpacked and the documentation files
% are generated. The files and directories obey the TDS standard.
% \begin{description}
% \item[\CTAN{install/macros/latex/contrib/oberdiek.tds.zip}]
% \end{description}
% \emph{TDS} refers to the standard ``A Directory Structure
% for \TeX\ Files'' (\CTAN{tds/tds.pdf}). Directories
% with \xfile{texmf} in their name are usually organized this way.
%
% \subsection{Bundle installation}
%
% \paragraph{Unpacking.} Unpack the \xfile{oberdiek.tds.zip} in the
% TDS tree (also known as \xfile{texmf} tree) of your choice.
% Example (linux):
% \begin{quote}
%   |unzip oberdiek.tds.zip -d ~/texmf|
% \end{quote}
%
% \paragraph{Script installation.}
% Check the directory \xfile{TDS:scripts/oberdiek/} for
% scripts that need further installation steps.
% Package \xpackage{attachfile2} comes with the Perl script
% \xfile{pdfatfi.pl} that should be installed in such a way
% that it can be called as \texttt{pdfatfi}.
% Example (linux):
% \begin{quote}
%   |chmod +x scripts/oberdiek/pdfatfi.pl|\\
%   |cp scripts/oberdiek/pdfatfi.pl /usr/local/bin/|
% \end{quote}
%
% \subsection{Package installation}
%
% \paragraph{Unpacking.} The \xfile{.dtx} file is a self-extracting
% \docstrip\ archive. The files are extracted by running the
% \xfile{.dtx} through \plainTeX:
% \begin{quote}
%   \verb|tex ifluatex.dtx|
% \end{quote}
%
% \paragraph{TDS.} Now the different files must be moved into
% the different directories in your installation TDS tree
% (also known as \xfile{texmf} tree):
% \begin{quote}
% \def\t{^^A
% \begin{tabular}{@{}>{\ttfamily}l@{ $\rightarrow$ }>{\ttfamily}l@{}}
%   ifluatex.sty & tex/generic/oberdiek/ifluatex.sty\\
%   ifluatex.pdf & doc/latex/oberdiek/ifluatex.pdf\\
%   test/ifluatex-test1.tex & doc/latex/oberdiek/test/ifluatex-test1.tex\\
%   test/ifluatex-test2.tex & doc/latex/oberdiek/test/ifluatex-test2.tex\\
%   test/ifluatex-test3.tex & doc/latex/oberdiek/test/ifluatex-test3.tex\\
%   ifluatex.dtx & source/latex/oberdiek/ifluatex.dtx\\
% \end{tabular}^^A
% }^^A
% \sbox0{\t}^^A
% \ifdim\wd0>\linewidth
%   \begingroup
%     \advance\linewidth by\leftmargin
%     \advance\linewidth by\rightmargin
%   \edef\x{\endgroup
%     \def\noexpand\lw{\the\linewidth}^^A
%   }\x
%   \def\lwbox{^^A
%     \leavevmode
%     \hbox to \linewidth{^^A
%       \kern-\leftmargin\relax
%       \hss
%       \usebox0
%       \hss
%       \kern-\rightmargin\relax
%     }^^A
%   }^^A
%   \ifdim\wd0>\lw
%     \sbox0{\small\t}^^A
%     \ifdim\wd0>\linewidth
%       \ifdim\wd0>\lw
%         \sbox0{\footnotesize\t}^^A
%         \ifdim\wd0>\linewidth
%           \ifdim\wd0>\lw
%             \sbox0{\scriptsize\t}^^A
%             \ifdim\wd0>\linewidth
%               \ifdim\wd0>\lw
%                 \sbox0{\tiny\t}^^A
%                 \ifdim\wd0>\linewidth
%                   \lwbox
%                 \else
%                   \usebox0
%                 \fi
%               \else
%                 \lwbox
%               \fi
%             \else
%               \usebox0
%             \fi
%           \else
%             \lwbox
%           \fi
%         \else
%           \usebox0
%         \fi
%       \else
%         \lwbox
%       \fi
%     \else
%       \usebox0
%     \fi
%   \else
%     \lwbox
%   \fi
% \else
%   \usebox0
% \fi
% \end{quote}
% If you have a \xfile{docstrip.cfg} that configures and enables \docstrip's
% TDS installing feature, then some files can already be in the right
% place, see the documentation of \docstrip.
%
% \subsection{Refresh file name databases}
%
% If your \TeX~distribution
% (\teTeX, \mikTeX, \dots) relies on file name databases, you must refresh
% these. For example, \teTeX\ users run \verb|texhash| or
% \verb|mktexlsr|.
%
% \subsection{Some details for the interested}
%
% \paragraph{Attached source.}
%
% The PDF documentation on CTAN also includes the
% \xfile{.dtx} source file. It can be extracted by
% AcrobatReader 6 or higher. Another option is \textsf{pdftk},
% e.g. unpack the file into the current directory:
% \begin{quote}
%   \verb|pdftk ifluatex.pdf unpack_files output .|
% \end{quote}
%
% \paragraph{Unpacking with \LaTeX.}
% The \xfile{.dtx} chooses its action depending on the format:
% \begin{description}
% \item[\plainTeX:] Run \docstrip\ and extract the files.
% \item[\LaTeX:] Generate the documentation.
% \end{description}
% If you insist on using \LaTeX\ for \docstrip\ (really,
% \docstrip\ does not need \LaTeX), then inform the autodetect routine
% about your intention:
% \begin{quote}
%   \verb|latex \let\install=y\input{ifluatex.dtx}|
% \end{quote}
% Do not forget to quote the argument according to the demands
% of your shell.
%
% \paragraph{Generating the documentation.}
% You can use both the \xfile{.dtx} or the \xfile{.drv} to generate
% the documentation. The process can be configured by the
% configuration file \xfile{ltxdoc.cfg}. For instance, put this
% line into this file, if you want to have A4 as paper format:
% \begin{quote}
%   \verb|\PassOptionsToClass{a4paper}{article}|
% \end{quote}
% An example follows how to generate the
% documentation with pdf\LaTeX:
% \begin{quote}
%\begin{verbatim}
%pdflatex ifluatex.dtx
%makeindex -s gind.ist ifluatex.idx
%pdflatex ifluatex.dtx
%makeindex -s gind.ist ifluatex.idx
%pdflatex ifluatex.dtx
%\end{verbatim}
% \end{quote}
%
% \section{Catalogue}
%
% The following XML file can be used as source for the
% \href{http://mirror.ctan.org/help/Catalogue/catalogue.html}{\TeX\ Catalogue}.
% The elements \texttt{caption} and \texttt{description} are imported
% from the original XML file from the Catalogue.
% The name of the XML file in the Catalogue is \xfile{ifluatex.xml}.
%    \begin{macrocode}
%<*catalogue>
<?xml version='1.0' encoding='us-ascii'?>
<!DOCTYPE entry SYSTEM 'catalogue.dtd'>
<entry datestamp='$Date$' modifier='$Author$' id='ifluatex'>
  <name>ifluatex</name>
  <caption>Provides the \ifluatex switch.</caption>
  <authorref id='auth:oberdiek'/>
  <copyright owner='Heiko Oberdiek' year='2007,2009,2010'/>
  <license type='lppl1.3'/>
  <version number='1.4'/>
  <description>
    The package looks for  LuaTeX regardless of its mode and provides
    the switch <tt>\ifluatex</tt>; it works with Plain TeX or LaTeX.
    <p/>
    The package is part of the <xref refid='oberdiek'>oberdiek</xref>
    bundle.
  </description>
  <documentation details='Package documentation'
      href='ctan:/macros/latex/contrib/oberdiek/ifluatex.pdf'/>
  <ctan file='true' path='/macros/latex/contrib/oberdiek/ifluatex.dtx'/>
  <miktex location='oberdiek'/>
  <texlive location='ifluatex'/>
  <install path='/macros/latex/contrib/oberdiek/oberdiek.tds.zip'/>
</entry>
%</catalogue>
%    \end{macrocode}
%
% \begin{History}
%   \begin{Version}{2007/12/12 v1.0}
%   \item
%     First public version.
%   \end{Version}
%   \begin{Version}{2009/04/10 v1.1}
%   \item
%     Test adopted for \LuaTeX\ 0.39.
%   \item
%     Makes \cs{luatexversion} available.
%   \end{Version}
%   \begin{Version}{2009/04/17 v1.2}
%   \item
%     Fixes (Manuel P\'egouri\'e-Gonnard).
%   \item
%     \cs{luatextrue} and \cs{luatexfalse} are no longer defined.
%   \item
%     Makes \cs{luatexrevision} available, too.
%   \end{Version}
%   \begin{Version}{2010/03/01 v1.3}
%   \item
%     Line ends fixed in case \cs{endlinechar} = \cs{newlinechar}.
%   \end{Version}
%   \begin{Version}{2016/05/16 v1.4}
%   \item
%     Documentation updates.
%   \end{Version}
% \end{History}
%
% \PrintIndex
%
% \Finale
\endinput
|
% \end{quote}
% Do not forget to quote the argument according to the demands
% of your shell.
%
% \paragraph{Generating the documentation.}
% You can use both the \xfile{.dtx} or the \xfile{.drv} to generate
% the documentation. The process can be configured by the
% configuration file \xfile{ltxdoc.cfg}. For instance, put this
% line into this file, if you want to have A4 as paper format:
% \begin{quote}
%   \verb|\PassOptionsToClass{a4paper}{article}|
% \end{quote}
% An example follows how to generate the
% documentation with pdf\LaTeX:
% \begin{quote}
%\begin{verbatim}
%pdflatex ifluatex.dtx
%makeindex -s gind.ist ifluatex.idx
%pdflatex ifluatex.dtx
%makeindex -s gind.ist ifluatex.idx
%pdflatex ifluatex.dtx
%\end{verbatim}
% \end{quote}
%
% \section{Catalogue}
%
% The following XML file can be used as source for the
% \href{http://mirror.ctan.org/help/Catalogue/catalogue.html}{\TeX\ Catalogue}.
% The elements \texttt{caption} and \texttt{description} are imported
% from the original XML file from the Catalogue.
% The name of the XML file in the Catalogue is \xfile{ifluatex.xml}.
%    \begin{macrocode}
%<*catalogue>
<?xml version='1.0' encoding='us-ascii'?>
<!DOCTYPE entry SYSTEM 'catalogue.dtd'>
<entry datestamp='$Date$' modifier='$Author$' id='ifluatex'>
  <name>ifluatex</name>
  <caption>Provides the \ifluatex switch.</caption>
  <authorref id='auth:oberdiek'/>
  <copyright owner='Heiko Oberdiek' year='2007,2009,2010'/>
  <license type='lppl1.3'/>
  <version number='1.4'/>
  <description>
    The package looks for  LuaTeX regardless of its mode and provides
    the switch <tt>\ifluatex</tt>; it works with Plain TeX or LaTeX.
    <p/>
    The package is part of the <xref refid='oberdiek'>oberdiek</xref>
    bundle.
  </description>
  <documentation details='Package documentation'
      href='ctan:/macros/latex/contrib/oberdiek/ifluatex.pdf'/>
  <ctan file='true' path='/macros/latex/contrib/oberdiek/ifluatex.dtx'/>
  <miktex location='oberdiek'/>
  <texlive location='ifluatex'/>
  <install path='/macros/latex/contrib/oberdiek/oberdiek.tds.zip'/>
</entry>
%</catalogue>
%    \end{macrocode}
%
% \begin{History}
%   \begin{Version}{2007/12/12 v1.0}
%   \item
%     First public version.
%   \end{Version}
%   \begin{Version}{2009/04/10 v1.1}
%   \item
%     Test adopted for \LuaTeX\ 0.39.
%   \item
%     Makes \cs{luatexversion} available.
%   \end{Version}
%   \begin{Version}{2009/04/17 v1.2}
%   \item
%     Fixes (Manuel P\'egouri\'e-Gonnard).
%   \item
%     \cs{luatextrue} and \cs{luatexfalse} are no longer defined.
%   \item
%     Makes \cs{luatexrevision} available, too.
%   \end{Version}
%   \begin{Version}{2010/03/01 v1.3}
%   \item
%     Line ends fixed in case \cs{endlinechar} = \cs{newlinechar}.
%   \end{Version}
%   \begin{Version}{2016/05/16 v1.4}
%   \item
%     Documentation updates.
%   \end{Version}
% \end{History}
%
% \PrintIndex
%
% \Finale
\endinput
|
% \end{quote}
% Do not forget to quote the argument according to the demands
% of your shell.
%
% \paragraph{Generating the documentation.}
% You can use both the \xfile{.dtx} or the \xfile{.drv} to generate
% the documentation. The process can be configured by the
% configuration file \xfile{ltxdoc.cfg}. For instance, put this
% line into this file, if you want to have A4 as paper format:
% \begin{quote}
%   \verb|\PassOptionsToClass{a4paper}{article}|
% \end{quote}
% An example follows how to generate the
% documentation with pdf\LaTeX:
% \begin{quote}
%\begin{verbatim}
%pdflatex ifluatex.dtx
%makeindex -s gind.ist ifluatex.idx
%pdflatex ifluatex.dtx
%makeindex -s gind.ist ifluatex.idx
%pdflatex ifluatex.dtx
%\end{verbatim}
% \end{quote}
%
% \section{Catalogue}
%
% The following XML file can be used as source for the
% \href{http://mirror.ctan.org/help/Catalogue/catalogue.html}{\TeX\ Catalogue}.
% The elements \texttt{caption} and \texttt{description} are imported
% from the original XML file from the Catalogue.
% The name of the XML file in the Catalogue is \xfile{ifluatex.xml}.
%    \begin{macrocode}
%<*catalogue>
<?xml version='1.0' encoding='us-ascii'?>
<!DOCTYPE entry SYSTEM 'catalogue.dtd'>
<entry datestamp='$Date$' modifier='$Author$' id='ifluatex'>
  <name>ifluatex</name>
  <caption>Provides the \ifluatex switch.</caption>
  <authorref id='auth:oberdiek'/>
  <copyright owner='Heiko Oberdiek' year='2007,2009,2010'/>
  <license type='lppl1.3'/>
  <version number='1.4'/>
  <description>
    The package looks for  LuaTeX regardless of its mode and provides
    the switch <tt>\ifluatex</tt>; it works with Plain TeX or LaTeX.
    <p/>
    The package is part of the <xref refid='oberdiek'>oberdiek</xref>
    bundle.
  </description>
  <documentation details='Package documentation'
      href='ctan:/macros/latex/contrib/oberdiek/ifluatex.pdf'/>
  <ctan file='true' path='/macros/latex/contrib/oberdiek/ifluatex.dtx'/>
  <miktex location='oberdiek'/>
  <texlive location='ifluatex'/>
  <install path='/macros/latex/contrib/oberdiek/oberdiek.tds.zip'/>
</entry>
%</catalogue>
%    \end{macrocode}
%
% \begin{History}
%   \begin{Version}{2007/12/12 v1.0}
%   \item
%     First public version.
%   \end{Version}
%   \begin{Version}{2009/04/10 v1.1}
%   \item
%     Test adopted for \LuaTeX\ 0.39.
%   \item
%     Makes \cs{luatexversion} available.
%   \end{Version}
%   \begin{Version}{2009/04/17 v1.2}
%   \item
%     Fixes (Manuel P\'egouri\'e-Gonnard).
%   \item
%     \cs{luatextrue} and \cs{luatexfalse} are no longer defined.
%   \item
%     Makes \cs{luatexrevision} available, too.
%   \end{Version}
%   \begin{Version}{2010/03/01 v1.3}
%   \item
%     Line ends fixed in case \cs{endlinechar} = \cs{newlinechar}.
%   \end{Version}
%   \begin{Version}{2016/05/16 v1.4}
%   \item
%     Documentation updates.
%   \end{Version}
% \end{History}
%
% \PrintIndex
%
% \Finale
\endinput
|
% \end{quote}
% Do not forget to quote the argument according to the demands
% of your shell.
%
% \paragraph{Generating the documentation.}
% You can use both the \xfile{.dtx} or the \xfile{.drv} to generate
% the documentation. The process can be configured by the
% configuration file \xfile{ltxdoc.cfg}. For instance, put this
% line into this file, if you want to have A4 as paper format:
% \begin{quote}
%   \verb|\PassOptionsToClass{a4paper}{article}|
% \end{quote}
% An example follows how to generate the
% documentation with pdf\LaTeX:
% \begin{quote}
%\begin{verbatim}
%pdflatex ifluatex.dtx
%makeindex -s gind.ist ifluatex.idx
%pdflatex ifluatex.dtx
%makeindex -s gind.ist ifluatex.idx
%pdflatex ifluatex.dtx
%\end{verbatim}
% \end{quote}
%
% \section{Catalogue}
%
% The following XML file can be used as source for the
% \href{http://mirror.ctan.org/help/Catalogue/catalogue.html}{\TeX\ Catalogue}.
% The elements \texttt{caption} and \texttt{description} are imported
% from the original XML file from the Catalogue.
% The name of the XML file in the Catalogue is \xfile{ifluatex.xml}.
%    \begin{macrocode}
%<*catalogue>
<?xml version='1.0' encoding='us-ascii'?>
<!DOCTYPE entry SYSTEM 'catalogue.dtd'>
<entry datestamp='$Date$' modifier='$Author$' id='ifluatex'>
  <name>ifluatex</name>
  <caption>Provides the \ifluatex switch.</caption>
  <authorref id='auth:oberdiek'/>
  <copyright owner='Heiko Oberdiek' year='2007,2009,2010'/>
  <license type='lppl1.3'/>
  <version number='1.4'/>
  <description>
    The package looks for  LuaTeX regardless of its mode and provides
    the switch <tt>\ifluatex</tt>; it works with Plain TeX or LaTeX.
    <p/>
    The package is part of the <xref refid='oberdiek'>oberdiek</xref>
    bundle.
  </description>
  <documentation details='Package documentation'
      href='ctan:/macros/latex/contrib/oberdiek/ifluatex.pdf'/>
  <ctan file='true' path='/macros/latex/contrib/oberdiek/ifluatex.dtx'/>
  <miktex location='oberdiek'/>
  <texlive location='ifluatex'/>
  <install path='/macros/latex/contrib/oberdiek/oberdiek.tds.zip'/>
</entry>
%</catalogue>
%    \end{macrocode}
%
% \begin{History}
%   \begin{Version}{2007/12/12 v1.0}
%   \item
%     First public version.
%   \end{Version}
%   \begin{Version}{2009/04/10 v1.1}
%   \item
%     Test adopted for \LuaTeX\ 0.39.
%   \item
%     Makes \cs{luatexversion} available.
%   \end{Version}
%   \begin{Version}{2009/04/17 v1.2}
%   \item
%     Fixes (Manuel P\'egouri\'e-Gonnard).
%   \item
%     \cs{luatextrue} and \cs{luatexfalse} are no longer defined.
%   \item
%     Makes \cs{luatexrevision} available, too.
%   \end{Version}
%   \begin{Version}{2010/03/01 v1.3}
%   \item
%     Line ends fixed in case \cs{endlinechar} = \cs{newlinechar}.
%   \end{Version}
%   \begin{Version}{2016/05/16 v1.4}
%   \item
%     Documentation updates.
%   \end{Version}
% \end{History}
%
% \PrintIndex
%
% \Finale
\endinput
